% !TEX encoding = UTF-8 Unicode
% !TEX TS-program = XeLaTex
% !TEX root = ../EMU2016_booklet.tex

\begin{flushright}

\large{
	\scshape{
	24 ottobre 2016 -- ore 15:30 -- 18:30
	}}

\medskip
	
\small{Conferenza
	\newline Sala Medaglioni}

\medskip

{\fontsize{42}{42} \svolk{\emph{Caso I}}}

\normalfont

\normalsize

\bigskip

Conferenza tenuta da \textsc{Giorgio Netti} con la partecipazione dell' \textsc{mdi ensemble}

\bigskip

\textbf{\emph{Sulla volontà del suono}}

Creazione e interpretazione musicale


\end{flushright}

\clearpage

\begin{flushleft}

~\vfill
\large{
	\scshape{
	25 ottobre 2016 -- ore 11:00 -- 13:00
	}}

\medskip
	
\small{Conferenza
	\newline Aula Bianchini}

\medskip

{\fontsize{42}{42} \svolk{\emph{Caso II}}}

\normalfont

\normalsize

\bigskip

Seminario tenuto da \textsc{Anna Terzaroli}

\bigskip

\textbf{\emph{Estrazione delle caratteristiche musicali del suono e loro utilizzo nella Composizione}}

Il seminario riguarda l'estrazione delle caratteristiche musicali del suono, un campo di ricerca di recente costituzione ed in ampia espansione. 

Più specificamente, l'argomento dell'incontro verterà sul significato e sull'utilizzo dell'estrazione delle caratteristiche musicali del suono nelle discipline compositive. A tal proposito verranno analizzate la teoria e la pratica dei principali algoritmi di estrazione e verrano illustrati esempi della loro applicazione alla composizione.

\end{flushleft}