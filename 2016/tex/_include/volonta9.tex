% !TEX encoding = UTF-8 Unicode
% !TEX TS-program = XeLaTex
% !TEX root = ../EMU2016_booklet.tex

\begin{flushright}

\large{
	\scshape{
	28 ottobre 2016 -- ore 20:30
	}}

\medskip
	
\small{Concerto
	\newline Sala Accademica}

\medskip

{\fontsize{42}{42} \svolk{\emph{Volontà IX}}}

\normalsize

\medskip

regia del suono \textsc{Pasquale Citera} e \textsc{Giuseppe Desiato}

\bigskip

\livel{Simone Cardini}{Dereistically}{7’30}{per chitarra}{2016}
\medskip

\livel{Roberto Zanata}{Sax Live}{8'30}{per sax tenore e live electronics}{2015}
\medskip

\livel{Luca Richelli}{Ricercare... e non trovare}{8'50}{per flauto in sol e live electronics}{2015}
\medskip

\livel{Levy Oliveira}{Por um triz!}{7'08}{for piano and tape}{2016}
\medskip

\livel{Reuben de Lautour}{Undertow}{9’12}{for flute and live electronics}{2016}
\medskip

\livel{Robert Scott Thompson}{METTĀ}{14’00}{for soprano saxophone, percussion and live electronics}{2014}

\bigskip
 	 
\esecutore{chitarra}{Arturo Tallini}
\esecutore{sax tenore}{Filippo Ansaldi}
\esecutore{flauto contralto}{Elena D'Alò}
\esecutore{pianoforte}{Sara Ferrandino}
\esecutore{sax soprano}{Enzo Filippetti}
\esecutore{percussioni}{Ivan Liuzzo}
\esecutore{live electronics}{Massimiliano Mascaro, Roberto Zanata, Luca Richelli, Reuben de Lautour, Federico Ripanti}

\vfill

\descrizione{Dereistically}{Viene definito dereistico quel pensiero che abbia smarrito i propri legami con la realtà, con la logica. La visione antinomica della realtà così proposta sembra ignorare l'evidenza (e gli sforzi di jankélévitchiana memoria a riguardo) di un mondo dalle molteplici possibilità crepuscolari; quasi la ragione stessa volesse persuaderci, per mezzo di questa confortante convizione, ed estromettere ciò che non le è utile per confermarsi, ipso facto, ragionevolmente ragione. Dereisticamente, l'interprete supera il progetto del brano stesso attualizzando questa sorta di trio per chitarra sola attraverso la propria alterità e responsabilità: sarà, però, l'organizzazione del mondo percittivo di ciascuno a realizzare e rendere reale il brano stesso. Le tre parti per voce, chitarra e percussioni e il necessario rapporto tra compositore, interprete e pubblico, offrono dei confini embricati che si amalgamano, si fondono e confondono fino a giungere ad una reciproca dissolvenza. Il sé interiorizzato rimane per me una trascendenza lacerata che necessita un'esperienza sociale per essere espressa.}

\end{flushright}

\clearpage

\begin{flushleft}

~\vfill

\descrizione{Sax Live}{Secondo brano della trilogia \emph{Live}, questa composizione è concepita come creazione di relazione e autonomia tra la notazione dello strumento e l’improvvisazione del suo trattamento elettronico dal vivo.}

\descrizione{Ricercare... e non trovare}{L’integrazione tra il suono acustico ed il suono elettronico è il punto di partenza della composizione Ricercare … e non trovare. Nella partitura i parametri tradizionali della scrittura musicale – melodia, armonia e ritmo - sono fusi con i parametri tipici del mondo elettroacustico – timbro, texture e densità. Il suono inizia il suo percorso di ibridazione già nel mondo acustico per proseguire nella controparte elettronica. Le trasformazione elettroacustiche si pongono su due differenti livelli: il primo, percettivo, di trasformazione timbrica - harmonizer, il secondo, formale, di componente strutturale – loop. Lo spazio sonoro multicanale è l’espansione virtuale dello strumento acustico. La composizione, per flauto in sol ed elettronica, esplora la zona di confine tra il suono acustico e quello elettronico. Il live electronics riveste principalmente il ruolo di amplificatore del gesto musicale, espandendo le possibilità espressive dello strumento acustico: le elaborazione elettroniche sono sempre fuse con la controparte acustica che le ha generate. La stratificazione progressiva dei materiali crea un'eco onirica e straniante. Il titolo allude in modo ironico alla condizione del compostore contemporaneo che spesso rischia di perdersi nella propria ricerca artistica.}

\descrizione{Por um triz!}{\emph{Por um triz!} (C'era quasi!) usa una gran quantità di suoni registrati ed elettronici che interagiscono con il pianoforte. La parte elettronica amplifica ciò che il pianista suona, arricchendo la complessità strutturale, esasperando i gesti ed enfatizzando le armonie. In alcune parti del pezzo si utilizzano suoni di piano registrati per raggiungere le timbriche dello strumento originale con l'elettronica. Il pezzo è stato creato nello studio personale del compositore e al Centro di Ricerca di Musica Contemporanea dell'Università Federale di Minas Gerais (Brasile).}

\descrizione{Undertow}{è una composizione per flauto ed elettronica diffusa attraverso un laptop in stereofonia o quadrifonia. Il brano trae ispirazione da un passo del romanzo \emph{Il museo dell'innocenza} di Orhan Pamuk in cui il protagonista, mentre nuota, descrive gli abissi del Bosforo. Al di sotto della superficie il personaggio nota una quantità di detriti: barche affondate, automobili sommerse, valige smarrite e biciclette. Questi oggetti sommersi, quasi fluttuanti su diversi livelli paralleli, formano una sorta di storia di ricordi persi e storie dimenticate. Formalmente la composizione segue la graduale evoluzione di un gesto del flauto: da una semplice frase dal profilo  increspato si giunge  a una scrittura dalle caratteristiche melodiche e figurative più ampie. L'elettronica supporta ed elabora le caratteristiche spettrali e transienti del flauto e, seguendo l'evoluzione del brano, introduce una serie di lunghe  trame stocastiche modulate. L'elettronica combina timbri che imitano e arricchiscono i suoni emessi dal flauto e una tavolozza di trame più astratte che tentano di rievocare l'idea dei diversi oggetti intrappolati al di sotto della superficie oceanica. I multifonici del flauto vengono riproposti in diverse versioni risintetizzate e modulate a diversi fini musicali; talvolta a queste multivonici vengono aggiunte parziali acute per sottolineare i registri più stabili del flauto in altri momenti questi vengono utilizzati per generare tessiture accordali al cui interno viene inglobata una scrittura flautistica maggiormente melodica. In altri casi ancora; singole parziali vengono estratte dal multifonico e quindi modulate e sovrapposte al flauto per intersecare trame maggiormente eterogenee. I suoni di natura percussiva del flatuo, come il pizzicato o i suoni di chiave, vengono campionati e trasformati in gesti musicali che rispondono alla linea del flauto principale.}

\descrizione{METTĀ}{Il concetto buddista del \emph{mettā} è centrato sulla ricerca dell'affettuosità ed è una pietra angolare della meditazione della compassione. Tra i vari significati associati al termine vi è il concetto dell'unione mentale - cioè dell' \emph{essere sulla stessa lunghezza d'onda}-.  La composizione mira a creare uno spazio sonoro rituale e meditativo per arrivare all'unione mentale attraverso la concentrazione, la contemplazione e la compassione. Campanelli, foglie soffiate, bastoncini rotti, gong e campane si fondono con suoni atmosferici di sassofono e percussioni nella parte elettroacustica. Questi suoni fortemente trasformati vengono combinati con un elaborazione sonora creata dai performer. Alle volte, la profonda integrazione delle componenti elettroacustiche e quelle dal vivo crea una sfumatura di confine tra il reale e l'immaginario, tra l'effimero ed il concreto, tra il misurabile ed il senza tempo. I materiali sonori per la composizione sono stati creati usando 'Metasynth' e provengono dallo studio di registrazione fatto dai performer. \emph{Mettā} è dedicato a Jan Berry Baker and Stuart Gerber.}

\end{flushleft}

\biografia{Simone Cardini}{Studia composizione con F. Telli e pianoforte con A. Torchiani; partecipa a masterclass e seminari tenuti da I. Fedele, A. Solbiati, S. Sciarrino, M. Andre, M. Lanza, T. Tulev, M. Trojahn, G. Giuliano, A. Gilardino, C. Antonelli, G. D'Alò, A. Di Pofi, S. Mastrangelo, L. Verdi, M. Silvi, G. Garrera, D. Przybylski, G. Westley, M. D. Cisneros, P. Mykietyn, P. Manoury. Sue composizioni sono state eseguite in Europa e USA in rassegne e festival quali ArteScienza (2012), Contemporanea (2013), Nuova Consonanza (2013, 2014, 2015), Rondò (2014), NYCEMF (2015) da ensemble internazionali come Divertimento Ensemble, PMCE e sono state selezionate e premiate in competizioni come AFAM (2014), Opus Dissonus (2015), T. K. International Guitar Competition (2016), Valentino Bucchi 37th ed. (2015), etc. L’elaborato Musica e Architettura – Implicazioni estetiche e sociologiche è stato pubblicato nel libro Musica & Architettura, Edizioni Nuova Cultura (2012). Il brano Threshold sarà pubblicato dalla Universal Edition.}

\biografia{Roberto Zanata}{nato a Cagliari, laureato in filosofia e diplomato in composizione e musica elettronica. Ha scritto opere per musica da camera, musica acusmatica e multimedia. Attualmente insegna musica elettronica presso il Conservatorio \emph{Monteverdi} di Bolzano.}

\biografia{Luca Richelli}{Live electronics performer e compositore. Diplomato in Pianoforte, Composizione, Musica Elettronica, Composizione e Nuove Tecnologie, Regia del Suono. Docente Composizione Musicale Elettroacustica di Como, di Informatica Musicale ed Elettroacustica presso il Conservatorio \emph{F.A. Bonporti} di Trento e coordinatore del SaMPL (Sound and Music Processing Lab) del Conservatorio di \emph{C. Pollini} di Padova. Svolge attività concertistica nell’ambito del live-electronics in numerose rassegne musicali e ha scritto, su commissione IRCAM, il manuale online della libreria OMChroma per l'ambiante grafico di aiuto alla composizione OpenMusic.}

\biografia{Levy Oliveira}{(1993) è un compositore Brasiliano. Studia composizione presso l'università federale di Minas Gerais(UFMG), seguito da João Pedro Oliveira. Levy è interessato sia alla musica elettronica che a quella acustica. Le sue composizioni sono state eseguite in diversi Festival, tra i più recenti: Monaco Electroacoustique 2015 (Monaco/Monaco), MUSLAB 2015 (Mexico City/Mexico), JIMEC 2015 (Amiens/France), Open Circuit 2016 (Liverpool/UK) and Tinta Fresca 2016 (Belo Horizonte/Brazil). Il suo lavoro Hyperesthesia ha ricevuto il primo premio del Destellos Electronic Composition Competition ed è stato finalista dell'Open Circuit Composition Prize. Il suo brano orchestrale Um ato de fé ha ricevuto una menzione  preso il festival Tinta Fresca del 2016.}

\biografia{Reuben de Lautour}{Il neozelandese Ruben de Lautour è un compositore, artista sonoro e pianista. È docente presso la Istanbul Technical University's Center for Advanced Studies in Music, dove ha fondato il Program in Sonic Arts in 2012. Compone musica per solisti o ensemble ed elettronica, pubblica saggi su musica, tecnologia e pratiche di ascolto. La sua musica è stata eseguita e registrata da artisti come Evelyn Glennie, the Nash Ensemble, the New Jersey Symphony and UMS 'n JIP. Ha conseguito un dottorato presso l'università di Princeton dove ha studiato con Paul Lansky e Steven Mackey; precedentemente ha studiato pianoforte e composizione presso l'università di Aukland in Nuova Zelanda con Bryan Sayer, John Rimmer e John Elmsly.}

\biografia{Robert Scott Thompson}{Robert Scott Thompson è un compositore di musica strumentale ed elettronica ed è docente di composizione presso la Georgia State University di Atlanta. Ha ricevuto diversi premi e riconoscimenti  per le sue composizioni, tra cui: il primo premio nel 2013 presso la Musica Nova Competition,  il primo premio nel 2001 presso la Pierre Schaeffer Competition e premi dal Concorso Internazionale \emph{Luigi Russolo}, Irino Prize Foundation Competition for Chamber Music, e Concours International de Musique Electroacoustique de Bourges tra cui la Commande Commission 2007. I suoi lavori sono stati presentati in diversi festival tra cui: Koriyama Bienalle, Helsinki Bienalle, Sound, Présences, Synthèse, Sonorities, ICMC, SEAMUS e il Cabrillo Music Festival, e diffusi Radio France, BBC, NHK, ABC, WDR, and NPR. Le sue composizioni sono state pubblicate su dischi solisti e raccolte di diverse etichette, tra cui: EMF Media, Neuma, Drimala, Capstone, Hypnos, Oasis/Mirage, Groove, Lens, Space for Music, Zero Music, Twelfth Root, Relaxed Machinery and Aucourant.}


