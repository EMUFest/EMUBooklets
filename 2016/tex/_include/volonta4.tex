% !TEX encoding = UTF-8 Unicode
% !TEX TS-program = XeLaTex
% !TEX root = ../EMU2016_booklet.tex

\begin{flushright}

\large{
	\scshape{
	26 ottobre 2016 -- ore 18:00
	}}

\medskip
	
\small{Concerto Acusmatico
	\newline Il Suono di Piero [Aula Bianchini]}

\medskip

{\fontsize{42}{42} \svolk{\emph{Volontà IV}}}

\normalsize

\bigskip

\livel{Wilfried Jentzsch}{Entre ciel et terre (between heaven and earth)}{13’45}{}{2015}
\medskip

\livel{Benjamin O’Brien}{OSCines}{6'08}{}{2013}
\medskip

\livel{Clelia Patrono}{Blue4Notes}{8'27}{}{2016}
\medskip

\livel{Daniele Pozzi}{Breakpoint}{6'24}{}{2016}
\medskip

%\brano{Christian Eloy}
%{La cicatrice d'Ulysse}{13'00''}
%{acusmatico}
%new version 2015\\

%\acusmatici{Ursula Meyer-K\"onig}
%{Allears}{2012-13}{8'}

%\vspace{6mm}

\vfill

\descrizione{Entre ciel et terre (between heaven and earth)}{\emph{L’intervallo tra Paradiso e Terra è come un grande tubo: vuoto, ma non crolla: si muove,  generando sempre più} (Daodejing, Part 1 (5), Wikipedia). Il verso è stato la fonte spirituale di questa composizione elettroacustica. Il paradiso, la terra nello spazio circostante hanno prodotto la concezione di un suono circolare multidimensionale. I movimenti spaziali del suono sono caratterizzati da varie configurazioni con il variare della velocità, della direzione e della distanza dall’ascoltatore. Il materiale del suono è basato su tre elementi: i cimbali cinesi, il canto degli uccelli e il canto medievale (Machaut). Questi suoni derivano da diverse culture, epoche e anche da diverse nature. Il suono è stato sintetizzato usando vari metodi di cross synthesis. Con l’aiuto di questi metodi evoluti di sviluppo digitale del suono del computer, si è in grado di produrre nuove qualità di suono. Questa composizione è stata premiata il 23 Marzo 2016 all’Espace Senghor Brussel ed è dedicata a Annette vande Gorne.}

\descrizione{OSCines}{\emph{OSCines} si basa sul processo di traduzione di melodie da canti di uccelli. l'usignolo appartiene alla famiglia dei \emph{Passeri}, anche conosciuti come Oscine, dal latino \emph{oscen} (uccello canterino). Il suo canto è composto da una vasta gamma di fischiettii, trilli e gorgoglii, i quali creano un profilo melodico ricco e movimentato. I campioni dell' usignolo e del clarinetto fungono - alternativamente - da sorgente e da obbiettivo sonoro per informazioni spettrali elaborate da un sistema di acquisizione del segnale processato. \emph{OSCines} esplora gli allineamenti e le collisioni di precise caratteristiche timbriche e topologie melodiche in una voliera virtuale di altoparlanti nello spazio stereofonico.}

%\bigskip

%\svolk{\emph{Ho cercato di incanalare quell’energia in un percorso che la rendesse percepibile senza snaturarne l’essenza: l’energia che abita i violini di Corelli, Tartini, Vivaldi, certo non citazioni ne dirette ne indirette ma l’imprevedibile vitalità delle loro articolazioni (tremoli, arpeggi, ribattuti, sincronie e fioriture improvvise) che hanno fatto della scuola italiana, elettrica ante litteram, un irraggiungibile modello di virtuosismo strumentale; poi la tensione, il vuoto attorno e dentro alla costruzione delle frasi, le imitazioni, i pedali, la continua sovrapposizione delle corde.}}
%
%Giorgio Netti
%
%\normalfont

\end{flushright}

\clearpage

\begin{flushleft}

~\vfill

\descrizione{Blue4Notes}{Brano composto in Quattro movimenti. I suoni utilizzati sono suoni concreti e suoni prodotti da una chitarra suonata con E-BOW(Electronic Bow). Tutti i suoni sono stati trattati e rielaborati con l’utilizzo di sintesi granulare, equalizzatori e filtri di risonanza.}

\descrizione{Breakpoint}{Interruzione intenzionale. Smembramento, segmentazione: esitazioni compositive si combinano anprocedure concatenative emergenti, sovrapposte e contrapposte in un incessante processo astratto di sgretolamento e riassemblamento che sempre viene interrotto prima di raggiungere il contorno di una costruzione. Singoli atomi sonori appaiono in fuggevoli ed effimeri cumuli musicali, esponendo brevemente la propria struttura e tensione morfologica, tirandosi e spingendosi l’un l’altro in dolorose torsioni plastiche. Storti e piegati vibrano in rigidi mucchi nervosi, incastrati in un gioco di forze che solo li conduce ad accartocciarsi su sé stessi, sgualciti e stropicciati.}


\end{flushleft}
