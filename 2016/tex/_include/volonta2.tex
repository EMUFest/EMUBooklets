% !TEX encoding = UTF-8 Unicode
% !TEX TS-program = XeLaTex
% !TEX root = ../EMU2016_booklet.tex

\begin{flushright}

\large{
	\scshape{
	25 ottobre 2016 -- ore 18:00
	}}

\medskip
	
\small{Concerto Acusmatico
	\newline Il Suono di Piero [Aula Bianchini]}

\medskip

{\fontsize{42}{42} \svolk{\emph{Volontà II}}}

\normalsize

\bigskip

\livel{Massimo Vito Avantaggiato}{ATLAS OF UNCERTAINTY}{7’00}{}{2015}
\medskip

\livel{Marcela Pavia}{Aleph}{9’30}{}{2013}
\medskip

\livel{Hiromi Ishii}{Ryojinfu}{10’09}{}{2013}
\medskip

\livel{David Ledoux}{Marfa}{13’20}{}{2014}
\medskip

%\brano{Christian Eloy}
%{La cicatrice d'Ulysse}{13'00''}
%{acusmatico}
%new version 2015\\

%\acusmatici{Ursula Meyer-K\"onig}
%{Allears}{2012-13}{8'}

%\vspace{6mm}

\vfill

%\bigskip

%\svolk{\emph{Ho cercato di incanalare quell’energia in un percorso che la rendesse percepibile senza snaturarne l’essenza: l’energia che abita i violini di Corelli, Tartini, Vivaldi, certo non citazioni ne dirette ne indirette ma l’imprevedibile vitalità delle loro articolazioni (tremoli, arpeggi, ribattuti, sincronie e fioriture improvvise) che hanno fatto della scuola italiana, elettrica ante litteram, un irraggiungibile modello di virtuosismo strumentale; poi la tensione, il vuoto attorno e dentro alla costruzione delle frasi, le imitazioni, i pedali, la continua sovrapposizione delle corde.}}
%
%Giorgio Netti
%
%\normalfont

\end{flushright}

\clearpage

\begin{flushleft}

~\vfill

\descrizione{Atlas of Uncertainty}{è un brano di musica concreta, nel quale un microcosmo di suoni, spesso assai distanti tra loro,  viene esplorato attraverso varie tecniche di manipolazione sonora: segnali sonori  generati da eletrodomestici ; texture create impiegando suoni di campana tibetana o altre percussioni; \emph{whooshes} di rumore bianco; accumulazioni granulari,  solo per citarne alcuni. Questi suoni sono combinati in gesti articolati  in vario modo e molto ben riconoscibili. Prima esecuzione: Casa del Suono di Parma, may 2016.}

\descrizione{Aleph}{è stato composto per l’Acusmonium Audior. Il titolo si riferisce all’omonimo breve racconto di Borges \emph{the projection of the Whole in one point of the Space -micro cosmo- which reflects the whole Universe -macro cosmos}. La continua trasformazione del materiale musicale non c'è, ma c'è una timeline per ognuno degli stadi di trasformazione; viceversa  \emph{Aleph} è il congelamento della trasformazione continua. Tempo e spazio diventano uniti, diventano la stessa cosa. Se fosse dato abbastanza tempo qualsiasi cosa potrebbe diventare qualcos'altro e questo potrebbe accadere gradualmente o in modo brusco, ad alta o a bassa velocità e con tutti gli stadi intermedi in un percorso esplicito o non esplicito. Il risultato della trasformazione potrebbe essere differente: le storie e la storia sono determinate dalla casualità, ma la serie degli eventi non è unica. L'elettronica ha permesso di entrare nella struttura interna del suono aprendo la mente alla possibilità di forma come \emph{evoluzione nel tempo} di questa intima struttura. Il suono diventa non solo il pilastro del discorso ma il discorso stesso apre altre possibilità anche per una composizione esclusivamente acustica.}

\descrizione{}{}

\descrizione{}{}


\end{flushleft}
