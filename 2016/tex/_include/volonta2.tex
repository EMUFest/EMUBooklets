% !TEX encoding = UTF-8 Unicode
% !TEX TS-program = XeLaTex
% !TEX root = ../EMU2016_booklet.tex

\begin{flushright}

\large{
	\scshape{
	25 ottobre 2016 -- ore 18:00
	}}

\medskip
	
\small{Concerto Acusmatico
	\newline Il Suono di Piero [Aula Bianchini]}

\medskip

{\fontsize{42}{42} \svolk{\emph{Volontà II}}}

\normalsize

\medskip

regia del suono \textsc{Francesco Bianco}

\bigskip

\livel{Massimo Vito Avantaggiato}{ATLAS OF UNCERTAINTY}{7’00}{}{2015}
\medskip

\livel{Marcela Pavia}{Aleph}{9’30}{}{2013}
\medskip

\livel{Hiromi Ishii}{Ryojinfu}{10’09}{}{2013}
\medskip

\livel{David Ledoux}{Marfa}{13’20}{}{2014}
\medskip

%\brano{Christian Eloy}
%{La cicatrice d'Ulysse}{13'00''}
%{acusmatico}
%new version 2015\\

%\acusmatici{Ursula Meyer-K\"onig}
%{Allears}{2012-13}{8'}

%\vspace{6mm}

\vfill

\descrizione{Atlas of Uncertainty}{è un brano di musica concreta, nel quale un microcosmo di suoni, spesso assai distanti tra loro,  viene esplorato attraverso varie tecniche di manipolazione sonora: segnali sonori  generati da eletrodomestici ; texture create impiegando suoni di campana tibetana o altre percussioni; \emph{whooshes} di rumore bianco; accumulazioni granulari,  solo per citarne alcuni. Questi suoni sono combinati in gesti articolati  in vario modo e molto ben riconoscibili. Prima esecuzione: Casa del Suono di Parma, may 2016.}

\descrizione{Aleph}{è stato composto per l’Acusmonium Audior. Il titolo si riferisce all’omonimo breve racconto di Borges \emph{the projection of the Whole in one point of the Space -micro cosmo- which reflects the whole Universe -macro cosmos}. La continua trasformazione del materiale musicale non c'è, ma c'è una timeline per ognuno degli stadi di trasformazione; viceversa  \emph{Aleph} è il congelamento della trasformazione continua. Tempo e spazio diventano uniti, diventano la stessa cosa. Se fosse dato abbastanza tempo qualsiasi cosa potrebbe diventare qualcos'altro e questo potrebbe accadere gradualmente o in modo brusco, ad alta o a bassa velocità e con tutti gli stadi intermedi in un percorso esplicito o non esplicito. Il risultato della trasformazione potrebbe essere differente: le storie e la storia sono determinate dalla casualità, ma la serie degli eventi non è unica. L'elettronica ha permesso di entrare nella struttura interna del suono aprendo la mente alla possibilità di forma come \emph{evoluzione nel tempo} di questa intima struttura. Il suono diventa non solo il pilastro del discorso ma il discorso stesso apre altre possibilità anche per una composizione esclusivamente acustica.}


\descrizione{Ryojinfu}{Questa fantasia sonora multicanale è stata ispirata da una leggenda di un imperatore giapponese religioso, e dedicata ad Imayo (canto Buddhista), ma ha dovuto combattere diverse battaglie. I materiali sonori sono: 1. Canto (maschile solo) voce di Imayo, 2. Suoni e rumori registrati durante la cerimonia Buddhista, 3. Suoni granulari di riso. Il tutto principalmente processato usando cross synthesis e sintesi granulare. I suoni processati offrono differenti caratteristiche di movimento e di disegno in uno spazio tridimensionale; il suono processato del 1. appare con variazioni (ma mai come il suono originale), e conduce alla fine ad un suono simile alla voce di un ragazzo. I suoni massicci 2. si muovono lentamente sviluppando una sorta di muro sonoro. I suoni prodotti dal 3. sono invece veloci ed irregolari come il volo di uccelli. Questo pezzo fu composto usando il sistema del suono spazializzato di Zirkonium.}

\descrizione{Marfa}{Scene e personaggi da film come 127 Hours, There Will Be Blood, The Road e No Country For Old Man hanno ispirato Marfa. Il titolo si riferisce ad un'area nel deserto del Texas (US) dove alcuni di questi film furono girati e anche conosciuti per le loro misteriose visioni di luci notturne. Come il lettore il quale sviluppa un'immagine mentale di quello che legge, questi tre movimenti sono una descrizione acustica di emozioni/ localizzazioni per la mente, viaggiando attraverso ansia, contemplazione ed elasticità. Come parte dei suoi studi compositivi elettroacustici, Marfa è il primo tentativo di Ledoux sulla composizione di un ambiente ad ascolto 3D - per cupola. Attraverso il corso semestrale the Fall 2014, Ledoux ha imparato ad usare il software ZKM di Zirkonium in compagnia del plugin ZirkOSCII con il prof. Robert Normandeau. Questa combinazione di strumenti gli permise di organizzare lo spazio e la musica simultaneamente, usando la tecnica di spazializzazione VBAP, e a creare un'esperienza musicale profonda che poteva essere suonata su ogni cupola tridimensionale preposta. Marfa è stato già presentato Durante gli Ultrasons series all’università di Montréal - con cupola a quattordici diffusori,  come pure ai concerti/conferenze InSonic 2015, ospitato da ZKM Center for Art and Media (Karlsruhe, Germania) per cupola a quarantatré diffusori ed è stato anche selezionato per l’imminente CUBE Fest 2016 di ICAT: Massively Multichannel Music (Blacksburg, VA) per cupola a struttura cubica a centoventotto diffusori, alti quattro piani.}

***
%\bigskip

%\svolk{\emph{Ho cercato di incanalare quell’energia in un percorso che la rendesse percepibile senza snaturarne l’essenza: l’energia che abita i violini di Corelli, Tartini, Vivaldi, certo non citazioni ne dirette ne indirette ma l’imprevedibile vitalità delle loro articolazioni (tremoli, arpeggi, ribattuti, sincronie e fioriture improvvise) che hanno fatto della scuola italiana, elettrica ante litteram, un irraggiungibile modello di virtuosismo strumentale; poi la tensione, il vuoto attorno e dentro alla costruzione delle frasi, le imitazioni, i pedali, la continua sovrapposizione delle corde.}}
%
%Giorgio Netti
%
%\normalfont

\end{flushright}

\clearpage

\begin{flushleft}

~\vfill

\biografia{Massimo Vito Avantaggiato}{è diplomato in musica elettronica a pieni voti presso Conservatorio G. Verdi di Milano. Si interessa di linguaggi programmazione applicati all’audio e al video. Più recentemente la sua attenzione si è spostata sugli ambienti interattivi e alla realtà virtuale.  è stato selezionato recentemente nelle seguenti manifestazioni: NYCEMF 2016, New York, USA; Sound Thought 2016, Glasgow, UK; Csound Conference 2015, Saint Petersburg, Russia; LINUX Audio Conference 2015, Mainz, Germany; ATMM 2014, Ankara, Turkey; International Computer Music Conference 2014, Athens, Greece; ICMPC-APSCOM2014, Seoul, South Korea; EMS 2014, Berlin, Germany; Music and Screen Media Conference 2014,  Liverpool, England; Music as a Process,  Christ Church University, Canterbury, England; FAS2013, San José, Costa Rica.}

\biografia{Marcela Pavia}{Laureata in Composizione presso l'Universidad Nacional of Rosario (Argentina) e in Musica Elettronica presso il Conservatorio \emph{G.Verdi} di Milano (Biennio II Livello). Masterclasses con Franco Donatoni (Italia) presso l’Accademia Chigiana di Siena, Giorgy Ligeti, Ennio Morricone, Henri Pousseur e con Javier Torres Maldonado in Musica Elettronica. Artista in residenza presso il Virginia Center for the Creative Arts (USA). Artista in residenza presso il Gästeatelier Krone di Aarau (Svizzera). Premi: 2016 WPTA Composition Competition, SONOM 2012 (Electronic Music), 2012 \emph{Terre di Puglia}, 2012 Erasmus Competition Universitè VIII (Elecronic Music- Paris), \emph{Trinac 2011} (Fondazione \emph{Encuentros Internacionales de Musica Contemporanea} of Alicia Terzian); \emph{Miriam Gideon} Prize 2010 (Research for New Music Competition by the International Alliance for Women in Music) ecc. Ha partecipato a numerosi Festivals quali (selezione) 2016 WPTA International Conference, 2016 North and South Consonance (New York), 2016 Suono italiano (Stuttgart), 2016 New Music on the Bayou (Louisiana, prossimamente), 2015 Festival Inaudita (Cagliari), 2014 Semaine della Musique Eletroacustique a Lille (France), 2014 (Wroclaw) e 2013 World New Music Days (Kosice-Bratislava-Vienna), Risuonanze 2013, 2013 International Computer Music Conference (Perth), 2009 Stagione del Teatro Colon (Buenos Aires), 2013 Italian Composers Forum ecc}

\biografia{Hiromi Ishii}{ha studiato composizione in Tokyo, musica elettroacustica in Dresda e più tardi alla City University London dove le fu conferito il dottorato. la sua ricerca, \emph{Composing electroacoustic music relating to Japanese traditional music}, fu supportata da una borsa di studio dell’ ORS Award Scheme del regno unito. I suoi pezzi sono stati presentati a festival musicali e istituti in tutto il mondo come Musica Viva Lisbona, MusicAcoustica Beijing, EMF Florida, EMUfest Rome, Cynetart Dresda, JSEM-20th Tokyo, Punto y Raya Reykjavik, NYCEMF, Musiques&Recherches, ZKM e trasmessi da WDR, MDR. Nel 2006 (ZKM) e 2013 fu invitata come compositrice  al ZKM Karlsruhe. I suoi lavori recenti si basano su musica acusmatica multicanale e musica visuale per la quale compose in parallelo musica e immagini in movimento. Vanta un CD retrospettivo con la Wergo (Wind Way ARTS 8112 2). Ishii attualmente vive in Colonia. http://www.hiromi-ishii.de}

\biografia{David Ledoux}{si è laureato da poco in musica digitale all’ università di Montréal. Batterista fin dall’infanzia,  ha suonato in diverse formazioni rock prima di intraprendere i suoi studi. Le possibilità offerte dalle nuove tecnologie lo attrassero verso nuove espressioni musicali che solo la batteria non poteva soddisfare. Egli inizierà a perseguire una laurea in composizione e sound design il prossimo autunno. Con il prof. Robert Normandeau, Ledoux cominciò a comporre musica acusmatica e sviluppò un crescente interesse per le composizioni spazializzate in 3D. Il suo stile musicale, il cui obiettivo è quello di immergere e influenzare il pubblico, consiste in linea di massima ad un mix di immagini cinematografiche e paesaggi sonori, oscillando tra realismo e onirismo, conducendo l’ascoltatore dentro le proprie fantasie.}


\end{flushleft}
