% !TEX encoding = UTF-8 Unicode
% !TEX TS-program = XeLaTex
% !TEX root = ../EMU2016_booklet.tex

\begin{flushright}

\large{
	\scshape{
	25 ottobre 2016 -- ore 18:00
	}}

\medskip
	
\small{Concerto Acusmatico
	\newline Il Suono di Piero [Aula Bianchini]}

\medskip

{\fontsize{42}{42} \svolk{\emph{Volontà II}}}

\normalsize

\medskip

regia del suono \textsc{Francesco Bianco}

\bigskip

\livel{Massimo Vito Avantaggiato}{ATLAS OF UNCERTAINTY}{7’00}{}{2015}
\medskip

\livel{Marcela Pavia}{Aleph}{9’30}{}{2013}
\medskip

\livel{Hiromi Ishii}{Ryojinfu}{10’09}{}{2013}
\medskip

\livel{David Ledoux}{Marfa}{13’20}{}{2014}
\medskip

%\brano{Christian Eloy}
%{La cicatrice d'Ulysse}{13'00''}
%{acusmatico}
%new version 2015\\

%\acusmatici{Ursula Meyer-K\"onig}
%{Allears}{2012-13}{8'}

%\vspace{6mm}

\vfill

\descrizione{Atlas of Uncertainty}{è un brano di musica concreta, nel quale un microcosmo di suoni, spesso assai distanti tra loro,  viene esplorato attraverso varie tecniche di manipolazione sonora: segnali sonori  generati da eletrodomestici ; texture create impiegando suoni di campana tibetana o altre percussioni; \emph{whooshes} di rumore bianco; accumulazioni granulari,  solo per citarne alcuni. Questi suoni sono combinati in gesti articolati  in vario modo e molto ben riconoscibili. Prima esecuzione: Casa del Suono di Parma, may 2016.}

\descrizione{Aleph}{è stato composto per l’Acusmonium Audior. Il titolo si riferisce all’omonimo breve racconto di Borges \emph{the projection of the Whole in one point of the Space -micro cosmo- which reflects the whole Universe -macro cosmos}. La continua trasformazione del materiale musicale non c'è, ma c'è una timeline per ognuno degli stadi di trasformazione; viceversa  \emph{Aleph} è il congelamento della trasformazione continua. Tempo e spazio diventano uniti, diventano la stessa cosa. Se fosse dato abbastanza tempo qualsiasi cosa potrebbe diventare qualcos'altro e questo potrebbe accadere gradualmente o in modo brusco, ad alta o a bassa velocità e con tutti gli stadi intermedi in un percorso esplicito o non esplicito. Il risultato della trasformazione potrebbe essere differente: le storie e la storia sono determinate dalla casualità, ma la serie degli eventi non è unica. L'elettronica ha permesso di entrare nella struttura interna del suono aprendo la mente alla possibilità di forma come \emph{evoluzione nel tempo} di questa intima struttura. Il suono diventa non solo il pilastro del discorso ma il discorso stesso apre altre possibilità anche per una composizione esclusivamente acustica.}

%\bigskip

%\svolk{\emph{Ho cercato di incanalare quell’energia in un percorso che la rendesse percepibile senza snaturarne l’essenza: l’energia che abita i violini di Corelli, Tartini, Vivaldi, certo non citazioni ne dirette ne indirette ma l’imprevedibile vitalità delle loro articolazioni (tremoli, arpeggi, ribattuti, sincronie e fioriture improvvise) che hanno fatto della scuola italiana, elettrica ante litteram, un irraggiungibile modello di virtuosismo strumentale; poi la tensione, il vuoto attorno e dentro alla costruzione delle frasi, le imitazioni, i pedali, la continua sovrapposizione delle corde.}}
%
%Giorgio Netti
%
%\normalfont

\end{flushright}

\clearpage

\begin{flushleft}

~\vfill

\descrizione{Ryojinfu}{Questa fantasia sonora multicanale è stata ispirata da una leggenda di un imperatore giapponese religioso, e dedicata ad Imayo (canto Buddhista), ma ha dovuto combattere diverse battaglie. I materiali sonori sono: 1. Canto (maschile solo) voce di Imayo, 2. Suoni e rumori registrati durante la cerimonia Buddhista, 3. Suoni granulari di riso. Il tutto principalmente processato usando cross synthesis e sintesi granulare. I suoni processati offrono differenti caratteristiche di movimento e di disegno in uno spazio tridimensionale; il suono processato del 1. appare con variazioni (ma mai come il suono originale), e conduce alla fine ad un suono simile alla voce di un ragazzo. I suoni massicci 2. si muovono lentamente sviluppando una sorta di muro sonoro. I suoni prodotti dal 3. sono invece veloci ed irregolari come il volo di uccelli. Questo pezzo fu composto usando il sistema del suono spazializzato di Zirkonium.}

\descrizione{Marfa}{Scene e personaggi da film come 127 Hours, There Will Be Blood, The Road e No Country For Old Man hanno ispirato Marfa. Il titolo si riferisce ad un'area nel deserto del Texas (US) dove alcuni di questi film furono girati e anche conosciuti per le loro misteriose visioni di luci notturne. Come il lettore il quale sviluppa un'immagine mentale di quello che legge, questi tre movimenti sono una descrizione acustica di emozioni/ localizzazioni per la mente, viaggiando attraverso ansia, contemplazione ed elasticità. Come parte dei suoi studi compositivi elettroacustici, Marfa è il primo tentativo di Ledoux sulla composizione di un ambiente ad ascolto 3D - per cupola. Attraverso il corso semestrale the Fall 2014, Ledoux ha imparato ad usare il software ZKM di Zirkonium in compagnia del plugin ZirkOSCII con il prof. Robert Normandeau. Questa combinazione di strumenti gli permise di organizzare lo spazio e la musica simultaneamente, usando la tecnica di spazializzazione VBAP, e a creare un'esperienza musicale profonda che poteva essere suonata su ogni cupola tridimensionale preposta. Marfa è stato già presentato Durante gli Ultrasons series all’università di Montréal - con cupola a quattordici diffusori,  come pure ai concerti/conferenze InSonic 2015, ospitato da ZKM Center for Art and Media (Karlsruhe, Germania) per cupola a quarantatré diffusori ed è stato anche selezionato per l’imminente CUBE Fest 2016 di ICAT: Massively Multichannel Music (Blacksburg, VA) per cupola a struttura cubica a centoventotto diffusori, alti quattro piani.}


\end{flushleft}
