% !TEX encoding = UTF-8 Unicode
% !TEX TS-program = XeLaTex
% !TEX root = ../EMU2016_booklet.tex

\begin{flushright}

\large{
	\scshape{
	27 ottobre 2016 -- ore 18:00
	}}

\medskip
	
\small{Concerto Acusmatico
	\newline Il Suono di Piero [Aula Bianchini]}

\medskip

{\fontsize{42}{42} \svolk{\emph{Volontà VI}}}

\normalsize

\medskip

regia del suono \textsc{Leonardo Mammozzetti}

\bigskip

\livel{Alexis Langevin-Tétrault}{Apax}{8'02}{}{2015}
\medskip

\livel{Gilles Gobeil}{Un cercle hors de l'arbre}{10'00}{}{2014-15}
\medskip

\livel{Demian Rudel Rey}{Che-toi}{8'15}{}{2016}
\medskip

\livel{Vanessa Massera}{Éclats de Feux}{10'07}{}{2016}
\medskip

%\brano{Christian Eloy}
%{La cicatrice d'Ulysse}{13'00''}
%{acusmatico}
%new version 2015\\

%\acusmatici{Ursula Meyer-K\"onig}
%{Allears}{2012-13}{8'}

%\vspace{6mm}

\vfill

\descrizione{Apax}{ riflette un processo creativo segnato dal desiderio di sconvolgere la mia solita immagine compositiva. Il pezzo è sostanzialmente costituito da differenti variazioni di un singolo suono. Ciò mostra una ricerca di variazione nella continuità tramite cambiamenti graduali di timbro e spazializzazione. Il processo compositivo è ispirato dalla fenomenologia del tempo e dalle letture: \emph{La Dialettica Della Durata}, \emph{Intuizione Dell'Istante} e \emph{Poetiche Dello Spazio} di Gaston Bachelard. Questo pezzo ottofonico è stato composto con gli strumenti di spazializzazione (GRIS) sviluppati dal gruppo di ricerca di Normandeau all'università di Montreal. Questa composizione ha vinto il premio \emph{Metamorphoses} 2016, categoria studenti.}

\descrizione{Un cercle hors de l'arbre}{Commissione: PANaroma Stusios (Sao Paulo, Brasile) a Flo Mendez. Liberamente ispirato dal film \emph{La Jetée} di Chris Marker (1921-2012). \emph{Un cercle hors de l'arbre} è stato realizzato nei PANaroma Studios a Sao Paulo (Brasile) tra settembre e ottobre 2014 e la prima fu eseguita il 24 ottobre 2014 durante il BIMESP Festival (Biennale Internazionale della Musica Elettroacustica di Sao Paulo). Si ringrazia il CCA (Canadian Council fo the Arts) per il supporto. \emph{Un cercle hors de l'arbre} vinse il secondo premio alle ottave \emph{Destellos Electroacoustic Composition Competition} (Mar Del Plata, Argentina, 2015).}

%\bigskip

%\svolk{\emph{Ho cercato di incanalare quell’energia in un percorso che la rendesse percepibile senza snaturarne l’essenza: l’energia che abita i violini di Corelli, Tartini, Vivaldi, certo non citazioni ne dirette ne indirette ma l’imprevedibile vitalità delle loro articolazioni (tremoli, arpeggi, ribattuti, sincronie e fioriture improvvise) che hanno fatto della scuola italiana, elettrica ante litteram, un irraggiungibile modello di virtuosismo strumentale; poi la tensione, il vuoto attorno e dentro alla costruzione delle frasi, le imitazioni, i pedali, la continua sovrapposizione delle corde.}}
%
%Giorgio Netti
%
%\normalfont

\end{flushright}

\clearpage

\begin{flushleft}

~\vfill

\descrizione{Che-toi}{ è un pezzo elettroacustico che basa la sua logica concettuale sulla fusione di materiali delle culture francese e argentina. Questi sono collegati tramite l'uso di parole monosillabiche come: \emph{che, no, toi, moi, temp}, ecc. Inoltre vi sono citazioni e frammenti del Barocco francese e del Tango argentino, e strumenti quali il bandoneón e l'accordion, i quali interagiscono su un altro livello di senso.}

\descrizione{Éclats de Feux}{ - un lavoro di transizione del mio viaggio compositivo - è cominciato con numerose registrazioni di oggetti e ambienti trovati in giro per Sheffield (Inghilterra). Come il primo pezzo del mio portfolio per il dottorato, questo funge da ponte tra la scuola di Montreal (Canada), da dove provengo, e l'effervescenza dell'acusmatico britannico. In questo pezzo, ho esplorato i contrasti tra le grandi masse e i singoli oggetti isolati, con una particolare sensibilità nell'uso dello spazio stereofonico, poiché la mia ricerca si basa sull'interpretazione e l'esecuzione di musica acusmatica. Il titolo fa riferimento alle straordinarie notti di falò e infiniti fuochi d'artificio alle quali ho assistito nelle prime settimane dopo il mio arrivo il Regno Unito. \emph{Schegge di Fuoco}, come si traduce, rappresenta inoltre l'estrema rapidità e intensità con le quali la vita di una persona può cambiare dopo un viaggio.}


\end{flushleft}
