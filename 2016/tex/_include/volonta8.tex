% !TEX encoding = UTF-8 Unicode
% !TEX TS-program = XeLaTex
% !TEX root = ../EMU2016_booklet.tex

\begin{flushright}

\large{
	\scshape{
	28 ottobre 2016 -- ore 18:00
	}}

\medskip
	
\small{Concerto Acusmatico
	\newline Il Suono di Piero [Aula Bianchini]}

\medskip

{\fontsize{42}{42} \svolk{\emph{Volontà VIII}}}

\normalsize

\bigskip

\livel{Marco Ferrazza}{CitiZen}{8'18}{}{2016}
\medskip

\livel{Paolo Pastorino}{Dimensione aggiuntiva}{3'25}{}{2016}
\medskip

\livel{Chester Udell}{Jorneta Stream}{11'00}{}{2014}
\medskip

\livel{Leo Cicala}{Khoisan}{10'45}{}{2015}
\medskip

%\brano{Christian Eloy}
%{La cicatrice d'Ulysse}{13'00''}
%{acusmatico}
%new version 2015\\

%\acusmatici{Ursula Meyer-K\"onig}
%{Allears}{2012-13}{8'}

%\vspace{6mm}

\vfill

\descrizione{CitiZen}{I materiali sonori di Citizen provengono essenzialmente da registrazioni ambientali effettuate in differenti città. Segnali tipici del traffico come il clacson convivono con suoni di campane, tessiture rumoristiche e presenze metalliche indistinte. Ne deriva un’orchestrazione intesa come riorganizzazione di più paesaggi sonori e condivisione di differenti spazialità (sia geografiche che acustiche).}

\descrizione{Khoisan}{Per \emph{Dimensione aggiuntiva} si intende una dimensione supplementare che viene generalmente indicata come una ulteriore estensione di un oggetto. L’obiettivo di questa composizione è stato quello di creare una connessione timbrica e temporale tra gli oggetti sonori impiegati.  Elementi provenienti da ambienti e contesti differenti, totalmente estranei tra loro, coesistono e dialogano insieme, dando così origine ad una forma \emph{viva} capace di muoversi in uno spazio immaginario.}

%\bigskip

%\svolk{\emph{Ho cercato di incanalare quell’energia in un percorso che la rendesse percepibile senza snaturarne l’essenza: l’energia che abita i violini di Corelli, Tartini, Vivaldi, certo non citazioni ne dirette ne indirette ma l’imprevedibile vitalità delle loro articolazioni (tremoli, arpeggi, ribattuti, sincronie e fioriture improvvise) che hanno fatto della scuola italiana, elettrica ante litteram, un irraggiungibile modello di virtuosismo strumentale; poi la tensione, il vuoto attorno e dentro alla costruzione delle frasi, le imitazioni, i pedali, la continua sovrapposizione delle corde.}}
%
%Giorgio Netti
%
%\normalfont

\end{flushright}

\clearpage

\begin{flushleft}

~\vfill

\descrizione{Jorneta Stream}{ Dovremmo vivere tanto a lungo quanto i nostri racconti sono umidi del nostro respiro}

\descrizione{Éclats de Feux}{Khoisan è un pezzo simbolico che gioca sugli elementi morfologici peculiari di questa lingua primordiale, Khoisan appunto,  ricca di consonanti dure e schioccanti. Rappresenta una esplorazione psicologica ed intima della spinta  alla migrazione, che da sempre per la nostra specie si ripete tra l’Africa e l’Europa. IL brano è organizzato metaforicamente in eventi che si susseguono come una serie di passi, di tappe; nella prima parte l’evoluzione degli eventi sonori è inserita nella scia di  un  gesto primario che rappresenta la necessità di fare qualcosa in risposta ad un’altra. Le restanti tre parti sono costruite intrecciando microeventi realizzati con varie tecniche tra cui la risintesi : lo scopo è quello di generare materiali nuovi sullo stampo dei materiali di partenza per rendere il contrasto tra il fascino di un mondo migliore e la paura dell’ignoto.}


\end{flushleft}
