% !TEX encoding = UTF-8 Unicode
% !TEX TS-program = XeLaTex
% !TEX root = ../EMU2016_booklet.tex

\begin{flushright}

\large{
	\scshape{
	29 ottobre 2016 -- ore 20:30
	}}

\medskip
	
\small{Concerto
	\newline Sala Accademica}

\medskip

{\fontsize{42}{42} \svolk{\emph{Volontà XI}}}

\normalsize

\medskip

regia del suono \textsc{Pasquale Citera} e \textsc{Federico Paganelli}

\bigskip

\livel{Franco Donatoni}{Ali}{12’00}{per viola sola}{1977}
\medskip

\livel{Mario MARY}{Sal}{9’11}{acousmatic}{2016}
\medskip

\livel{Giuseppe Desiato}{tocco materia}{8’00}{per chitarra e supporto elettronico}{2016}
\medskip

\livel{Francis Dhomont}{Here and There}{10’10}{acuosmatique avec spatialisation}{2003}
\medskip

\livel{Giorgio Nottoli}{Specchi risonanti: Scoperta-Riflessione-Canto}{14’00}{per viola elettrica ed acustica con live electronics}{2011-2016}

\bigskip
 	 
\esecutore{viola}{Luca Sanzò}
\esecutore{chitarra}{Jacopo Lazzaretti}
\esecutore{live electronics}{Federico Paganelli, Giorgio Nottoli}

\vfill

\descrizione{Ali}{Ali, due pezzi per viola sola, richiama il suono dei fruscii e fremiti di ali. Scritto nel 1977, fa parte di un gruppo di quattro opere, insieme a Algo, due pezzi per chitarra, e Argot per violino, che confluiscono in About, un trio per violino viola e chitarra, dove Donatoni utilizza materiale già utilizzato per ciascuno dei pezzi solistici.
A livello esecutivo, è interessante notare una serie di volute contraddizioni presenti dei due pezzi: la prima è proprio nel carattere e nella scrittura tra le due ali, Ali I è esatto, con notazione molto precisa ed una struttura perfettamente riconosciblie, in cui all'inizio si nota una reminiscenza della sonata op. 25 numero 1 di Paul Hindemith, scritta una sessantina di anni prima. Nel quarto movimento, Hindemith suona dei pedali ritmici sul do al tempo di 660 di metronomo, alternati a dei bicordi sparsi e sempre più contratti. In Ali Franco Donatoni percorre una strada simile, sostituendo i do vuoti  (anche graficamente, in quanto scrive delle stanghette senza note) con delle pause, lasciando così nudi dei bicordi che via via si contraggono. Altra contraddizione presente in Ali I è certamente il fatto che, pur accumulando materiale (ai bicordi iniziali si aggiungono acccordi, scale veloci e armonici) il pezzo tende a perdere consistenza, terminando con un grande diminuendo intenzionale e musicale.
Ali II contraddice Ali I. È un pezzo certamente opposto, solo effettistico, qui le note perdono la loro importanza e cedono il passo ad un nuovo linguaggio, in cui l'esecutore si diverte a cercare timbriche decisamente nuove per l'epoca, e suggestive anche oggi. Ali II lo considero un pezzo volto al futuro, perfettamente integrabile in un ipotetico evento visuale, in quanto colgo in esso, a differenza di Ali I, pezzo disciplinato per eccellenza, un invito ad una folle teatralità. (L. Sanzò)}

\end{flushright}

\clearpage

\begin{flushleft}

~\vfill

\descrizione{Sal}{ è una composizione elettroacustica realizzata grazie a una residenza presso Iberomùsicas sede del CMMAS in Messico. Il brano utilizza due personali tecniche compositive che chiamo \emph{Orchestrazione Elettroacustica} e \emph{Polifonia dello Spazio}. La struttura di questo lavoro è complessa ma può essere riassunta con una forma suddivisa in due macro-sezioni riccamente articolate al loro interno. La natura generale della composizione è vitale ed energetica; la musica sembra congelarsi in due occasioni creando un contrasto inaspettato ma che non da luogo a un calo di  tensione musicale grazie alle attese che questa sospensione crea. Durante la stesura del brano, svoltasi in Messico, un elemento extramusicale locale è stato assorbito nel pezzo: il sale di vermi. Sia il sale (e le spezie in genere) che il verme (chapuline) rivestono una posizione importante nella cultura messicana, da questo deriva il titolo della composizione.}

\descrizione{tocco materia}{L’idea portante alla base di questo progetto è il suono inteso come gesto. Lo strumento- chitarra, pur mantenendo la sua forte identità, diventa generatore di una vasta gamma di timbri carichi di materialità. Proprio per questo, il titolo è una vera e propria dichiarazione d’intenti. Il musicista, suonando la chitarra, ne libera attraverso il tocco le potenzialità timbriche, costruendo un percorso fatto di elementi che sfociano nel concretismo sonoro. L’esecuzione dal vivo è affiancata dalla diffusione di suoni di natura elettronica che, ponendosi in maniera dialettica con i gesti chitarristici, enfatizzano l’aspetto tattile-materico del brano.}

\descrizione{Here and There}{Versione stereofonica dell'omonima composizione in otto tracce dedicata a Darren Copeland e David Eagle, ha come tema lo spazio. La  composizione acusmatica affida a un \emph{interprete} la spazializzazione del suono dal vivo che, in questo caso, è stata realizzata attraverso la pre-programazione del software Audiobox di Darren Copeland, che ha permesso a David Eagle di eseguire il brano utilizzando il programma informatico aXio. Questa versione,tuttavia, è stereofonica, presentandosi quindi come uno studio; uno studio comunque comunque molto libero, non teso a dimostrare qualcosa o a negare lirismo alla composizione. Il principio di articolazione, sebbene rappresenti l'oggetto centrale del brano, non deve ostacolare il semplice piacere dell'immersione sonora. Questa composizione è stata commissionata da \emph{New Adventures in Sound Art}, la sua realizzazione fu resa possibile dal  Consiglio Canadese per le Arti. La  prima esecuzione, di David Eagle e Darren  Copeland, si tenne l'undici maggio 2003 in ocasione dell' \emph{Open Ears Festival of Music and Sound} presso Kitchener, Ontario,Canada. Ha ricevuto il primo premio presso il V International Computer Music Competition \emph{Pierre Schaeffer} 2005, Pescara, Italia ed è pubblicato sul CD empreintes DIGITALes …et autres utopies, IMED 0682 \copyright SACEM France.}

\descrizione{Specchi risonanti: Scoperta-Riflessione-Canto}{\emph{Specchi risonanti} estende la sonorità e le modalità di articolazione dello strumento per mezzo di un dispositivo elettro-acustico virtuale. Il  lavoro è basato sul  dialogo fra lo strumento e quattro sue copie elaborate (specchi) che, oltre ad applicare al materiale sonoro un insieme di trattamenti convenzionali, fanno risuonare altrettante corde virtuali tese idealmente nello spazio d'ascolto. Il lavoro si articola in tre episodi: \emph{Scoperta} e \emph{Riflessione}, per viola elettrica, \emph{Canto}, per viola acustica. Scoperta è stato composto nel 2011 ed eseguito molte volte prima che i due episodi seguenti fossero terminati, 5 anni dopo. La composizione è dedicata al violista Luca Sanzò.}

\end{flushleft}

\biografia{Franco Donatoni}{Fu allievo di E. Desderi al conservatorio di Milano passando quindi a Bologna dove studiò con A. Zecchi e L. Liviabella. Subito dopo seguì all'Accademia di Santa Cecilia il corso di perfezionamento di I. Pizzetti ma per i suoi orientamenti di compositore fu determinante il suo incontro con B. Maderna nel 1953, l'anno in cui seguì i Ferienkurse di Darmstadt. Partendo dalle suggestioni di B. Bartók, il suo cammino stilistico è stato progressivamente condizionato dall'approfondita conoscenza delle opere di A. Webern che, filtrate attraverso le esperienze dei maestri di Darmstadt, di P. Boulez e di K. Stockhausen, lo hanno fatto approdare a un originale, fantasioso strutturalismo.}

\biografia{Mario MARY}{dottore in \emph{Estetica, Scienza e Tecnologia delle Arti} (Università di Parigi VIII, Francia), Professore di Composizione di Musica Elettroacustica presso Academia Ranieri III di Monte-Carlo, e Direttore artistico di Monaco Electroacoustique - Incontri Internazionali di Musica Elettroacustica. Ha lavorato come ricercatore presso l'IRCAM e insegnato all'Università Parigi VIII, Ha vinto una ventina di premi in concorsi di composizione. Ha dato nomerosos conferenze e corsi in diversi paesi. http://ipt.univ-paris8.fr/mmary/}

\biografia{Giuseppe Desiato}{(1987) è un compositore e sassofonista italiano. Il suo approccio alla composizione è da sempre una continua ricerca sulle possibilità di modellazione del suono. I suoi materiali sono scelti meticolosamente per creare strutture stratificate e polifoniche, la cui aggregazione punta costantemente ad una forte chiarezza. Molte delle sue composizioni sono influenzate da forme, architetture e video arte. I lavori presenti e passati di Desiato includono esecuzioni da parte di Dario Calderone, Anna Clementi, Hugh Watkins, Alexandra Wood, Septura Brass Septet, Chroma Ensemble e Resonanz ensemble tra gli altri. Giuseppe Desiato è diplomato presso la Royal Academy of Music di Londra e il Conservatorio \emph{Licinio Refice} di Frosinone. Ha recentemente studiato con Salvatore Sciarrino e sta frequentando Il corso di musica elettronica presso il Conservatorio \emph{Santa Cecilia} di Roma sotto la guida di Michelangelo Lupone e Nicola Bernardini.}

\biografia{Francis Dhomont}{(Parigi 1926) Compositore francese e canadese, dottore honoris causa di l' Università di Montreal, Canada. Convinto dell’ originalità dell’arte acusmatica, la sua produzione è, da 1960, esclusivamente costituita d’opere per nastro. Ha insegnato la composizione elettroacustica, tra 1980 e 96, all’ Università di Montreale e ha ricevuto molti onorificenze internazionali: cinque volte premiato al concorso internazionale di musica elettroacustica di Bourges (Francia); Premio \emph{magistere} nel 1988; Premio SACEM 2007 della migliore creazione contemporanea elettroacustica; borsa di carriera del Consiglio delle arti e delle lettere del Québec (2000). Nel 1999, otteneva cinque primi prezzi internazionali per quattro delle sue opere; Prezzo Lynch-Staunton del Consiglio delle arti del Canada, ed invitato del DAAD a Berlino. Si assume la direzione di numerosi speciali \emph{L'espace du son} per le edizioni \emph{Musiques et recherches} (Ohain, Belgio). Dal 1978 Francis Dhomont a condiviso la sua attività tra la Francia ed il Québec e svolge una carriera internazionale. Tornato in Francia nel 2004, vive oggi in Avignon e si dedica alla composizione ed alla ricerca. http://www.electrocd.com/fr/bio/dhomont_fr/discog/}

\biografia{Giorgio Nottoli}{(compositore, nato a Cesena, Italia nel 1945) è stato docente di Musica Elettronica al Conservatorio di Roma \emph{S.Cecilia} sino al 2013. Attualmente è docente di Composizione elettroacustica all’Università di Roma \emph{Tor Vergata}. La maggior parte delle sue opere utilizza mezzi elettronici sia per la sintesi che per l'elaborazione del suono. Il centro della sua ricerca di musicista riguarda il timbro concepito quale parametro principale e \emph{unità costruttiva} delle sue opere attraverso la composizione della microstruttura del suono. Nei suoi lavori per strumenti ed elettronica Giorgio Nottoli punta ad estendere la sonorità degli strumenti acustici mediante complesse elaborazioni del suono. Ha progettato vari sistemi elettronici per la musica utilizzando sia tecnologie analogiche che digitali in collaborazione con varie università e centri di ricerca.}


