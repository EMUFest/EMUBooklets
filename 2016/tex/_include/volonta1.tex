% !TEX encoding = UTF-8 Unicode
% !TEX TS-program = XeLaTex
% !TEX root = ../EMU2016_booklet.tex

\begin{flushright}

\large{
	\scshape{
	24 ottobre 2016 -- ore 20:30
	}}

\medskip
	
\small{Concerto
	\newline Sala Accademica}

\medskip

{\fontsize{42}{42} \svolk{\emph{Volontà I}}}

\normalsize

%\medskip
%
%Apre il ciclo della settimana di eventi EMUFest un percorso incentrato sulla natura intima e rituale del suono. Un omaggio ai molteplici universi del contemporaneo, condotti dalla musica di Giorgio Netti. Musiche di confine che trasfigurano la natura d’origine del violino e della viola, in un cammino sospeso fra i glissandi di Xenakis, i battimenti di Scelsi, le maglie interne del suono di Murail e Dusapin, fino al concretismo strumentale di Lachenmann: differenti modalità di fare vuoto, e ripartire. Una prima volontà eseguita dall’mdi ensemble.

\bigskip

\livel{Tristan Murail}{C’est un jardin secret, ma soeur, ma fiancée, \newline  une source scellée, una fontaine close…}{5’00}{per viola}{1976}

\medskip

\livel{Iannis Xenakis}{Mikka}{4’25}{violino}{1971}
\medskip

\livel{Pascal Dusapin}{Inside}{8’32}{per viola}{1980}
\medskip

\livel{Giorgio Netti}{Dalla tentazione di Sant’Antonio}{9’00}{per violino}{1986}
\medskip

\livel{Giacinto Scelsi}{Manto II}{5’10}{per viola}{1967}
\medskip

\livel{Helmut Lachenmann}{Toccatina}{4’50}{per violino}{1986}
\medskip

\livel{Giorgio Netti}{Inoltre}{17’00}{per due violini}{2005-2006}

%\brano{Christian Eloy}
%{La cicatrice d'Ulysse}{13'00''}
%{acusmatico}
%new version 2015\\

%\acusmatici{Ursula Meyer-K\"onig}
%{Allears}{2012-13}{8'}

%\vspace{6mm}

\bigskip

\textbf{mdi ensemble} \\
\esecutore{violino}{Lorenzo Gentili-Tedeschi}
\esecutore{viola}{Paolo Fumagalli}

\vfill

\bigskip

\svolk{\emph{Ho cercato di incanalare quell’energia in un percorso che la rendesse percepibile senza snaturarne l’essenza: l’energia che abita i violini di Corelli, Tartini, Vivaldi, certo non citazioni ne dirette ne indirette ma l’imprevedibile vitalità delle loro articolazioni (tremoli, arpeggi, ribattuti, sincronie e fioriture improvvise) che hanno fatto della scuola italiana, elettrica ante litteram, un irraggiungibile modello di virtuosismo strumentale; poi la tensione, il vuoto attorno e dentro alla costruzione delle frasi, le imitazioni, i pedali, la continua sovrapposizione delle corde.}}

Giorgio Netti

\normalfont

\end{flushright}

\clearpage

\begin{flushleft}

~\vfill

\descrizione{C’est un jardin secret, ma soeur, ma fiancée, une source scellée, una fontaine close…}{Fondatore insieme a Gerard Grisey della musica “spettrale”, Murail utilizza l’informatica per approfondire le ricerche d’analisi e sintesi dei fenomeni ascustici, costruendo una musica basata sulle micro-variazioni interne al suono. C’est un jardin secret… presenta questo percorso di ricerca: una miniatura per viola attraversata da molteplici processi, trasformazioni progressive e ambiguità tra armonia e timbro. Un brano, afferma il compositore, costruito attorno al ritmo di un battito cardiaco, costantemente accellerato e rallentato, vivo e risonante.}

\descrizione{Mikka}{Riconosciuto tra gli esponenti più radicali della musica del 900, il compositore greco Iannis Xenakis compone il brano Mikka nel 1971 e l’anno successivo eseguito al Festival D’automne de Paris. Il brano è un impressionante fusione di glissati del violino che pongono l’accento su una materia incisa in limite estremo dell’accezione di melodia. Un suono/canto dell’arco che oscilla tra i suoni più sottili fino alle dinamiche più accese.}

\descrizione{Inside}{Diviso in tre parti, il brano mette in luce il grande ventaglio timbrico della viola, in una articolazione frenetica ed immersiva. Anche questo brano presenta micro-variazioni del suono instaurando un dialogo quarti-tonale tra le corde.}

\descrizione{Dalla tentazione di Sant’Antonio}{…Ho cercato di ripensare lo strumento a partire dalla sua memoria, dalla memoria delle dita che per secoli l’hanno attraversato: ho cercato di ascoltarlo come voce di voci e dallo stratificarsi nell'aria di queste, dal loro sovrapporsi, ha preso corpo il volume dello spazio nel quale è contenuto. Le diverse “apparizioni” vorrebbero arrivare a comporre via via un’unità, che non determina una direzione quanto un sostare: stato incandescente della materia sonora, non solidifica, s’addensa e, sospeso, prossimo a saturarsi viene nell’ultimo respiro infine travasato…}

\descrizione{Manto II}{Secondo movimento per Viola, Manto prende il nome da un profeta dell’antica Grecia. Scelsi incentra il brano sulle soglie del battimento e sugli aspetti liminari del suono. Qui il rito è alla base dell’estetica del compositore, percepibile come vero e proprio ponte/distanza tra il tempo immobile e il successivo canto umano che accompagna il terzo movimento.}

\descrizione{Toccatina}{Helmut Lachenmann consegna in eredità al fare musicale una continua nuova percezione dell’ascolto. Questo studio per violino, apre una nuova finestra, indicando *un possibile oltre a ciò che abitualmente chiamiamo musica* (G. Netti): una nuova sensibilità dello strumento, che parte da un idea di musica concreta strumentale, fattasi  sottile e cristallina in questa breve dischiusura musicale}

\descrizione{Inoltre}{Starting from a muted blow, INOLTRE becomes a continuous attempt to bring the sound matter nearer to the pitched musical world, creating an acoustic experience coming from somewhere else. (Quotation from the Composer) This composition makes the matter white hot and in its boiling, articulations, fragments, nightmares and wanderings reappear and melting become something else. A voice-violin, a sound-man, sacred in their uniqueness, with no more articulation, tense and suspended on the extreme and guarded electric background.}

\end{flushleft}
