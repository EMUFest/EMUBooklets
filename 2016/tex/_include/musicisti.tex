% !TEX encoding = UTF-8 Unicode
% !TEX TS-program = XeLaTex
% !TEX root = ../EMU2016_booklet.tex

\flushright
~\vfill

\section*{Biografie Interpreti}

%2016-701:
\biografia{Filippo Ansaldi}{ nasce a Savigliano nel 1992, frequenta il Conservatorio *G. Verdi* di Torino nella classe del M° Pietro Marchetti, ottenendo la laurea di II livello nel 2014 con il massimo dei voti e la lode. Ha frequentato Master Class con musicisti di caratura internazionale. Ha esperienza in diverse formazioni sassofonistiche, quali il duo, il quartetto e l'ensemble, con cui collabora con importanti enti concertistici. Ha ricevuto diversi premi e riconoscimenti in concorsi nazionali ed internazionali. Attualmente frequenta il Master di II livello in Interpretazione della Musica Contemporanea presso il Conservatorio *Santa Cecilia* di Roma grazie al sostegno dell'associazione torinese De Sono.}

%2016-702:
\biografia{Sofia Bandini}{ (classe 1999) viene ammesa al Conservatorio di Musica Santa Cecilia di Roma nel 2010 nella classe di violino del Maestro Giuseppe Crosta, dove ha appena sostenuto l’esame di compimento medio. Nel corso degli anni, ha frequentato numerose Masterclass  con i maestri Georg Mönch, Alina Company, Felice Cusano, David Romano, Roberto Gonzàlez-Monjas e Marlène Prodigo. Fin dal 2007 svolge un’intensa attività sinfonica che le ha dato la possibilità di suonare con musicisti del calibro di Antonio Pappano, Shlomo Mintz, Francesca Dego, Salvatore Accardo, Ezio Bosso. Dal 2013 studia quartetto d’archi sotto la guida del Maestro Alberto Mina, e dal settembre 2016 ha iniziato gli studi presso l’Accademia Walter Stauffer con il Quartetto di Cremona. Il quartetto ha avuto modo di esibirsi presso la Camera dei Deputati, Villa Madama, Castel Sant’Angelo, ed ha vinto il primo premio assoluto di musica da camera all’edizione 2016 del Concorso Nazionale Riviera Etrusca.}

%2016-703:
\biografia{Sauro Berti}{ Clarinetto Basso del Teatro dell’Opera di Roma, ha collaborato con le orchestre Italiane più importanti (Teatro alla Scala, Maggio Musicale Fiorentino, RAI National Orchestra), con il Royal Scottish National Orchestra e  Sinfonia Finlandia Jyväskylä all’estero. Ha suonato con direttori come G. Prêtre, R. Chailly, M. W. Chung, R. Muti, W. Sawallisch, V. Gergiev, L. Maazel, P. Boulez and Z. Mehta. Ha partecipato come solista in vari festival musicali in tutto il mondo. Nel 2009 si è laureato in Direzione d’orchestra con D. Renzetti. Ha pubblicato *Venti Studi per Clarinetto basso*, *Tuning* per fiati (Suvini Zerboni), la sua versione da concerto di V. Bucchi e i CD: *Suggestions*(Edipan) e *SoloNonSolo*(ParmaRecords).}

%2016-704:
\biografia{Francesco Bianco}{ Musicista e cofondatore dell’etichetta Studiolo Laps. All’attività di compositore affianca quella di musicista in diverse formazioni che spaziano dalla sperimentazione alla musica alternativa alla performance. La sua formazione è arricchita da numerose esperienze di studio e confronto con il mondo accademico ed extra accademico. La sua ricerca artistica si rivolge alle profonde relazioni fra l'arte e la vita, la società, il tempo, gli spazi, la comunicazione, il linguaggio.}

%2016-705:
\biografia{Tiziano Capponi}{ nasce il 21/03/1992 a Roma. Inizia a studiare teoria musicale e percussioni classiche presso la scuola Musica S.Cecilia sotto la guida del maestro Aldo Tamantini. Parallelamente si avvicina allo studio del pianoforte e della batteria. Inizia gli studi in Percussioni al Conservatorio S. Cecilia di Roma, inizialmente con il maestro Michele Iannaccone e in seguito con il maestro Gianluca Ruggeri, con il quale si diploma nel 2015 con il massimo dei voti.}

%2016-706:
%\biografia{Maureen Chowining}{ Soprano, ha studiato ai Conservatori di Boston e del New England. Con i suoi concerti ha fatto il giro del mondo; tra le performance più importanti ricordiamo quelle all’International Electronic Music Festival a Bourges, dove nel 1990 e nel 1997 ha eseguito in prima assoluta, rispettivamente *Solemn Songs for Evening* di Richard Boulanger e *Sea Songs* di Dexter Morrill. Nel 2005 ha eseguito in prima assoluta *Voices*, composta per lei da John Chowning alla Maison de Radio a Parigi. Nota per la duttilità della sua voce che le consente di affrontare stili e repertori diversi e intonazioni alternative, come la scala di Pierce. In 27 anni ha dato lezioni di canto privatamente e tenuto molte Master class in tutto il mondo.}

\clearpage
\flushleft
~\vfill

%\flushright
%~\vfill

%2016-707:
\biografia{Pasquale Citera}{ (1981) Ha studiato pianoforte con il M° Gemma D’Alessio, Composizione con i M° Luciano Pelosi e Giovanni Piazza e Musica Elettronica con il M° Giorgio Nottoli. Da anni collabora con diverse compagnie teatrali e case di produzione cinematografica oltre che con scultori e fotografi. Ha composto musiche di scena per spettacoli classici e contemporanei. Tra gli altri: L’Alcesti di Euripide, Lisistrata di Aristofane, Anfitrione di Plauto, la Locandiera di Goldoni, l’Avaro di Molière, Da quale parte del vetro di Silvio Nanni, Il dito sulla bocca di Donatella Ferrara, Certe Notti non accadono mai di Patrizia Masi. Ha scritto colonne sonore per la Nero-Film, è Assistente Musicale in diverse scuole di Roma ed è stato docente di Tecnologie Musicali. Dalla collaborazione con lo scultore Arturo Ianniello sono nate diverse sonorizzazioni di opere visuali raccolte in due esposizioni. È attualmente Compositore e Sound Designer per musiche di scena al Teatro Anfitrione ed all’Anfiteatro della Quercia del Tasso.}

%2016-708:
\biografia{Eleonora Claps}{ Lucana di nascita, consegue gli studi con E. Scatarzi presso il Conservatorio *G. Martucci* (SA). Si perfeziona sotto la guida di A. Caiello, frequenta il *Corso di Specializzazione in Canto Lirico per l’Opera Contemporanea* (Verona Opera Academy) e quelli dell‟ Internationales Musikinstitute di Darmstadt, esibendosi in concerto ufficiale IMD2016. Finalista del *Premio Bucchi Interpretazione – Parco della Musica 2015*, interprete vocale dello *ScarlattiLab/Electronics*, svolge regolare attività concertistica su repertorio del '900 e Contemporaneo, acustico ed elettroacustico.}

%2016-709:
\biografia{Alice Cortegiani}{ Nata a Rieti nel 1994, inizia a studiare clarinetto a 8 anni. Nel 2014 si Diploma con il massimo dei voti presso il Conservatorio S. Cecilia di Roma, sotto la guida dei maestri G. Amato, G. Russo, D. Rossi. Attualmente si perfeziona con A. Carbonare e frequenta il Biennio Specialistico in Musica da Camera sotto la guida di R. Galletto. Ottiene la borsa di studio Erasmus+ per la Royal Academy School of Music di Londra e per il Real Conservatorio Superior de Musica de Madrid sotto la guida di A. Garcès. Ha partecipato a masterclass tenute da: F. Meloni, K. Leister, V. Alberola Ferrando. Dal 2011 intraprende una ricca attività concertistica a tutto tondo: in orchestra, come solista e nella musica da camera. Tra i progetti in attivo ci sono: Imago Sonora Ensemble, il Duo Essentia con il fisarmonicista S. Telari, PentElios Quintet (Quintetto di Fiati), Trio Alpha (Clarinetto, Violoncello, Pianoforte), Duo con il pianista A. Viale.}

%2016-710:
\biografia{Elena D'Alò}{ flautista (dall’ottavino al flauto basso), si laurea cum laude al biennio in Flauto, dopo un brillante diploma, presso il Conservatorio *Santa Cecilia* di Roma, con Deborah Kruzansky, studiando anche con Edda Silvestri, Bruno Paolo Lombardi e Paolo Taballione. Ha affiancato gli studi musicali con quelli scientifici, laureandosi in Fisica acustica presso *La Sapienza* con Paolo Camiz. Attualmente è iscritta al triennio di Musica Elettronica. Si esibisce in formazioni cameristiche e orchestrali, in un repertorio che va dal barocco al contemporaneo, per il quale ha suonato a festival come Nuova Consonanza, Atlante Sonoro XXsecolo, ArteScienza ed EMUFest.}

%2016-711:
\biografia{Michele D'Auria}{ Born in Salerno in 1993, Michele D'Auria started playing saxophone at the age of 8. In 2013 he graduated from the Conservatory of Salerno with the highest honors. He attended a number of master classes lectured by worldwide famous musicians. He took part in many home and international competitions, always arriving  first: he won the *V. Cammarota* 2014 award form contemporary music, the *Claudio Ceschini* award for the best saxophonist and he received a special mention from the Ministry of Human Resources of the Hungarian Republic. He performed several concerts where the sax was the solo instrument, playing both in chamber and in orchestral ensemble, in Italy as well as abroad.}

%2016-712:
\biografia{Marco De Martino}{ Nato a Roma. Dopo i primi studi di Pianoforte e il diploma di Liceo scientifico, prosegue presso la facoltà di lettere e filosofia all’Università di Roma Tor Vergata. Laureato in Musicologia, ha studiato Composizione Elettroacustica con Giorgio Nottoli, Michelangelo Lupone e Informatica Musicale con Nicola Bernardini. Inizia il suo percorso di docenza come assistente di Informatica Musicale per il conservatorio e per il Master di II livello in Interpretazione della Musica Contemporanea, sempre del Conservatorio di Roma.}

%2016-713:
\biografia{Sara Ferrandino}{ si è diplomata in pianoforte nel 2005 presso il Conservatorio di Perugia nella classe del Mº Tanganelli, conseguendo nel 2009, con votazione di 110 e Lode, la Laurea per il Biennio Specialistico. Nel 2012 ha ottenuto il diploma del Corso di Perfezionamento tenuto dal Mº Perticaroli, presso l’Accademia Nazionale di Santa Cecilia in Roma. Ha partecipato a numerosi concorsi nazionali e internazionali ottenendo sempre piazzamenti nelle prime posizioni. Si è esibita in molteplici concerti solistici e cameristici in prestigiose sale in Italia e all’estero. Ha collaborato e c ollabora presso il Conservatorio di Perugia con le classi di corno, tromba, flauto, oboe, sassofono e violino. È docente di pianoforte principale per i corsi pre-accademici presso l'Accademia AIMART in Roma.}

%2016-714:
\biografia{Enzo Filippetti}{ è professore di Sassofono e del Master Annuale di II Livello in Interpretazione della Musica Contemporanea al Conservatorio *S. Cecilia* di Roma e da più di trent’anni tiene concerti in tutto il mondo. Si è esibito alla Biennale di Venezia, al Mozarteum di Salisburgo, a Roma, Milano, Parigi, Londra, Berlino, Vienna, Madrid, Bruxelles, Buenos Aires, Caracas, Riga, Birmingham, Köln, Lyon, St. Etienne (Francia), Principato di Monaco, Yeosu (Korea), Kawasaki, Adis Abeba, Chisnau, Taormina, Ravello. Ha collaborato con Claude Delangle, Alda Caiello e Bruno Canino e molti tra i più importanti compositori hanno scritto per lui più di cento opere e gli hanno affidato numerose prime esecuzioni. Come solista e con il Quartetto di Sassofoni Accademia ha inciso per Nuova Era, Dynamic, Rai Trade e Cesmel. Ha pubblicato studi per Riverberi Sonori e cura una collana per le edizioni Sconfinarte.}

%2016-735:
\biografia{Paolo Fumagalli}{ Nato nel 1978, si è diplomato in violino sotto la guida di Elena Ponzoni presso il Conservatorio *Cantelli* di Novara e in viola con Roberto Tarenzi presso il Conservatorio *Nicolini* di Piacenza, perfezionandosi poi con Maja Jokanovic, Claudio Pavolini e Simonide Braconi in Italia e in Svizzera. Ha tenuto concerti come camerista invitato dalle Settimane Musicali di Stresa, dal festival Ligeti-Milano Musica, dalla Fondazione *Fernando Rielo* di Roma, da Contemporaneamente Lodi, Teatro Bibiena di Mantova, Teatro Municipale Piacenza, Amici della Musica Palermo, Nuova Consonanza Roma, Staatsoper Stuttgart, Europa Musica, Musica y Escena Mexico City, WDR Koln. Prima viola dell’Orchestra Giovanile *Luigi Cherubini* diretta da Riccardo Muti dal 2005 al 2008, con lo stesso incarico viene chiamato dall’Orchestra La Fenice di Venezia e dal Teatro G. Verdi di Trieste. Collabora stabilmente con diverse formazioni orchestrali diretto da L. Maazel, Y. Temirkanov, R. Barshai, K. Masur, E. Inball. E’ stato membro del sestetto d’archi dell’accademia del Teatro alla Scala, col quale si è esibito al Festival di Ravello, al ridotto dei palchi Teatro alla Scala partecipando poi a tourneè internazionali in Asia e in Sud America. Collabora con le più importanti formazioni da camera italiane che si occupano del repertorio contemporaneo, come Divertimento Ensemble, Sentieri Selvaggi, Ensemble Icarus, Ensemble Risognanze, Xenia Ensemble. Attualmente insegna viola presso la scuola *Dedalo* di Novara. Ha effettuato registrazioni per Stradivarius, Ricordi Oggi, Aeon Paris e per la WestDeutscheRundfunk ha inciso un duo inedito per viola, arpa e elettronica  di Robert HP Platz.}

%2016-734:
\biografia{Lorenzo Gentili-Tedeschi}{ Nato a Milano nel 1988, si diploma con lode a soli sedici anni presso l’Istituto Musicale Pareggiato Donizetti di Bergamo, laureandosi due anni dopo con 110 e lode nel Biennio Specialistico del Conservatorio di Milano. Si perfeziona con Francesco De Angelis presso l’Haute Ecole de Musique di Losanna-Sion, dove consegue il MasterSoliste nel 2010 eseguendo il concerto di Beethoven op.61 con l’Orchestre de Chambre de Lausanne. Per i successivi due anni insegna come assistente di De Angelis presso la medesima istituzione. Dal 2014 è membro dei London Philharmonic Orchestra, con cui suona alla Royal Festival Hall e alla Royal Albert Hall di Londra per i BBC Proms, effettua tour in Europa, Stati Uniti, Cina e partecipa al prestigioso festival operistico di Glyndebourne, Inghilterra.
 Da anni collabora stabilmente con alcune delle più importanti orchestre italiane: Filarmonica della Scala, Orchestra del Teatro alla Scala e del Teatro Regio di Torino, Orchestra da Camera di Mantova e Solisti di Pavia, diretto da grandi direttori quali Gustavo Dudamel, Daniel Barenboim, Riccardo Chailly, Yuri Temirkanov e molti altri. È invitato regolarmente come violino di spalla presso l’orchestra del Teatro Petruzzelli di Bari, dopo essere stato per cinque anni primo violino di spalla dell’Orchestra dell’Accademia del Teatro alla Scala, suonando anche come solista nel Kammerkonzert di Alban Berg per la stagione de I concerti del Quirinale in diretta su Radio3. Nel 2012 ha preso parte alla Lucerne Festival Academy, lavorando con Pierre Boulez e Pablo Heras Casado come violino di spalla dell’ensemble.}

%2016-715:
\biografia{Arianna Granieri}{ Pianista, si diploma con il massimo dei voti e consegue con lode e menzione d’onore la laurea di II livello in Pianoforte indirizzo Solistico presso il Conservatorio Santa Cecilia di Roma, sotto la guida del M° Cinzia Damiani, con un impegnativo programma di musica italiana moderna e contemporanea. Durante il periodo di studi, ha partecipato in vari masterclass con musicisti di fama mondiale come Boris Berman. Ha conseguito con lode la laurea magistrale in Filosofia presso l’Università di Roma Tor Vergata, con una tesi sull’estetica del Giappone e del samurai. Si è esibita sia in qualità di solista che in formazioni cameristiche presso vari festival ed eventi musicali tra i quali Concerti Accademici, Orvieto Festival of Strings, Domeniche Estive a Castel Sant’Angelo, AnemosArts, Novantenario della nascita di Franco Evangelisti, la programma della Rai I fatti vostri inoltre ha eseguito in prima mondiale la riduzione per due pianoforti del Concerto per pianoforte e orchestra di Henry Cowell. È interessata all’unione della musica con le altre arti, in particolare il teatro (ha suonato in diversi spettacoli).}

%2016-716:
\biografia{Virginia Guidi}{ Si diploma in Canto Lirico e in Musica Vocale da Camera al Conservatorio S. Cecilia dove si specializza con lode con Silvia Schiavoni con una tesi sul rapporto tra interprete e compositore nella musica elettroacustica. Spazia dalla musica da camera a quella contemporanea con attenzione per la musica di sperimentazione. Ha collaborato con numerosi compositori eseguendo spesso pezzi a lei dedicati. Si è esibita in Italia e all’estero (Pechino, Washington DC) e ha partecipato ad importanti festival (EMUfest, Biennale di Venezia, ArteScienza) e ad installazioni di famosi artisti (Allora\&Caladilla, Thomas De Falco).}

%2016-717:
\biografia{Filiz Karapınar}{ Ha suonato in molte delle principali orchestre, tra cui la Philharmonic, Istanbul State Symphony, Bilkent Symphony,  Antalya State Symphony, e Bursa State Symphony, e Çukurova Symphony. Interprete di musica contemporanea, è membro del MIAM Modern Music Ensemble e si è esibita in prime esecuzioni di compositori come Fred Lerdahl and Kamran İnce. Ha vinto the special jury price come migliore interprete di musica contemporanea alla Cahit Koparal National Flute Competition nel 2007. Ha partecipato a masterclass con Robert Winn, Patrick Gallois, Andras Adorjan, Emmanuel Pahud, Davide Formisano, Julien Beaudiment e Sophie Cherrier  ed è stata membro della Turkish-Greek Youth Orchestra diretta da Vladimir Ashkenazy. Si è laureata presso la Istanbul Technical University's Dr Erol Üçer Center for Advanced Studies in Music, dove ha studiato con Bülent Evcil, ed è dottoranda in Flute Performance studiando con Jülide Gündüz.}

%2016-718:
\biografia{Jacopo Lazzaretti}{ Nato a Roma nel 1994, frequenta il decimo anno del corso di chitarra con il M° Arturo Tallini presso il Conservatorio di Santa Cecilia a Roma. Ha studiato come studente Erasmus con il M° Matthew McAllister presso il Royal Conservatoire of Scotland nell’anno 2015/2016. Ha partecipato come allievo effettivo a numerose Masterclasses con artisti di fama internazionale tra i quali Oscar Ghiglia, Marcin Dylla, Pavel Steidl e molti altri. Ha partecipato e vinto in concorsi nazionali e internazionali che gli hanno permesso di esibirsi anche in altre occasioni. Ha tenuto concerti in Italia, Scozia e Turchia.}

%2016-719:
\biografia{Ivan Liuzzo}{ was born on february 26, 1993. In 2007 he began his studies of percussions at the Conservatorio di Musica *L. Refice* with C. Di Blasi. He continued his study of jazz drums with R. PISTOLESI and G. HUTCHINSON. Active Member and co-founder of Phthorà Collective, along with F. Ferazzoli and F. Abbate has collaborated with artists come: Lisa Mezzacappa, Stefano Costanzo, Vincenzo core, Wound, Ron Grieco, Achille Succi etc.}

%2016-720:
\biografia{Leonardo Mammozzetti}{ nasce a Chapecó (Brasile) l'11 ottobre 1985. Residente a Roma, frequenta il corso di laurea Musica Elettronica al Conservatorio Santa Cecilia di Roma. Gli aspetti tecnici acustici della materia, in particolare i metalli, hanno interessato sempre i suoi studi e sperimentazioni nei suoi progetti tecnico musicali; é interessato dunque alla composizione di brani nell'ambito della musica elettroacustica. Assistente occasionale del CRM sotto la guida del Maestro Michelangelo Lupone.}

%2016-721:
\biografia{Massimiliano Mascaro}{ compositore. Nato a Roma nel 1986. Allievo del M° Michelangelo Lupone e del M° Nicola Bernardini, si è formato presso il Conservatorio *A. Casella* di L'Aquila e successivamente presso il Conservatorio *S. Cecilia* di Roma affrontando gli studi della Composizione elettroacustica e della Composizione classica. La musica elettroacustica è il settore nel quale svolge la sua principale attività musicale.}

%2016-736:
\biografia{mdi ensemble}{ nasce nel 2002 su iniziativa di sei giovani musicisti uniti dalla passione per la musica contemporanea, allora grazie all’appoggio dell’associazione Musica d’Insieme di Milano. Nel corso della sua decennale attività l’ensemble si è sviluppato lavorando al fianco di celebri compositori quali Helmut Lachenmann, Sofia Gubaidulina, Dai Fujikura, Gérard Pesson, Pierluigi Billone, Fabio Vacchi e Mauro Lanza, e proponendo contemporaneamente prime esecuzioni di giovani compositori emergenti del panorama internazionale. Diverse le collaborazioni di prestigio con direttori come Beat Furrer, Pierre André Valade, Yoichi Sugiyama e Robert HP Platz.}

%2016-722:
\biografia{Alessandro Pace}{ laureato in Flauto con il M°Carlo Morena con la votazione di 110 e lode presso il Conservatorio di Santa Cecilia di Roma. Prosegue gli studi in flauto, affiancati dagli studi in Composizione tradizionale nello stesso conservatorio. Ha fatto e continua a fare molti concerti nei vari generi. Suona con diversi ensemble: Orchestra Ars Ludi Romana (anche come solista); Broadway Musical Orchestra (es. Festival di Todi); Indivenire Ensemble (repertorio contemporaneo). Ha suonato nell'orchestra nazionale di Panama a Panama City. Ha preso parte al festival Contaminazioni sia come flautista che come compositore. Ha seguito il progetto del M° Antonio Di Pofi sulla musica dei film muti (anche qui sia come flautista che compositore). Suona molta musica da camera in diverse formazione ed è in continua ricerca di nuove esperienze.}

%2016-723:
\biografia{Federico Paganelli}{ nato a Roma, studia musica elettronica presso il Conservatorio di Santa Cecilia con i maestri Bernardini e Lupone. Precedentemente ha studiato con il Maestro Giorgio Nottoli.}

%2016-724:
\biografia{Danilo Perticaro}{ Nato a Cosenza nel 1992, intraprende giovanissimo lo studio del Sassofono, al Conservatorio della città, laureandosi sotto la guida di Luigi Grisolia con lode. Ha partecipato e vinto numerosi concorsi e manifestazioni internazionali. Ha seguito masterclasses e corsi di alto perfezionamento tenuti dai Maestri: Salime, Moretti, Delangle, Espinoza, Marzi, Filippetti, Mlekusch, Bornkamp. Collabora con orchestre e ensemble, soprattutto in ambito contemporaneo ed elettronico in un'intensa attività concertistica nazionale ed internazionale. Ha avviato un progetto di incisione di tutte le opere per sassofono di Stockhausen. Collabora con diversi compositori nella realizzazione di opere inedite a lui dedicate.}

%2016-725:
\biografia{Alessandro Pirchio}{ Studia presso il Conservatorio di Santa Cecilia con il M° Franz Albanese. Ha partecipato da solo o in formazioni cameristiche a la Rassegna *Musica a Roma per Roma*; il *Sutri Beethoven Festival; Stagione cameristica del Museo della ceramica di Viterbo. Ha suonato per lo spettacolo *La dodicesima notte* (Premio *Le maschere del teatro 2015* per le musiche originali del M° Piovani) in numerosi teatri italiani (Donizzetti di Bergamo, Ponchielli di Cremona, Verdi di Padova sono tra i più importanti). Attualmente ricopre la parte di Primo Flauto nella Banda della Gendarmeria Vaticana e dell’Ass. Nazionale Carabinieri.}

%2016-726:
\biografia{Federico Ripanti}{ Nato a Roma nel 1987, studia Musica Elettronica presso il Conservatorio *S. Cecilia* di Roma. Nel 2009 si diploma in Fonia e Music Technology presso la Saint Louis Music College. Ha studiato privatamente pianoforte, chitarra elettrica e percussioni africane.}

%2016-727:
\biografia{Alice Romano}{ nata a Roma il 27 Luglio 1995, ha iniziato lo studio del violoncello con il M° M. Scarpelli. Ha preso parte alle attività orchestrali presso l'Accademia Nazionale di S. Cecilia, come elemento della Juniorchestra! e attualmente come tutor del settore Education. Inoltre, è da poco membro della Ljubljana International Orchestra (BSA). Diplomata al Conservatorio di S. Cecilia (M° M. Massarelli), ha intrapreso gli studi presso La Sapienza di Roma, corso di laurea in Lingue, cultura, letteratura, traduzione.}

%2016-728:
\biografia{Matteo Rossi}{ percussionista versatile nel repertorio classico e nel repertorio contemporaneo, si diploma con il massimo dei voti presso il Conservatorio *S.Cecilia* di Roma con Gianluca Ruggeri. Frequenta il corso di Alta Formazione in Timpani presso l’Accademia Nazionale di Santa Cecilia con Antonio Catone; segue il corso di perfezionamento presso l’Accademia Musicale Chigiana con Antonio Caggiano, e come membro del Chigiana Percussion Ensemble, si esibisce al CHIGIANA INTERNATIONAL FESTIVAL, RAVELLO FESTIVAL e MAXXI di Roma. Collabora con diverse formazioni orchestrali, come la  World Youth Orchestra con la quale si esibisce anche sul suolo internazionale, e cameristiche quali PMCE,  InDivenire Ensemble ed ensemble di percussioni quali Ars Ludi, Blow-Up Roma Percussion, Aere Silente con cui si esibisce in un repertorio percussionistico moderno e contemporaneo in diversi eventi quali Le esperienze del minimalismo, Le Forme del Suono, Artescienza, EMUFest, RomAFAMfest.}

%2016-729:
\biografia{Luca Sanzò}{ allievo di Bruno Giuranna, svolge attività concertistica, discografica e didattica. È molto attento alla produzione e alla diffusione della nuova musica, della quale è un apprezzato esecutore. È fondatore del Quartetto Michelangelo, inoltre è regolarmente invitato, insieme a musicisti di tutto il mondo, all'annuale Rome Chamber Music Festival. Ha collaborato, in qualità di prima viola solista, con il Teatro dell’Opera di Roma, il Teatro Lirico di Cagliari e Concerto Italiano. Ha pubblicato per Ricordi una revisione dei 41 Capricci di Campagnoli per viola sola ed è titolare della cattedra di viola presso il Conservatorio di S. Cecilia di Roma. Fra le sue incisioni si segnala, per Brilliant Classics, l'integrale delle sonate per viola e pianoforte e per viola d'amore e pianoforte di Hindemith, e quello delle sonate di Brahms.}

%2016-730:
\biografia{Franco Sbacco}{ diplomato in Composizione, Direzione d'Orchestra, Musica Elettronica e Percussion, rispettivamente con D. Guaccero, D. Paris, G. Nottoli e L. Torrebruno. Perfezionato nel 1979 in Composizione presso l'Accademia Nazionale di S. Cecilia a Roma con G. Petrassi e F. Donatoni. Nel 1989 vince una borsa di studio canadese per attività di ricerca sulla computer music presso la Simon Fraser University di Vancouver con il sistema PODX di B. Truax. Si è dedicato al teatro musicale, le sue composizioni, eseguite in Italia e all'estero nei principali festival di musica contemporanea. È inoltre vincitore del 3° Concorso Internazionale  di Musica Elettroacustica di Varese e del 18° Concorso Internazionale di Musica Elettroacustica di Bourges. Ha inciso per BMG Ariola – Roma, per DOMANI MUSICA – Roma, che nel 1999 e nel 2008 gli ha dedicato due cd monografici e nel 2015 per ALBANY RECORDS -  New York. Insegna Armonia e Analisi presso il Conservatorio S. Cecilia di Roma.}

%2016-731:
\biografia{Arturo Tallini}{ È docente al Conservatorio Santa Cecilia e tiene regolarmente Master class nei conservatori italiani e università straniere. Considerato un riferimento per il repertorio contemporaneo, collabora con artisti di fama internazionale tra cui Michiko Hirayama, il gruppo di musica contemporanea Modus Novus di Madrid, il Coro dell’Accademia Nazionale di Santa Cecilia e con il flautista Carlo Morena. È coordinatore del Master Annuale di II Livello in Interpretazione della Musica Contemporanea del Conservatorio Santa Cecilia in cui è anche docente di Chitarra. Si è esibito in Europa, negli Stati  Uniti, in Egitto, Algeria e Tunisia.}

%2016-732:
\biografia{Gianni Trovalusci}{ suona la gamma dei flauti moderni, flauti aumentati, il traversiere e strumenti d’invenzione; è attivo nel campo della musica contemporanea e antica, del teatro musicale, di sound art e performance d'avanguardia. Ha collaborato con numerosi artisti e si è esibito in importanti luoghi di riferimento in Italia e nel contesto internazionale, come Festival Musica Elettronica New York, Mills College San Francisco, Stockholm New Music, Munchener Biennale, Ars Electronica Linz, Cafe Oto Londra, EMUFest, Nuova Consonanza, etc.}

%2016-733:
\biografia{Francesco Ziello}{ Si è laureato nel 2012 in Musica Elettronica al Conservatorio di Roma *Santa Cecilia*, dove parallelamente ha studiato Pianoforte e Composizione. Come compositore scrive in ambito teatrale e cinematografico, collaborando anche con l’Accademia Nazionale di Danza, per una serie di spettacoli, di cui il primo presentato al CRM (centro di Ricerche Musicali), nel luglio 2015. Polistrumentista e performer, lavora principalmente nell’ambito della musica contemporanea, partecipando in importanti festival come EMUFest e ArteScienza. Nel 2016 partecipa come operatore musicale nell’ambito di un progetto di riabilitazione psichiatrica, in collaborazione con l’Università di Tor Vergata di Roma.}

%2016-737:
%\biografia{SAXATILE [modulable sax ensemble]}{Di recente costituzione, è un progetto realizzato da un’idea di Enzo Filippetti specificamente rivolto all’esecuzione della musica contemporanea, nei suoi molteplici aspetti, con un orientamento diretto alla ricerca e all’instaurazione di un rapporto dialettico con altri artisti e con i compositori. I componenti, emanazione del Conservatorio di Roma, possono vantare una solida, variegata esperienza.}
