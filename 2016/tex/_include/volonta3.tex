% !TEX encoding = UTF-8 Unicode
% !TEX TS-program = XeLaTex
% !TEX root = ../EMU2016_booklet.tex

\begin{flushleft}

\large{
	\scshape{
	25 ottobre 2016 -- ore 20:30
	}}

\medskip
	
\small{Concerto
	\newline Sala Accademica}

\medskip

{\fontsize{42}{42} \svolk{\emph{Volontà III}}}

\normalsize

\medskip

regia del suono \textsc{Pasquale Citera} e \textsc{Marco De Martino}

\bigskip

\livel{Karlheinz Stockhausen}{Solo}{10’39}{for a soloist with live electronics}{}
\medskip

\livel{Maura Capuzzo}{Arcipelagos}{9’30}{per violino, violoncello, clarinetto e live electronics}{}
\medskip

\livel{James Dashow}{Soundings in Pure Duration n. 2b for percussive and octophonic electronic sounds}{12’26}{acusmatic}{}
\medskip

\livel{Marco Marinoni}{Still}{15’00}{live performance}{}
\medskip

\livel{Barry Truax}{The Garden of Sonic Delights}{11’15}{acousmatic}{}
\medskip

\livel{Mario Duarte}{Achtli}{6’10}{for flute, piano, two percussionists and live electronics}{}
\medskip

\bigskip

\esecutore{direttore}{Franco Sbacco}
\esecutore{sassofono}{Danilo Perticaro}
\esecutore{violino}{Sofia Bandini}
\esecutore{clarinetto}{Alice Cortegiani}
\esecutore{violoncello}{Alice Romano}
\esecutore{flauto}{Alessandro Pace}
\esecutore{pianoforte}{Francesco Ziello}
\esecutore{percussioni}{Tiziano Capponi, Matteo Rossi}
\esecutore{live electronics}{Leonardo Mammozzetti, \newline Massimiliano Mascaro}


\vfill

\descrizione{Solo}{L'esecuzione di questo brano del 1965-66 prevede uno strumentista e 4 assistenti musicali che ne devono interpretare la partitura ed eseguire in maniera puntuale ogni aspetto indicato. A corredo delle prime esecuzioni c'erano rumori di fondo del nastro, possibili errori di sincronismo e molte altre criticità. Oggi i mezzi sono cambiati: uno strumentista, in questo caso il sassofonista, e un assistente musicale, che si occupa della parte elettronica. L'attenzione per lo stile e la precisione sono ancor più esasperati, una sfida quindi, con l'obiettivo di fare sempre meglio. (L. Mammozzetti)}

\descrizione{Arcipelagos}{Un arcipelago è un paesaggio di isole nel medesimo mare. Simili o assai diverse nelle forme, nei colori o negli umori ma un solo mare, lo stesso vento.}

\end{flushleft}

%\clearpage

\begin{flushright}

~\vfill

\descrizione{Soundings in Pure Duration n. 2b}{per suoni percussivi pre-registrati e suoni elettronici esafonici, nasce dalla sovrabbondanza di materiale sonoro che ho creato per Soundings n. 2a.  Come per quest'ultimo, il brano è stato composto per sviluppare la spazializzazione approfittando dei punti d'attacco percussivi come precisi indicatori di posizionamento nello spazio.   Oltre alle interazioni tra suoni percussivi e suoni elettronici, ho cominciato a lavorare anche con 2 concetti della spazializzazione... quella definita dalle traiettorie nello spazio dei suoni percussivi, e quella definita dai fasci sonori elettronici. Queste due componenti sono molto piu' integrate rispetto al n. 2a, per cercare di creare due strati di spazio simultaneamente, oppure una specie di contrappunto  spaziale.  Durante la composizione di questo brano continuo i miei  tentativi di esplorare il mio secondo concetto di spazializzazione cioè il movimento DELLO spazio, anziché movimento NELLO spazio.}

\descrizione{Still}{è una live performance che utilizza registrazioni catturate durante le missioni spaziali Voyager mediante strumenti di indagine in radioastronomia planetaria (PRA) in grado di registrare segnali di emissione dai pianeti, le loro lune ed i loro sistemi ad anello catturando fenomeni elettromagnetici quali le interazioni del vento solare con la magnetosfera del pianeta, la magnetosfera stessa, le emissioni di particelle polarizzate e le loro interazioni, trasformando tali fenomeni in segnali elettrici, amplificati e utilizzati per eccitare la membrana di un altoparlante.}

\descrizione{The Garden of Sonic Delights}{è un paesaggio sonoro composto da molteplici tracce sonore. Il pezzo invita l'ascoltatore in un ambiente sonoro immaginario (descritto da Murray Schafer come un \emph{giardino risonante}) pregno di suoni che dovrebbero ricordarci quelli dell'acqua, del vento, degli insetti, degli animali e degli uccelli. Il nostro viaggio comincerà il pomeriggio per finire al mattino seguente, lasciandoci - si spera - lieti e riposati. Il pezzo è stato commissionato dal \emph{Birmingham ElectroAcoustic Sound Theatre} (BEAST) per il BEAST FEaST 2016, e ralizzato con 48 canali alla \emph{Technical University} di Berlino e allo studio privato del compositore con l'ausilio del TiMax2 Soundhub della Outboard per la spazializzazione.}

\descrizione{Achtli}{\emph{Ci volevano seppellire ma non sapevano che noi fossimo semi.} Nel settembre 2014, scomparvero 43 studenti del \emph{Ayotzinapa Teacher Training College} in seguito a degli scontri con le forze dell'ordine. Si crede che i poliziotti abbiano consegnato gli studenti ai \emph{drug cartel} i quali, a loro volta, li avrebbero uccisi e poi bruciati in una discarica nella periferia di Cocula, Guerrero (Messico). \emph{Achtli} significa \emph{seme} in Nàhuatl (un' antica lingua messicana). Ho scritto questo pezzo in segno di protesta per richiedere una mobilitazione in merito alla scomparsa degli studenti. Il pezzo è stato creato inserendo in una patch del software MAX/MSP i nomi dei 43 studenti in modo tale da affidare ad ogni personaggio un parametro musicale come timbro, altezza, durata, intensità e gesto. Questo pezzo è un pianto di giustizia ed è parte dell'azione globale in favore di Ayotzinapa. Potrete rivedere la performance del pezzo a questo link: https://www.youtube.com/watch?v=BJLKTs1ygZU.}

\end{flushright}

\clearpage

\begin{flushleft}

~\vfill

***

%\biografia{Karlheinz Stockhausen}{è stato un compositore tedesco. Dal 1947 al 1951 ha studiato pedagogia della musica e pianoforte alla Musikhochschule di Colonia e scienza della musica, germanistica e filosofia all'università di Colonia. Come docente universitario ed autore di numerose pubblicazioni sulla teoria della musica, attraverso le sue attività per la radio e grazie a più di 300 proprie composizioni ha partecipato in modo significativo a influenzare la musica del XX secolo.}

\biografia{Maura Capuzzo}{Nata a Padova, studia con C.Benati, G. Bonato, N. Bernardini ed si diploma in Musica Corale,  Composizione, Musica elettronica. Segue corsi di perfezionamento con F.Valdambrini, M.Bonifacio e S.Sciarrino, seminari con G.Grisey, H.Lachenmann, M.Stroppa A. Vidolin. 1997 vince l’ European Women Composers Contest \emph{Kaleidoscope Programm of European Union}.  2000 Borsa di Studio al corso del M° S.Sciarrino,Città di Castello, Festival delle Nazioni. 2001, III premio, al Concorso Internazionale di Composizione Corale \emph{A Cappella} Germania. Nel 2009 Borsa di Studio della Fondazione Lerici dell'IIC, in collaborazione con il KTH di Stoccolma. 2011 II premio concorso di composizione organistica,Mantova (in giuria Adriano Guarnieri). 2012, premiata con esecuzione al Festival Biennale Koper, Slovenia (V. Globokar,Fabio Nieder in giuria). Sue composizioni son state eseguite in Italia ed all'estero: Festival Urticanti,ISCM Festival ( Hong Kong 2007) MiTo, Musikpodium Zurig, Biennale Koper, Visioni del Suono, SpazioMusica, CIM (Trieste 2012,Roma 2014),Camino Controcorrente Udine, Piano Fazioli Series (IIC Los Angeles) Festival Germi, Festival Mixtur, Segnali Sonori, New Made Week-Siae Classici d’oggi, Festival 5 Giornate,Astra Concert Season Melbourne.Sue composizioni sono state trasmesse anche da Radio3, WDR3, Radio4(Hong Kong) Radio DRS2, Radio Cemat, Radio e Tv Koper. I suoi lavori sono editi da ArsPublica.Insegna Teoria Ritmica e Percezione musicale al B. Marcello,Venezia}

\biografia{James Dashow}{compositore, dedica la sua principale attività compositiva alla computer music, spesso con esecutori dal vivo, pur non trascurando la musica per strumenti tradizionali. La sua attività di ricerca sfocia nella creazione di un suo linguaggio di sintesi, MUSIC30, ed un suo metodo di composizione, il Sistema Diadi. Uno dei fondatori del Centro di Sonologia Computazionale di Padova. Nel 2011, la Fondazione CEMAT (Roma) gli ha conferito il premio CEMAT per la Musica per il suo contributo allo sviluppo della musica elettronica. Ha insegnato al MIT dove ha ricoperto il ruolo di direttore supplente dello Studio di Musica Sperimentale, e alla Princeton University. È stato vice-presidente nel primo comitato direttivo dell'International Computer Music Association, e per molti anni ha condotto il programma radiofonico \emph{Il Forum Internazionale di Musica Contemporanea} per RAI Radio 3. I suoi lavori sono registrati su DVD, CD e LP di varie case discografiche italiane ed estere: BMG Ariola - RCA, Wergo, EdiPan, Capstone, Neuma, ProViva, CRI, Scarlatti Classica, BVHAAST e Centaur.}

\biografia{Marco Marinoni}{Nasce a Monza nel 1974. Prix du Trivium nel 29e Concours International de Musique et d'Art Sonore Electroacoustiques - Bourges 2002. Finalista dell'International Gaudeamus Composition Prize 2002 e 2003. Vincitore della Seconda Call per Opere Elettroacustiche indetta dalla Federazione CEMAT. Primo Premio nel Primo Concorso di Composizione per Iperviolino - Genova 2007. Primo Premio nel VIII Concorso Internazionale di Composizione \emph{Città di Udine}. Le partiture dei suoi brani sono edite da ARSPUBLICA e TAUKAY. È docente di Musica Elettronica presso il Conservatorio di Como. Nel 2015 esce il suo primo romanzo, La Confraternita di Ecate - Cauda Draconis (ed. Nerocromo).}

\end{flushleft}

\clearpage

~\vfill

\begin{flushright}

\biografia{Barry Truax}{è professore emerito della School of Communication(e in precedenza della School for the Contemporary Arts) presso la Simon Fraser University, dove ha tenuto corsi di comunicazione acustica e musica elettroacustica. Ha lavorato con il World Soundscape Project, rivedendone il testo Acoustic Ecology, e ha pubblicato il libro Acoustic Communication che tratta di suono e tecnologia. Come compositore Truax è noto soprattutto per i suoi lavori con il sistema per la computer music PODX che ha utilizzato per pezzi per solo supporto elettronico, pezzi di teatro musicale e composizioni con esecutori dal vivo e grafica virtuale. Nel 1991, il suo brano, Riverrun, è stato insignito del Magisterium presso il concorso internazionale di musica elettroacustica di Brouges, Francia. I suoi paesaggi sonori multicanale sono frequentemente proposti all'interno di festival e concerti internazionali. Dal suo pensionamento, nel 2015, Barry è stato Edgard Varèse Guest Professor presso la Technical University di Berlino. www.sfu.ca/~truax}

\biografia{Mario Duarte}{Nato a Città del Messico, Mario ha studiato chitarra, musicologia e composizione presso il Musical Studies and Research Centre (CIEM) e letteratura Ispanica presso la Autonomous National University of Mexico (UNAM). Dopo aver concluso i suoi studi musicali ha lavorato come compositore, sceneggiatore, produttore e presentatore per la radio Opus 94.5FM. Nel 2010 ha fondato, con il supporto dell'università, il programma Comunidades Sonoras (Sound Communities), un progetto socio-musicale che si occupa dei bambini socialmente esclusi ed emarginati di Città del Messico. Nel 2013 ha iniziato un dottorato in composizione presso il centro di ricerca NOVARS sotto la supervisione del professor Ricardo Climent. Il consiglio nazionale per la scienza e la tecnologia Messicano sponsorizza la sua ricerca sull'utilizzo delle catene di DNA e RNA come modelli per la composizione musicale.}

\end{flushright}