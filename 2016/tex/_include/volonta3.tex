% !TEX encoding = UTF-8 Unicode
% !TEX TS-program = XeLaTex
% !TEX root = ../EMU2016_booklet.tex

\begin{flushright}

\large{
	\scshape{
	25 ottobre 2016 -- ore 20:30
	}}

\medskip
	
\small{Concerto
	\newline Sala Accademica}

\medskip

{\fontsize{42}{42} \svolk{\emph{Volontà III}}}

\normalsize

\medskip

regia del suono \textsc{Pasquale Citera} e \textsc{Marco De Martino}

\bigskip

\livel{Karlheinz Stockhausen}{Solo}{10’39}{for a soloist with live electronics}{}
\medskip

\livel{Maura Capuzzo}{Arcipelagos}{9’30}{per violino, violoncello, clarinetto e live electronics}{}
\medskip

\livel{James Dashow}{Soundings in Pure Duration n. 2b for percussive and octophonic electronic sounds}{12’26}{acusmatic}{}
\medskip

\livel{Marco Marinoni}{Still}{15’00}{live performance}{}
\medskip

\livel{Barry Truax}{The Garden of Sonic Delights}{11’15}{acousmatic}{}
\medskip

\livel{Mario Duarte}{Achtli}{6’10}{for flute, piano, two percussionist and live electronics}{}
\medskip

\bigskip

\esecutore{direttore}{Franco Sbacco}
\esecutore{sassofono}{Danilo Perticaro}
\esecutore{violino}{Sofia Bandini}
\esecutore{clarinetto}{Alice Cortegiani}
\esecutore{violoncello}{Alice Romano}
\esecutore{flauto}{Alessandro Pace}
\esecutore{pianoforte}{Francesco Ziello}
\esecutore{percussioni}{Tiziano Capponi, \newline Matteo Rossi}
\esecutore{live electronics}{Leonardo Mammozzetti, \newline Massimiliano Mascaro}


\vfill

\descrizione{Solo}{L'esecuzione di questo brano del 1965-66 prevede uno strumentista e 4 assistenti musicali che ne devono interpretare la partitura ed eseguire in maniera puntuale ogni aspetto indicato. A corredo delle prime esecuzioni c'erano rumori di fondo del nastro, possibili errori di sincronismo e molte altre criticità. Oggi i mezzi sono cambiati: uno strumentista, in questo caso il sassofonista, e un assistente musicale, che si occupa della parte elettronica. L'attenzione per lo stile e la precisione sono ancor più esasperati, una sfida quindi, con l'obiettivo di fare sempre meglio. (L. Mammozzetti)}

\end{flushright}

\clearpage

\begin{flushleft}

~\vfill

\descrizione{Arcipelagos}{Un arcipelago è un paesaggio di isole nel medesimo mare. Simili o assai diverse nelle forme, nei colori o negli umori ma un solo mare, lo stesso vento.}

\descrizione{Soundings in Pure Duration n. 2b}{per suoni percussivi pre-registrati e suoni elettronici esafonici, nasce dalla sovrabbondanza di materiale sonoro che ho creato per Soundings n. 2a.  Come per quest'ultimo, il brano è stato composto per sviluppare la spazializzazione approfittando dei punti d'attacco percussivi come precisi indicatori di posizionamento nello spazio.   Oltre alle interazioni tra suoni percussivi e suoni elettronici, ho cominciato a lavorare anche con 2 concetti della spazializzazione... quella definita dalle traiettorie nello spazio dei suoni percussivi, e quella definita dai fasci sonori elettronici. Queste due componenti sono molto piu' integrate rispetto al n. 2a, per cercare di creare due strati di spazio simultaneamente, oppure una specie di contrappunto  spaziale.  Durante la composizione di questo brano continuo i miei  tentativi di esplorare il mio secondo concetto di spazializzazione cioè il movimento DELLO spazio, anziché movimento NELLO spazio.}

\descrizione{Still}{è una live performance che utilizza registrazioni catturate durante le missioni spaziali Voyager mediante strumenti di indagine in radioastronomia planetaria (PRA) in grado di registrare segnali di emissione dai pianeti, le loro lune ed i loro sistemi ad anello catturando fenomeni elettromagnetici quali le interazioni del vento solare con la magnetosfera del pianeta, la magnetosfera stessa, le emissioni di particelle polarizzate e le loro interazioni, trasformando tali fenomeni in segnali elettrici, amplificati e utilizzati per eccitare la membrana di un altoparlante.}

\descrizione{The Garden of Sonic Delights}{è un paesaggio sonoro composto da molteplici tracce sonore. Il pezzo invita l'ascoltatore in un ambiente sonoro immaginario (descritto da Murray Schafer come un \emph{giardino risonante}) pregno di suoni che dovrebbero ricordarci quelli dell'acqua, del vento, degli insetti, degli animali e degli uccelli. Il nostro viaggio comincerà il pomeriggio per finire al mattino seguente, lasciandoci - si spera - lieti e riposati. Il pezzo è stato commissionato dal \emph{Birmingham ElectroAcoustic Sound Theatre} (BEAST) per il BEAST FEaST 2016, e ralizzato con 48 canali alla \emph{Technical University} di Berlino e allo studio privato del compositore con l'ausilio del TiMax2 Soundhub della Outboard per la spazializzazione.}

\descrizione{Achtli}{\emph{Ci volevano seppellire ma non sapevano che noi fossimo semi.} A settembre 2014, scomparvero 43 studenti del \emph{Ayotzinapa Teacher Training College} in seguito a degli scontri con le forze dell'ordine. Si crede che i poliziotti abbiano consegnato gli studenti ai \emph{drug cartel} i quali, a loro volta, li avrebbero uccisi e poi bruciati in una discarica nella periferia di Cocula, Guerrero (Messico). \emph{Achtli} significa \emph{seme} in Nàhuatl (un' antica lingua messicana). Ho scritto questo pezzo in segno di protesta per richiedere una mobilitazione in merito alla scomparsa degli studenti. Il pezzo è stato creato inserendo in una patch del software MAX/MSP i nomi dei 43 studenti in modo tale da affidare ad ogni personaggio un parametro musicale come timbro, altezza, durata, intensità e gesto. Questo pezzo è un pianto di giustizia ed è parte dell'azione globale in favore di Ayotzinapa. Potrete rivedere la performance del pezzo a questo link: https://www.youtube.com/watch?v=BJLKTs1ygZU.}

\end{flushleft}
