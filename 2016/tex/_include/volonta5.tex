% !TEX encoding = UTF-8 Unicode
% !TEX TS-program = XeLaTex
% !TEX root = ../EMU2016_booklet.tex

\begin{flushleft}

\large{
	\scshape{
	26 ottobre 2016 -- ore 20:30
	}}

\medskip
	
\small{Concerto
	\newline Sala Accademica}

\medskip

{\fontsize{42}{42} \svolk{\emph{Volontà V}}}

\normalsize

\medskip

regia del suono \textsc{Pasquale Citera} e \textsc{Elena D'Alò}

\bigskip

\livel{Cesare Saldicco}{Spire V}{8’00}{per flauto discendente al si amplificato e supporto elettronico}{2016}
\medskip

\livel{Nicoletta Andreuccetti}{>=< cpd (In Two Minds)}{7’02}{per sax soprano e live electronics}{2015}
\medskip

\livel{Vittorio Montalti}{Labyrinthes}{12’16}{per flauto basso ed elettronica}{2012}
\medskip

\livel{Roberto Doati}{Antidinamica}{10’ ~ 18’}{version for Four saxophones and live electronics}{2015-2016}
\medskip

\livel{Rouzbeh Rafie}{Melancholia}{9’00}{for flutist and percussionist}{2016}

\bigskip
 	 
\esecutore{flauti}{Alessandro Pirchio, \newline Gianni Trovalusci}
\esecutore{SAXATILE [modulable sax ensemble]}{\newline Danilo Perticaro, \newline Enzo Filippetti, \newline Filippo Ansaldi, \newline Michele D’Auria}
\esecutore{percussions}{Ivan Liuzzo}
\esecutore{live electronics}{Massimiliano Mascaro}

\vfill

\descrizione{Spire V}{è, in ordine cronologico, il quarto lavoro di un progetto più ampio dedicato allo strumento solista, ispirato al concetto di spirale. La spirale, simbolo universale presente in natura e nell'arte, è rintracciabile in molteplici situazioni: nella forma delle galassie, nella doppia elica del DNA, nel simbolo dello Yin-Yang cinese o nel simbolo del caduceo, la bacchetta di Mercurio sulla quale due serpenti si attorcigliano intorno ad un asse: l'asse del mondo. E in altre forme ancora. Il compositore ha riflettuto su queste immagini, dopo aver osservato a lungo l’ipnotica curva descritta dalla puntina di un giradischi. In Spire V il compositore ha lavorato sull'idea di ripetibilità e interpolazione del principale elemento diacronico, così nella scrittura si confrontano due gestualità e due idee contrastanti, che tuttavia sono legate dallo stesso spettro armonico. Due soggetti musicali diversi - soprattutto nel timbro e nella direzionalità - ma che in comune conservano la lettura e rilettura del materiale sonoro. Nella fase precedente la scrittura, il compositore ha realizzato alcune patch per la formalizzazione di processi di letture spiroidali di stringhe complesse, che hanno consentito una rapida ed efficace verifica delle procedure adottate.}

\end{flushleft}

\clearpage

\begin{flushright}

~\vfill

\descrizione{>=< cpd (In Two Minds)}{narra l’alternanza di stati mentali trasfigurati in suono mediante un’interferenza continua che accosta antitesi estreme, rappresentate da sonorità che privilegiano i registri estremi del sax in una dialettica imprevedibile tra overtones/subtones e distorsioni ‘acide’/sinusoidi ‘pure’. Convergent Parallel Divergent sono i modelli di sviluppo che strutturano la relazione tra sax e live elettronics: sinusoidi pure che ‘nascono’ dal sax evolvono in differenti direzioni, producendo battimenti e modulando, secondo differenti prospettive, la forma generale. Il sax diviene così, attraverso l’uso della dinamica, del gesto, dell’attacco e dello sviluppo del suono, un tool di controllo che plasma il live electronics.}

\descrizione{Labyrinthes}{è un brano per flauto basso ed elettronica commissionato dall’ Ex Novo Ensemble. Nella composizione si sviluppa la descrizione di un ipotetico labirinto e di possibili itinerari in cui perdersi. Il viaggio all’interno di questa rete costruisce una narrazione in cui gli elementi musicali si alternano, si sovrappongono e si trasformano l’uno nell’altro con ritorni in luoghi già visitati. L’elettronica amplifica il percorso strumentale, divenendone un’estensione. Si creano così labirinti paralleli, diramazioni che aprono la porta su dimensioni altre. Il brano è inoltre un omaggio allo scrittore e poeta Jorge Luis Borges e ai suoi labirinti. \emph{Ossessivamente sogno di un labirinto piccolo, pulito, al cui centro c'è un'anfora che ho quasi toccato con le mani, che ho visto con i miei occhi, ma le strade erano così contorte, così confuse, che una cosa mi apparve chiara: sarei morto prima di arrivarci} (J.L. Borges)}

\descrizione{Antidinamica}{La composizione ha inizio con l’elaborazione, in tempo reale, della registrazione di un’improvvisazione di Gianpaolo Antongirolami (sax contralto) per la mia opera Il domestico di Edgar. L’analisi spettrale di questa elaborazione, che non contiene più alcun suono riferibile alla sorgente acustica, definisce la partitura per i sassofoni in ANTIDINAMICA. Il sassofonista può scegliere e cambiare il metronomo (semiminima da 20 a 120) ad ogni pagina, così come quali e quanti pentagrammi eseguire, fra i 6 di ogni pagina, ma entro una durata stabilita (minimo 6’, massimo 12’).  Nei restanti 4’ (massimo 6’) prosegue improvvisando liberamente sulla memoria di quanto precedentemente letto. L’interprete al live electronics improvvisa liberamente sui parametri di un ambiente costituito da una convoluzione con impulsi filtrati e inviati a 4 ritardi in parallelo con traspositore nel feedback, dapprima sulla registrazione usata per generare la partitura e poi sui sassofoni dal vivo.}

\descrizione{Melancholia}{il pezzo è stato commissinato da Gianni Trovalusci, Melancholia si è ispirato dall’omonimo film di Lars von Trier. Nel corso del pezzo il flautista recita un brevo testo di Giulia Laurenzi scritto originalmente per questo brano. \emph{Sanguino, sacra madre impietrita, l’horror vacui latra.}}

\end{flushright}

\clearpage

\begin{flushleft}

***

\biografia{Cesare Saldicco}{è diplomato in Pianoforte, Musica Elettronica e Composizione. Successivamente segue lezioni e master class con A. Hultqvist, O. Lützow-Holm, P. Hurel, U. Chin, O. Strasnoy, H. Lachenmann, G. Bryars, S. Gervasoni, S. Sciarrino e Ivan Fedele, con il quale ha conseguito nel 2012 il diploma di alto perfezionamento presso l’Accademia Nazionale di S. Cecilia a Roma. Già vincitore di diverse borse di studio è stato selezionato e premiato in concorsi prestigiosi quali Bourges, EmuFest, Destellos e Mùsica Viva Pourtugal. Nel 2012 *La Biennale di Venezia* gli ha commissionato un nuovo lavoro elettroacustico realizzato e allestito durante la 56° edizione del Festival. Recentemente il lavoro audiovideo *I camminatori. Resconto audiovisivo per isole erranti* sulla sequenza di poesie di Italo Testa (Premio Ciampi 2013) e presentato a EXPO2015, ha vinto il premio Best of audience al Los Angeles Film Festival ed è entrato nella Official selection di diversi Festival internazionali tra i quali il Phoenix Film Festival, New York City Electroacoustic Music Festival e Toronto Film Week. La sua musica è edita e pubblicata da ArsPublica e Sconfinarte ed è stata eseguita in Italia, Australia, Albania, Argentina, Belgio, Bulgaria, Canada, Chile, Danimarca, Francia, Germania, Grecia, Finlandia, Inghilterra, Malta, Portogallo, Romania, Russia, Spagna, Stati Uniti, Svezia, Svizzera. Dal 2012 fa parte del MoA Project, un collettivo di compositori che sviluppa progetti site-specific.}

\biografia{Nicoletta Andreuccetti}{ha approfondito la sua formazione musicale sviluppando una varietà di interessi che spaziano dalla composizione alla musicologia. Dopo essersi affermata in numerosi concorsi internazionali di composizione (Primo Premio all’International Electroacoustic Music Competition Musicanova di Praga, Primo Premio Dutch Harp Composition Contest di Utrecht), la sua musica è stata eseguita nei più importanti festivals internazionali: Achantes 2009 (Metz, Paris), ISCM World New Music Days 2011 (Music Biennale Zagreb), International Gaudeamus Music Week 2012, Biennale di Venezia 2012, New Horizons Music Festival (USA 2013), Festival Music and Performing Arts (New York University 2013), Orchestra Sinfonica di Lecce 2013, Mixtur 2014 (Barcelona), ICMC World New Music Days 2014 (Athens), Festival Alla battaglia! 2014 in collaborazione con l’RSI (Radio-televisione Svizzera Italiana), Bienal de Fin del Mundo 2015 (Chile), Expo 2015 (Milan), Muslab 2015 (Mexico), I Pomeriggi Musicali 2016 (Milano), INTER/actions 2016 Symposium (Bangor), New York City Electroacoustic Music Festival 2016, 12th International Symposium on Computer Music (CMMR) San Paolo. http://www.nicolettaandreuccetti.it}

\biografia{Vittorio Montalti}{(Roma, 1984) si è diplomato in pianoforte con Aldo Tramma al Conservatorio S. Cecilia di Roma e in composizione con Alessandro Solbiati al conservatorio G. Verdi di Milano. Si è poi perfezionato all’Accademia Nazionale di Santa Cecilia, sotto la guida di Ivan Fedele ed ha studiato musica elettronica presso l’ IRCAM-Centre Pompidou di Parigi. Nel 2010, nell’ambito de La Biennale di Venezia-54. Festival Internazionale di Musica Contemporanea,  gli è stato conferito il Leone d’Argento per la Creatività. Nel 2016 il Teatro La Fenice gli ha assegnato il premio Una vita nella musica - giovani. La sua musica è stata commissionata da importanti istituzioni ed eseguita in festival e stagioni concertistiche quali New York Philharmonic, Gran Teatro La Fenice, Teatro dell’Opera di Roma, IRCAM-Centre Pompidou, La Biennale di Venezia, I Teatri di Reggio Emilia, Teatro Lirico di Spoleto, Accademia Filarmonica Romana, Orchestra Regionale della Toscana, e molte altre. È stato inoltre compositore in residenza presso l'Istituto Italiano di Cultura di Parigi e l'Accademia Americana di Roma ed attualmente insegna composizione al Conservatorio di Rodi Garganico e composizione elettroacustica in Francia presso il Conservatorio e l'Università di Tours. Le sue partiture sono pubblicate dalle edizioni Suvini Zerboni-SugarMusic S.p. A. Milano. www.vittoriomontalti.com}

\end{flushleft}

\clearpage

\begin{flushright}

\biografia{Roberto Doati}{(Genova, 1953). Svolge gli studi di musica elettronica a Firenze con Albert Mayr e Pietro Grossi e a Venezia, dove si diploma con Alvise Vidolin. Dal 1979 al 1989 svolge attività di compositore e ricercatore presso il Centro di Sonologia Computazionale dell'Università di Padova. Dal 1983 al 1993 collabora con il Laboratorio di Informatica Musicale della Biennale di Venezia (L.I.M.B.) Dal 2005 è docente di Musica Elettronica presso il Conservatorio *Niccolò Paganini* di Genova per cui ha ideato e realizzato numerosi progetti fra cui la costituzione di GEO (Galata Electroacoustic Orchestra) che ha diretto, insieme al collega Tolga Tüzün, al Festival di Musica Contemporanea 2014 de La Biennale di Venezia ricevendo il XXXIV Premio della critica musicale *Franco Abbiati*. È stato compositore residente presso diverse istituzioni, fra cui la Fondazione Rockefeller. Fra le sue opere ricordiamo Allegoria dell’opinione verbale (2000) per attrice, elettronica e sistema interattivo EyesWeb su testi di Gianni Revello, Un avatar del diavolo, opera di teatro musicale commissionata da La Biennale di Venezia (2005), Sindrome scamosciata per video e live electronics (2008-2009). Dal 2013 si occupa di estetica del gusto producendo videomusiche quali Seppie senz’osso (video: Paolo Pachini) e Il suono bianco (video: Maurizio Goina).}

\biografia{Rouzbeh Rafie}{si è diplomato in composizione nell’università dell’arte di Tehran con maestro Kiawasch SahebNassagh, in 2011 si è trasferito a Roma e ha iniziato lo studio di composizione in conservatorio di Santa Cecilia con Maestro Rosario Mirigliano, in 2014 ha partecipato in corsi di composizione di Salvatore Sciarrino (in conservatorio di Latina) e Tuivo Tulev. Nel 2015 ha iniziato a studiare presso l’ Accademia Nazionale di Santa Cecilia con Ivan Fedele. I suoi brani sono stati eseguiti da Ensemble contemporanea del Parco della mMusica, ensemble Imago Sonora,  e trio 3:00, ensemble Navak e ha collaborato con  musicisti come Paolo Ravaglia, Gianni Trovalusci, Daniele Dian. Ha vinto premio dello studio *Adrianna Giannuzi* per il brano *Lentezza* e secondo premio del concorso *Goffredo detrassi* per *Eslimi 1*}

\end{flushright}
