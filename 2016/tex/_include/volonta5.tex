% !TEX encoding = UTF-8 Unicode
% !TEX TS-program = XeLaTex
% !TEX root = ../EMU2016_booklet.tex

\begin{flushright}

\large{
	\scshape{
	26 ottobre 2016 -- ore 20:30
	}}

\medskip
	
\small{Concerto
	\newline Sala Accademica}

\medskip

{\fontsize{42}{42} \svolk{\emph{Volontà V}}}

\normalsize

\bigskip

\livel{Cesare Saldicco}{Spire V}{8’00}{per flauto discendente al si amplificato e supporto elettronico}{}
\medskip

\livel{Nicoletta Andreuccetti}{>=< cpd (In Two Minds)}{7’02}{per sax soprano e live electronics}{}
\medskip

\livel{Vittorio Montalti}{Labyrinthes}{12’16}{per flauto basso ed elettronica}{}
\medskip

\livel{Roberto Doati}{Antidinamica}{10’ ~ 18’}{version for Four saxophones and live electronics}{}
\medskip

\livel{Rouzbeh Rafie}{Melancholia}{9’00}{for flutist and percussionist}{}

\bigskip
 	 
\esecutore{flauti}{Alessandro Pirchio, \newline Gianni Trovalusci}
\esecutore{SAXATILE [modulable sax ensemble]}{Danilo Perticaro, \newline Enzo Filippetti, \newline Filippo Ansaldi, \newline Michele D’Auria}
\esecutore{percussions}{Ivan Liuzzo}
\esecutore{live electronics}{Massimiliano Mascaro}

\vfill

\descrizione{Spire V}{è, in ordine cronologico, il quarto lavoro di un progetto più ampio dedicato allo strumento solista, ispirato al concetto di spirale. La spirale, simbolo universale presente in natura e nell'arte, è rintracciabile in molteplici situazioni: nella forma delle galassie, nella doppia elica del DNA, nel simbolo dello Yin-Yang cinese o nel simbolo del caduceo, la bacchetta di Mercurio sulla quale due serpenti si attorcigliano intorno ad un asse: l'asse del mondo. E in altre forme ancora. Il compositore ha riflettuto su queste immagini, dopo aver osservato a lungo l’ipnotica curva descritta dalla puntina di un giradischi. In Spire V il compositore ha lavorato sull'idea di ripetibilità e interpolazione del principale elemento diacronico, così nella scrittura si confrontano due gestualità e due idee contrastanti, che tuttavia sono legate dallo stesso spettro armonico. Due soggetti musicali diversi - soprattutto nel timbro e nella direzionalità - ma che in comune conservano la lettura e rilettura del materiale sonoro. Nella fase precedente la scrittura, il compositore ha realizzato alcune patch per la formalizzazione di processi di letture spiroidali di stringhe complesse, che hanno consentito una rapida ed efficace verifica delle procedure adottate.}

\end{flushright}

\clearpage

\begin{flushleft}

~\vfill

\descrizione{>=< cpd (In Two Minds)}{narra l’alternanza di stati mentali trasfigurati in suono mediante un’interferenza continua che accosta antitesi estreme, rappresentate da sonorità che privilegiano i registri estremi del sax in una dialettica imprevedibile tra overtones/subtones e distorsioni ‘acide’/sinusoidi ‘pure’. Convergent Parallel Divergent sono i modelli di sviluppo che strutturano la relazione tra sax e live elettronics: sinusoidi pure che ‘nascono’ dal sax evolvono in differenti direzioni, producendo battimenti e modulando, secondo differenti prospettive, la forma generale. Il sax diviene così, attraverso l’uso della dinamica, del gesto, dell’attacco e dello sviluppo del suono, un tool di controllo che plasma il live electronics.}

\descrizione{Labyrinthes}{è un brano per flauto basso ed elettronica commissionato dall’ Ex Novo Ensemble. Nella composizione si sviluppa la descrizione di un ipotetico labirinto e di possibili itinerari in cui perdersi. Il viaggio all’interno di questa rete costruisce una narrazione in cui gli elementi musicali si alternano, si sovrappongono e si trasformano l’uno nell’altro con ritorni in luoghi già visitati. L’elettronica amplifica il percorso strumentale, divenendone un’estensione. Si creano così labirinti paralleli, diramazioni che aprono la porta su dimensioni altre. Il brano è inoltre un omaggio allo scrittore e poeta Jorge Luis Borges e ai suoi labirinti. \emph{Ossessivamente sogno di un labirinto piccolo, pulito, al cui centro c'è un'anfora che ho quasi toccato con le mani, che ho visto con i miei occhi, ma le strade erano così contorte, così confuse, che una cosa mi apparve chiara: sarei morto prima di arrivarci} (J.L. Borges)}

\descrizione{Antidinamica}{La composizione ha inizio con l’elaborazione, in tempo reale, della registrazione di un’improvvisazione di Gianpaolo Antongirolami (sax contralto) per la mia opera Il domestico di Edgar. L’analisi spettrale di questa elaborazione, che non contiene più alcun suono riferibile alla sorgente acustica, definisce la partitura per i sassofoni in ANTIDINAMICA. Il sassofonista può scegliere e cambiare il metronomo (semiminima da 20 a 120) ad ogni pagina, così come quali e quanti pentagrammi eseguire, fra i 6 di ogni pagina, ma entro una durata stabilita (minimo 6’, massimo 12’).  Nei restanti 4’ (massimo 6’) prosegue improvvisando liberamente sulla memoria di quanto precedentemente letto. L’interprete al live electronics improvvisa liberamente sui parametri di un ambiente costituito da una convoluzione con impulsi filtrati e inviati a 4 ritardi in parallelo con traspositore nel feedback, dapprima sulla registrazione usata per generare la partitura e poi sui sassofoni dal vivo.}

\descrizione{Melancholia}{il pezzo è stato commissinato da Gianni Trovalusci, Melancholia si è ispirato dall’omonimo film di Lars von Trier. Nel corso del pezzo il flautista recita un brevo testo di Giulia Laurenzi scritto originalmente per questo brano. \emph{Sanguino, sacra madre impietrita, l’horror vacui latra.}}

\end{flushleft}
