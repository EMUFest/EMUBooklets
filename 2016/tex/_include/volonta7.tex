% !TEX encoding = UTF-8 Unicode
% !TEX TS-program = XeLaTex
% !TEX root = ../EMU2016_booklet.tex

\begin{flushright}

\large{
	\scshape{
	27 ottobre 2016 -- ore 20:30
	}}

\medskip
	
\small{Concerto
	\newline Sala Accademica}

\medskip

{\fontsize{42}{42} \svolk{\emph{Volontà VII}}}

\normalsize

\medskip

regia del suono \textsc{Pasquale Citera} e \textsc{Massimiliano Mascaro}

\bigskip

\livel{Massimo Massimi}{Il senso ottuso}{13’00}{per voce femminile e live electronic}{2014}
\medskip

\livel{Alessio Gabriele}{Âme lie}{12’00}{per voce femminile, sassofono contralto con WindBack\footnote{il sistema Windback è stato ideato da Michelangelo Lupone, i brani sono stati prodotti dal Centro Ricerche Musicali}, elettronica}{2016}
\medskip

\livel{Antonio Russo}{Ridefinizione}{8’00}{per voce femminile ed elettronica}{2016}
\medskip

\livel{Giuseppe Silvi}{S4FE}{13’00}{per sax alto, WindBack\footnote{} \& S.T.ONE}{2016}
\medskip

\livel{Silvia Lanzalone}{Feedback for Two}{10’00}{per 2 voci, 2 megafoni e 2 sistemi di feedback}{2016}

\bigskip

\esecutore{mezzosoprano}{Virginia Guidi}
\esecutore{soprano}{Eleonora Claps}
\esecutore{sassofono e windback}{Enzo Filippetti}
\esecutore{live electronics}{Massimo Massimi, Alessio Gabriele, Antonio Russo, Marco Matteo Markidis}

\vfill

\descrizione{Il senso ottuso}{Dal punto di vista di una ricerca sull’emissione vocale, \emph{Il senso ottuso} diviene un lavoro sulla voce nel senso più restrittivo e più ampio allo stesso tempo; il materiale sonoro è costruito a partire da ciò che più naturalmente con l’apparato vocale si possa produrre piegandosi poi ad una necessità espressiva legata a forme innaturali dell’emissione nella direzione di un effetto puramente acustico. La conseguente scarsa intelligibilità del testo tende a svincolare l’ascoltatore dai significati letterali dirigendolo verso un ascolto della parola o morfema come un contenitore acustico/psicologico da manipolare e deformare a ridosso di un urgenza emotiva dell’interprete. La parte elettronica viene generata a partire dai materiali prodotti dal vivo e elaborati con la stessa volontà di annichilire il testo letterale e mettere in luce una dimensione psichica.}

\end{flushright}

\clearpage

\begin{flushleft}

~\vfill

\descrizione{Âme lie}{Un ideale respiro che fluisce ininterrotto è l’anima che lega la voce e il sassofono, come nel rapporto indissolubile tra entità generatrice e generata, in un percorso che esplora stadi diversi di interazione e compenetrazione fra corpi vibranti. Il WindBack, oltre che dispositivo non convenzionale di eccitazione del sassofono, è inteso come strumento mediante cui le intenzioni musicali degli esecutori possono fondersi nel corpo dello strumento o proiettarsi all’esterno di esso, arricchite delle deformate acustiche e timbriche indotte dal feedback. Il brano - costruito in tre sezioni a partire da una successione di 6 note corrispondenti ad un crittogramma delle lettere che compongono il titolo del brano – è un quadro sonoro in cui voce, strumento ed elettronica dal vivo sono posti reciprocamente in dialogo, sovrapposizione, fusione. Il WindBack è stato ideato da Michelangelo Lupone. Il brano è stato prodotto dal CRM - Centro Ricerche Musicali di Roma.}

\descrizione{Ridefinizione}{Ridefinizione si fonda sul continuo tentativo di definire e capire la relazione che esiste con il suono elettronico, tra l’algoritmo ed il segnale, tra il suono elettronico e la voce. Nel brano si fa uso di suoni di sintesi e di suoni generati dall’elaborazione in tempo reale della voce, realizzata mediante algoritmi di analisi dell’inviluppo d’ampiezza e filtri digitali.}

\descrizione{S4FE}{La \emph{Song for Enzo Filippetti} racconta di una visione, di una speculazione sul Tempo, di una storia di liberazione. Il Tempo vive la sua vita, percezione eterna ed impalpabile. Lo vedevo consumarsi, il Tempo, senza mai rigenerarsi. Un Tempo. In questa storia il Tempo è libero, libero di consumarsi, di tornare (di fermarsi?). Ha una forma. È lo spettro della percezione stessa, legata, scandita, vissuta. Il racconto non ha una direzione. Il racconto ha ogni direzione. In ogni direzione, dentro il volto di una coincidenza c’è il corpo della concatenazione. Il racconto va e viene, guarda in volto una coincidenza per scorgerne, nel fondo degli occhi, la concatenazione. Non esiste bidimensionalità quando si guardano fissi due occhi e se ne scorge lo spettro. In un Tempo libero.}

\descrizione{Feedback for Two}{Un gioco, una sfida, un intreccio, un duello. Le due voci, a volte in contrappunto, si fondono e con-fondono. Il brano sperimenta alcune possibili relazioni tra voci e sistemi di feedback. Ciascun sistema di feedback realizza due controreazioni: la prima tra microfono e megafono e la seconda tra megafono e altoparlante. Ciascun megafono è posto al centro del processo di feedback e si inserisce come \emph{strumento} di elaborazione dei suoni di voce. Il brano è stato realizzato presso il CRM-Centro Ricerche Musicali di Roma.}

***

\biografia{Massimo Massimi}{si è formato musicalmente presso il Conservatorio Santa Cecilia di Roma, diplomato in liuto e musica elettronica, ha affrontando lo studio della musica antica e successivamente si è dedicato alla composizione elettroacustica con particolare attenzione all’interazione tra strumento e macchina.}

\biografia{Alessio Gabriele}{Compositore e violinista, compie gli studi musicali presso i Conservatori di Frosinone e L’Aquila dove si diploma in Violino e Musica Elettronica. Consegue la Laurea in Informatica presso l’Università di L’Aquila. I suoi lavori, commissionati ed eseguiti in Italia e all’estero, comprendono opere strumentali, brani acusmatici, installazioni sonore d’arte interattive e adattive. Dal 2004 collabora con il CRM - Centro Ricerche Musicali di Roma, in qualità di compositore e ricercatore senior. Come interprete svolge attività concertistica in Italia e all’estero. È attualmente docente nei corsi accademici di Musica Elettronica presso i Conservatori di L’Aquila e Salerno.}

\biografia{Antonio Russo}{(Capua, 1992) Si è diplomato al Liceo Scientifico, studia pianoforte, frequenta attualmente l’ultimo anno del corso triennale di Musica Elettronica presso il Conservatorio di Musica *G. Martucci* di Salerno. Ha seguito diversi seminari e masterclass di Composizione Elettroacustica presso il Conservatorio.}

\biografia{Giuseppe Silvi}{[1981] è musicista elettroacustico e sassofonista. Studia sassofono con Enzo Filippetti, musica elettronica con  Giorgio Nottoli, Nicola Bernardini and Michelangelo Lupone, elettroacustica con Piero Schiavoni presso il  Conservatorio di Musica S. Cecilia di Roma. La sua ricerca sullo spazio sonoro e le dimensioni musicali lo hanno portato alla costruzione di prototipi elettroacustici e software per la produzione musicale elettroacustica. Collabora con il  Conservatorio di Musica S. Cecilia di Roma, il Centro Ricerche Musicali di Roma e l'Università di Roma Tor Vergata. È membro dello staff di EMUFest per il quale è regista del suono. È tecnico del suono specializzato in produzioni e registrazioni surround, incide per Tactus, Naxos, Brilliant Classic e Sony.}

\biografia{Silvia Lanzalone}{(Salerno 1970), compositrice e ricercatrice. Diploma di Flauto, Composizione e Musica Elettronica presso i Conservatori di Salerno, L’Aquila e Roma. Dal 1997 collabora con il CRM - Centro Ricerche Musicali di Roma. Dal 2009 è coordinatore del Dipartimento di Nuove Tecnologie e Linguaggi Musicali e docente di Composizione Elettroacustica presso il Conservatorio di Salerno. La sua produzione musicale prevede opere con live electronics, strumenti aumentati, installazioni sonore d’arte, opere multimediali e interattive. Ha pubblicato con Ars Publica, Taukay e Suvini Zerboni. (http://www.silvialanzalone.it/)}

\end{flushleft}
