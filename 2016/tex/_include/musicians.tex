% !TEX encoding = UTF-8 Unicode
% !TEX TS-program = XeLaTex
% !TEX root = EMU2015_booklet.tex

%2016-701:
\biografia{Filippo Ansaldi}{ was born in Savigliano in 1992, and has attended M° Pietro Marchetti’s class in the Turin Conservatorio *G. Verdi*, obtaining his masters degree in 2014 with the highest possible marks. He has attended Master Class with internationally recognized musicians. He has experience in multiple saxophone playing styles, such as duo, quartet and ensemble, and he works with reputable musical institutions. He has received multiple prizes and awards in national and international competitions. At the moment he is attending the 2nd level Masters course in *Interpretazione della Musica Contemporanea* at the Conservatorio *Santa Cecilia* in Rome, thanks to the help of the De Sono association from Turin.}

%2016-702:
\biografia{Sofia Bandini}{ (1999) entered the Conservatorio di Musica Santa Cecilia in Rome in 2010 to study violin with Maestro Giuseppe Crosta; she has just been admitted to the 9th year of the course. In the last years, she has attended several Masteclasses with Maestros Georg Mönch, Alina Company, Felice Cusano, David Romano, Roberto Gonzàlez-Monjas and Marlène Prodigo. Since 2007, she has played in orchestras and had the opportunity to play together with top level musicians such as Antonio Pappano, Shlomo Mintz, Francesca Dego, Salvatore Accardo, Ezio Bosso. Since 2013 she studies string quartet with Maestro Alberto Mina and in September 2016 she has been admitted with her fellows to the Accademia Walter Stauffer in Cremona. The quartet has been invited to play at Camera dei Deputati, Villa Madama, Castel Sant’Angelo in Rome and was awarded the first price for chamber music at the Concorso Nazionale Riviera Etrusca in 2016.}

%2016-703:
\biografia{Sauro Berti}{ Bass clarinet of the *Teatro dell’Opera di Roma*, collaborated with the most important Italian orchestras (Teatro alla Scala, Maggio Musicale Fiorentino, RAI National Orchestra), the Royal Scottish National Orchestra and the Sinfonia Finlandia Jyväskylä. He played under G. Prêtre, R. Chailly, M. W. Chung, R. Muti, W. Sawallisch, V. Gergiev, L. Maazel, P. Boulez and Z. Mehta. He participated as a soloist in various renown music festivals worldwide. In 2009 he obtained his conducting diploma with D. Renzetti. He published *Venti Studi per Clarinetto Basso*, *Tuning* for winds (Suvini Zerboni), his version of V.Bucchi’s concerto and the CDs: *Suggestions*(Edipan) and *SoloNonSolo*(ParmaRecords).}

%2016-704:
\biografia{Francesco Bianco}{ Musician and co-founder of Studiolo Laps netlabel. He is a composer and musician in various groups ranging from experimental to alternative music and performance. His training is enhanced by numerous study experiences and comparison with academic and extra-academic field. His research aims at the deep relationship between art and life, society, time, spaces, communication, language.}

%2016-705:
\biografia{Tiziano Capponi}{ nasce il 21/03/1992 a Roma. Inizia a studiare teoria musicale e percussioni classiche presso la scuola Musica S.Cecilia sotto la guida del maestro Aldo Tamantini. Parallelamente si avvicina allo studio del pianoforte e della batteria. Inizia gli studi in Percussioni al Conservatorio S. Cecilia di Roma, inizialmente con il maestro Michele Iannaccone e in seguito con il maestro Gianluca Ruggeri, con il quale si diploma nel 2015 con il massimo dei voti.}

%2016-706:
%\biografia{Maureen Chowining}{ Soprano, she has studied in Boston and New England conservatories. Her performances were held around the world, among those most notable were the International Electronic Music festival in Bourges, where in 1990 and 1997 she performed the premieres of *Solemn Songs for Evening* by Richard Boulanger and *Sea Songs* by Dexter Morrill. In  2005 she first performed *Voices*, a composition dedicated to her by John Chowning, at Maison de Radio in Paris. The soprano is renown for the flexibility of her voice, which allows her to meet different styles and repertories, alternative intonations, as the scale of Pierce.  During 27 years of her career she has given a number of private vocal lessons, as well as many Master classes around the world.}

%2016-707:
\biografia{Pasquale Citera}{ (1981) He studied piano with M°Gemma D'Alessio, Composition with M°Luciano Pelosi and M°Giovanni Piazza and Electronic Music with M°Giorgio Nottoli. He has been working with several theatre companies, movie studios, sculptors and photographers. He has composed music for classical and contemporary theater shows, like: Alcestis (Euripides), Lysistrata (Aristophanes), Amphitryon (Plautus), Locandiera (Goldoni), L’Avare (Molière), Da quale parte del vetro (Silvio Nanni), Il dito sulla bocca (Donatella Ferrara), Certe notti non accadono mai (Patrizia Masi). He wrote soundtracks for Nero-Film, he is Assistant Music in several schools of Rome and has been Professor of Music Technology. From the collaboration with the sculptor Arturo Ianniello are born different soundtracks of visual works collected in two exhibitions. He is currently Composer and Sound Designer for incidental music at Anfitrione Theatre and *Quercia Del Tasso* Amphiteatre.}

%2016-708:
\biografia{Eleonora Claps}{ Born in Basilicata, she studied with E. Scatarzi at the Conservatory *G. Martucci *(SA). She perfect herself under the guidance of A. Caiello, attended the *Corso di specializzazione in Canto Lirico per l’Opera Contemporanea* (Verona Opera Academy) and the ‘Internationales Musikinstitute of Darmstadt’ Summer Courses, performing in IMD2016 official concert. She was one of the finalist in the * Premio Bucchi Interpretazione – Parco della Musica 2015*, she is the vocal interpreter of *ScarlattiLab/Electronics*, and regurarly performs concerts about the '900 and contemporary music, acoustic and electroacoustic.}

%2016-709:
\biografia{Alice Cortegiani}{ Born in 1994, she began studying Clarinet at 8 years old. In 2014 she graduated at Conservatory S. Cecilia of Rome under the guidance of D. Rossi with the Highest Vote. Actually she continues her specialize's studies with A. Carbonare and frequents Postgraduates Master of Chamber Music with R. Galletto. She obtained scholarship Erasmus+ at the Royal Academy School of Music of London and at the Real Conservatorio Superior de Musica de Madrid, studying with A. Garcès. She has participated in Masterclass with F. Meloni, K. Leister, V. Alberola Ferrando. She immediately began a rich and full concert activity in the orchestra, as a soloist and in chamber music. She has plays in chamber ensemble: Imago Sonora Ensemble, Duo Essentia with the accordionist S. Telari, PentElios Quintet (woodwind quintet), Trio Alpha (clarinet, cello and piano), Duo with the pianist A. Viale.}

%2016-710:
\biografia{Elena D'Alò}{ is a flutist, from piccolo to bass flute. She studied Flute (Diploma and Master's degree) in *Santa Cecilia* Conservatory in Roma, with teachers: Edda Silvestri, Bruno Paolo Lombardi, Deborah Kruzansky and Paolo Taballione. She also graduated in Acoustic Physics (Bachelor's degree) in *La Sapienza* University with Paolo Camiz as supervisor. Now she studies Electronic Music. She plays chamber music and in orchestral concerts too: from barocco to contemporary repertory.}

%2016-711:
\biografia{Michele D'Auria}{ Nato a Salerno nel 1993, intraprende all’età di otto anni lo studio del sassofono. Si diploma nel 2013 in Sassofono con il massimo dei voti al Conservatorio di Salerno. Frequenta numerose masterclasses tenute da insegnanti di fama internazionale,  partecipa a numerosi concorsi nazionali ed internazionali classificandosi sempre come primo assoluto,  è stato vincitore del *Premio V. Cammarota* 2014 per la musica contemporanea, Premio *Claudio Ceschini* per miglior sassofonista,  riceve un riconoscimento dal Ministry of Human Resources of the Hungarian Republic. Si è esibito in numerosi concerti solistici, in formazioni cameristiche e orchestrali, in Italia e all’estero.}

%2016-712:
\biografia{Marco De Martino}{ Born in Rome. After the early studies in Piano and the high school degree, the education path follows with studies at the University of Rome Tor Vergata. Graduated in Musicology, he studied electroacoustic composition with Giorgio Nottoli and Michelangelo lupone, then Computer Music with Nicola Bernardini. His teachings begins as assistant in Computer Music for the conservatoire and for the 2nd level Master in Interpretation of contemporary music, always for Rome’s conservatoire.}

%2016-713:
\biografia{Sara Ferrandino}{ graduated in piano in 2005 at the Conservatory of Perugia, in the class of M° L. Tanganelli. At the same institution, in March 2009, she passed the level II Academical Diploma with top marks and special mention. In July 2012 she obtained the specialist diploma for the postgraduated course held by Mº S. Perticaroli at the Santa Cecilia Academy in Rome. She has partecipated in more than 30 national and international piano competitions, always reaching the top. She plays both as a piano soloist and in chamber ensembles with important musicians in prestigious classical concert halls in Italy and abroad. She is a piano teacher for the pre-academical courses at AIMART. She also collaborates with the Conservatory of Perugia for the courses of horn, trumpet, flute, violin, saxophone, oboe and works as an artistic consultants in other important musical institutions in Rome, organizing masterclasses and pianistic competitions at international level.}

%2016-714:
\biografia{Enzo Filippetti}{ is professor of Saxophone and of Second Level Master of Art Course in the Interpretation of Contemporary Music at Conservatorio *S. Cecilia* in Rome. In more than thirty years he gives concerts all over the world. He has performed at Biennale di Venezia, Mozarteum di Salisburgo, Rome, Milan, Paris, London, Berlin, Wien, Madrid, Bruxelles, Buenos Aires, Caracas, Riga, Birmingham, Köln, Lyon, St. Etienne (Francia), Principaute-Monaco-Monte-Carlo, Yeosu (Korea), Kawasaki, Adis Abeba, Chisnau, Taormina, Ravello. He has collaborated with Claude Delangle, Alda Caiello and Bruno Canino and many of the most important composers wrote for him more than a hundred works. As a soloist and with the Quartetto di Sassofoni Accademia he has recorded for the Nuova Era, Dynamic, Rai Trade and Cesmel. He has published studies for Riverberi Sonori and he direct a collection for the Sconfinarte editions.}

%2016-735:
\biografia{Paolo Fumagalli}{ Born in 1978, he took a diploma in performance on the violin under the guidance of Elena Ponzoni at the *Cantelli* Conservatory in Novara and on the viola with Roberto Tarenzi at the *Nicolini* Conservatory in Piacenza, going on to study further with Maja Jokanovic, Claudio Pavolini and Simonide Braconi in Italy and Switzerland. He has been invited to give concertsof chamber music by the Settimane Musicali di Stresa, the Ligeti-Milano Musica Festival, the Fondazione *Fernando Rielo* in Rome, Contemporaneamente Lodi, the Teatro Bibiena in Mantua, the Teatro Municipale Piacenza, Amici della Musica Palermo, Nuova Consonanza in Rome, the Staatsoper Stuttgart, Europa Musica, Musica y Escena in Mexico City and WDR in Cologne. First viola of the *Luigi Cherubini* Youth Orchestra conducted by Riccardo Muti from 2005 to 2008, he held the same positionat the Orchestra La Fenice in Venice and the Teatro G. Verdi in Trieste. He works permanently with various orchestras, conducted by L. Maazel, Y. Temirkanov, R. Barshai, K. Masur and E. Inball. He has been a memberof the string sextet of the academyof theTeatro alla Scala, with which he performed at the Ravello Festival and in the foyerof La Scala, going on to take part in international toursof Asia and South America. He works with some of the most important chamber music ensembles inItaly that play the contemporary repertoire, such as Divertimento Ensemble, Sentieri Selvaggi, Ensemble Icarus, Ensemble Risognanze and Xenia Ensemble. He currently teaches viola at the *Dedalo* school in Novara. He has made recordings for Stradivarius, Ricordi Oggi, Aeon and Limen, and of a previously unrecorded duo for viola, harp and electronic instruments by Robert HP Platz for West Deutsche Rundfunk.}

%2016-734:
\biografia{Lorenzo Gentili-Tedeschi}{ was born in Milan on December 31st 1988 and graduated cum laude at the age of 16 from *Gaetano Donizetti* Conservatory in Bergamo. He attended lessons with Francesco de Angelis at the Haute école de musique de Sion (Switzerland), where he obtained his Master degree performing Beethoven violin concerto with Orchestre de Chambre de Lausanne. He afterwards taught in Sion as Francesco De Angelis’ assistant for two years. From 2014 he is a member of the London Philharmonic Orchestra, with whom he performs at Royal Festival Hall and Royal Albert Hall for the BBC Proms in London, tours extensively Europe, America, China and plays at Glyndebourne Opera Festival. Lorenzo performed with the main italian orchestras as Orchestra del Teatro alla Scala, Filarmonica della Scala, Orchestra del Teatro Regio di Torino, Orchestra da Camera di Mantova and I Solisti di Pavia, under the baton of Gustavo Dudamel, Daniel Barenboim, Valery Gergiev, Riccardo Chailly, Yuri Temirkanov and many others. He is regularly guest leader at Orchestra del Teatro Petruzzelli in Bari after being leader of Orchestra dell’Accadmia del Teatro alla Scala for five years, appearing as a soloist too in the Kammerkonzert by Alban Berg in Rome and Milan. In 2012 Lorenzo was a member of the Lucerne Festival Academy, working with Pierre Boulez and Pablo Heras Casado as leader of the ensemble.}

%2016-715:
\biografia{Arianna Granieri}{ She graduated Piano studies of II level under the guidance of M° Cinzia Damiani, at Conservatorio Santa Cecilia of Rome, winning the highest grades, an honorary mention, after having presented a sofisticated repertoire of modern and contemporary Italian music. During the period of studies, she has participated in various Master classes with worldwide known musicians, as Boris Berman.  She also graduated with masters degree with honors in Philosophy at University of Roma Tor Vergata,  having presented a thesis on Japanese and Samurai aesthetics. She has performed as a solo pianist as well as in ensembles in various musical events and festivals, among these the Accademic Concerts,  Orvieto Festival of Strings, Domeniche Estive a Castel Sant’Angelo, AnemosArts, The 90th anniversary of Franco Evangelisti’s birth, the program of Rai | fatti vostri, moreover has performed the premiere of Henry Cowell’s Concerto per pianoforte e orchestra version for two pianos. She is interested in unity of music and other arts, particularly with theater (she has performed in various theatrical plays).}

%2016-716:
\biografia{Virginia Guidi}{ Graduated in Singing and in Vocal Chamber Music at the Conservatory S. Cecilia, where she specialized with honors with Silvia Schiavoni with a thesis on the relationship between performer and composer in electroacoustic music. Ranging from chamber music to contemporary with an emphasis on experimental music. She has collaborated with many composers often performing pieces dedicated to her. She has performed in Italy and abroad (Washingont DC, Bejing) and has participated in important festivals (EMUfest, Venice’s Biennale, ArteScienza) and installations by famous artists (Allora&Caladilla, Thomas De Falco).}

%2016-717:
\biografia{Filiz Karapınar}{ has played in many of Turkey's leading orchestras including the Borusan Istanbul Philharmonic, the Istanbul State Symphony, the Bilkent Symphony, the Antalya State Symphony, the Bursa State Symphony, and the Çukurova Symphony. An active performer of contemporary music, she is a core member of the MIAM Modern Music Ensemble, and has given Turkish premieres of works by composers such as Fred Lerdahl and Kamran İnce. She won the special jury prize for the best performer of contemporary music at the inaugural Cahit Koparal National Flute Competition in 2007. She has participated in master classes with Robert Winn, Patrick Gallois, Andras Adorjan, Emmanuel Pahud, Davide Formisano, Julien Beaudiment and Sophie Cherrier and was a member of the Turkish-Greek Youth Orchestra conducted by Vladimir Ashkenazy. She holds a Master's degree from Istanbul Technical University's Dr Erol Üçer Center for Advanced Studies in Music, where she studied with Bülent Evcil, and is currently a PhD candidate in Flute Performance studying with Jülide Gündüz.}

%2016-718:
\biografia{Jacopo Lazzaretti}{ Born in Rome in 1994, Jacopo Lazzaretti  is studying with Arturo Tallini at the Conservatoire of Santa Cecilia in Rome and this year he will complete his Diploma. He has studied as an Erasmus student at the Royal Conservatoire of Scotland with Matthew McAllister. He has performed in masterclasses with some of the worlds most well-known guitarists such as Oscar Ghiglia, Marcin Dylla, Pavel Steidl to name a few. He has won several prizes in national and international guitar competitions with or without age limitations. He has played in many concerts for several music Festivals in Italian cities and has also performed abroad in Istanbul, Glasgow and Aberdeen.}

%2016-719:
\biografia{Ivan Liuzzo}{ nasce a Frosinone il 26 febbraio 1993 ed inizia a studiare la batteria all’età di 9 anni con M. FIOCCO, successivamente con G.GUIDONI e A.BLASI. Nel 2007 intraprende gli studi delle percussioni presso il Conservatorio *L.Refice* di Frosinone con il M° C. DI BLASI, conseguendo il diploma (V.O.). Ha approfondito lo studio della batteria jazz con R. PISTOLESI e G. HUTCHINSON. Membro attivo e co-fondatore del Collettivo Phthorà, assieme a F. Ferazzoli e F. Abbate ha collaborato con artisti come: Lisa Mezzacappa, Stefano Costanzo, Vincenzo Core, Wound, Ron Grieco, Achille Succi ecc.}

%2016-720:
\biografia{Leonardo Mammozzetti}{ was born in Chapecó (Brazil) on 11 October 1985. Based in Rome, he attended the degree course Electronic Music at the Santa Cecilia Conservatory in Rome. Acoustics technical aspects of the materials, particularly metals, have always been part of his studies and experiments in his musical/technical projects; he is focused on the composition of electroacoustic music. He is occasionally a musical assistant of the CRM under the direction of Maestro Michelangelo Lupone.}

%2016-721:
\biografia{Massimiliano Mascaro}{ composer. He was born in Rome in 1986. He studies with M° Michelangelo Lupone and M° Nicola Bernardini. He studied at the Conservatory *A. Casella* in L'Aquila and He currently attends The Concervatory of *S. Cecilia * in Rome. He attends courses of Electroacoustic Composition and Classical Composition. The Electroacoustic music is the field in which he mainly carries out his musical activity.}

%2016-736:
\biografia{mdi ensemble}{ was formed in Milan in 2002 from the idea of six young musicians sharing their common passion for contemporary music, thanks to the support of Musica d’Insieme Association. From the very beginning, the ensemble established close collaborations with many renewed composers such as Helmut Lachenmann, Sofia Gubaidulina, Gérard Pesson, Pierluigi Billone, as well as premiering several new works by young composers emerging from the international scene. At the same time, the ensemble frequently collaborates with eminent conductors such as Beat Furrer, Yoichi Sugiyama, Robert HP Platz, Marino Formenti and Pierre-André Valade.}

%2016-722:
\biografia{Alessandro Pace}{ graduated in Flute with M°Carlo Morena with score of *110 e lode* at the music academy of Santa Cecilia. Now following his studies with flute and traditional composition in the same music academy. He did and does a conspicuous concerts activity of various genres. He plays in the following ensembles: Orchestra Ars Ludi Romana (also as a soloist); Broadway Musical Orchestra (es. Festival di Todi); Indivenire Ensemble (contemporary repertoire). He played with the Panama's national orchestra in Panama City. He performed in the Contaminazioni festival both as flutist and composer. He joined the project of M° Antonio Di Pofi  about Silent film's music, both composing and playing for it. He played quite a lot of chamber music with various ensembles and continuously search for new experiences.}

%2016-723:
\biografia{Federico Paganelli}{ born in Rome studies electroacoustic music at Conservatorio Santa Cecilia with Nicola Bernardini and Michelangelo Lupone, formerly he studied with Giorgio Nottoli.}

%2016-724:
\biografia{Danilo Perticaro}{ Born in Cosenza in 1992, young undertook the study of the Saxophone at the Conservatorio S. Giacomantonio of Cosenza, earning under the guidance of M° Luigi Grisolia, with 110/110 cum laude with a degree in Saxophone 2st level. He has participated in numerous competitions and national and international events resulting in many of them the winner, receiving praise and positive reviews from the various juries. Has participated in masterclasses held by F. Salime, F. Moretti, C. Delangle, D. Espinoza, M. Marzi, E. Filippetti, L. Mlekusch, A. Bornkamp. He is very active teaching and concert, soloist,chamber,orchestre and ensemble, particularly in the field of contemporary music.  He has launched a project of CD recording of all the works for saxophone of the renowned Stockhausen. He collaborates with many composers in the realization of works dedicated to him.}

%2016-725:
\biografia{Alessandro Pirchio}{ He studies at the Santa Cecilia Conservatory in Rome with M° Albanese. He has performed as soloist and in chamber music ensembles for: Musica a Roma per Roma*; *Sutri Beethoven Festival*; Chamber Music season of the Viterbo Ceramic Museum. Moreover, he played for the theatrical performance *Twelfth Night* (*Le maschere del teatro 2015* Award for original soundtrack by M° Piovani) in several Italian theaters (*Donizetti* in Bergamo, *Ponchielli* in Cremona, *Verdi* in Padova, Pistoia and Ravenna, among the most important). He is currently First Flute in the band of the Vatican Gendarmerie and Carabinieri National Association.}

%2016-726:
\biografia{Federico Ripanti}{ Born in Rome in 1987, Federico Ripanti is currently enrolled in Electronic Music at the *S. Cecilia* Conservatory of Music. In 2009 he graduated in Audio and Music Technology at the Saint Louis Music College. He attended private classes of piano, electric guitar and African percussions.}

%2016-727:
\biografia{Alice Romano}{ Born on July 27th, 1995, Alice Romano started cello studies with M° M. Scarpelli. She took part to several orchestra projects by Accademia Nazionale di S. Cecilia, as a member of Juniorchestra! and now as an Education sector tutor. Moreover, she's a new member of the Ljubljana International Orchestra (BSA). Graduated from Conservatorio di S. Cecilia (M° M. Massarelli), she's just started her studies by La Sapienza in Rome, in Languages, cultures, literature, translation.}

%2016-728:
\biografia{Matteo Rossi}{ is a percussionist player. He studied percussions in *S.Cecilia* Conservatory  with teacher Gianluca Ruggeri. Attends a perfectioning course with the Chigiana Musical Accademy directed by Antonio Caggiano and performs as a member of Chigiana Percussion Ensemble at CHIGIANA INTERNATIONAL FESTIVAL, RAVELLO FESTIVAL and MAXXI in Rome. Also collaborates with orchestras and chamber ensembles such PMCE, InDivenire Ensemble and percussion ensembles which Ars Ludi, Blow-Up Roma Percussion, Aere Silente with whom he performs in a modern and contemporary percussionistic repertory during the events such as Le esperienze del minimalismo, Le Forme del Suono, Artescienza, EMUFest.}

%2016-729:
\biografia{Luca Sanzò}{ pupil of Bruno Giuranna, conducts an intense professional activity dividing his time between concerts, recordings and teaching. He is particularly interested in the production and dissemination of new music, of which he is an esteemed performer. Many composers have dedicated works to him and consider him a reference point in the interpretation of their works. He has played as Principal Solo Viola, with the Rome Opera House, the Teatro Lirico in Cagliari and the Concerto Italiano, a Group which has allowed him, thanks to a collaboration with  amongst the best instrumentalists in this sector, to deepen his knowledge of philological performance of Baroque Music using original instruments. He has published a revision of Campagnoli’s 41 Caprices for solo viola, for Ricordi.  He is viola professor at the Rome Conservatorium of Santa Cecilia. He has recorded for the following labels:  Nuova Era, Bottega Discantica, BMG Ricordi, Opus 111, Tactus, Edi Pan, Stradivarius, Naïve, Chandos, Naxos and the Hindemith viola and piano sonatas and Brahms viola and piano sonatas with Brilliant Classics.}

%2016-730:
\biografia{Franco Sbacco}{ Graduated in Composition, Conducting, Electronic Music and Percussions, respectively with D. Guaccero, D, Paris, G, Nottoli and L. Torrebruno. Specialized in Composition at Accademia Nazionale of S. Cecilia in Rome with G. Petrassi and F. Donatoni in 1979. In 1989 he won a Canadian bursary for research activities on computer music with the system PODX of B. Truax at the Simon University of Vancouver. He worked in musical theatre. His compositions are executed in Italy and all over the world in the main contemporary music festivals. He also won the Varese's 3rd International Competition of Electroacoustic Music and the 18the Bourges's International Contest of Electroacoustic Music. He recorded for BMG Ariola - Roma, for DOMANI MUSICA - Roma, which has named after him two monographic cds in 1999 and in 2008 moreover, he recorded for ALBANY RECORDS too (New York). Nowdays he teachs analysis an harmony at the Conservatory of Santa Cecilia in Rome.}

%2016-731:
\biografia{Arturo Tallini}{ A professor of Conservatorio Santa Cecilia, he regularly holds Master classes in conservatories of  Italy and universities abroad. Considered to be a reference of contemporary repertoire, he collaborates with internationally renown artists such as Michiko Hirayama, the contemporary music group Modus Novus from Madrid, the Coro dell’Accademia Nazionale di Santa Cecilia, the flutist Carlo Morena. He also works as a coordinator of Annual Master course of Contemporary Music Interpretation at Conservatorio di Santa Cecilia, where he gives guitar lessons as well. He performs across the Europe, USA, Egypt, Algeria and Tunisia.}

%2016-732:
\biografia{Gianni Trovalusci}{ Contemporary repertoire with Pierre-Yves Artaud in Paris and Performing Practices of Early Music with Jesper Christensen and Traversiere with Oskar Peter at the Schola Cantorum Basileensis. He performed in the field of contemporary and ancient music, in music theatre and avant-garde performance at very important Festivals as NYCEMF New York City Electroacoustic Music Festival; Munich Biennale; Nuova Consonanza, Museo Casa Scelsi, Musica e Scienza, EMUFest Rome; M.A.N.C.A. Festival, Nice; GAS Festival, Goteborg; Udine Jazz Festival; REC Reggio Emilia Contemporanea, British Film Institute London, Nancy Opera, Flanders Opera, Ars Electronica - BrucknerHaus Linz, Neue Alte Musik Cologne, CCA Glasgow, Stockholm New Music, Nits de Musica Mirò Foundation Barcelona, etc.}

%2016-733:
\biografia{Francesco Ziello}{ He’s graduated in 2012 in Electronic Music at the Conservatory of Rome *Santa Cecilia*, where he also studied piano and composition. As a composer, works for theater and the National Dance Academy of Rome for a several shows, the first presented at the CRM (Music Research Centre) in July 2015. As multi instrumentalist and performer he works mainly in the contemporary music, playing in important festival as EMUFest and ArteScienza. In 2016 he’s working as a musical operator in psychiatric rehabilitation in collaboration with Tor Vergata University of Rome.}


%2016-737:
%\biografia{SAXATILE [modulable sax ensemble]}{Di recente costituzione, è un progetto realizzato da un’idea di Enzo Filippetti specificamente rivolto all’esecuzione della musica contemporanea, nei suoi molteplici aspetti, con un orientamento diretto alla ricerca e all’instaurazione di un rapporto dialettico con altri artisti e con i compositori. I componenti, emanazione del Conservatorio di Roma, possono vantare una solida, variegata esperienza.}
