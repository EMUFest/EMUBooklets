% !TEX encoding = UTF-8 Unicode
% !TEX TS-program = XeLaTex
% !TEX root = ../EMU2016_booklet.tex

\begin{flushright}

\large{
	\scshape{
	29 ottobre 2016 -- ore 18:00
	}}

\medskip
	
\small{Concerto Acusmatico
	\newline Il Suono di Piero [Aula Bianchini]}

\medskip

{\fontsize{42}{42} \svolk{\emph{Volontà X}}}

\normalsize

\medskip

regia del suono \textsc{Massimiliano Mascaro}

\bigskip

\livel{Gustavo Adolfo Delgado}{Microelektra I}{5'00}{}{1999}
\medskip

\livel{Kazuya Ishigami}{Genshi No Umi ver2.0 -primitive sea ver2.0-}{7'14}{}{2016}
\medskip

\livel{Filippo Mereu}{Lamie}{7'05}{}{2015}
\medskip

\livel{Michele Papa}{Imenottero Aculeato - Studio preparatorio per interazione vocale}{5'40}{per mezzosoprano e sfera}{2016}
\medskip

%\brano{Christian Eloy}
%{La cicatrice d'Ulysse}{13'00''}
%{acusmatico}
%new version 2015\\

%\acusmatici{Ursula Meyer-K\"onig}
%{Allears}{2012-13}{8'}

%\vspace{6mm}

\vfill

\descrizione{Microelektra I}{La composizione si sviluppa una fantasia fra complessi ritmi e azioni costruiti avvalendosi dell’utilizzo di campioni di percussioni varie e oggetti per la casa, isolati, frammentati, tagliati fino a pochi millisecondi, alcune volte quasi senza elaborazioni ma che attraverso un arduo lavoro di montaggio servono a creare delle microstrutture elettroniche ad incastro, molto articolate e contrastanti, coordinate da numerosi prototipi di curve direzionali di tipo sospensivo/risolutivo.}

\descrizione{Genshi No Umi ver2.0 -primitive sea ver2.0-}{Circa quaranta milioni di anni fa i nostri antenati nacquero dall'antico mare. La nuova vita, già possedeva gli elementi della guerra. Questi elementi sono la sorgente degli animali e dell'essere umano. In altre parole, la guerra che continua tutt'oggi, nacque insieme alla vita.}

\descrizione{Lamie}{ Il brano trae ispirazione dalle figure mitologiche dell'antichità greca, le Lamie, creature in parte umane e in parte animalesche. Il lavoro è costruito con l’intenzione di voler compiere un viaggio nella mente, tra i pensieri onirici. I suoni diventano così informazioni che alimentano stati emotivi e ci proiettano attraverso e oltre il tempo. La vocalità è l’elemento più rilevante; le voci presenti nel brano sono eterogenee, riconducibili a voci immaginifiche, spettrali e penetranti. Il brano presenta tre momenti differenti: il primo in cui prevale una certa gestualità; il secondo momento più riflessivo, in contrapposizione al primo; il terzo è la riproposizione del materiale già presente nella prima parte con l’aggiunta di nuovo materiale gestuale e un finale che presenta una stratificazione di carattere essenzialmente vocale. Ogni singolo oggetto sonoro è stato sottoposto a diverse procedure di elaborazione: modifica del tempo e dell’intonazione, sovrapposizione multipla e ripetizione di frammenti, modulazione, filtraggio e riverbero. Sono stati utilizzati i seguenti materiali: 1. Suoni naturali registrati: materiali metallici, acqua e flora. 2. Suoni estrapolati dai media (voci liriche).}

\end{flushright}

\clearpage

\begin{flushleft}

~\vfill

\descrizione{Imenottero Aculeato - Studio preparatorio per interazione vocale}{Lo studio preparatorio che si va a presentare è basato sull’interazione tra voce ed elementi elettronici elaborati in tempo reale. Una ricerca sulla scomposizione del lemma \emph{MORPHOPOLIS} e delle caratteristiche linguistiche che una sola parola può esplicare nella sua scomposizione. L’elaborazione sarà un tramite tra voce e suoni elettronici, che, spazializzati, muoveranno l’essere metaforico che vive dentro di noi. L’\emph{Imenottero Aculeato} che si annida all’interno dell’essere umano (la sfera) è l’immagine della liberazione dai propri idoli, i quali, una volta espulsi da noi stessi si presentano con la loro forma reale: quella di insetto.}

***

\biografia{Gustavo Adolfo Delgado}{Buenos Aires (1976) – Roma.
 Ottenne il \emph{Diploma di Secondo Livello specialistico in Musica Elettronica} presso il Conservatorio di Musica Santa Cecilia di Roma con il massimo dei voti cum laude sotto la guida del M° Giorgio Nottoli e la Laurea in Musica Elettronica presso l’Università Nazionale di Quilmes (Argentina) sotto la guida dei Maestri Pablo Di Liscia, Carmelo Saitta e Maria Teresa Luengo. Ha studiato Composizione Musicale Elettronica con James Dashow e Francisco Kröpfl. La sua produzione artistica studia argomenti quali l’orchestrazione e il transfer elettroacustico in concomitanza dell’uso del morphing spettrale insieme alla sintesi del suono. Il suo linguaggio si caratterizza per la presenza di microstrutture sonore a incastro dinamicamente articolate su molteplici spazi virtuali. Gustavo A. Delgado collabora con l’INA GRM come beta tester della suite di plugin GRM Tools. Docente di Composizione di Musica Elettronica presso il Conservatori Statale A. Vivaldi di Alessandria e di Informatica Musicale presso il Conservatorio Statale O. Respighi di Latina.}

\biografia{Kazuya Ishigami}{è un compositore, performer e tecnico del suono nato ad Osaka, Giappone, nel 1992. Ha ricevuto una laurea in Ingegneria del Suono presso la University Of Arts di Osaka e un master in Urban Informatics dalla Osaka City University. Ha studiato composizione elettroacustica presso l'INA-GRM nel 1997. Le sue composizioni sono state eseguite da diverse radio e presso diversi festival tra cui: DR(DeutschlandRadio Germany), WDR(westdeutscher rundfunk Germany), FUTURA(France), MUSLAB(Mexico), SR(Radio Saarbruecken Germany), SILENCE(Italy), VII-International-FKL-Symposium(Italy), ICMC(2015 USA TEXAS). Attualmente è docente presso la Osaka University of Arts, la Kyoto Seika University e il Doshisha Women's College.}

\end{flushleft}

\clearpage

\begin{flushright}

~\vfill

\biografia{Filippo Mereu}{Nato a Gavoi nel 1983. Ha conseguito il Diploma accademico di II livello in Musica Elettronica, presso il Conservatorio G. P. da Palestrina di Cagliari. A.A. 2008-2010 Partecipazione a corsi di formazione seminari, tenuti dai seguenti musicisti e o compositori: Xabier Iriondo, Luigi Ceccarelli, Stefano Zorzanello, Marco Donnarumma, Romeo Scaccia, Marcel Wierckx, Bernard Fort, Tim Hodgkinson, Lionel Marchetti e altri. Co-fondatore e membro del duo Terminale3 con il quale ha pubblicato l’album \emph{I-son} (TiConZero, 2014). Live electronics nei seguenti Festival internazionali: Signal, Music In Touch, Miniere Sonore, Acusmatica in Movimento, Spaziomusica, Contemporary, DI Stanze 2013, Multiversal 2015. Menzione d’onore Ambienti Sonori 2008, presso il Conservatorio G. P. da Palestrina. Collaborazione con l’Associazione Amici della musica, Festival internazionale Acusmatica, 2008-2009. Composizione ed esecuzione di musiche originali per spettacolo di danza contemporanea Paesaggi interrotti, Festival internazionale Time in jazz, Berchidda 2009. Collaborazione con la compagnia teatrale Carovana S.M.I., 2009-2010. Composizione musiche originali, Festival internazionale Acusmatica in Movimento 2011, Auditorium del Conservatorio G. P. da Palestrina; in collaborazione con il Festival Suono Francese, promosso dall’Ambasciata di Francia in Italia, in collaborazione con l’IRCAM e il M.I.U.R.. Composizione acusmatica \emph{Non Consumiamo Cage}, per supporto digitale multicanale, eseguita in prima assoluta al Festival internazionale Spazio Musica 2012, al Festival Miniere Sonore 2012, selezionata al XIX CIM Trieste, Conservatorio G. Tartini 2012. Composizione acusmatica \emph{Come rifiuti sparsi a caso}, Festival internazionale Music In Touch 2013, Conservatorio G. P. da Palestrina. Sonorizzazione live del film \emph{The Phantom of Regular Size}, del regista Shinya Tsukamoto, Festival internazionale Spazio Musica Winter 2014. Selezionato al Tempo Reale Festival 2013 (lavoro acusmatico \emph{Opera Macchina}) e 2015 (lavoro multimediale \emph{(Im)mobilitas)}.}

\biografia{Michele Papa}{Nasce a Formia nell'aprile del 1987. Abbraccia la musica all'età di 13 anni, studia pianoforte e composizione, ma dopo il liceo decide di fermarsi per intraprendere la carriera universitaria. Dopo un diploma in tecniche di missaggio e fonia e una laurea in Lettere e Filosofia, riprende gli studi musicali imbattendosi nella musica elettronica e dal 2014 riserva buona parte del suo tempo alla sperimentazione di tecniche compositive elettroniche e alla produzione di testi poetici. Attualmente è studente al Conservatorio di Santa Cecilia iscritto al dipartimento di musica elettronica con gli insegnanti Michelangelo Lupone e Nicola Bernardini.}

\end{flushright}
