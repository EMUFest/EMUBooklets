% !TEX encoding = UTF-8 Unicode
% !TEX TS-program = XeLaTex
% !TEX root = ../EMU2016_booklet.tex

\begin{flushright}

\large{
	\scshape{
	29 ottobre 2016 -- ore 18:00
	}}

\medskip
	
\small{Concerto Acusmatico
	\newline Il Suono di Piero [Aula Bianchini]}

\medskip

{\fontsize{42}{42} \svolk{\emph{Volontà X}}}

\normalsize

\bigskip

\livel{Gustavo Adolfo Delgado}{Microelektra I}{5'00}{}{1999}
\medskip

\livel{Kazuya Ishigami}{Genshi No Umi ver2.0 -primitive sea ver2.0-}{7'14}{}{2016}
\medskip

\livel{Filippo Mereu}{Lamie}{7'05}{}{2015}
\medskip

\livel{Michele Papa}{Imenottero Aculeato - Studio preparatorio per interazione vocale}{5'40}{per mezzosoprano e sfera}{2016}
\medskip

%\brano{Christian Eloy}
%{La cicatrice d'Ulysse}{13'00''}
%{acusmatico}
%new version 2015\\

%\acusmatici{Ursula Meyer-K\"onig}
%{Allears}{2012-13}{8'}

%\vspace{6mm}

\vfill

\descrizione{Microelektra I}{La composizione si sviluppa una fantasia fra complessi ritmi e azioni costruiti avvalendosi dall’utilizzo di campioni di percussioni varie e oggetti per la casa, isolati, frammentati, tagliati fino a pochi millisecondi, alcune volte quasi senza elaborazioni ma che attraverso un arduo lavoro di montaggio servono a creare delle microstrutture elettroniche ad incastro, molto articolate e contrastanti, coordinate da numerose prototipi di curve direzionali di tipo sospensivo/risolutivo.}

\descrizione{Genshi No Umi ver2.0 -primitive sea ver2.0-}{Circa quaranta milioni di anni fa i nostri antenati nacquero dall'antico mare. La nuova vita, già possedeva gli elementi della guerra. Questi elementi sono la sorgente degli animali e dell'essere umano. In altre parole, la guerra che continua tutt'oggi, nacque insieme alla vita.}

%\bigskip

%\svolk{\emph{Ho cercato di incanalare quell’energia in un percorso che la rendesse percepibile senza snaturarne l’essenza: l’energia che abita i violini di Corelli, Tartini, Vivaldi, certo non citazioni ne dirette ne indirette ma l’imprevedibile vitalità delle loro articolazioni (tremoli, arpeggi, ribattuti, sincronie e fioriture improvvise) che hanno fatto della scuola italiana, elettrica ante litteram, un irraggiungibile modello di virtuosismo strumentale; poi la tensione, il vuoto attorno e dentro alla costruzione delle frasi, le imitazioni, i pedali, la continua sovrapposizione delle corde.}}
%
%Giorgio Netti
%
%\normalfont

\end{flushright}

\clearpage

\begin{flushleft}

~\vfill

\descrizione{Lamie}{ Il brano trae ispirazione dalle figure mitologiche dell'antichità greca, le Lamie, creature in parte umane e in parte animalesche. Il lavoro è costruito con l’intenzione di voler compiere un viaggio nella mente, tra i pensieri onirici. I suoni diventano così informazioni che alimentano stati emotivi e ci proiettano attraverso e oltre il tempo. La vocalità è l’elemento più rilevante; le voci presenti nel brano sono eterogenee, riconducibili a voci immaginifiche, spettrali e penetranti. Il brano presenta tre momenti differenti: il primo in cui prevale una certa gestualità; il secondo momento più riflessivo, in contrapposizione al primo; il terzo é la riproposizione del materiale già presente nella prima parte con l’aggiunta di nuovo materiale gestuale e un finale che presenta una stratificazione di carattere essenzialmente vocale. Ogni singolo oggetto sonoro è stato sottoposto a diverse procedure di elaborazione: modifica del tempo e dell’intonazione, sovrapposizione multipla e ripetizione di frammenti, modulazione, filtraggio e riverbero. Sono stati utilizzati i seguenti materiali: 1. Suoni naturali registrati: materiali metallici, acqua e flora. 2. Suoni estrapolati dai media (voci liriche).}

\descrizione{Imenottero Aculeato - Studio preparatorio per interazione vocale}{Lo studio preparatorio che si va a presentare è basato sull’interazione tra voce ed elementi elettronici elaborati in tempo reale. Una ricerca sulla scomposizione del lemma \emph{MORPHOPOLIS} e delle caratteristiche linguistiche che una sola parola può esplicare nella sua scomposizione. L’elaborazione sarà un tramite tra voce e suoni elettronici, che, spazializzati, muoveranno l’essere metaforico che vive dentro di noi. L’\emph{Imenottero Aculeato} che si annida all’interno dell’essere umano (la sfera) è l’immagine della liberazione dai propri idoli, i quali, una volta espulsi da noi stessi si presentano con la loro forma reale: quella di insetto.}


\end{flushleft}
