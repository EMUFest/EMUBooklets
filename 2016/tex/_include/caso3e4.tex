% !TEX encoding = UTF-8 Unicode
% !TEX TS-program = XeLaTex
% !TEX root = ../EMU2016_booklet.tex

\begin{flushright}

\large{
	\scshape{
	26 ottobre 2016 -- ore 11:00 -- 13:00
	}}

\medskip
	
\small{Conferenza
	\newline Aula Bianchini}

\medskip

{\fontsize{42}{42} \svolk{\emph{Caso III}}}

\normalfont

\normalsize

\bigskip

Conferenza tenuta da \textsc{Marco Matteo Markidis} e \textsc{Giuseppe Silvi}

\bigskip

\textbf{\emph{Path to path~}}

Questa presentazione descrive \pa, %un'external
un external per \emph{Pure Data} che implementa un sistema di analisi e sintesi concatenativa basata su un \emph{corpus} audio, ossia una collezione di file audio. % qui ci andrebbe una brevissima facilitazione sul senso di corpus
Il suono \`e acquisito % e sintetizzato tramite un riconoscimento di similarit\`a tra i grani audio, usando un insieme di descrittori audio estratti in tempo differito per i corpora sonora e in tempo reale%i
ed analizzato mediante un insieme di descrittori audio estratti in tempo differito. Il processo di sintesi in tempo reale avviene mediante riconoscimento di similarit\`a tra i grani audio dei \emph{corpora sonora} analizzati e il segnale in tempo reale.

La catena di processo avviene, in tempo differito, segmentando il file e generando i grani, eseguendo successivamente un'analisi di ognuno di questi ed estraendone i descrittori che contribuiscono alla creazione di un albero \emph{k}-dimensionale e una lista di \emph{k}-primi vicini per ogni grano generato, dove ogni grano \`e rappresentato da un insieme di punti. La sorgente entra in \pa, ne vengono estratti i descrittori in tempo reale ed il grano pi\`u simile viene trovato nell'albero. Partendo dall'elemento trovato ed usando la sua lista di primi vicini si rende disponibile un insieme di grani per la sintesi. L'obiettivo principale del lavoro \`e l'audio mosaicing in tempo reale. 
Questa presentazione descrive anche come \pa ~sia usato nella composizione di Giuseppe Silvi per sassofono aumentato ed elettronica dal vivo *S4EF*. \\
Questa lavoro \`e stato scritto considerando \pa ~il principale algoritmo per l'elaborazione numerica del segnale, l'analisi sonora e la sintesi in Pure Data. 
S4EF si focalizza su due aspetti principali:
\begin{itemize}
\item la forma esterna della diffusione del sassofono e la sua sintesi utilizzando il sistema omnidirezionale di diffusori S.T.ONE sviluppato da uno degli Autori;
\item l'acustica del fluido dentro il sassofono e la sua relazione con il feedback acustico attraverso la sintesi in \pa .
\end{itemize}

\noindent
La presentazione \`e organizzata come segue:
\begin{itemize}
\item presentazione di \pa ;
\item funzionalit\`a di \pa ;
\item dimostrazione di \pa ~ed esempli di sassofono estratti da S4EF;
\item sviluppi futuri;
\item domande.
\end{itemize}

\noindent
La presentazione finir\`a con un'improvvisazione:
\begin{itemize}
\item Marco Matteo Markidis, elettronica;
\item Giuseppe Silvi, sassofono baritono e contralto.
\end{itemize}

\end{flushright}

\clearpage

\begin{flushleft}

~\vfill
\large{
	\scshape{
	26 ottobre 2016 -- ore 14:00 -- 16:00
	}}

\medskip
	
\small{Workshop
	\newline Aula Bianchini}

\medskip

{\fontsize{42}{42} \svolk{\emph{Caso IV}}}

\normalfont

\normalsize


\bigskip

Workshop tenuto da \textsc{Marco Matteo Markidis}

\bigskip

\textbf{\emph{Writing externals in PD: an introdaction}}

Questo laboratorio \`e incentrato sulla scrittura di external in ambiente Pure Data (Pd). Gli external sono degli oggetti non nativi all'interno di Pd ma che possono essere caricati ed integrati in questo ambiente di lavoro. Questo permette la scrittura di plug-in estendendo le funzionalit\`a native del linguaggio Pd. \\
Il laboratorio vuole essere \emph{hands-on}. \`E consigliato a tutti i partecipanti di venire muniti del proprio portatile. Sul repository del seminario github.com/amurtet/emufest gli interessati possono trovare tutti le indicazioni preliminari e il materiale disponibile. Non sono richieste conoscenze preliminari; tuttavia i concetti chiave della programmazione in C saranno dati per acquisiti. \\
Alla fine del laboratorio, l'Autore presenter\`a \pa ~come esempio di external realizzata. \\

\end{flushleft}
