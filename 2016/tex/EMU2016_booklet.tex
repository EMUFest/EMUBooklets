% !TEX encoding = UTF-8 Unicode
% !TEX TS-program = XeLaTex

\documentclass[9pt,
			   twoside
			   ]{extreport}

\usepackage[document]{ragged2e}
			   
\usepackage[paperwidth=10.5cm,
			paperheight=29.7cm,
			%margin=.707cm
			top=1cm,
			bottom=1cm,
			outer=1.2cm,
			inner=.707cm,
			%headsep=14pt
			]{geometry}

\usepackage{latexsym}

\usepackage[polutonikogreek,
			italian,
			english
			]{babel}

\usepackage[hang,
			small,
			labelfont=bf,
			up,
			textfont=it,
			]{caption}
			
\usepackage{float,
			afterpage,
			graphicx,
			amssymb,
			epstopdf,
			pst-barcode,
			hyperref,
			titlesec,
			tcolorbox,
			color
			}

\definecolor{supercolor}{RGB}{3,39,36}

\usepackage{fontspec,xltxtra,xunicode}
\defaultfontfeatures{Mapping=tex-text}
\setromanfont[Mapping=tex-text]{Source Sans Pro}
\setsansfont[Scale=MatchLowercase,Mapping=tex-text]{Source Sans Pro}
\setmonofont[]{Source Code Pro}

\newfontfamily{\svolk}{Volkhov}

\linespread{1.03}

\renewcommand\thesection{\Roman{section}} % Roman numerals for the sections
\renewcommand\thesubsection{\Roman{subsection}} % Roman numerals for subsections
\renewcommand{\thefootnote}{\textasteriskcentered}

\newtcbox{\mybox}{nobeforeafter,
				  colframe=supercolor,
				  colback=supercolor,
				  boxrule=0.5pt,
				  arc=8pt,
				  boxsep=0pt,
				  left=2pt,
				  right=2pt,
				  top=6pt,
				  bottom=2pt,
				  tcbox raise base
				  }

\newcommand\blankpage{%
    \null
    \thispagestyle{empty}%
    \addtocounter{page}{-1}%
    \newpage}

% !TEX encoding = UTF-8 Unicode
% !TEX TS-program = XeLaTex
% !TEX root = ../EMU2016_booklet.tex

%----------------------------------------------------------------------------------------
%	NEW COMMANDS 2015
%----------------------------------------------------------------------------------------

\newcommand{\greco}[1]{%
\begin{otherlanguage*}{greek}#1\end{otherlanguage*}}


%----------------------------------------------------------------------------------------
\newcommand{\livel}[5]{%
\noindent \textsc{#1}\\ %
\noindent \textbf{\textit{#2}} -- #3\\%
\noindent #4\\ %
\noindent #5%
\\
}%

%: #1 autore, #2 titolo, #3 anno ed esecuzione, #4 tipologia, #5 strumenti, #6 esecutori

%\livel{}
%{}{}
%{}
%{}{}


%-----------------------------------------------------------------------------------------------------------------
\newcommand{\brano}[4]{%
\noindent \textsc{#1}\\ %
\noindent \textbf{\textit{#2}} -- #3\\%
\noindent #4%
\\
}%

%: #1 autore, #2 titolo, #3 anno ed esecuzione, #4 tipologia
%\acusmatico{}
%{}{}
%{}

%-----------------------------------------------------------------------------------------------------------------
\newcommand{\acusmatici}[4]{%
\noindent \textsc{#1}\\ %
\noindent \textbf{\textit{#2}} -- #4\\%
#3\\%
%\\
}%

%: #1 autore, #2 titolo, #3 anno ed esecuzione
%\acusmatici{}
%{}{}

%-----------------------------------------------------------------------------------------------------------------
\newcommand{\esecutore}[2]{%
\noindent #1:  %
\textsc{#2} %
\\
}%

%: #1 strumento, #2 nome e cognome
% \esecutore{}{}

%-----------------------------------------------------------------------------------------------------------------
\newcommand{\descrizione}[2]{%
\noindent \textbf{\textit{#1}} %
#2 %
\medskip
}%

%: #1 titolo, #2 abstract
% \descrizione{}{}


%-----------------------------------------------------------------------------------------------------------------
\newcommand{\biografia}[2]{%
\noindent \textbf{\textsc{#1}} %
#2 %
\medskip
}%

%: #1 autore, #2 biografia
% \biografia{}{}



%----------------------------------------------------------------------------------------
%	TITLE SECTION
%----------------------------------------------------------------------------------------

\title{\flushright{
	\svolk{CONSERVATORIO DI MUSICA S. CECILIA} \\
	\vspace{.1em}
	\fontsize{58}{58}
	\svolk{\emph{EMUFest 2016}}} \\
		\bigskip 30 giugno 2015 \\
		3 ottobre 2015 \\
		20 ottobre 2015 \\
		24 -- 29 ottobre 2015 \\
		Roma}

\author{}

\vfill

\date{}

%\dedica{.2\textwidth}{\small A Monica:\\
%per tutto ciò che mi hai insegnato\\
%e per tutto ciò che ancora avresti dovuto insegnarmi.}

%----------------------------------------------------------------------------------------

\begin{document}

\pagestyle{empty}
\maketitle 

\cleardoublepage

~\vfill

% !TEX encoding = UTF-8 Unicode
% !TEX TS-program = XeLaTex
% !TEX root = EMU2015_booklet.tex

{\fontsize{30}{30} \svolk{\emph{Passare all'atto}}}

% \svolk{\emph{Passer à l’acte)}

\begin{quote}
\begin{it}
	\svolk{Il programma EMUfest 2015 ricava dalle suggestioni del titolo il criterio
di selezione delle opere e  intende offrire al pubblico una panoramica
delle attuali esperienze compositive: quelle che nascono dalla ricerca,
dalla scoperta, dall’invenzione.

Nessuna pratica artistica è fine solo a se stessa e anche la musica,
astratta e immateriale, porta in se questa responsabilità. L’idea è ciò che
il compositore incarna nella musica ma è pure la conseguenza di una
interpretazione della realtà, una visione dei valori o delle derive della
nostra civiltà.

L’elemento essenziale che permette ad un’opera d’arte di compiersi e di
trasmettere i suoi contenuti  è la correlazione tra la necessità interiore
dell’artista e la scelta del modo di esprimerla. \emph{Passer à l’acte} è
dunque il gesto che definisce la pratica artistica ma, allo stesso tempo e
in modo indiretto, è anche l’esortazione per noi fruitori, a cogliere gli
stimoli offerti dall’opera e a rendere attiva la nostra riflessione.}

\hfill \emph{Michelangelo Lupone}

\end{it}
\end{quote}

\newpage

%-------------------------------------------------------------------------------------

% !TEX encoding = UTF-8 Unicode
% !TEX TS-program = XeLaTex
% !TEX root = ../EMU2016_booklet.tex

\begin{flushright}

\large{
	\scshape{
	24 ottobre 2016 -- ore 20:30
	}}

\medskip
	
\small{Concerto
	\newline Sala Accademica}

\medskip

{\fontsize{42}{42} \svolk{\emph{Volontà I}}}

\normalsize

\bigskip

\livel{Tristan Murail}{C’est un jardin secret, ma soeur, ma fiancée, \newline  une source scellée, una fontaine close…}{5’00}{per viola}{1976}

\medskip

\livel{Iannis Xenakis}{Mikka}{4’25}{violino}{1971}
\medskip

\livel{Pascal Dusapin}{Inside}{8’32}{per viola}{1980}
\medskip

\livel{Giorgio Netti}{Dalla tentazione di Sant’Antonio}{9’00}{per violino}{1986}
\medskip

\livel{Giacinto Scelsi}{Manto II}{5’10}{per viola}{1967}
\medskip

\livel{Helmut Lachenmann}{Toccatina}{4’50}{per violino}{1986}
\medskip

\livel{Giorgio Netti}{Inoltre}{17’00}{per due violini}{2005-2006}

%\brano{Christian Eloy}
%{La cicatrice d'Ulysse}{13'00''}
%{acusmatico}
%new version 2015\\

%\acusmatici{Ursula Meyer-K\"onig}
%{Allears}{2012-13}{8'}

%\vspace{6mm}

\bigskip

\textbf{mdi ensemble} \\
\esecutore{violino}{Lorenzo Gentili-Tedeschi}
\esecutore{viola}{Paolo Fumagalli}

\vfill

\bigskip

\svolk{\emph{Ho cercato di incanalare quell’energia in un percorso che la rendesse percepibile senza snaturarne l’essenza: l’energia che abita i violini di Corelli, Tartini, Vivaldi, certo non citazioni ne dirette ne indirette ma l’imprevedibile vitalità delle loro articolazioni (tremoli, arpeggi, ribattuti, sincronie e fioriture improvvise) che hanno fatto della scuola italiana, elettrica ante litteram, un irraggiungibile modello di virtuosismo strumentale; poi la tensione, il vuoto attorno e dentro alla costruzione delle frasi, le imitazioni, i pedali, la continua sovrapposizione delle corde.}}

Giorgio Netti

\normalfont

\end{flushright}

\clearpage

\begin{flushleft}

~\vfill

\descrizione{C’est un jardin secret, ma soeur, ma fiancée, une source scellée, una fontaine close…}{Fondatore insieme a Gerard Grisey della musica “spettrale”, Murail utilizza l’informatica per approfondire le ricerche d’analisi e sintesi dei fenomeni ascustici, costruendo una musica basata sulle micro-variazioni interne al suono. C’est un jardin secret… presenta questo percorso di ricerca: una miniatura per viola attraversata da molteplici processi, trasformazioni progressive e ambiguità tra armonia e timbro. Un brano, afferma il compositore, costruito attorno al ritmo di un battito cardiaco, costantemente accellerato e rallentato, vivo e risonante.}

\descrizione{Mikka}{Riconosciuto tra gli esponenti più radicali della musica del 900, il compositore greco Iannis Xenakis compone il brano Mikka nel 1971 e l’anno successivo eseguito al Festival D’automne de Paris. Il brano è un impressionante fusione di glissati del violino che pongono l’accento su una materia incisa in limite estremo dell’accezione di melodia. Un suono/canto dell’arco che oscilla tra i suoni più sottili fino alle dinamiche più accese.}

\descrizione{Inside}{Diviso in tre parti, il brano mette in luce il grande ventaglio timbrico della viola, in una articolazione frenetica ed immersiva. Anche questo brano presenta micro-variazioni del suono instaurando un dialogo quarti-tonale tra le corde.}

\descrizione{Dalla tentazione di Sant’Antonio}{…Ho cercato di ripensare lo strumento a partire dalla sua memoria, dalla memoria delle dita che per secoli l’hanno attraversato: ho cercato di ascoltarlo come voce di voci e dallo stratificarsi nell'aria di queste, dal loro sovrapporsi, ha preso corpo il volume dello spazio nel quale è contenuto. Le diverse “apparizioni” vorrebbero arrivare a comporre via via un’unità, che non determina una direzione quanto un sostare: stato incandescente della materia sonora, non solidifica, s’addensa e, sospeso, prossimo a saturarsi viene nell’ultimo respiro infine travasato…}

\descrizione{Manto II}{Secondo movimento per Viola, Manto prende il nome da un profeta dell’antica Grecia. Scelsi incentra il brano sulle soglie del battimento e sugli aspetti liminari del suono. Qui il rito è alla base dell’estetica del compositore, percepibile come vero e proprio ponte/distanza tra il tempo immobile e il successivo canto umano che accompagna il terzo movimento.}

\descrizione{Toccatina}{Helmut Lachenmann consegna in eredità al fare musicale una continua nuova percezione dell’ascolto. Questo studio per violino, apre una nuova finestra, indicando *un possibile oltre a ciò che abitualmente chiamiamo musica* (G. Netti): una nuova sensibilità dello strumento, che parte da un idea di musica concreta strumentale, fattasi  sottile e cristallina in questa breve dischiusura musicale}

\descrizione{Inoltre}{Starting from a muted blow, INOLTRE becomes a continuous attempt to bring the sound matter nearer to the pitched musical world, creating an acoustic experience coming from somewhere else. (Quotation from the Composer) This composition makes the matter white hot and in its boiling, articulations, fragments, nightmares and wanderings reappear and melting become something else. A voice-violin, a sound-man, sacred in their uniqueness, with no more articulation, tense and suspended on the extreme and guarded electric background.}

\end{flushleft}


\clearpage

% !TEX encoding = UTF-8 Unicode
% !TEX TS-program = XeLaTex
% !TEX root = ../EMU2016_booklet.tex

\begin{flushright}

\large{
	\scshape{
	25 ottobre 2016 -- ore 18:00
	}}

\medskip
	
\small{Concerto Acusmatico
	\newline Il Suono di Piero [Aula Bianchini]}

\medskip

{\fontsize{42}{42} \svolk{\emph{Volontà II}}}

\normalsize

\medskip

regia del suono \textsc{Francesco Bianco}

\bigskip

\livel{Massimo Vito Avantaggiato}{ATLAS OF UNCERTAINTY}{7’00}{}{2015}
\medskip

\livel{Marcela Pavia}{Aleph}{9’30}{}{2013}
\medskip

\livel{Hiromi Ishii}{Ryojinfu}{10’09}{}{2013}
\medskip

\livel{David Ledoux}{Marfa}{13’20}{}{2014}
\medskip

%\brano{Christian Eloy}
%{La cicatrice d'Ulysse}{13'00''}
%{acusmatico}
%new version 2015\\

%\acusmatici{Ursula Meyer-K\"onig}
%{Allears}{2012-13}{8'}

%\vspace{6mm}

\vfill

\descrizione{Atlas of Uncertainty}{è un brano di musica concreta, nel quale un microcosmo di suoni, spesso assai distanti tra loro,  viene esplorato attraverso varie tecniche di manipolazione sonora: segnali sonori  generati da eletrodomestici ; texture create impiegando suoni di campana tibetana o altre percussioni; \emph{whooshes} di rumore bianco; accumulazioni granulari,  solo per citarne alcuni. Questi suoni sono combinati in gesti articolati  in vario modo e molto ben riconoscibili. Prima esecuzione: Casa del Suono di Parma, may 2016.}

\descrizione{Aleph}{è stato composto per l’Acusmonium Audior. Il titolo si riferisce all’omonimo breve racconto di Borges \emph{the projection of the Whole in one point of the Space -micro cosmo- which reflects the whole Universe -macro cosmos}. La continua trasformazione del materiale musicale non c'è, ma c'è una timeline per ognuno degli stadi di trasformazione; viceversa  \emph{Aleph} è il congelamento della trasformazione continua. Tempo e spazio diventano uniti, diventano la stessa cosa. Se fosse dato abbastanza tempo qualsiasi cosa potrebbe diventare qualcos'altro e questo potrebbe accadere gradualmente o in modo brusco, ad alta o a bassa velocità e con tutti gli stadi intermedi in un percorso esplicito o non esplicito. Il risultato della trasformazione potrebbe essere differente: le storie e la storia sono determinate dalla casualità, ma la serie degli eventi non è unica. L'elettronica ha permesso di entrare nella struttura interna del suono aprendo la mente alla possibilità di forma come \emph{evoluzione nel tempo} di questa intima struttura. Il suono diventa non solo il pilastro del discorso ma il discorso stesso apre altre possibilità anche per una composizione esclusivamente acustica.}

%\bigskip

%\svolk{\emph{Ho cercato di incanalare quell’energia in un percorso che la rendesse percepibile senza snaturarne l’essenza: l’energia che abita i violini di Corelli, Tartini, Vivaldi, certo non citazioni ne dirette ne indirette ma l’imprevedibile vitalità delle loro articolazioni (tremoli, arpeggi, ribattuti, sincronie e fioriture improvvise) che hanno fatto della scuola italiana, elettrica ante litteram, un irraggiungibile modello di virtuosismo strumentale; poi la tensione, il vuoto attorno e dentro alla costruzione delle frasi, le imitazioni, i pedali, la continua sovrapposizione delle corde.}}
%
%Giorgio Netti
%
%\normalfont

\end{flushright}

\clearpage

\begin{flushleft}

~\vfill

\descrizione{Ryojinfu}{Questa fantasia sonora multicanale è stata ispirata da una leggenda di un imperatore giapponese religioso, e dedicata ad Imayo (canto Buddhista), ma ha dovuto combattere diverse battaglie. I materiali sonori sono: 1. Canto (maschile solo) voce di Imayo, 2. Suoni e rumori registrati durante la cerimonia Buddhista, 3. Suoni granulari di riso. Il tutto principalmente processato usando cross synthesis e sintesi granulare. I suoni processati offrono differenti caratteristiche di movimento e di disegno in uno spazio tridimensionale; il suono processato del 1. appare con variazioni (ma mai come il suono originale), e conduce alla fine ad un suono simile alla voce di un ragazzo. I suoni massicci 2. si muovono lentamente sviluppando una sorta di muro sonoro. I suoni prodotti dal 3. sono invece veloci ed irregolari come il volo di uccelli. Questo pezzo fu composto usando il sistema del suono spazializzato di Zirkonium.}

\descrizione{Marfa}{Scene e personaggi da film come 127 Hours, There Will Be Blood, The Road e No Country For Old Man hanno ispirato Marfa. Il titolo si riferisce ad un'area nel deserto del Texas (US) dove alcuni di questi film furono girati e anche conosciuti per le loro misteriose visioni di luci notturne. Come il lettore il quale sviluppa un'immagine mentale di quello che legge, questi tre movimenti sono una descrizione acustica di emozioni/ localizzazioni per la mente, viaggiando attraverso ansia, contemplazione ed elasticità. Come parte dei suoi studi compositivi elettroacustici, Marfa è il primo tentativo di Ledoux sulla composizione di un ambiente ad ascolto 3D - per cupola. Attraverso il corso semestrale the Fall 2014, Ledoux ha imparato ad usare il software ZKM di Zirkonium in compagnia del plugin ZirkOSCII con il prof. Robert Normandeau. Questa combinazione di strumenti gli permise di organizzare lo spazio e la musica simultaneamente, usando la tecnica di spazializzazione VBAP, e a creare un'esperienza musicale profonda che poteva essere suonata su ogni cupola tridimensionale preposta. Marfa è stato già presentato Durante gli Ultrasons series all’università di Montréal - con cupola a quattordici diffusori,  come pure ai concerti/conferenze InSonic 2015, ospitato da ZKM Center for Art and Media (Karlsruhe, Germania) per cupola a quarantatré diffusori ed è stato anche selezionato per l’imminente CUBE Fest 2016 di ICAT: Massively Multichannel Music (Blacksburg, VA) per cupola a struttura cubica a centoventotto diffusori, alti quattro piani.}


\end{flushleft}


\clearpage

% !TEX encoding = UTF-8 Unicode
% !TEX TS-program = XeLaTex
% !TEX root = ../EMU2016_booklet.tex

\begin{flushright}

\large{
	\scshape{
	25 ottobre 2016 -- ore 20:30
	}}

\medskip
	
\small{Concerto
	\newline Sala Accademica}

\medskip

{\fontsize{42}{42} \svolk{\emph{Volontà III}}}

\normalsize

\bigskip

\livel{Karlheinz Stockhausen}{Solo}{10’39}{for a soloist with live electronics}{}
\medskip

\livel{Maura Capuzzo}{Arcipelagos}{9’30}{per violino, violoncello, clarinetto e live electronics}{}
\medskip

\livel{James Dashow}{Soundings in Pure Duration n. 2b for percussive and octophonic electronic sounds}{12’26}{acusmatic}{}
\medskip

\livel{Marco Marinoni}{Still}{15’00}{live performance}{}
\medskip

\livel{Barry Truax}{The Garden of Sonic Delights}{11’15}{acousmatic}{}
\medskip

\livel{Mario Duarte}{Achtli}{6’10}{for flute, piano, two percussionist and live electronics}{}
\medskip

\bigskip

\esecutore{direttore}{Franco Sbacco}
\esecutore{sassofono}{Danilo Perticaro}
\esecutore{violino}{Sofia Bandini}
\esecutore{clarinetto}{Alice Cortegiani}
\esecutore{violoncello}{Alice Romano}
\esecutore{flauto}{Alessandro Pace}
\esecutore{pianoforte}{Francesco Ziello}
\esecutore{percussioni}{Tiziano Capponi, \newline Matteo Rossi}
\esecutore{live electronics}{Leonardo Mammozzetti, \newline Massimiliano Mascaro}


\vfill

\descrizione{Solo}{L'esecuzione di questo brano del 1965-66 prevede uno strumentista e 4 assistenti musicali che ne devono interpretare la partitura ed eseguire in maniera puntuale ogni aspetto indicato. A corredo delle prime esecuzioni c'erano rumori di fondo del nastro, possibili errori di sincronismo e molte altre criticità. Oggi i mezzi sono cambiati: uno strumentista, in questo caso il sassofonista, e un assistente musicale, che si occupa della parte elettronica. L'attenzione per lo stile e la precisione sono ancor più esasperati, una sfida quindi, con l'obiettivo di fare sempre meglio. (L. Mammozzetti)}

\end{flushright}

\clearpage

\begin{flushleft}

~\vfill

\descrizione{Arcipelagos}{Un arcipelago è un paesaggio di isole nel medesimo mare. Simili o assai diverse nelle forme, nei colori o negli umori ma un solo mare, lo stesso vento.}

\descrizione{Soundings in Pure Duration n. 2b}{per suoni percussivi pre-registrati e suoni elettronici esafonici, nasce dalla sovrabbondanza di materiale sonoro che ho creato per Soundings n. 2a.  Come per quest'ultimo, il brano è stato composto per sviluppare la spazializzazione approfittando dei punti d'attacco percussivi come precisi indicatori di posizionamento nello spazio.   Oltre alle interazioni tra suoni percussivi e suoni elettronici, ho cominciato a lavorare anche con 2 concetti della spazializzazione... quella definita dalle traiettorie nello spazio dei suoni percussivi, e quella definita dai fasci sonori elettronici. Queste due componenti sono molto piu' integrate rispetto al n. 2a, per cercare di creare due strati di spazio simultaneamente, oppure una specie di contrappunto  spaziale.  Durante la composizione di questo brano continuo i miei  tentativi di esplorare il mio secondo concetto di spazializzazione cioè il movimento DELLO spazio, anziché movimento NELLO spazio.}

\descrizione{Still}{è una live performance che utilizza registrazioni catturate durante le missioni spaziali Voyager mediante strumenti di indagine in radioastronomia planetaria (PRA) in grado di registrare segnali di emissione dai pianeti, le loro lune ed i loro sistemi ad anello catturando fenomeni elettromagnetici quali le interazioni del vento solare con la magnetosfera del pianeta, la magnetosfera stessa, le emissioni di particelle polarizzate e le loro interazioni, trasformando tali fenomeni in segnali elettrici, amplificati e utilizzati per eccitare la membrana di un altoparlante.}

\descrizione{The Garden of Sonic Delights}{è un paesaggio sonoro composto da molteplici tracce sonore. Il pezzo invita l'ascoltatore in un ambiente sonoro immaginario (descritto da Murray Schafer come un \emph{giardino risonante}) pregno di suoni che dovrebbero ricordarci quelli dell'acqua, del vento, degli insetti, degli animali e degli uccelli. Il nostro viaggio comincerà il pomeriggio per finire al mattino seguente, lasciandoci - si spera - lieti e riposati. Il pezzo è stato commissionato dal \emph{Birmingham ElectroAcoustic Sound Theatre} (BEAST) per il BEAST FEaST 2016, e ralizzato con 48 canali alla \emph{Technical University} di Berlino e allo studio privato del compositore con l'ausilio del TiMax2 Soundhub della Outboard per la spazializzazione.}

\descrizione{Achtli}{\emph{Ci volevano seppellire ma non sapevano che noi fossimo semi.} A settembre 2014, scomparvero 43 studenti del \emph{Ayotzinapa Teacher Training College} in seguito a degli scontri con le forze dell'ordine. Si crede che i poliziotti abbiano consegnato gli studenti ai \emph{drug cartel} i quali, a loro volta, li avrebbero uccisi e poi bruciati in una discarica nella periferia di Cocula, Guerrero (Messico). \emph{Achtli} significa \emph{seme} in Nàhuatl (un' antica lingua messicana). Ho scritto questo pezzo in segno di protesta per richiedere una mobilitazione in merito alla scomparsa degli studenti. Il pezzo è stato creato inserendo in una patch del software MAX/MSP i nomi dei 43 studenti in modo tale da affidare ad ogni personaggio un parametro musicale come timbro, altezza, durata, intensità e gesto. Questo pezzo è un pianto di giustizia ed è parte dell'azione globale in favore di Ayotzinapa. Potrete rivedere la performance del pezzo a questo link: https://www.youtube.com/watch?v=BJLKTs1ygZU.}

\end{flushleft}


\clearpage

\end{document}
