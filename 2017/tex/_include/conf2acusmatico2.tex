% !TEX encoding = UTF-8 Unicode
% !TEX TS-program = XeLaTex
% !TEX root = ../EMU2017_booklet.tex

\begin{flushright}

\large{
	\scshape{
	24 ottobre 2017 -- ore 10:00 - 13:30
	}}

\medskip
	
\small{Masterclass
	\newline Aula Bianchini}

\medskip

{\fontsize{20}{20} \svolk{\emph{GRM Tools\\e la spazializzazione}}}

\normalfont

\normalsize

\bigskip

Masterclass tenuta da \textsc{Emmanuel Favreau}\\{\footnotesize direttore scientifico dell'INA-GRM di Parigi}


\bigskip

Dopo una breve introduzione storica sul GRM e il suo approccio alla musica concreta, presenterò nei dettagli alcuni dispositivi GRM Tools. L’accento è posto, in particolare, sugli ultimi dispositivi dedicati alla spazializzazione 2D, 3D e binaurale con alcune dimostrazioni sulla cupola di 22 altoparlanti.

\bigskip

\small{Evento ArteScienza2017 realizzati in collaborazione\\con il CRM - Centro Ricerche Musicale}

\vfill

\large{
	\scshape{
	24 ottobre 2017 -- ore 18
	}}

\medskip

\small{Concerto Acusmatico
	\newline Il Suono di Piero [Aula Bianchini]}

\medskip


{\fontsize{20}{20} \svolk{\emph{Concerto Acusmatico II}}}

\normalsize

\medskip

regia del suono \textsc{Federico Paganelli}

\bigskip

\livel{Francis Dhomont}{PHŒNIX XXI}{16'31}{}{2016}
\medskip

\livel{Giuseppe Pisano}{Termiti}{4'02}{}{2017}
\medskip

\livel{Diego Ratto}{Echoss}{8'15}{}{2017}
\medskip

\livel{Roberto Begini}{Kymbalon 3}{8'30}{}{2017}
\medskip

\livel{Alessandro Perini}{Étude Tendu}{8'35}{}{2017}
\medskip

\vfill


\descrizione{PHŒNIX XXI}{A Inés Wickmann. Una traccia/conferma della vitalità della scena acustica a inizio XXI secolo ma, per me, rappresenta anche una nuova vita date ad antiche ceneri musicali: registrazioni molto antiche recuperate e trasformate estratte da lavori strumentali. Una sorta di eterno ritorno.PHŒNIX XXI è stato commissionato dall’INA-GRM la cui prima si è tenuta l’8 ottobre 2016 a Parigi.}

\descrizione{Termiti}{ è uno studio sulla decomposizione della materia. Suoni reali, concreti e umani lasciati a marcire senza alcun tipo di cura museale, trasformati in un nuovo ecosistema e ospiti di una vita che fa della saprofagia la sua caratteristica saliente. Il brulicare organico dei microeventi, sciamante e randomico, assale il materiale scarnificandolo e impastandolo e plasma nuove forme in cui, con attenzione, è ancora possibile intuire i resti laceri delle componenti originarie.}


\end{flushright}

\clearpage

\begin{flushleft}


\descrizione{Echoss}{Il brano, in forma tripartita, vede un alternarsi di gesti violenti che interrompono un’apparente tranquillità. Più in profondità, vi è un’eccitazione  turbolenta che si mostra completamente nella parte finale. L’utilizzo del silenzio, la gestione dello spazio virtuale e l’organizzazione degli elementi figura-sfondo, sono alcune delle caratteristiche fondamentali della composizione.}

\descrizione{Kymbalon 3}{è uno studio sulla composizione a partire da sorgenti sonore non strumentali e dal suono concreto, con particolare riferimento ai corpi metallici. È un brano interamente composto da un unica sorgente, un grosso piatto metallico, da cui sono state ricavate diverse sonorità.
Lo sviluppo è incentrato sull’elaborazione ed il trattamento sonoro, con particolare riferimento ai processi di generazione delle texture e con un algoritmo specifico sviluppato per il progetto (algoritmo per la generazione di tessiture frattali).}

\descrizione{Étude Tendu}{ Dopo aver costruito quattro altalene con cavo d'acciaio, amplificate con microfoni a contatto, e averle usate per un brano live chiamato \textit{Steel String Quartet}, ho registrato varie famiglie di suoni prodotti con differenti tecniche esecutive applicate alle funi metalliche. \textit{Étude tendu} è una composizione interamente costruita con questi suoni.}


\end{flushleft}

\vfill

