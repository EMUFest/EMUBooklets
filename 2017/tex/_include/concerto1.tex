% !TEX encoding = UTF-8 Unicode
% !TEX TS-program = XeLaTex
% !TEX root = ../EMU2017_booklet.tex

\begin{flushleft}

\large{
	\scshape{
	23 ottobre 2017 -- ore 20:00
	}}

\medskip
	
\small{Concerto
	\newline Sala Accademica}

\medskip

{\fontsize{20}{20} \svolk{\emph{Concerto I}}}

\normalsize

\medskip

regia del suono \textsc{Pasquale Citera} e \textsc{Marco De Martino}

\bigskip

 

\livel{Franck Bedrossian}{Propaganda}{10’00}{per quartetto di sassofoni e dispositivo elettronico}{2008}
\medskip

\livel{Giorgio Nottoli}{Ellenikà}{6’00}{acusmatico}{2013}
\medskip

\livel{James Dashow}{Soundings in Pure Duration n.9 *}{13’40}{per flauto basso e suoni elettronici ottofonici}{2017}
\medskip

\livel{Pierre Jodlowski}{Time \& Money}{14’00}{per percussioni ed elettronica}{2004}
\medskip

\livel{Demien Rudel Rey}{Khēmia I}{5’30}{per contrabbasso ed elettronica}{2016}
\medskip

\livel{John Cage}{FOUR\textsuperscript{5}}{12’00}{per ensemble di sassofoni}{1991}
\medskip

\bigskip
* prima esecuzione assoluta

\bigskip

\esecutore{Saxatile\textsuperscript{4} [\textbf{m}odulable \textbf{s}axophone \textbf{e}nsemble]}{Enzo Filippetti, Danilo Perticaro, Alessandro Scalone, Maurizio Schifitto,\\Marzia Marianantoni, Giulia De Mico, Cristiano Cerroni,\\Francesco De Cicco}
\esecutore{flauto}{Manuel Zurria}
\esecutore{percussioni}{Ivan Liuzzo}
\esecutore{contrabbasso}{Mauro Tedesco}

\medskip

\esecutore{live electronics}{James Dashow, Federico Scalas}


\vfill

\descrizione{Propaganda}{Il quartetto di sassofoni ha sempre stimolato la mia curiosità in quanto costituisce in sé quasi uno strumento elettronico. L’omogeneità dei timbri del quartetto, la loro elasticità e capacità di fondersi sono tali che, a tratti, si potrebbe credere che i sassofoni siano sottoposti naturalmente a trasformazioni elettroacustiche. L’idea di fondere questo ensemble con l’elaborazione di suoni elettronici, quindi, mi ha permesso di sviluppare quest’impressione di flessibilità. La scrittura strumentale stessa, inoltre, favorisce l’ampliarsi di tessiture tra tra gli strumenti acustici e l’elettronica provocando incontri inaspettati o conflittuali. Questo brano è stato commissionato da James Giroudon a cui è anche dedicata la composizione.}

\end{flushleft}


%\clearpage

\begin{flushright}

~\vfill

\descrizione{Ellenikà}{Questo lavoro è costituito da registrazioni effettuate in Grecia durante due permanenze presso l’isola di Thassos, nel 2012 e nel 2013. Le registrazioni includono sia suoni naturali che, sebbene ripresi in Grecia, potrebbero provenire da qualsiasi zona del Mediterraneo che eventi sonori parlati in Greco. Alcune delle  registrazioni di parlato provengono da luoghi affollati quali i mercati del paese in cui i richiami dei mercanti prevalgono sul brusio della folla altre dalla lettura di versi di poeti Greci del periodo classico tra cui Saffo e Alceo. \\
Il materiale parlato è stato analizzato in modo da ottenere l’andamento di alcuni parametri base, relativi a intonazione e durate, che vanno a costituire l’articolazione base della composizione. Le altezze sono usate per filtrare i materiali concreti ottenendo complesse mutazioni tonali; per esempio all’inizio del brano il suono del mare, filtrato attraverso risonatori intonati sulle altezze di un canone, viene lentamente trasformato in timbro chiaro e variato. La composizione è divisa in due sezioni: nella prima parte dominano suoni naturali, nella seconda quelli della voce umana che si fa via via più intelligibile. Il risultato musicale si basa su un lento e graduale gioco di alternanze tra astrazione e rivelazione degli elementi del paesaggio sonoro.}


\descrizione{Soundings in Pure Duration n. 9}{Questo particolare "Soundings" continua le esplorazioni del compositore nelle tre dimensioni principali del suo lavoro:  le strutture delle altezze, il timbro e, soprattutto, la spazializzazione. Il timbro è costruito grazie alle procedure pilotate dalle altezze, che a loro volta sono derivate da un uso rigoroso del suo Sistema Diadi. Queste strutture sono poi sincronizzate con varie elaborazioni dello spazio che circonda l'ascoltatore. Spazi e settori di spazio di diverse dimensioni, spesso simultanei oltre che in succesione, sono utilizzati insieme a traiettorie multiple (la "coreografia" dei suoni) nello spazio. Tutti questi elementi sono integrati in un complesso concetto musicale di spazio-timbro. Come al solito, il solista è il punto focale di tutta questa attività, e spesso è lui l'innesco per il movimento spaziale. Soundings in Pure Duration n.9 è composto e dedicato a Manuel Zurria.}

\descrizione{Time \& Money}{Questo brano è stato composto dopo "People /Time", lavoro di musica da camera commissionato dal festival di Donaueschingen nel 2003. Condivide con esso lo stesso in interrogativo che riguarda la nostra società, i nostri comportamenti con il tempo e il denaro. È un sorta di contestazione al nostro sistema economico, dove l'argomento diventa una sorta di quadro musicale [...] la musica inizia con cicli e cicli di modelli ritmici in cui la presenza di suoni radio e film (soprattutto dialoghi) stanno producendo un secondo strato di percezione per il pubblico, attivando alcune memorie collettive. E la musica inizia ad andare sempre più veloce, proprio come a volte accade  nelle nostre vite, così spaventati nel perdere il tempo ...}

\descrizione{Khēmia I}{il titolo del brano si riferisce a degli assi concettuali (\textit{Khēmia} vuol dire \textit{trasmutazione}, un termine collegato all’alchimia). Il contrabbasso e i materiali elettroacustici sono \textit{trasmutati} attraverso processi di saturazione, dove si mischiano in qualcosa di complesso, in un instabile mondo di suoni. D’altra parte, il processo opposto applica la separazione di queste due sorgenti di suono (lo strumento e gli altoparlanti), a elementi ancora differenti. Ci sono materiali consonanti (armonicamente stabili, o semplici spettralmente) che esprimono onirismo opposto agli oggetti sonori caotico, saturati e altamente complessi.}

\descrizione{FOUR\textsuperscript{5}}{Note singole in un intervallo di tempo flessibile per quattro sassofoni, o multipli di quattro. Le frequenze non suoneranno come scritto. L’intonazione è propria di ogni singolo musicista. Se il suono è duraturo, la dinamica è leggera. I suoni brevi hanno dinamica libera, anche lo sforzato se voluto, o il pianissimo..}

\end{flushright}

\clearpage

%\begin{flushleft}
%
%~\vfill
%
%%***
%
%
%\end{flushleft}
%
%\clearpage
%
%~\vfill
%
%
