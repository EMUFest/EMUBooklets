% !TEX encoding = UTF-8 Unicode
% !TEX TS-program = XeLaTex
% !TEX root = ../EMU2017_booklet.tex

\begin{flushleft}

\large{
	\scshape{
	25 ottobre 2017 -- ore 20:00
	}}

\medskip
	
\small{Concerto
	\newline Sala Accademica}

\medskip

{\fontsize{20}{20} \svolk{\emph{Concerto II}}}

\normalsize

\medskip

regia del suono \textsc{Pasquale Citera}

\bigskip

 
\livel{Lorenzo Pagliei}{Lontano Interiore *}{13’00}{Madrigale invisibile per 2 voci e live electronics}{2017}
\medskip

\livel{Levy Oliveira}{Golden Aspen}{8’00}{per flauto ed elettronica}{2017}
\medskip

\livel{Claudio Panariello}{Theatrum Insectourm}{7’00}{per clarinetto basso, pianoforte ed elettronica}{2017}
\medskip

\livel{Maura Capuzzo}{Come arriva l'amore così ti si libera il naso,\\
improvvisamente, e a caso}{5’00}{per voce, fixed media e live electronics}{2012, revisione 2017}
\medskip

\livel{Roderik De Man}{Yuxtaposiciones}{11’00}{per clarinetto basso ed elettronica}{2008}
\medskip

\livel{Gene Coleman}{Kokhlos IV *}{9’00}{Per voci, ensemble e suoni registrati}{2017}
\medskip

\bigskip
\textbf{*} prima esecuzione assoluta

\bigskip

\esecutore{\textsc{CumTempora Ensemble}}{}
\smallskip

\esecutore{soprano}{Elisa Prosperi}
\esecutore{mezzosoprano}{Virginia Guidi}
\esecutore{flauto}{Elena D'Alò}
\esecutore{clarinetto}{Sauro Berti}
\esecutore{pianoforte}{Sara Ferrandino}

\medskip

\esecutore{live electronics}{Federico Paganelli}


\vfill

\descrizione{Lontano Interiore}{ fa parte di un ciclo di madrigali dedicati all’onirico, al sogno e ai differenti stati di coscienza intitolato \textit{Il Libro dei Margini}. Schegge di eventi sonori allusivi e connotati emergono da fondali sonori (scenosonie) creati interamente dal vivo. Le voci vagano in modo non lineare fra emissioni quotidiane, suoni al limite dell’udibile magnificati e frammenti di linguaggio, come le memorie imprecise e le incerte apparenze del sogno.}

\descrizione{Golden Aspen}{ è la specie dell’albero Pando. Questa pianta è ritenuta pesare circa 6,000,000 chili rendendolo l’organismo conosciuto più pesante. Questo albero si riproduce attraverso un sistema definito spollonatura. Un singolo fusto può generare radici laterali che, nelle giuste condizioni, generano altri fusti.; visto al di fuori del terreno questi nuovi fusti sembrano alberi separati. Cosi come il Pando il brano (per flauto amplificato ed elettronica) prova as espandersi partendo dalle minuscole idee musicali presentate all’inizio del pezzo. Il brano è stato composto tra lo studio personale del compositore e il centro di ricerche per la musica contemporanea dell’univeristà federale di Minas Gerais ( Belo Horizonte/Brasile).}

\end{flushleft}


%\clearpage

\begin{flushright}

~\vfill

\descrizione{Theatrum Insectourm}{È un lavoro concepito sulla dialettica tra il mondo organico identificato dagli “esecutori umani” e quello prettamente meccanico identificato dall'elettronica; dopo averli messi in funzione ho lasciato che i due interagissero conducendo il pezzo ad un stadio tale in cui elettronica e strumenti invertono i loro ruoli.\\
Il mondo sonoro che il pezzo va costruendo richiama il soundscape di una natura microscopica e l'elettronica, è costituita da catene di feedback interno, una sorta di nascosta entomologia meccanica che viene alla luce e affronta gli strumenti.}

\descrizione{Come arriva l'amore così ti si libera il naso,\\improvvisamente, e a caso}{Il sapore della quotidianità nei gesti sonori del timer del proprio scaldabagno o della centrifuga della lavatrice o la propria voce. Accanto ad essi la voce della solista che è suono ma è anche rumorosa, è respiro, è espressione ma non parola. Quando tutto è stato detto, quando la forma ha fatto il suo percorso la voce può diventare parola e cantare: \textit{tipota}, che in greco moderno \textit{nulla}. \\...ma piace anche la pasta in brodo, piacciono i complimenti e il colore azzurro, piace una vecchia scarpa, piace averla vinta, piace accarezzare un cane.. ma cosa è mai la poesia?Più di una risposta incerta è stata già data in proposito. Ma io non lo so e mi aggrappo a questo come alla salvezza di un corrimano (W.Szymborska)}

\descrizione{Yuxtaposiciones}{In questo brano, dedicato al clarinettista Harry Sparnaay, il solista è accompagnato da un ensemble pre-registrato di clarinetti basso,clarinetti contrabbasso e di una parte elettronica. Insieme formano una "band". Questa parte ha due versioni, una dove la parte cd è fissa e una per cui il solista ha più libertà nel tempo e tramite una path di max msp è  possibile innescare le parti separate con un pedale. Il titolo Yuxtaposiciones si riferisce al modo in cui solista e parte elettronica funzionano fianco a fianco dove il tempo della "band" sul supporto digitale si incastra con l'esecutore dal vivo. Il pezzo richiede al musicista di essere fortemente lirico e virtuoso.}

\descrizione{Kokhlos IV}{Nei ultimi anni ha composto brani modellati su ciò che chiama \textit{Architettura Neurologica dell’Ascolto} o anche \textit{Percorso uditivo del cervello}. Questa serie di composizioni, intitolate “Kokhlos” (che in greco antico significa “spirale”), esplora i modelli di ascolto dell’orecchio interno umano. La serie Kohklos è parte della new-media opera  “Dreamlives of Debris” basata su testi dello scrittore americano Lance Olsen. L’elettronica di questo pezzo è stata composta da Adam Vidiksis e Gene Coleman per The Algo. Insitute – una piattaforma di lavoro e di interscambio tra scienza, arte e tecnologia.}

\end{flushright}

\clearpage

%\begin{flushleft}
%
%~\vfill
%
%%***
%
%
%\end{flushleft}
%
%\clearpage
%
%~\vfill
%
%
