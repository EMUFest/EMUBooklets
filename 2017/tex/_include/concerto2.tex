% !TEX encoding = UTF-8 Unicode
% !TEX TS-program = XeLaTex
% !TEX root = ../EMU2017_booklet.tex

\begin{flushleft}

\large{
	\scshape{
	25 ottobre 2017 -- ore 20:00
	}}

\medskip
	
\small{Concerto
	\newline Sala Accademica}

\medskip

{\fontsize{20}{20} \svolk{\emph{Concerto II}}}

\normalsize

\medskip

regia del suono \textsc{Pasquale Citera}

\bigskip

 
\livel{Lorenzo Pagliei}{Lontano Interiore *}{13’00}{Madrigale invisibile per 2 voci e live electronics}{2017}
\medskip

\livel{Levy Oliveira}{Golden Aspen}{8’00}{per flauto ed elettronica}{2017}
\medskip

\livel{Claudio Panariello}{Theatrum Insectourm}{7’00}{per clarinetto basso, pianoforte ed elettronica}{2017}
\medskip

\livel{Maura Capuzzo}{Come arriva l'amore così ti si libera il naso,\\
improvvisamente, e a caso}{5’00}{per voce, fixed media e live electronics}{2012, revisione 2017}
\medskip

\livel{Roderik De Man}{Yuxtaposiciones}{11’00}{per clarinetto basso ed elettronica}{2008}
\medskip

\livel{Gene Coleman}{Kokhlos IV *}{9’00}{Per voci, ensemble e suoni registrati}{2017}
\medskip

\bigskip
\textbf{*} prima esecuzione assoluta

\bigskip

\esecutore{\textsc{CumTempora Ensemble}}{}
\smallskip

\esecutore{soprano}{Elisa Prosperi}
\esecutore{mezzosoprano}{Virginia Guidi}
\esecutore{flauto}{Elena D'Alò}
\esecutore{clarinetto}{Sauro Berti}
\esecutore{pianoforte}{Sara Ferrandino}

\medskip

\esecutore{live electronics}{Federico Paganelli}


\vfill

\descrizione{Lontano Interiore}{ fa parte di un ciclo di madrigali dedicati all’onirico, al sogno e ai differenti stati di coscienza intitolato \textit{Il Libro dei Margini}. Schegge di eventi sonori allusivi e connotati emergono da fondali sonori (scenosonie) creati interamente dal vivo. Le voci vagano in modo non lineare fra emissioni quotidiane, suoni al limite dell’udibile magnificati e frammenti di linguaggio, come le memorie imprecise e le incerte apparenze del sogno.}

\descrizione{Golden Aspen}{ is the specie of the Pando tree. The plant is estimated to weigh collectively 6,000,000 kilograms (6,600 short tons), making it the heaviest known organism. This kind of tree reproduces via a process called suckering. An individual stem can send out lateral roots that, under the right conditions, send up other erect stems; from all above-ground appearances the new stems look just like individual trees. Such as the tree, the piece Golden Aspen (for amplified flute and electronics) try to expend itself by the smallest musical ideas presented in the beginning of the music. The piece was composed in the composer’s personal studio and in the Research Center of Contemporary Music of the Federal University of Minas Gerais (Belo Horizonte/Brazil).}

\end{flushleft}


%\clearpage

\begin{flushright}

~\vfill

\descrizione{Theatrum Insectourm}{È un lavoro concepito sulla dialettica tra il mondo organico identificato dagli “esecutori umani” e quello prettamente meccanico identificato dall'elettronica; dopo averli messi in funzione ho lasciato che i due interagissero conducendo il pezzo ad un stadio tale in cui elettronica e strumenti invertono i loro ruoli.\\
Il mondo sonoro che il pezzo va costruendo richiama il soundscape di una natura microscopica e l'elettronica, è costituita da catene di feedback interno, una sorta di nascosta entomologia meccanica che viene alla luce e affronta gli strumenti.}

\descrizione{Come arriva l'amore così ti si libera il naso, improvvisamente, e a caso}{Il sapore della quotidianità nei gesti sonori del timer del proprio scaldabagno o della centrifuga della lavatrice o la propria voce. Accanto ad essi la voce della solista che è suono ma è anche rumorosa, è respiro, è espressione ma non parola. Quando tutto è stato detto, quando la forma ha fatto il suo percorso la voce può diventare parola e cantare: \textit{tipota}, che in greco moderno \textit{nulla}. \\...ma piace anche la pasta in brodo, piacciono i complimenti e il colore azzurro, piace una vecchia scarpa, piace averla vinta, piace accarezzare un cane.. ma cosa è mai la poesia?Più di una risposta incerta è stata già data in proposito. Ma io non lo so e mi aggrappo a questo come alla salvezza di un corrimano (W.Szymborska)}

\descrizione{Yuxtaposiciones}{}

\descrizione{Kokhlos IV}{Nei ultimi anni ha composto brani modellati su ciò che chiama \textit{Architettura Neurologica dell’Ascolto} o anche \textit{Percorso uditivo del cervello}. Questa serie di composizioni, intitolate “Kokhlos” (che in greco antico significa “spirale”), esplora i modelli di ascolto dell’orecchio interno umano. La serie Kohklos è parte della new-media opera  “Dreamlives of Debris” basata su testi dello scrittore americano Lance Olsen. L’elettronica di questo pezzo è stata composta da Adam Vidiksis e Gene Coleman per The Algo.Insitute – una piattaforma di lavoro e di interscambio tra scienza, arte e tecnologia.}

\end{flushright}

\clearpage

%\begin{flushleft}
%
%~\vfill
%
%%***
%
%
%\end{flushleft}
%
%\clearpage
%
%~\vfill
%
%
