% !TEX encoding = UTF-8 Unicode
% !TEX TS-program = XeLaTex
% !TEX root = ../EMU2017_booklet.tex

\begin{flushright}

\large{
	\scshape{
	25 ottobre 2017 -- ore 20:00
	}}

\medskip
	
\small{Concerto
	\newline Sala Accademica}

\medskip

{\fontsize{20}{20} \svolk{\emph{Concerto III}}}

\normalsize

\medskip

regia del suono \textsc{Pasquale Citera} e \textsc{Michele Papa}

\bigskip


\livel{Alessandro Solbiati}{Studio V *}{5’00}{per chitarra}{2017}
\medskip

\livel{Maurizio Pisati}{Chaha\textsuperscript{X}}{8’00}{chaconne hack per chitarra}{2017}
\medskip

\livel{Pasquale Citera}{Come ascoltare i dormiveglia delle vedove}{15’00}{per pianoforte e live electronics}{2017}
\medskip

\livel{Olga Neuwirth}{Spleen III}{10’00}{per sassofono baritono solo}{2001}
\medskip

\livel{Matej Bunderla}{Barrytmic (conceptual improvisation)}{12’00}{per sassofono baritono}{2017}
\medskip

\livel{Simone Santi Gubini}{Klangrelief (Relief II)}{11’00}{per sassofono baritono amplificato}{2017}
\medskip

\bigskip
\textbf{*} prima esecuzione assoluta

\bigskip

\esecutore{chitarra}{Arturo Tallini}
\esecutore{pianoforte}{Arianna Granieri}
\esecutore{sassofono}{Matej Bunderla}

\medskip

\esecutore{live electronics}{Marco De Martino}


\vfill

\descrizione{Studio V}{, dedicato ad Arturo Tallini costituisce il mio ritorno alla chitarra dopo il lavoro di scavo nelle sue possibilità timbrico-articolative effettuato nel 2015 in Sonata. Si tratta anche della riapertura forse ormai imprevista del progetto di otto Studi per chitarra nato nel 1997 e arrestatosi nel 2004. Un omaggio ad una particolare attitudine di Arturo Tallini, è l’aggiunta della possibilità di trattenere con la voce, a bocca chiusa, alcune note della melodia finale, creando così la propria innere Stimme.}

\descrizione{Chaha\textsuperscript{X}}{ è una nuova visione del precedente CHAHACK, hackeraggio della ciaccona dalla Partita N.2 di J. S. Bach, dove scrittura e interprete alternavano fantasiose intrusioni e fughe. Ora, sul disegno di quella partitura, si articola il percorso a rapidi scatti di una luce tra i pentagrammi. È lei che legge per l’interprete. Richiama nelle mani ciò che la memoria già conosce, dettandole i tempi. Una scansione continua, circolare, mentre alcune note diventano “interruttori”: la luce le sfiora e nella traccia audio si attivano nuove aperture, fantasie sull’immagine poetica che è ogni partitura, coi suoi segni e oltre ogni loro eventuale significato.}

\end{flushright}


%\clearpage

\begin{flushleft}

~\vfill


\descrizione{Come ascoltare i dormiveglia delle vedove}{I luoghi sonori riconducibili all'intero vissuto possono esser svolti lungo una mappa dinamica di eventi dalla lettura non univoca, non lineare, che può avere percorsi sempre diversi. Così come nei dormiveglia di chi ha più passato che futuro, si ripercorrono immagini frammentarie modificate da e nel tempo, ognuna di queste immagini - luoghi sonori d'affetto o d'esperienza- appaiono dapprima discontinui, paralleli, a sé stanti. Le variazioni di questi luoghi tra sé stessi, creano rapporti dialogici dove ciò che appare per prima non è per forza l'inizio, dove la causa è successiva all'effetto, in una dimensione di temporalità allargata analoga a quella onirica.\\
La composizione è altresì frutto di studî dei rapporti tra risonanza dello strumento e la sua propagazione nell'ambiente; ogni tempo metronomico ed indicazioni di velocità in partitura sono parametricamente dipendenti dall'acustica della sala, in primo luogo dal tempo di riverberazione, al punto che all'esecutore è chiesto di modificare i tempi seguendo una tabella di valori di riferimento, in modo da avere velocità e risonanze appositamente modellate allo spazio sonoro dell'atto musicale.}



\descrizione{Spleen III}{Terzo pezzo del ciclo spleen per sax Baritono, con prima esecuzione del Maggio del 2001 a Zurigo. L'eccesso e il grottesco hanno sempre costituito il fondamento stesso del servatoio musicale di Olga Neuwith. Qui, niente è risparmiato. non l'individuo compositore, non l'ascoltatore, e nemmemo l'esecutore. La compositrice già interrogata sul ciclo partito con Spleen I per clarinetto Basso (1994) risponde: \textit{Non c'è Niente da dire, non voglio semplicemente dire nulla}}

\descrizione{Barrytmic (conceptual improvisation)}{è un brano di improvvisazione concettuale dove il musicista prova a combinare tecniche musicali estese a una struttura poliritmica.}

\descrizione{Klangrelief (Relief II)}{Klangrelief è suono, soltanto suono incentrato sulla purezza della difettosità strumentale. L'esecuzione è fisica, “fuori controllo”, come il suono è in rilievo (relief), in eccesso, presente e comprensibile sul corpo dell'ascoltatore.\\Una musica quindi da ascoltare dal di dentro, dall'incavo dello strumento, nella cavità del suono che diviene sorgente. Un suono liberato, estremamente instabile, in squilibrio, ispessito da una timbrica raschiata e fatto di materia prima; composto da una forma fluida e naturale che si condensa nell'articolazione, in continua - graduale o immediata - sovraesposizione di energia (timbro e tempo).\\Forma che diviene articolazione, materia grezza torchiata, è come affettare il vetro: un suono sovraesposto, il suono per ogni forma che forma ogni suono. Come questa si disperde o smette di essere dinamica, massima, tensiva, perciò fatta di testure contrastanti che creano nuove, sorprendenti relazioni, immediatamente non c'è più nulla da ascoltare: l'attitudine al fuori controllo diviene musica informe, come non nata.\\L’ascoltare diventa quindi cercare, un’attiva, dinamica inesprimibilità, l’inafferrabilità del suono. Si ha come la sensazione di percorrere un vasto, illimitato territorio dinamico, estremamente astratto eppure naturale, gradualmente proprio. Una monumentalità la cui origine è composta di “polvere sonora” (ancora materia prima), un’immagine residua: ciò che risuona continuamente. L’espressività di una grumosa materia che si sedimenta nel senso di una memoria inconsistente, olfattiva quasi, antica e inattuale: il territorio estremo di una serenissima ambiguità. La “pelle della musica” e il rinnovamento dei materiali, degli attrezzi scelti, questo è ciò che interessa, insieme alla facoltà di vivere una musica nei nervi e nelle viscere che l’hanno creata.}

\end{flushleft}

\clearpage

%\begin{flushleft}
%
%~\vfill
%
%%***
%
%
%\end{flushleft}
%
%\clearpage
%
%~\vfill
%
%
