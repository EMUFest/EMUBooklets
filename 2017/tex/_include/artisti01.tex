% !TEX encoding = UTF-8 Unicode
% !TEX TS-program = XeLaTex
% !TEX root = ../EMU2017_booklet.tex

%\flushright

\vfill

\textbf{\emph{Biografie Artisti}}


%2016-701:
%\biografia{}{ }

\biografia{Franck Bedrossian}{ Ha studiato con Gaussin, Leroux, Ferneyhough, Murail, Manoury e Lachenmann. I suoi lavori sono eseguiti in Festival internazionali da ensemble tra cui: l’Itineraire, 2e2m, Court-Circuit, Ensemble Modern, Intercontemporain, Orchestre National de Lyon, San Francisco Contemporary Music Players. Nel 2001 riceve una borsa di studio dalla Meyer Foundation e nel 2004 vince il premio Hervé-Dugardin (Sacem). Nel 2005 l’Institut de France (Académie des Beaux-Arts) gli conferisce il “Prix Pierre Cardin” per la composizione. Nel 2006-08 è in residenza a Villa Medici. Da settembre 2008 è assistente professore di composizione alla University of California-Berkeley. I suoi lavori sono pubblicati dalla Editions Bilaudot.}


\biografia{John Cage}{Studia con Schoenberg, Cowell, Suzuki, Fuller e Duchamp. Docente alla Cornish School di Seattle, al Mills College di Oakland, al Chi- cago Institute of Design, al Black Mountain College, alla New School for Social Research di New York dal 1936 al 1960. Grazie alla sua grande inventiva come compositore, pensatore e scrittore, ha un posto in prima fila nell’avanguardia internazionale del Novecento. Elabora un linguaggio intimo e rivoluzionario partendo dalla dissacrazione totale delle "regole" musicali classiche. Inventa le composizioni per "pianoforte preparato" e introduce la “casualità” in musica. Fra le sue opere più importanti si citano: Music for Marcel Duchamp (1947); Concerto for prepared piano and chamber orchestra (1951); Music of changes (1951); Atlas eclipticalis (1961); Etudes Australes (1974- 75); Quartets I-VIII (1976) per orchestra; Thirty pieces for five orchestras (1981).}


\biografia{James Dashow}{ compositore, dedica la sua principale attività compositiva alla computer music, spesso con esecutori dal vivo, pur non trascurando la musica per strumenti tradizionali. La sua attività di ricerca sfocia nella creazione di un suo linguaggio di sintesi, MUSIC30, ed un suo metodo di composizione, il Sistema Diadi. Uno dei fondatori del Centro Sonologia Computazionale di Padova. Nel 2011, la Fondazione CEMAT (Roma) gli ha conferito il premio CEMAT per la Musica per il suo contributo allo sviluppo della musica elettronica. Ha insegnato al MIT dove ha ricoperto il ruolo di direttore supplente dello Studio di Musica Sperimentale, e alla Princeton University. È stato vice-presidente nel primo comitato direttivo dell'International Computer Music Association, e per molti anni ha condotto il programma radiofonico *Il Forum Internazionale di Musica Contemporanea* per RAI Radio 3. I suoi lavori sono registrati su DVD, CD e LP di varie case discografiche italiane ed estere: BMG Ariola - RCA, Wergo, EdiPan, Capstone, Neuma, ProViva, CRI, Scarlatti Classica, BVHAAST e Centaur.}

\clearpage
%\flushleft
%~\vfill

\flushright
~\vfill





%2016-714:
\biografia{Enzo Filippetti}{ è professore di Sassofono al Conservatorio di Musica “Santa Cecilia”. In oltre trent’anni di attività ha tenuto concerti in tutto il mondo esibendosi alla Biennale di Venezia, al Mozarteum di Salisburgo, al Conservatorio di Parigi, a Roma, Milano, New York, Londra. È molto attivo nel campo della musica contemporanea, di cui è apprezzato interprete, e molti importanti compositori hanno scritto per lui più di cento opere. Come solista e come membro del Quartetto di Sassofoni Accademia ha inciso per Nuova Era, Dynamic, Rai Trade e Cesmel. Ha pubblicato studi per Riverberi Sonori e dirige una collana per le edizioni Sconfinarte.}

\clearpage
\flushleft
~\vfill

\biografia{Pierre Jodlowski}{ Dopo gli studi presso il Conservatorio di Lione e l'IRCAM, Pierre Jodlowski fonda il collettivo l'éOle  e il Novelum Festival a Tolosa. Le sue attività di compositore lo hanno portato in molti luoghi della Francia e all'ester, per la nuova musica assieme alla danza, al teatro, le arti plastiche e la musica elettronica. Oltre al suo universo musicale Jodlowski lavora con immagini, programmazione interattiva e sceneggiatura. Asserisce la pratica di una musica "attiva" sia nella sua dimensione fisica (gesti, energie e spazi) che nella sua dimensione psicologica (evocazione, memoria e aspetto cinematografico). Ha ricevuto commissioni da IRCAM, l'Ensemble Intercontemporain, il Ministero della Cultura, il CIRM, il Festival di Donaueschingen, la Radio France e il Concorso Pianistico di Orléans.}


%2016-719:
\biografia{Ivan Liuzzo}{ nasce a Frosinone il 26 febbraio 1993 ed inizia a studiare la batteria all’età di 9 anni con M. FIOCCO, successivamente con G.GUIDONI e A.BLASI. Nel 2007 intraprende gli studi delle percussioni presso il Conservatorio *L.Refice* di Frosinone con il M° C. DI BLASI, conseguendo il diploma (V.O.). Ha approfondito lo studio della batteria jazz con R. PISTOLESI e G. HUTCHINSON. Membro attivo e co-fondatore del Collettivo Phthorà, assieme a F. Ferazzoli e F. Abbate ha collaborato con artisti come: Lisa Mezzacappa, Stefano Costanzo, Vincenzo Core, Wound, Ron Grieco, Achille Succi ecc.}

\clearpage
\flushleft
~\vfill

%\flushright
%~\vfill


\biografia{Giorgio Nottoli}{ (compositore, nato a Cesena, Italia nel 1945) è stato docente di Musica Elettronica al Conservatorio di Roma "S.Cecilia" sino al 2013. Attualmente è docente di Composizione elettroacustica all’Università di Roma "Tor Vergata". La maggior parte delle sue opere utilizza mezzi elettronici sia per la sintesi che per l'elaborazione del suono. Il centro della sua ricerca di musicista riguarda il timbro concepito quale parametro principale e *unità costruttiva* delle sue opere attraverso la composizione della microstruttura del suono. Nei suoi lavori per strumenti ed elettronica Giorgio Nottoli punta ad estendere la sonorità degli strumenti acustici mediante complesse elaborazioni del suono. Ha progettato vari sistemi elettronici per la musica utilizzando sia tecnologie analogiche che digitali in collaborazione con varie università e centri di ricerca.}

\biografia{Demien Rudel Rey}{(Argentina, 1987) Si diploma in chitarra. Laurea in Composizione con Santiago Satero e ha conseguito il Master in Arti Combinate all’UNA (Argentina). Ha presenziato a seminari con Dhomont, Vaggione, Tutschku, etc. È iscritto al Master in Composizione al CNSMD di Lione con Martin Matalon. È stato coordinatore del Festival Bahía[in]sonora. È stato premiato/menzionato in occasione dei seguenti eventi; TRINAC, SADAIC, Destellos, FAUNA, IndieFEST, Konex Mozart Award, Martirano Award, Sagarik Award, CICEM, Métamorphoses, MA/IN, Prix Jolivet, Forum Wallis, Earplay, etc.}

%2016-737:
\biografia{Saxatile\textsuperscript{4} [\textbf{m}odulable \textbf{s}axophone \textbf{e}nsemble]}{ è un progetto realizzato da un’idea di Enzo Filippetti nell’ambito del Conservatorio di Musica “S. Cecilia” specificatamente rivolto all’esecuzione della musica contemporanea, nei suoi molteplici aspetti, con un orientamento diretto alla ricerca e all’instaurazione di un rapporto dialettico con altri artisti e con i compositori. I componenti possono vantare una solida, variegata esperienza.}


\biografia{Mauro Tedesco}{ Allievo del M°Daniele Roccato presso il conservatorio S.Cecilia di Roma. Membro dell'ensamble di contrabbassi Ludus Gravis (www.ludusgravis.com) dedito nel repertorio contemporaneo il quale ha collaborato con compositori di fama internazionale come Terry Riley, Sofia Gubaidulina , Julio Estrada etc. Si è esibito con artisti di nota importanza come Dominique Pifarely e Michele Rabbia presso l'Auditorium parco della musica di Roma durante il festival "Una striscia di terra feconda" organizzato da Paolo Damiani e Armand Meignan; con Luca Sanzò presso l'Accademia Filarmonica Romana nel festival "Paesaggi Sonori" organizzato da Michelangelo Lupone e Giorgio Nottoli; Ha recentemente fatto parte del progetto "Assedio, frammenti di un reportage" organizzato da Lucia Goracci, reporter di Rainews24 presso la Siria, che vede coinvolta anche l'Istituzione Sinfonica Abruzzese con musiche di Tonino Battista e Daniele Roccato presso l'Auditorium del parco di L' Aquila e il Teatro dei Rinnovati a Siena.}

\biografia{Manuel Zurria}{ Sono nato a Catania nel 1962. Pur avendo tentato altre esperienze in giro per il mondo, sono sempre ritornato a vivere a Roma, mio malgrado. Ho partecipato a prime assolute di Pennisi, Bussotti, Clementi, Guarnieri, Donatoni, Vacchi e Francesconi. Collaboro da molti anni con Sciarrino e Lucier. Ho partecipato a Festival internazionali, in tutto il mondo. La mia discografia conta attualmente 40 tra CDs e vinili per etichette quali Ricordi, Capstone, EdiPan, Stradivarius, Die Schachtel, Mazagran, Mode, Megadisc, God, Atopos, Touch, Another Timbre.}

