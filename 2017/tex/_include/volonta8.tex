% !TEX encoding = UTF-8 Unicode
% !TEX TS-program = XeLaTex
% !TEX root = ../EMU2017_booklet.tex

\begin{flushright}

\large{
	\scshape{
	28 ottobre 2016 -- ore 18:00
	}}

\medskip
	
\small{Concerto Acusmatico
	\newline Il Suono di Piero [Aula Bianchini]}

\medskip

{\fontsize{42}{42} \svolk{\emph{Volontà VIII}}}

\normalsize

\medskip

regia del suono \textsc{Francesco Ziello}

\bigskip

\livel{Marco Ferrazza}{CitiZen}{8'18}{}{2016}
\medskip

\livel{Paolo Pastorino}{Dimensione aggiuntiva}{3'25}{}{2016}
\medskip

\livel{Chester Udell}{Jorneta Stream}{11'00}{}{2014}
\medskip

\livel{Leo Cicala}{Khoisan}{10'45}{}{2015}
\medskip

%\brano{Christian Eloy}
%{La cicatrice d'Ulysse}{13'00''}
%{acusmatico}
%new version 2015\\

%\acusmatici{Ursula Meyer-K\"onig}
%{Allears}{2012-13}{8'}

%\vspace{6mm}

\vfill

\descrizione{CitiZen}{I materiali sonori di Citizen provengono essenzialmente da registrazioni ambientali effettuate in differenti città. Segnali tipici del traffico come il clacson convivono con suoni di campane, tessiture rumoristiche e presenze metalliche indistinte. Ne deriva un’orchestrazione intesa come riorganizzazione di più paesaggi sonori e condivisione di differenti spazialità (sia geografiche che acustiche).}

\descrizione{Dimensione aggiuntiva}{Per \emph{Dimensione aggiuntiva} si intende una dimensione supplementare che viene generalmente indicata come una ulteriore estensione di un oggetto. L’obiettivo di questa composizione è stato quello di creare una connessione timbrica e temporale tra gli oggetti sonori impiegati.  Elementi provenienti da ambienti e contesti differenti, totalmente estranei tra loro, coesistono e dialogano insieme, dando così origine ad una forma \emph{viva} capace di muoversi in uno spazio immaginario.}

\descrizione{Jorneta Stream}{ Dovremmo vivere tanto a lungo quanto i nostri racconti sono umidi del nostro respiro}

\descrizione{Khoisan}{ è un pezzo simbolico che gioca sugli elementi morfologici peculiari di questa lingua primordiale, Khoisan appunto,  ricca di consonanti dure e schioccanti. Rappresenta una esplorazione psicologica ed intima della spinta  alla migrazione, che da sempre per la nostra specie si ripete tra l’Africa e l’Europa. IL brano è organizzato metaforicamente in eventi che si susseguono come una serie di passi, di tappe; nella prima parte l’evoluzione degli eventi sonori è inserita nella scia di  un  gesto primario che rappresenta la necessità di fare qualcosa in risposta ad un’altra. Le restanti tre parti sono costruite intrecciando microeventi realizzati con varie tecniche tra cui la risintesi : lo scopo è quello di generare materiali nuovi sullo stampo dei materiali di partenza per rendere il contrasto tra il fascino di un mondo migliore e la paura dell’ignoto.}
%\bigskip

%\svolk{\emph{Ho cercato di incanalare quell’energia in un percorso che la rendesse percepibile senza snaturarne l’essenza: l’energia che abita i violini di Corelli, Tartini, Vivaldi, certo non citazioni ne dirette ne indirette ma l’imprevedibile vitalità delle loro articolazioni (tremoli, arpeggi, ribattuti, sincronie e fioriture improvvise) che hanno fatto della scuola italiana, elettrica ante litteram, un irraggiungibile modello di virtuosismo strumentale; poi la tensione, il vuoto attorno e dentro alla costruzione delle frasi, le imitazioni, i pedali, la continua sovrapposizione delle corde.}}
%
%Giorgio Netti
%
%\normalfont

\end{flushright}

\clearpage

\begin{flushleft}

~\vfill



***

\biografia{Marco Ferrazza}{Compositore di musica acusmatica e audio performer, Marco Ferrazza ha studiato arte contemporanea e musica elettronica. I suoi lavori, eseguiti in varie competizioni e festival, indagano costantemente le relazioni tra musica concreta e computer music, arti elettroniche e registrazioni ambientali, improvvisazione e nuove tecnologie.}

\biografia{Paolo Pastorino}{(1983) è un chitarrista, sound designer e compositore. Si è diplomato col massimo dei voti e la lode in musica elettronica presso il Conservatorio di Sassari. Attualmente frequenta il biennio di specializzazione in M.E. presso il Conservatorio di Cagliari. Suoi lavori sono stati presentati al CIM, 3ème Concours International de Composition pour un instrument acoustique et dispositif électronique (Bourges, Francia), DronesTruck Como (Midway Parkway St. Paul, Minnesota - USA), CIRMMT (Centre for Interdisciplinary Research in Music Media and Technology - Montréal), Galleria comunale d’arte di Cagliari in occasione di Musei Aperti, Festival Suona Italiano Suona Francese 2015, Inter \#6: experimental sound for loudspeakers (Glasgow - UK).}

\biografia{Chester Udell}{Dalle antiche paludi di cipressi di Wewahitchka, Chester (Chet) Udell conseguì il dottorato in Composizione Musicale all'Università della Florida con una particolare attenzione nell'ambito dell' ingegneria elettronica. Ha conseguito la laurea di primo livello in Arti Musicali/Digitali all'Università di Stetson (2005) e in seguito il master in Composizione Musicale all'Università della Florida. I suoi interessi includono: l'ecologia acustica, registrazioni d'ambienti, la composizione, l'ingegneria elettronica (sistemi digitali, interfacce, comunicazione wireless), il \emph{circuit bending}, e nella costruzione di sistemi musicali autonomi. La sua tesi di ricerca sulla progettazione di nuove interfacce musicali compare in un brevetto americano registrato (depositato) e in una compagnia di startup tecnologiche: la \emph{eMotion Technologies LLC}. Alcuni dei suoi riconoscimenti comprendono: un premio presso i Walt Disney's Dreamers e Doers Award, la qualificazione tra i primi otto finalisti mondiali al Georgia Tech's Margaret Guthman Musical Instrument Competition 2014, la nomina tra gli University of Florida 20 under 30 Gators to Watch, la menzione d'onore alle International Composition Competition 2011 e finalista al Sound in Space 2011. Gli fu assegnato il primo premio alle SEAMUS/ASCAP Student Commission Competition 2010 - una delle più alte onoreficenze americane per studenti di musica elettroacustica -.  La sua musica è stata presentata in tutto il mondo ed è disponibile presso le etichette Summit e SEA.}

\biografia{Leo Cicala}{Leonardo \emph{Leo} Cicala Compositore, interprete acusmatico, live performer, insegnante. Ha compiuto gli studi in Musica Elettronica e Strumentazione per Banda presso il Conservatorio Musicale \emph{T. Schipa} di Lecce, ha conseguito la laurea in Biologia ed in Infermieristica ed ha studiato Batteria e Musica Jazz. Ha studiato proiezione sonora all’acusmonium con Jonatan Prager. Ha interpretato all’acusmonium più di cento opere, ed ha tenuto diverse conferenze su vari aspetti della spazializzazione delle opere acusmatiche in Italia e all’estero. Nel 2015 Ha Pubblicato il \emph{Manuale di Interpretazione Acusmatica} per la Salatino Edizioni Musicali, ed una serie di video didattici ad esso collegati. Nel 2014 ha pubblicato il cd \emph{Rust} per l’etichetta pugliese \emph{Art \& Classica}. Le sue composizioni sono state eseguite in importanti manifestazioni in Italia, Francia, Giappone, UK, Stati Uniti, Belgio. Vincitore del primo premio in composizione elettroacustica \emph{Bangor Dylan Thomas Prize} in UK.}

\end{flushleft}
