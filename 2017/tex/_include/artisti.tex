% !TEX encoding = UTF-8 Unicode
% !TEX TS-program = XeLaTex
% !TEX root = ../EMU2017_booklet.tex

%\flushright

~\vfill

\textbf{\emph{Biografie Artisti}}


%2016-701:
%\biografia{}{ }

\medskip

\biografia{Franck Bedrossian}{ Ha studiato con Gaussin, Leroux, Ferneyhough, Murail, Manoury e Lachenmann. I suoi lavori sono eseguiti in Festival internazionali da ensemble tra cui: l’Itineraire, 2e2m, Court-Circuit, Ensemble Modern, Intercontemporain, Orchestre National de Lyon, San Francisco Contemporary Music Players. Nel 2001 riceve una borsa di studio dalla Meyer Foundation e nel 2004 vince il premio Hervé-Dugardin. Nel 2005 l’Institut de France (Académie des Beaux-Arts) gli conferisce il \textit{Prix Pierre Cardin} per la composizione. Nel 2006-08 è in residenza a Villa Medici. Da settembre 2008 è assistente professore di composizione alla University of California-Berkeley. I suoi lavori sono pubblicati dalla Editions Bilaudot.}

\biografia{Roberto Begini}{ è un sound artist e compositore di musica elettronica. La sua ricerca è orientata verso la composizione a partire da sorgenti sonore non strumentali, con particolare interesse per la generazione di texture e bordoni sonori. Ha all’attivo pubblicazioni per Oak Editions, Manyfeetunder/ Concrete e Stochastic Resonance. Da diversi anni lavora come sound designer in campo musicale e cinematografico, in collaborazione con artisti e produzioni internazionali (London Film School, Goldsmiths University, Royal Central School of Speech and Drama, Royal College of Music).}

\biografia{Sauro Berti}{ Clarinetto Basso del Teatro dell’Opera di Roma, ha collaborato con le orchestre Italiane più importanti (Teatro alla Scala, Maggio Musicale Fiorentino, RAI National Orchestra), con il Royal Scottish National Orchestra e  Sinfonia Finlandia Jyväskylä all’estero. Ha suonato con direttori come G. Prêtre, R. Chailly, M. W. Chung, R. Muti, W. Sawallisch, V. Gergiev, L. Maazel, P. Boulez e Z. Mehta. Ha partecipato come solista in vari festival musicali in tutto il mondo. Nel 2009 si è laureato in Direzione d’orchestra con D. Renzetti. Ha pubblicato \emph{Venti Studi per Clarinetto basso}, \emph{Tuning} per fiati (Suvini Zerboni), la sua versione da concerto di V. Bucchi e i CD: \emph{Suggestions}(Edipan) e \emph{SoloNonSolo} (ParmaRecords).}

\biografia{Lo\"{i}se Bulot}{ ha iniziato la sua carriera come artista visiva, il suo lavoro crea quindi un universo attraverso le arti visive e la musica: composizioni miste ed elettroacustica ibrida. Esplorando poeticamente la complessità di fenomeni inediti per il suono L. Bulot crea un’affascinante interazione tra trame sonore con passaggi da tessiture pontillistiche a sonorità più ampie. Dopo aver studiato piano e arte a Parigi ha proseguito i suoi studi presso l’istituto di belle arti di Marsiglia, quindi presso l’accademia di musica (CNR) dove nel 2015 ha ottenuto un premio di composizione elettroacustica. Attualmente sviluppa il suo lavoro attraverso diversi progetti: composizioni elettroniche e miste, performance dal vivo, pittura, laboratori di gruppo e altro. Lavori di Loïse Bulot sono stati commissionati e presentati nei principali festival internazionali tra cui CIRMMT (Canada), Festival Les Musiques, Festival Reevox (Francia), Festival MUSLAB (Messico), Banc d’essai-GRM (Francia), Festival Futura (Francia), Heroines of Sound Festival (Germania), Festival LEM (Spagna), Musica Electronic Nova (Polonia).}

\biografia{Matej Bunderla}{ è nato nel 1985 nella città di Maribor in Slovenia. Studia il sassofono dal 2004 al 2013, pedagogia strumentale e musica contemporanea con i maestri Peter Straub, Gerald Preinfalk e suona nell'orchestra da camera Klangforum Wien. Partecipa alle MasterClass di Claude Delangle, Marcus Weiss, Arno Bornkamp e Matjaž Drevenšek. Dall'orchestra classica alle formazioni jazz come la Tot Big Band Maribor, la European Master Orchestra, giungendo poi all'Orchestra of the Opera Graz, esegue parti solistiche in tutta Europa, nel campo della musica classica, contemporaneo e jazz. Lavora come membro freelance in varie band jazz, orchestre da camera come lo Schalfeld Ensemble, il quartetto di sassofoni WAP e per progetti coreutici e teatrali off-scene. Dopo aver lavorato constantemente in stretta collaborazione con molti compositori, ha iniziato anche lui a realizzare composizioni di musica contemporanea per sassofono.}

\biografia{John Cage}{Studia con Schoenberg, Cowell, Suzuki, Fuller e Duchamp. Docente alla Cornish School di Seattle, al Mills College di Oakland, al Chi- cago Institute of Design, al Black Mountain College, alla New School for Social Research di New York dal 1936 al 1960. Grazie alla sua grande inventiva come compositore, pensatore e scrittore, ha un posto in prima fila nell’avanguardia internazionale del Novecento. Elabora un linguaggio intimo e rivoluzionario partendo dalla dissacrazione totale delle "regole" musicali classiche. Inventa le composizioni per "pianoforte preparato" e introduce la “casualità” in musica. Fra le sue opere più importanti si citano: Music for Marcel Duchamp (1947); Concerto for prepared piano and chamber orchestra (1951); Music of changes (1951); Atlas eclipticalis (1961); Etudes Australes (1974- 75); Quartets I-VIII (1976) per orchestra; Thirty pieces for five orchestras (1981).}

\flushright

\biografia{Maura Capuzzo}{Nata a Padova, studia con C.Benati, G. Bonato, N. Bernardini ed si diploma in Musica Corale,  Composizione, Musica elettronica. Segue corsi di perfezionamento con F.Valdambrini, M.Bonifacio e S. Sciarrino, seminari con G.Grisey, H.Lachenmann, M.Stroppa A. Vidolin. 1997 vince l’ European Women Composers Contest \textit{Kaleidoscope Programm of European Union}.  2000 Borsa di Studio al corso di Sciarrino, Città di Castello, Festival delle Nazioni. 2001, III premio, al Concorso Internazionale di Composizione Corale \textit{A Cappella} Germania. Nel 2009 Borsa di Studio della Fondazione Lerici dell'IIC, in collaborazione con il KTH di Stoccolma. 2011 II premio concorso di composizione organistica,Mantova (in giuria Adriano Guarnieri). 2012, premiata con esecuzione al Festival Biennale Koper,Slovenia, (V. Globokar,Fabio Nieder in giuria) Sue composizioni son state eseguite in Italia ed all'estero: Festival Urticanti,ISCM Festival ( Hong Kong 2007) MiTo, Musikpodium Zurig, Biennale Koper, Visioni del Suono, SpazioMusica,CIM (Trieste 2012,Roma 2014), Camino Controcorrente Udine, Piano Fazioli Series (IIC Los Angeles) Festival Germi, Festival Mixtur, Segnali Sonori, New Made Week-Siae Classici d’oggi,Festival 5 Giornate,Astra Concert Season Melbourne.Sue composizioni sono state trasmesse anche da Radio3, WDR3, Radio4(Hong Kong) Radio DRS2,Radio Cemat, Radio e Tv Koper. I suoi lavoro sono editi da ArsPublica.Insegna Teoria Ritmica e Percezione musicale al B. Marcello,Venezia}


\biografia{Antonio Carvallopinto}{Parallelo a suoi studi di pianoforte studia contrappunto e armonia con Rodolfo Norambuena, studiando dopo presso l’ “Universidad de Chile”, dove ottiene la Laurea in Composizione. I suoi insegnati furono Pablo Aranda, Cirilo Vila, Aliosha Solovera e Miguel Letelier, tra Poi si trasferisce a Roma, dove studia Musica Elettronica presso il Conservatorio Santa Cecilia. Ottiene il Diploma Accademico di primo e secondo livello. Al suo ritorno in Cile ottiene il Master in Arti menzione Composizione Musicale e il Dottorato in Filosofia menzione Estetica e Teoria delle Arti presso l'Univesidad de Chile. Le sue composizioni sono state eseguite in Cile, Italia, Francia, Olanda Svezia e Germania e sono state pubblicate su cd e sul libro e partitura. La sua labore come insegnante commincia l'anno 2000, insegnando Analisi Musicale e Teoria e Solfeggio presso l'Universidad de Chile. Oggi insegna Composizione, Analisi Musicale, Contrappunto e Armonia presso la Facoltà d'Arte dell' Universidad de Chile e Musica Elettroacustica presso la Pontificia Universidad Católica de Chile. È il Coordinatore del Gabinetto di Elettroacustica per la Musica d'Arte e del Master in Arte menzione Composizione Musicale dell'Universidad de Chile. É il Presidente dell'Associazione di Compositori del Cile.}



%2016-707:
\biografia{Pasquale Citera}{ Compositore, Sound Designer, Sound Engineer. 
Dopo aver studiato Pianoforte, Composizione sperimentale, Lingue orientali e Musica elettronica, si dedica alla composizione per spettacoli teatrali classici e contemporanei, colonne sonore, installazioni d’arte. Dal 2016 è Cultore della Materia in Musica Elettronica presso il Conservatorio Santa Cecilia di Roma.}

\biografia{Gene Coleman}{Compositore, musicista e direttore d’orchestra. Membro del Guggenheim nel 2014 e vincitore del Premio dell’Accademia Americana a Berlino nel 2013, ha creato oltre 70 composizioni per organici diversi e media.  L’innovazione nella manipolazione di suono, immagine, spazio e tempo ha permesso a Coleman di espandere la percezione dell’ascoltatore attraverso le sue opere. Il suo lavoro è incentrato principalmente sul mutamento globale della cultura e della musica in relazione alle nuove tecnologie. Ha studiato con importanti registi di cinema di sperimentazione, come Stan Brakhage e Ernie Gehr e col compositore Robert Snyder presso l’Istituto d’Arte di Chicago.}

\biografia{CumTempora Ensemble}{Il CumTempora Ensemble nasce in risposta al vuoto di valori che la nostra società sta vivendo. \textit{Musica sola fovet cum tempora fusca videntur} è il suo manifesto ed esprime a pieno la volontà di restituire centralità alla cultura nelle sue forme artistiche e musicali, forme essenziali perché costitutive dell’essenza umana. Forza trainante del CumTempora è la profonda relazione con la contemporaneità, l’epoca di cui si è figli, espressa attraverso la passione per la musica contemporanea, in un legame di stima ed amicizia, che unisce ogni elemento dell’ensemble.}

\clearpage
\flushleft
~\vfill


\biografia{Chaya Czernowin}{di origine israeliana vive e lavora negli USA. A 25 anni riceve una borsa dal DAAD per studiare in Germania a Berlino e successivamente a Stoccarda e a Vienna, negli Stati Uniti e  a Tokyo in Giappone. Prima donna ad essere nominata Docente di Composizione all’Università di Musica e Arti Performative di Vienna, Austria (2006 – 2009) e all’Università di Harvard (2009 in poi). La sua produzione include musica da camera e orchestrale, con e senza elettronica. I suoi lavori sono eseguiti in tutto il mondo dai migliori interpreti e in importanti festival di musica contemporanea. Ha ricevuto numerosi premi e riconoscimenti, tra i più importanti:  Composer Unesco Rostrum 1980; Darmstadt Ferienkurse; IRCAM (Parigi); Fondazione Siemens (’03); Fondazione Rockefeller (’04); Fondazione Fromm (’09); Fondazione Guggenheim (’11). La sua musica è edita dalla Casa editrice Schott.}

%2016-710:
\biografia{Elena D'Alò}{ flautista (dall’ottavino al flauto basso), si laurea cum laude al biennio in Flauto, dopo un brillante diploma, presso il Conservatorio \emph{Santa Cecilia} di Roma, con Deborah Kruzansky, studiando anche con Edda Silvestri, Bruno Paolo Lombardi, Paolo Taballione, Kathinka Pasveer. Ha affiancato gli studi musicali con quelli scientifici, laureandosi in Fisica acustica presso \emph{La Sapienza} con Paolo Camiz. Attualmente è iscritta al triennio di Musica Elettronica. Si esibisce in formazioni cameristiche e orchestrali, in un repertorio che va dal barocco al contemporaneo, per il quale ha partecipato a festival come Nuova Consonanza, Atlante Sonoro XXsecolo, ArteScienza ed EMUFest.}

\biografia{James Dashow}{ compositore, dedica la sua principale attività compositiva alla computer music, spesso con esecutori dal vivo, pur non trascurando la musica per strumenti tradizionali. La sua attività di ricerca sfocia nella creazione di un suo linguaggio di sintesi, MUSIC30, ed un suo metodo di composizione, il Sistema Diadi. Uno dei fondatori del Centro Sonologia Computazionale di Padova. Nel 2011, la Fondazione CEMAT (Roma) gli ha conferito il premio CEMAT per la Musica per il suo contributo allo sviluppo della musica elettronica. Ha insegnato al MIT dove ha ricoperto il ruolo di direttore supplente dello Studio di Musica Sperimentale, e alla Princeton University. È stato vice-presidente nel primo comitato direttivo dell'International Computer Music Association, e per molti anni ha condotto il programma radiofonico *Il Forum Internazionale di Musica Contemporanea* per RAI Radio 3. I suoi lavori sono registrati su DVD, CD e LP di varie case discografiche italiane ed estere: BMG Ariola - RCA, Wergo, EdiPan, Capstone, Neuma, ProViva, CRI, Scarlatti Classica, BVHAAST e Centaur.}


\biografia{Roderik De Man}{Compositore olandese, nato in Indonesia, si occupa di opere da camera, elettroacustica e multimedia,  eseguite in tutto il mondo. Ha studiato percussioni con Frans van der Kraan e teoria musicale presso il Conservatorio Koninklijk a Den Haag, successivamente  composizione con Kees van Baaren lavorando come studente nello studio di musica elettronica di Dick Raaijmakers. È stato premiato al Concours International de Musique Électroacoustique de Bourges (Deuxième Prix, 1991, per Chordis Canam, Deuxième Prix, 1999, per Air to Air, Premier Prix, 2005, per Cordes invisible , Euphonie d'Or, 2010 , per gli invisibili Cordes)}


%2016-712:
\biografia{Marco De Martino}{ Nato a Roma. Dopo i primi studi di Pianoforte e il diploma di Liceo scientifico, prosegue presso la facoltà di Lettere e Filosofia all’Università di Roma Tor Vergata. Laureato in Musicologia, ha studiato Composizione Elettroacustica con Giorgio Nottoli, Michelangelo Lupone e Informatica Musicale con Nicola Bernardini. Inizia il suo percorso di docenza come assistente di Informatica Musicale per il conservatorio e per il Master di II livello in Interpretazione della Musica Contemporanea, sempre del Conservatorio di Roma.}

\biografia{Francis Dhomont}{ (Parigi 1926) Compositore francese e canadese, dottore honoris causa di l' Università di Montreal, Canada. Convinto dell’ originalità dell’arte acusmatica, la sua produzione è, da 1960, esclusivamente costituita d’opere per nastro. Ha insegnato la composizione elettroacustica, tra 1980 e 96, all’ Università di Montreale e ha ricevuto molti onorificenze internazionali: cinque volte premiato al concorso internazionale di musica elettroacustica di Bourges (Francia); Premio \textit{magistere} nel 1988; Premio SACEM 2007 della migliore creazione contemporanea elettroacustica; borsa di carriera del Consiglio delle arti e delle lettere del Québec (2000). Nel 1999, otteneva cinque primi prezzi internazionali per quattro delle sue opere; Prezzo Lynch-Staunton del Consiglio delle arti del Canada, ed invitato del DAAD a Berlino. Si assume la direzione di numerosi speciali \textit{L'espace du son} per le edizioni \textit{Musiques et recherches} (Ohain, Belgio). Dal 1978 Francis Dhomont a condiviso la sua attività tra la Francia ed il Québec e svolge una carriera internazionale. Tornato in Francia nel 2004, vive oggi in Avignon e si dedica alla composizione ed alla ricerca. \href{http://www.electrocd.com/fr/bio/dhomont_fr/discog/}}

\clearpage
\flushright
~\vfill

%2016-713:
\biografia{Sara Ferrandino}{ si è diplomata in pianoforte nel 2005 presso il Conservatorio di Perugia nella classe del Mº Tanganelli, conseguendo nel 2009, con votazione di 110 e Lode, la Laurea per il Biennio Specialistico. Nel 2012 ha ottenuto il diploma del Corso di Perfezionamento tenuto dal Mº Perticaroli, presso l’Accademia Nazionale di Santa Cecilia in Roma. Ha partecipato a numerosi concorsi nazionali e internazionali ottenendo sempre piazzamenti nelle prime posizioni. Si è esibita in molteplici concerti solistici e cameristici in prestigiose sale in Italia e all’estero. Ha collaborato e collabora presso il Conservatorio di Perugia con le classi di corno, tromba, flauto, oboe, sassofono e violino. È docente di pianoforte principale per i corsi pre-accademici presso l'Accademia AIMART in Roma.}

%2016-714:
\biografia{Enzo Filippetti}{ è professore di Sassofono al Conservatorio di Musica “Santa Cecilia”. In oltre trent’anni di attività ha tenuto concerti in tutto il mondo esibendosi alla Biennale di Venezia, al Mozarteum di Salisburgo, al Conservatorio di Parigi, a Roma, Milano, New York, Londra. È molto attivo nel campo della musica contemporanea, di cui è apprezzato interprete, e molti importanti compositori hanno scritto per lui più di cento opere. Come solista e come membro del Quartetto di Sassofoni Accademia ha inciso per Nuova Era, Dynamic, Rai Trade e Cesmel. Ha pubblicato studi per Riverberi Sonori e dirige una collana per le edizioni Sconfinarte.}

%\clearpage
%\flushleft
%~\vfill

\biografia{Francesco Galante}{Ha studiato musica elettronica in Italia e in Francia. 
Compositore,ricercatore,saggista. Co-fondatore della Societa di Informatica Musicale(1982-1990). E’ stato ricercatore nel campo tecnologico (ICMC1984 e 1986, CIM 1988).E’ co-autore dei volumi "Musica espansa" e "Metafonie". Nel 1997 è stato "composer in residence" presso IIME Bourges (Francia). Dal 1998 al 2000, ha curato per il Teatro Alla Scala, assieme a Luigi Pestalozza,  l’ ideazione e la realizzazione del ciclo biennale di concerti Metafonie e un convegno internazionale sulla  Musica e la Tecnologia (1999). Ha realizzato conferenze in Italia, Olanda,Francia,Cuba,Spagna; le sue composizioni sono state eseguite in circa 20 paesi ( tra Europa, USA, Canada, Asia, Australia, America Latina) e all’interno di numerose edizioni della ICMC. Ha inciso per Fonit Cetra,Ricordi,Eshock Edts-Mosca,EMI-Italia,LIMEN,CEMAT. Negli ultimi anni è interessato ad approfondire la ricerca storico critica sulla musica elettronica europea e italiana degli anni 50/70. Dal 1990 è collaboratore scientifico della rivista Musica/Realtà.  E’ titolare della cattedra di composizione musicale elettroacustica presso il Conservatorio di Musica di Cosenza.}

\biografia{Gilles Gobeil}{Dopo gli studi di teoria musicale, Gilles Gobeil ha completato il Master in Composizione presso l'Université de Montréal. Dal 1985 si è concentrato sulla creazione di opere acusmatiche e miste. Gobeil ha ricevuto più di venti premi in Canada e a livello internazionale. Attualmente è professore di Tecnologia musicale a Drummondville CEGEP, ed è stato professore ospite di Elettroacustica presso l'Université de Montréal e presso il Conservatorio di Montréal.}

%2016-715:
\biografia{Arianna Granieri}{ Pianista, si diploma con il massimo dei voti e consegue con lode e menzione d’onore la Laurea di II livello in Pianoforte indirizzo Solistico presso il Conservatorio Santa Cecilia di Roma, sotto la guida del M° Cinzia Damiani. Ha conseguito con lode la Laurea magistrale in Filosofia presso l’Università di Roma Tor Vergata. Si è esibita sia in qualità di solista che in formazioni cameristiche presso vari festival ed eventi musicali tra i quali Concerti Accademici, Orvieto Festival of Strings, Domeniche Estive a Castel Sant’Angelo, Novantenario della nascita di Franco Evangelisti, inoltre ha eseguito in prima mondiale la riduzione per due pianoforti del Concerto per pianoforte e orchestra di Henry Cowell. E’ interessata all’unione della musica con le altre arti, in particolare il teatro.}

\clearpage
\flushleft
~\vfill

\biografia{Quartetto Guadagnini}{ Il Quartetto Guadagnini nasce nel 2012. L’affiatamento di quattro giovani musicisti provenienti da Ravenna, Pistoia, Roma e Bari porterà la giovane formazione a vincere, nel 2014 il premio Farulli, in seno al XXXIII Premio Franco Abbiati. Si qualifica attualmente tra le più promettenti formazioni cameristiche d’Europa, dedite al grande repertorio quartettistico e contemporaneo. Si è già esibito per le più importanti società concertistiche e sale da concerto italiane e in Francia, Austria, Germania, Svizzera e Cina. Nel 2015 ha suonato con la pianista Beatrice Rana all’Istituto italiano di cultura di Parigi, luogo che lo ha visto poi come ensemble in residenza. Nel 2016 è stato impegnato in una tournée nazionale promossa dal CIDIM, ha debuttato al Teatro La Pergola di Firenze e al Festival dei 2Mondi di Spoleto con musiche dedicate alla formazione stessa da Silvia Colasanti. E’ stato poi scelto dalla Fondazione Stauffer per rappresentare l’eccellenza italiana esibendosi a Shanghai in diversi concerti e alcune masterclass. Il Quartetto Guadagnini collabora con i compositori Silvia Colasanti, Domenico Turi, Paolo Cavallone e Raffaele Bellafronte registrando per Tactus diversi lavori. Vincitore di Concorsi Internazionali, sul fronte della formazione ha completato l’Accademia Stauffer con il Quartetto di Cremona e sta seguendo il maestro Hatto Beyerle, storico violista del Quartetto Alban Berg e loro grande estimatore.  Il quartetto è stato selezionato per il progetto Le dimore del Quartetto in collaborazione con l’Associazione Dimore Storiche Italiane. Si è esibito su RAI 5 accanto a Sandro Cappelletto, su RAI 3 accanto a Corrado Augias ed è ospite regolare di emittenti radiofoniche dedicate alla grande musica. Su RADIO 3 è stato invitato a Radio 3 Suite e alle Lezioni di Musica di Giovanni Bietti con il quale porta avanti una cospicua collaborazione.}

%\biografia{Quartetto Guadagnini}{Formatosi nel 2012, è attualmente una delle formazioni cameristiche più promettenti d’Europa, dedite al grande repertorio quartettistico classico e romantico, con particolare attenzione al repertorio del Novecento e alla musica del nostro tempo. L’affiatamento dei quattro giovani musicisti ha portato la formazione a vincere, nel 2014, il premio Piero Farulli, in seno al XXXIII Premio Franco Abbiati. Il Guadagnini si è già esibito nelle più importanti sale da concerto italiane, nel 2016 è stato impegnato in una tournée nazionale promossa dal CIDIM e, nello stesso anno, è stato scelto per rappresentare l’eccellenza italiana in Cina, esibendosi in diversi concerti a Shanghai}


%2016-716:
\biografia{Virginia Guidi}{ Si diploma in Canto Lirico e in Musica Vocale da Camera al Conservatorio S. Cecilia dove si specializza con lode con Silvia Schiavoni con una tesi sul rapporto tra interprete e compositore nella musica elettroacustica. Spazia dalla musica da camera a quella contemporanea con attenzione per la musica di sperimentazione. Ha collaborato con numerosi compositori eseguendo spesso pezzi a lei dedicati. Si è esibita in Italia e all’estero (Pechino, Washington DC) e ha partecipato ad importanti festival (EMUFest, Biennale di Venezia, ArteScienza) e ad installazioni di famosi artisti (Allora\&Caladilla, Thomas De Falco).}

\biografia{Sean Harold}{ (1984) è un compositore e  artista che lavora tra New York e il Connecticut. Una volta da bambino è riuscito a conquistare una base durante una partita di baseball, ma poi è diventato arrogante, ha riprovato ed è stato eliminato.}

\biografia{Pierre Jodlowski}{ Dopo gli studi presso il Conservatorio di Lione e l'IRCAM, Pierre Jodlowski fonda il collettivo l'éOle  e il Novelum Festival a Tolosa. Le sue attività di compositore lo hanno portato in molti luoghi della Francia e all'ester, per la nuova musica assieme alla danza, al teatro, le arti plastiche e la musica elettronica. Oltre al suo universo musicale Jodlowski lavora con immagini, programmazione interattiva e sceneggiatura. Asserisce la pratica di una musica "attiva" sia nella sua dimensione fisica (gesti, energie e spazi) che nella sua dimensione psicologica (evocazione, memoria e aspetto cinematografico). Ha ricevuto commissioni da IRCAM, l'Ensemble Intercontemporain, il Ministero della Cultura, il CIRM, il Festival di Donaueschingen, la Radio France e il Concorso Pianistico di Orléans.}

\biografia{Carlo Laurenzi}{Studia Chitarra privatamente con diversi insegnanti e poi Musica Elettronica presso il conservatorio dell’Aquila, con il M° Michelangelo Lupone, con il quale ha poi cominciato a lavorare attivamente in Italia e all’estero, completando sul campo la sua formazione come compositore, prima e dopo il diploma. Attualmente lavora come Computer Music Designer all’IRCAM di Parigi, dove collabora con compositori di fama, provenienti da molti paesi diversi, per la concezione e realizzazione dei loro progetti di musica mista (strumenti ed elettronica).}

%2016-719:
\biografia{Ivan Liuzzo}{ nasce a Frosinone il 26 febbraio 1993 ed inizia a studiare la batteria all’età di 9 anni con M. FIOCCO, successivamente con G.GUIDONI e A.BLASI. Nel 2007 intraprende gli studi delle percussioni presso il Conservatorio *L.Refice* di Frosinone con il M° C. DI BLASI, conseguendo il diploma (V.O.). Ha approfondito lo studio della batteria jazz con R. PISTOLESI e G. HUTCHINSON. Membro attivo e co-fondatore del Collettivo Phthorà, assieme a F. Ferazzoli e F. Abbate ha collaborato con artisti come: Lisa Mezzacappa, Stefano Costanzo, Vincenzo Core, Wound, Ron Grieco, Achille Succi}

\clearpage
\flushright
~\vfill

%\flushright
%~\vfill


%2016-721:
\biografia{Massimiliano Mascaro}{ compositore. Nato a Roma nel 1986. Allievo del M° Michelangelo Lupone e del M° Nicola Bernardini, si è formato presso il Conservatorio \emph{A. Casella} di L'Aquila e successivamente presso il Conservatorio \emph{S. Cecilia} di Roma affrontando gli studi della Composizione elettroacustica e della Composizione classica. La musica elettroacustica è il settore nel quale svolge la sua principale attività musicale.}

\biografia{Augusto Meijer}{ è un compositore di musica eletroacustica olandese. Ha conseguito master di secondo livello tra cui l’ “European Media Master of Arts” e successivamente alla “Utrecht School of the Arts”. Le sue composizioni sono state eseguite in varie sedi internazionali, tra le quali il “San Francisco Tape Music Festival”, il “New York City Electroacoustic Music Festival”, la “Linux Audio Conferences”,l’ “International Computer Music Conferences”, e molti altri.}

\biografia{Olga Neuwirth}{Olga Neuwirth è nata a Graz, in Austria, nel 1968. Ha studiato all'Academy of Music di Vienna e al Conservatorio di Musica di San Francisco. Durante il suo soggiorno negli Stati Uniti ha anche frequentato un college d'arte, dove ha studiato pittura e film. I suoi insegnanti privati ​​in composizione comprendono Adriana Hölszky, Tristan Murail e Luigi Nono. È apparsa per la prima volta sulla scena internazionale nel 1991, all'età di 22 anni, quando due delle sue mini-opera sono state eseguite al Wiener Festwochen. Le sue opere sono state presentate in tutto il mondo}

\biografia{Giorgio Nottoli}{ (compositore, nato a Cesena, Italia nel 1945) è stato docente di Musica Elettronica al Conservatorio di Roma "S.Cecilia" sino al 2013. Attualmente è docente di Composizione elettroacustica all’Università di Roma "Tor Vergata". La maggior parte delle sue opere utilizza mezzi elettronici sia per la sintesi che per l'elaborazione del suono. Il centro della sua ricerca di musicista riguarda il timbro concepito quale parametro principale e *unità costruttiva* delle sue opere attraverso la composizione della microstruttura del suono. Nei suoi lavori per strumenti ed elettronica Giorgio Nottoli punta ad estendere la sonorità degli strumenti acustici mediante complesse elaborazioni del suono. Ha progettato vari sistemi elettronici per la musica utilizzando sia tecnologie analogiche che digitali in collaborazione con varie università e centri di ricerca.}

\biografia{Levy Oliveira}{ (1993) è un compositore Brasiliano. Studia composizione presso l'università federale di Minas Gerais(UFMG), seguito da João Pedro Oliveira. Levy è interessato sia alla musica elettronica che a quella acustica. Le sue composizioni sono state eseguite in diversi Festival, tra i più recenti: Monaco Electroacoustique 2015 (Monaco/Monaco), MUSLAB 2015 (Mexico City/Mexico), JIMEC 2015 (Amiens/France), Open Circuit 2016 (Liverpool/UK) and Tinta Fresca 2016 (Belo Horizonte/Brazil). Il suo lavoro Hyperesthesia ha ricevuto il primo premio del Destellos Electronic Composition Competition ed è stato finalista dell'Open Circuit Composition Prize. Il suo brano orchestrale Um ato de fé ha ricevuto una menzione  preso il festival Tinta Fresca del 2016.}

%2016-723:
\biografia{Federico Paganelli}{ nato a Roma, studia musica elettronica presso il Conservatorio Santa Cecilia con i maestri Bernardini e Lupone. Precedentemente ha studiato con il Maestro Giorgio Nottoli.}

\biografia{Lorenzo Pagliei}{Ha studiato con S. Sciarrino, A. Corghi, I. Vandor e G. Nottoli. La sua attività si svolge su diversi livelli: composizione, creazione di nuovi strumenti elettroacustici, musica elettronica dal vivo, improvvisazione al pianoforte, direzione d’orchestra, collaborazione con coreografi e altri artisti. La sua musica si concentra sulla creazione di diversi tipi di tempo che non sono mai regolari ma curvi; il principio della curvatura si estende al dettaglio musicale che spesso è costituito da micro-variazioni di materiale semplice. Attualmente è compositore in ricerca e docente di musica elettronica all’Ircam dove effettua ricerche sul rilevamento del gesto e il controllo della sintesi sonora per modelli fisici in tempo reale. La sua musica è pubblicata da Casa Ricordi.}

\biografia{John Palmer}{Si è diplomato in pianoforte al Conservatorio di Lucerna dove ha studiato composizione nelle classi di Vinko Globokar e Edison Denisov.Ha continuato gli studi di composizione al Trinity College of Music, e poi alla City University dove ha ottenuto il Dottorato di Ricerca (PhD) in composizione acustica ed elettroacustica. Ha continuato gli studi in composizione con Jonathan Harvey e direzione d'orchestra con Alan Hazeldine alla \textit{Guidhall School of Music}. Sin dal 1985 la sua attività musicale è concentrata sulla composizione di musica orchestrale, strumentale, vocale e da camera; dal 1991 anche sulla musica elettroacustica.}

\clearpage
\flushleft
~\vfill

\biografia{Claudio Panariello}{Nato a Napoli nel 1989, ha studiato Composizione con G. Panariello e Musica Elettronica con A. Di Scipio e E. Martusicello. Attualmente è laureando al biennio di Musica Elettronica persso il Conservatorio S. Cecilia di Roma. È anche laureato in Fisica all'Università di Pisa.\\ La sua musica è stata eseguita dal Divertimento Ensemble, Ensemble SuonoGiallo, Ensemble Mise-En, Morphosis Ensemble, Ensemble Via Nova, etc., e selezionata in festival come il “Mise-En Music Festival” a New York, “Sound of Wander” a Milano, “Festival 5 Giornate” a Milano e “Risuonanze” a Udine.}

\biografia{Paolo Pastorino}{(Italia, 1983) è un compositore elettroacustico. Ha studiato musica elettronica presso il Conservatorio di Sassari sotto la guida del M° Walter Cianciusi (2015), successivamente si è specializzato in musica elettronica presso il conservatorio di Cagliari (2017). Attualmente si occupa di composizione elettroacustica. I suoi lavori sono stati eseguiti in Italia, Francia, Spagna, Germania, UK, Svizzera, USA, Messico, Argentina, Giappone.}

\biografia{Alessandro Perini}{ è nato a Cantù nel 1983. Ha studiato Composizione (con Luca Francesconi e Ivan Fedele tra gli altri), Musica Elettronica e Scienze della Comunicazione Musicale in Italia e Svezia. La sua produzione artistica spazia dalla musica elettronica e strumentale a lavori audiovisuali e basati sulle luci, alla net-art, all'arte ambientale e a dispositivi tattili. La documentazione dei suoi lavori si trova su www.alessandroperini.com }

%2016-724:
%\biografia{Danilo Perticaro}{ Nato a Cosenza nel 1992, intraprende giovanissimo lo studio del Sassofono, al Conservatorio della città, laureandosi sotto la guida di Luigi Grisolia con lode. Ha partecipato e vinto numerosi concorsi e manifestazioni internazionali. Ha seguito masterclasses e corsi di alto perfezionamento tenuti dai Maestri: Salime, Moretti, Delangle, Espinoza, Marzi, Filippetti, Mlekusch, Bornkamp. Collabora con orchestre e ensemble, soprattutto in ambito contemporaneo ed elettronico in un'intensa attività concertistica nazionale ed internazionale. Ha avviato un progetto di incisione di tutte le opere per sassofono di Stockhausen. Collabora con diversi compositori nella realizzazione di opere inedite a lui dedicate.}

\biografia{Giuseppe Pisano}{ attualmente studia musica elettronica al conservatorio di Napoli con il M° Elio Martusciello. Nel 2015 studia in Olanda Wave Field Synthesis e tecniche di spazializzazione con Wouter Snoei. Il suo lavoro fa uso di registrazioni e suoni concreti assieme a strumenti elettroacustici autocostruiti che utilizza anche in performance dal vivo, prevalentemente in contesti improvvisativi. E’ membro fondatore dell’OEOAS (Orchestra Elettroacustica Officina Arti Soniche)}

\biografia{Maurizio Pisati}{Nato a Milano nel 1959, è presente con propri lavori in festival d’Europa, Australia, USA, Giappone, America Latina. Sue composizioni sono state premiate in concorsi nazionali e internazionali (tra cui: Bucchi’83; Contilli’83; Rass. B. Brecht’85; Gaudeamus’86; ICONS’86; Petrassi’89), sono pubblicate da Casa-Ricordi, trasmesse da emittenti radiofoniche europee ed extraeuropee, sono incise su CD Ricordi-Fonit Cetra, Edipan, BMG, CavalliRecordsBamberg, Victor, Limen, ArsPublica, SiltaClassics e LArecords, etichetta indipendente da lui fondata nel 1997. ha compiuto gli studi musicali al Conservatorio di Milano, oltre che ai corsi estivi di Darmstadt e all’Accademia di Città di Castello, diplomandosi con il massimo dei voti in Composizione con S.Sciarrino, A. Guarnieri e G.Manzoni, e in seguito anche in Chitarra svolgendo attività concertistica in Europa dal1983 al1989 col gruppo Laboratorio Trio. Al Conservatorio di Bologna insegna di Composizione per la Musica Applicata, Elementi di Composizione per la Didattica, Invenzione \& Interpretazione, e nella stessa sede nel 2014 fonda CRS - Centro di ricerche musicali.}

\biografia{Elisa Prosperi}{- soprano. Allieva di M. Benvenuti, si diploma all'Ist.“Mascagni” di Livorno. Frequenta: London Masterclasses (RoyalCollege of Music) con R.Plowright; Voice Workshop (ILV- Fond.Cini Venezia) con B.Hannigan, T.Wishart e D.Moss; S.Marino Courses e Stockhausen Kurze Kürten)con N.Isherwood. Esegue repertorio del'900 e contemporaneo, anche con prime esecuzioni. Tra cui: Corpi da musica: vita e arte di Bussotti; TempoReale Festival; OA-5 atti teatrali sull'opera d'arte, C.A.N.T.O, CrashTroades, di G. Cauteruccio; In the midst of things, Voxnova Italia (Biennale Arte); Moving Out di R.Riccardi, IIC WashingtonDC. Nel 2013 di trasferisce a Leuven (Belgio) per uno stage al LUCA.}

\clearpage
\flushright
~\vfill

%\flushright
%~\vfill


\biografia{Diego Ratto}{nasce ad Alessandria il 14/01/1988. Musicista e compositore, nel 2017 si laurea in Composizione Elettroacustica (sotto la guida di Gustavo A. Delgado) presso il Conservatorio “A.Vivaldi” di Alessandria. Precedentemente, si laurea in Chitarra Jazz (sotto la guida di Pino Russo e Paolo Silvestri) e in Musicoterapia. Attualmente frequenta il primo anno del Master in Composizione Elettroacustica presso il KMH - Royal College of Music di Stoccolma. Ha studiato, tra gli altri, con Matteo Franceschini, Cesare Saldicco, Riccardo Piacentini e Antonio Galanti.}

\biografia{Jean-Claude Risset}{Compositore, pioniere della Computer Music internazionale. Ha studiato Fisica, Pianoforte e Composizione con  André Jolivet, a metà degli anni Sessanta si è negli USA per seguire l’amico Max Matthews, musicista anch’egli pioniere della Computer Music. Insieme hanno creato il software MUSIC IV per ricreare digitalmente il suono dei fiati. Nel 1969 pubblica il suo catalogo di suoni sintetizzati al computer. Sperimenta  l’uso di tecniche di sintesi come la Frequency Modulation (FM)  e la Waveshaping per scopi musicali. Importante anche il suo contributo per la formazione dell’IRCAM, che ha diretto insieme a Pierre Boulez dal 1975 al 1979. Ha avuto numerosi riconoscimenti e rivestito importanti ruoli scientifici, didattici e artistici. Ha scritto numerosi articoli e saggi artistico-scientifici. E’ stato “Directeur de recherche” dal 1985 al CNRS di Marsiglia dove ha continuato a lavorare in veste di “Directeur de recherche émérite” fino a qualche tempo prima della sua scomparsa.}

\biografia{Nicola Rodriguez}{Nicola Rodriguez nato a Buenos Aires, Argentina. È uno studente di composizione del conservatorio “Alberto Ginastera. Studia composizione con il maestro Jorge Sad.
La sua composizione “El diálogo entre los diálogos”, per soprano ed elettronica è stato eseguito la prima volta durante la quarta edizione del festival  “Nuevas músicas por la memoria” (2014) e ha partecipato al "New York City Electroacoustic Music Festival" (2015),supportato e dichiarato di interesse culturale dal ministero della cultura argentino. Sue diverse composizioni elettroacustiche hanno partecipato a festival internazionali quali  l’“Exnihilo festival” (Messico, aprile 2015), il “Soundscape Internacional Symposium” (Italia maggio 2015); l’ “Art \& Science Days”, (Francia giugno 2015); il festival  “Zeppelin 2015”, (Spagna ottobre 2015); il “Muslab”( Messico dicembre  2015 - 2016) e l’Ai - Maako festival (Cile 2016).}

\biografia{Demien Rudel Rey}{(Argentina, 1987) Si diploma in chitarra. Laurea in Composizione con Santiago Satero e ha conseguito il Master in Arti Combinate all’UNA (Argentina). Ha presenziato a seminari con Dhomont, Vaggione, Tutschku, etc. È iscritto al Master in Composizione al CNSMD di Lione con Martin Matalon. È stato coordinatore del Festival Bahía[in]sonora. È stato premiato/menzionato in occasione dei seguenti eventi; TRINAC, SADAIC, Destellos, FAUNA, IndieFEST, Konex Mozart Award, Martirano Award, Sagarik Award, CICEM, Métamorphoses, MA/IN, Prix Jolivet, Forum Wallis, Earplay, etc.}

\biografia{Simone Scarazza}{Compositore, musicista e tecnico audio. frequenta il corso di Musica Elettronica presso il Conservatorio dell’Aquila con i Maestri Michelangelo Lupone, Maria Cristina De Amicis e Agostino Di Scipio. Ha collaborato, in qualità di assistente tecnico, con il Centro Ricerche Musicali CRM di Roma per l’allestimento di spettacoli di musica elettroacustica. Ha studiato contrabbasso con Ares Tavolazzi (membro del gruppo progressive Area). Dal 1995 compone ed esegue con diverse realtà musicali brani di musica moderna in spettacoli live. Da sempre dedito alla ricerca sonora e amante di musica contemporanea, ha partecipato a diversi stages di produzione sonora e di musica elettronica e a vari festival, fra cui: Premio delle Arti (Avellino. 2017) classificandosi al secondo posto con il brano acusmatico "Dal Profondo", Festa della Musica (L’Aquila, 2015) con il brano acusmatico "Forme Spettrali", ArteScienza (Roma, 2015) con il brano acusmatico “Flussi”, ElettroAQustica (L’Aquila, 2015) con il Live- Electronics “Dialoghi”.}

\biografia{Simone Santi Gubini}{(1980, Roma) ha studiato composizione, organo e pianoforte al Conservatorio Francesco Morlacchi di Perugia (2002-2007) e storia e tecniche della musica eletronica con Giorgio Notoli all ́Università degli Studi di Roma “Tor Vergata” (2000-2003). Ha frequentato Masterclass con Luciano Berio, Brice Pauset, Hans Werner Henze, Wolfgang Rihm, Helmut Lachenmann, Beat Furrer e Tristan Murail (2004-2007). Nell ́inverno del 2006 Simone Sant Gubini ha preso parte, come Artst in Residence, in Villa Medici, Roma, scrivendo il suo primo lavoro (Abstrakte Natur Lebendig, 2006) in un nuovo lessico musicale da lui elaborato nel saggio “Composizione a Part Uguali” (Vienna, 2011). NeIla primavera del 2010 gli viene commissionato un vasto lavoro per ensemble, voci e amplifcazione (Als Oben, 2010-2011) da parte dell ́ente Vatcano Unio Apostolatus Catholici, con prima esecuzione assoluta all ́Auditorium}

\flushleft

\biografia{}{Parco della Musica (Marzo 2012). I suoi lavori sono pubblicat dalle edizioni Gallimard, Parigi e La Camera Verde, Roma. Atualmente sta scrivendo il secondo capitolo della sua azione scenica in 3D per ensemble amplifcato (»E«) ed un quarteto di percussionist e 27 percussioni amplifcate (Hyperklang), commissionato da Les Percussions de Strasbourg. Prima esecuzione assoluta Uluru, Australia, estate 2018. Simone Sant Gubini vive e lavora a Berlino.}

%2016-737:
\biografia{Saxatile\textsuperscript{4} [\textbf{m}odulable \textbf{s}axophone \textbf{e}nsemble]}{ è un progetto realizzato da un’idea di Enzo Filippetti nell’ambito del Conservatorio di Musica “S. Cecilia” specificatamente rivolto all’esecuzione della musica contemporanea, nei suoi molteplici aspetti, con un orientamento diretto alla ricerca e all’instaurazione di un rapporto dialettico con altri artisti e con i compositori. I componenti possono vantare una solida, variegata esperienza.}

%%2016-730:
%\biografia{Franco Sbacco}{ diplomato in Composizione, Direzione d'Orchestra, Musica Elettronica e Strumenti a percussione, rispettivamente con D. Guaccero, D. Paris, G. Nottoli e L. Torrebruno. Corso di perfezionamento nel 1979-80 in Composizione presso l'Accademia Nazionale di S. Cecilia a Roma con G. Petrassi e F. Donatoni. Nel 1989 vince una borsa di studio canadese per attività di ricerca sulla computer music presso la Simon Fraser University di Vancouver con il sistema PODX di B. Truax. Si è dedicato al teatro musicale, le sue composizioni, eseguite in Italia e all'estero nei principali festival di musica contemporanea. È inoltre stato premiato al 3° Concorso Internazionale  di Musica Elettroacustica di Varese e del 18° Concorso Internazionale di Musica Elettroacustica di Bourges. Ha inciso per BMG Ariola – Roma, per DOMANI MUSICA – Roma, che nel 1999 e nel 2008 gli ha dedicato due cd monografici e nel 2015 per ALBANY RECORDS -  New York. Insegna Armonia e Analisi presso il Conservatorio S. Cecilia di Roma.}

\biografia{Federico Scalas}{Studia contrabbasso e Musica Elettronica presso il Conservatorio “S. Cecilia” di Roma, diplomandosi sotto la guida di Giorgio Nottoli. Sue composizioni sono state eseguite in diversi concerti e in festival tra cui: “EMUFest”, “Monaco Electroacoustique”, “SOMA”, “Scatole Sonore”, Sala Uno Teatro, Accademia di San Luca, MAXXI. E’ docente di Elettroacustica al Conservatorio di Roma “S. Cecilia” e al Master SonicArts dell’Università di Roma Tor Vergata.}

\biografia{Alessadndro Solbiati}{Si è diplomato presso il Conservatorio di Milano in pianoforte con Eli Perrotta e in composizione con Sandro Gorli, dopo aver frequentato per due anni la Facoltà di Fisica. Contemporaneamente, ha studiato per quattro anni (1977-80) con Franco Donatoni all'Accademia Chigiana di Siena. Ha ricevuto commissioni dal Teatro alla Scala, dalla RAI, dal Ministero della Cultura francese, da Radio France, dall’Università di Parigi, dal Mozarteum, dal South Bank di Londra, dalla Fondazione Gulbenkian di Lisbona, dalla Biennale di Venezia, dal Festival Milano Musica, dal Teatro Comunale di Bologna, dalla Basilica di San Petronio per il VII Centenario della fondazione, dall'Orchestra Sinfonica G. Verdi di Milano.}

%2016-731:
\biografia{Arturo Tallini}{ Arturo Tallini ha suonato in tutta Europa, negli Stati Uniti, in Egitto, Algeria e Tunisia. È docente al Conservatorio di Santa Cecilia in Roma e tiene regolarmente masterclass nei conservatori e italiani e università straniere. Unanimemente considerato un riferimento per il repertorio contemporaneo, collabora con artisti ed enti di fama internazionale, come Magnus Andersson, Michiko Hirayama, il flautista Carlo Moren, il sassofonista Enzo Filippetti, e il gruppo di musica contemporanea Modus Novus di Madrid, il Coro dell’Accademia Nazionale di Santa Cecilia e il Teatro dell'Opera di Roma. È fondatore e coordinatore del Master Annuale di II Livello in Interpretazione della Musica Contemporanea del Conservatorio di Santa Cecilia in cui è anche docente di Chitarra.}

\biografia{Mauro Tedesco}{ Allievo del M°Daniele Roccato presso il conservatorio S.Cecilia di Roma. Membro dell'ensamble di contrabbassi Ludus Gravis (www.ludusgravis.com) dedito nel repertorio contemporaneo il quale ha collaborato con compositori di fama internazionale come Terry Riley, Sofia Gubaidulina , Julio Estrada etc. Si è esibito con artisti di nota importanza come Dominique Pifarely e Michele Rabbia presso l'Auditorium parco della musica di Roma durante il festival "Una striscia di terra feconda" organizzato da Paolo Damiani e Armand Meignan; con Luca Sanzò presso l'Accademia Filarmonica Romana nel festival "Paesaggi Sonori" organizzato da Michelangelo Lupone e Giorgio Nottoli; Ha recentemente fatto parte del progetto "Assedio, frammenti di un reportage" organizzato da Lucia Goracci, reporter di Rainews24 presso la Siria, che vede coinvolta anche l'Istituzione Sinfonica Abruzzese con musiche di Tonino Battista e Daniele Roccato presso l'Auditorium del parco di L' Aquila e il Teatro dei Rinnovati a Siena.}

\biografia{Clements von Reusner}{ è un compositore e soundartist residente in Germania, il cui lavoro si concentra sulla musica acusmatica. Le sue composizioni hanno ottenuto trasmissioni ed esecuzioni a livello internazionale.}

\biografia{Manuel Zurria}{ Sono nato a Catania nel 1962. Pur avendo tentato altre esperienze in giro per il mondo, sono sempre ritornato a vivere a Roma, mio malgrado. Ho partecipato a prime assolute di Pennisi, Bussotti, Clementi, Guarnieri, Donatoni, Vacchi e Francesconi. Collaboro da molti anni con Sciarrino e Lucier. Ho partecipato a Festival internazionali, in tutto il mondo. La mia discografia conta attualmente 40 tra CDs e vinili per etichette quali Ricordi, Capstone, EdiPan, Stradivarius, Die Schachtel, Mazagran, Mode, Megadisc, God, Atopos, Touch, Another Timbre.}

