% !TEX encoding = UTF-8 Unicode
% !TEX TS-program = XeLaTex
% !TEX root = ../EMU2017_booklet.tex

\begin{flushright}

\large{
	\scshape{
	26 ottobre 2016 -- ore 9:30 - 13:00
	}}

\medskip
	
\small{Masterclass
	\newline Aula Bianchini}

\medskip

{\fontsize{20}{20} \svolk{\emph{Il linguaggio Faust\\per la composizione musicale}}}

\normalfont

\normalsize

\bigskip

Masterclass tenuta da \textsc{Yann Orlarey}\\{\footnotesize direttore scientifico del GRAME cncm di Lione}


\bigskip

Faust è un linguaggio di programmazione appositamente progettato per descrivere processi di sintesi e di elaborazione del suono. Faust può essere utilizzato per progettare strumenti musicali elettronici su una vasta gamma di piattaforme, dai pedali di effetti alle applicazioni audio per il web o smartphone. L'obiettivo dell'intervento è di presentare Faust e il suo ecosistema in modo semplice e accessibile.

\bigskip

\small{Evento ArteScienza2017 realizzati in collaborazione\\con il CRM - Centro Ricerche Musicale}

\vfill


~\vfill
\large{
	\scshape{
	26 ottobre 2017 -- ore 15:00 - 18:00
	}}

\medskip
	
\small{Conferenza
	\newline Aula Bianchini}

\medskip

{\fontsize{18}{18} \svolk{\emph{Shock e Ambiguità musicale\\tecniche di un suono liberato}}}

\normalfont

\normalsize

\bigskip

Conferenza tenuta da \textsc{Simone Santi Gubini}

\bigskip

Il compositore presenterà gli elementi costitutivi della sua estetica musicale. L’indagine sarà rivolta al suono e alle forma, con esemplificazioni tratte dalle sue opere e in un susseguirsi di concetti che dichiarano la necessità di una musica “dinamica, ipertrofica, di espressione immediata, formulata in una dialettica strettissima di opposte articolazioni…”

\end{flushright}

\vfill