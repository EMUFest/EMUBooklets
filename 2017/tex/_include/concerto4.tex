% !TEX encoding = UTF-8 Unicode
% !TEX TS-program = XeLaTex
% !TEX root = ../EMU2017_booklet.tex

\begin{flushright}

\large{
	\scshape{
	28 ottobre 2017 -- ore 10:00 - 13:30
	}}

\medskip
	
\small{Masterclass
	\newline Aula Bianchini}

\medskip

{\fontsize{20}{20} \svolk{\emph{HIDDEN.\\Composing sound and space of sound II}}}

\normalfont

\normalsize

\bigskip

Masterclass tenuta da \textsc{Carlo Laurenzi}\\{\footnotesize Computer music specialist presso l'IRCAM di Parigi}


\bigskip

Sarà presentato il lavoro di ricerca, svolto presso IRCAM di Parigi, che si incentra sulla composizione e l’elaborazione dei suoni elettronici e strumentali nello spazio d’ascolto. La percezione spaziale del suono, intesa come parametro costruttivo musicale, al pari dei parametri di altezza, durata e intensità, sarà esaminata in funzione delle caratteristiche timbriche e delle tecnologie che ne permettono il trattamento compositivo.

\bigskip

\small{Evento \textit{ArteScienza2017} realizzato in collaborazione\\con il CRM - Centro Ricerche Musicale}

~\vfill

%\clearpage


\large{
	\scshape{
	28 ottobre 2017 -- ore 20:30
	}}

\medskip
	
\small{Concerto
	\newline Sala Accademica}

\medskip

{\fontsize{20}{20} \svolk{\emph{Concerto IV}}}

\medskip

{\fontsize{40}{40} \svolk{\emph{HIDDEN}}}

\normalsize

\medskip

regia del suono \textsc{Pasquale Citera}

\bigskip

\livel{Chaya Czernowin}{HIDDEN *}{50’00}{per quartetto d'archi e live electronics}{2014}
\medskip


{\footnotesize \textbf{*} prima esecuzione italiana}

\bigskip

\textsc{Quartetto Guadagnini:}\\
\smallskip
\esecutore{violino}{Fabrizio Zoffoli , Cristina Papini} 
\esecutore{viola}{Matteo Rocchi}
\esecutore{violoncello}{Alessandra Cefaliello}

\medskip

\esecutore{live electronics}{Carlo Laurenzi}

\bigskip

\small{Evento \textit{ArteScienza2017} realizzato in collaborazione\\con il CRM - Centro Ricerche Musicale}

\bigskip


\vfill

\descrizione{HIDDEN}{ È un tentativo di arrivare a ciò che è nascosto al di sotto dell’espressività o della musica. La composizione è un tentativo di spingersi fino al punto in si ha solo una presenza appena udibile, al limite della percezione umana. Non conosciamo questa presenza e potrebbe risultarci estranea, indecifrabile. HIDDEN e un’esperienza di 45 minuti in cui con una lenta evoluzione l’orecchio viene trasformato in occhio. All’orecchio è dato il tempo e il modo di di osservare e orientarsi nel imprevedibile ambientazione sonora. È un paesaggio subacqueo, formato da rocce sommerse, abitato da vibrazioni gravi che sono percepite più che ascoltate circondate da strati e strati di nebbia che si dissipano. Blocchi monolitici di ‘rocce  sonore’ sono osservati/ascoltati da diverse prospettive. La composizione si basa sull’osservazione; prova a rintracciare e percepire l’emergere dell’espressività.}


\end{flushright}