% !TEX encoding = UTF-8 Unicode
% !TEX TS-program = XeLaTex
% !TEX root = ../EMU2017_booklet.tex

\begin{flushleft}

\large{
	\scshape{
	26 ottobre 2016 -- ore 18:00
	}}

\medskip
	
\small{Concerto Acusmatico
	\newline Il Suono di Piero [Aula Bianchini]}

\medskip

{\fontsize{42}{42} \svolk{\emph{Volontà IV}}}

\normalsize

\medskip

regia del suono \textsc{Federico Paganelli}

\bigskip

\livel{Wilfried Jentzsch}{Entre ciel et terre (between heaven and earth)}{13’45}{}{2015}
\medskip

\livel{Benjamin O’Brien}{OSCines}{6'08}{}{2013}
\medskip

\livel{Clelia Patrono}{Blue4Notes}{8'27}{}{2016}
\medskip

\livel{Daniele Pozzi}{Breakpoint}{6'24}{}{2016}
\medskip

%\brano{Christian Eloy}
%{La cicatrice d'Ulysse}{13'00''}
%{acusmatico}
%new version 2015\\

%\acusmatici{Ursula Meyer-K\"onig}
%{Allears}{2012-13}{8'}

%\vspace{6mm}

\vfill

\descrizione{Entre ciel et terre (between heaven and earth)}{\emph{L’intervallo tra Paradiso e Terra è come un grande tubo: vuoto, ma non crolla: si muove,  generando sempre più} (Daodejing, Part 1 (5), Wikipedia). Il verso è stato la fonte spirituale di questa composizione elettroacustica. Il paradiso, la terra nello spazio circostante hanno prodotto la concezione di un suono circolare multidimensionale. I movimenti spaziali del suono sono caratterizzati da varie configurazioni con il variare della velocità, della direzione e della distanza dall’ascoltatore. Il materiale del suono è basato su tre elementi: i cimbali cinesi, il canto degli uccelli e il canto medievale (Machaut). Questi suoni derivano da diverse culture, epoche e anche da diverse nature. Il suono è stato sintetizzato usando vari metodi di cross synthesis. Con l’aiuto di questi metodi evoluti di sviluppo digitale del suono del computer, si è in grado di produrre nuove qualità di suono. Questa composizione è stata premiata il 23 Marzo 2016 all’Espace Senghor Brussel ed è dedicata a Annette van de Gorne.}

\descrizione{OSCines}{\emph{OSCines} si basa sul processo di traduzione di melodie da canti di uccelli. l'usignolo appartiene alla famiglia dei \emph{Passeri}, anche conosciuti come Oscine, dal latino \emph{oscen} (uccello canterino). Il suo canto è composto da una vasta gamma di fischiettii, trilli e gorgoglii, i quali creano un profilo melodico ricco e movimentato. I campioni dell' usignolo e del clarinetto fungono - alternativamente - da sorgente e da obbiettivo sonoro per informazioni spettrali elaborate da un sistema di acquisizione del segnale processato. \emph{OSCines} esplora gli allineamenti e le collisioni di precise caratteristiche timbriche e topologie melodiche in una voliera virtuale di altoparlanti nello spazio stereofonico.}

\descrizione{Blue4Notes}{Brano composto in Quattro movimenti. I suoni utilizzati sono suoni concreti e suoni prodotti da una chitarra suonata con E-BOW(Electronic Bow). Tutti i suoni sono stati trattati e rielaborati con l’utilizzo di sintesi granulare, equalizzatori e filtri di risonanza.}



%\bigskip

%\svolk{\emph{Ho cercato di incanalare quell’energia in un percorso che la rendesse percepibile senza snaturarne l’essenza: l’energia che abita i violini di Corelli, Tartini, Vivaldi, certo non citazioni ne dirette ne indirette ma l’imprevedibile vitalità delle loro articolazioni (tremoli, arpeggi, ribattuti, sincronie e fioriture improvvise) che hanno fatto della scuola italiana, elettrica ante litteram, un irraggiungibile modello di virtuosismo strumentale; poi la tensione, il vuoto attorno e dentro alla costruzione delle frasi, le imitazioni, i pedali, la continua sovrapposizione delle corde.}}
%
%Giorgio Netti
%
%\normalfont

\end{flushleft}

\clearpage

\begin{flushright}

~\vfill

\descrizione{Breakpoint}{Interruzione intenzionale. Smembramento, segmentazione: esitazioni compositive si combinano a procedure concatenative emergenti, sovrapposte e contrapposte in un incessante processo astratto di sgretolamento e riassemblamento che sempre viene interrotto prima di raggiungere il contorno di una costruzione. Singoli atomi sonori appaiono in fuggevoli ed effimeri cumuli musicali, esponendo brevemente la propria struttura e tensione morfologica, tirandosi e spingendosi l’un l’altro in dolorose torsioni plastiche. Storti e piegati vibrano in rigidi mucchi nervosi, incastrati in un gioco di forze che solo li conduce ad accartocciarsi su sé stessi, sgualciti e stropicciati.}

***

\biografia{Wilfried Jentzsch}{Nato nel 1941, ha studiato composizione a Dresda, Berlino e Colonia. Dal '76 al '81 ha studiato alla Sorbona di Parigi con Xenakis dove gli è stato conferito il dottorato; fa ricerca della sintesi digitale del suono presso l'IRCAM e il CEMAMu. Dopo aver fondato uno studio di computer-music a Norimberga, è stato direttore dello Studio Elettronico presso la Musikhochschule di Dresda dal '93 al '06. Premi internazionali nelle città di Boswil, Parigi, Bourges e ZKM.}

\biografia{Benjamin O’Brien}{compone, ricerca, ed esegue musica acustica e elettroacustica che si concentra su questioni di trasformazione e l'ascolto delle macchine. Ha conseguito un dottorato in musica presso l'Università della Florida, un MA in composizione musicale al Mills College, e una laurea in Matematica presso l'Università della Virginia. La sua opera è pubblicata dalla Oxford University Press, Taukay Edizioni Musicali, Canadian Electroacoustic Community, e Seamus. Vive a Marsiglia, in Francia.}

\biografia{Clelia Patrono}{musicista e compositrice diplomata nel Marzo 2016 presso il Conservatorio di Musica \emph{L. Refice} di Frosinone in Discipline Musicali e Musica Elettronica. Lavori selezionati: \emph{Tension and Relaese} (2013) risonorizzazione del film \emph{Rhythmus21} di Hans Richter selezionato ICMC 2015 International Computer Music Conference, University of North Texas, Denton, NYCEMF 2016 New York City Electroacoustic Music Festival. \emph{Blue4Notes} (2016) NYCEMF 2016 New York City Electroacustic Music Festival(NY).}

\biografia{Daniele Pozzi}{(Padova, 1990) ha studiato Musica Elettronica al Conservatorio di Padova con Giorgio Klauer e Nicola Bernardini, diplomandosi nel 2015 col massimo dei voti (BA). Frequenta attualmente il master in Computer music all’Institute für Elektronische Musik und Akustik (Graz) sotto la guida di Gerhard Eckel e Marko Ciciliani. All’interno della sua personale ricerca artistica sviluppa originali ambienti informatici ed interfacce fisiche reattive destinati alla performance ed alla composizione. Suoi principali interessi nel campo dell’informatica musicale includono temi come l’interazione uomo-macchina, l’intelligenza artificiale, i sistemi dinamici e l’uso creativo di tecniche di Music Information Retrieval. È attivo inoltre come performer, specialmente a Graz (Mumuth, Forum Stadtpark, MUWA, Cube, Volkshaus etc.): qui si esibisce nell’esecuzione di opere personali, utilizzando soprattutto strumenti elettroacustici da lui stesso progettati e realizzati. Tra i suoi lavori figurano principalmente composizioni acusmatiche o per strumento ed elettronica, installazioni, performance elettroacustiche, alcune delle quali sono state presentate in festival di musica ed arte contemporanea come: Impulse Minute Konzert (Graz 2016), Contrasti2016 (Trento), Soundcraft2015 (Treviso), XX CIM ed VII EMUFest (Roma 2014), distanze (Salerno 2014), PulsArt2014 (Vicenza), Living Lab VI (Padova).}

\end{flushright}
