% !TEX encoding = UTF-8 Unicode
% !TEX TS-program = XeLaTex
% !TEX root = ../EMU2017_booklet.tex

\begin{flushright}

\large{
	\scshape{
	27 ottobre 2017 -- ore 10:00 - 13:30
	}}

\medskip
	
\small{Masterclass
	\newline Aula Bianchini}

\medskip

{\fontsize{20}{20} \svolk{\emph{HIDDEN.\\Composing sound and space of sound I}}}

\normalfont

\normalsize

\bigskip

Masterclass tenuta da \textsc{Chaya Czernowin}\\{\footnotesize Docente di composizione presso all'Università di Harvard}


\bigskip

Il lavoro di Chaya Czernowin, compositrice di fama mondiale si caratterizza per l’uso della metafora come mezzo privilegiato per raggiungere un mondo sonoro non familiare, una musica per il subconscio, che si estenda oltre le convenzioni stilistiche o la razionalità. La composizione della tessitura sonora, l’attenzione alla fluidità o alla rugosità del timbro fino al confine con il rumore, la dilatazione del tempo, l’elaborazione e la spazializzazione del suono, sono tecniche sapientemente utilizzate per la creazione di una vitale, viscerale e diretta esperienza musicale. Nel corso della masterclass sarà descritta anche la realizzazione dell’opera HIDDEN, per quartetto d’archi ed elettronica, prodotta nel 2014 con Carlo Laurenzi presso l’IRCAM.

\bigskip

\small{Evento \textit{ArteScienza2017} realizzato in collaborazione\\con il CRM - Centro Ricerche Musicale}

\vfill

\large{
	\scshape{
	27 ottobre 2017 -- ore 18
	}}

\medskip

\small{Concerto Acusmatico
	\newline Il Suono di Piero [Aula Bianchini]}

\medskip


{\fontsize{20}{20} \svolk{\emph{Concerto Acusmatico IV}}}

\normalsize

\medskip

regia del suono \textsc{Edoardo Bellucci}

\bigskip


\livel{Antonio Carvallopinto}{Kustéata}{8'55}{}{2017}
\medskip

\livel{Simone Scarazza}{Integrale}{7'02}{}{2017}
\medskip

\livel{John Palmer}{Present Otherness}{10'51}{}{2017}
\medskip

\livel{Augusto Meijer}{Matera}{14'58}{}{2016}
\medskip

\livel{Paolo Pastorino}{Matérica}{4'33}{}{2017}
\medskip

\vfill


\descrizione{Kustéata}{Attraverso il linguaggio Yámana, ormai quasi del tutto estinto, la composizione si introduce all'interno delle possibilità espressive dei suoni vocali che perdono il loro connotato di significato sintattico. Il pezzo si presenta in una scena di emergenza di nuovi significati, dai suoni della singola parola e dall'elaborazione elettroacustica della stessa. La composizione è interamente sviluppata a partire da elaborazioni di suoni vocali.}

\descrizione{Integrale}{ Basato sulla sintesi granulare, Integrale è un brano in cui si esplorano le varie possibili implementazioni utilizzabili da questa tecnica. La composizione del brano avviene tramite la raccolta di microforme algoritmiche, strutturate si, per la generazione della sintesi granulare, ma attuate per avere comportamenti estremi, anomalie, singolarità, e aberrazioni implementative. La ricerca formale per la composizione, si è sviluppata su una domanda essenziale: cosa può succedere, in un irrisorio istante sonoro? In omaggio agli studi di Dennis Gabor e alla sua visione scientifica corpuscolare il brano è stato intitolato Integrale, ovvero: un elemento che fa parte di un tutto, che concorre alla costituzione di un intero.}


\end{flushright}

\clearpage

\begin{flushleft}


\descrizione{Present Otherness}{Nel 2007 Jonathan Harvey mi ha inviato a scrivere un pezzo acusmatico a partire da campioni di tromba suonati da Markus Stockhausen. Ho deciso di smembrare completamente le frasi originali e di manipolare le caratteristiche spettrali di ogni singolo suono per poter creare nuovi suoni e ricostruire una nuova musica a partire da zero. Ho voluto investigare la dicotomia tra una percezione \textit{presente} della realtà e un' \textit{altra} più sfuggente forma di percezione estesa legata ad una dimensione metafisica della vita.}

\descrizione{Matera}{è il risultato di una sperimentazione sonica la quale si concentra su nuovi tipi di approcci nella creazione di suoni elettronici. Questo pezzo è stato sviluppato attraverso l'utilizzo di una mistura di tecniche precedentemente utilizzate per altre composizioni. Nuove sperimentazioni nella creazione sonora hanno portato ad un florido mondo di strutture sonore, le quali spesso rimandano ad estetiche di tipo naturale e cosmico. Il pezzo è stato eseguito in prima assoluta all' \textit{Ecos Urbanos Festival}.}

\descrizione{Matérica}{ è la terza composizione che fa parte di uno studio sulla musica concreta e sull’organicità del suono. L’idea al principio di questa composizione è stata quella di creare delle connessioni timbriche e temporali tra i diversi oggetti sonori utilizzati. Così, elementi sonori provenienti da ambienti e contesti differenti coesistono, si intrecciano e interagiscono tra loro generando delle forme vive e reattive capaci di muoversi in uno spazio immaginario.}


\end{flushleft}

\vfill

