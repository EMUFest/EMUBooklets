% !TEX encoding = UTF-8 Unicode
% !TEX TS-program = XeLaTex
% !TEX root = ../EMU2017_booklet.tex

\begin{flushleft}

\large{
	\scshape{
	25 ottobre 2017 -- ore 10:00 -- 13:30
	}}

\medskip
	
\small{Masterclass
	\newline Aula Bianchini}

\medskip

{\fontsize{18}{18} \svolk{\emph{Jean-Claude Risset. Metafore e archetipi del linguaggio elettroacustico}}}

\normalfont

\normalsize

\bigskip

Masterclass tenuta da \textsc{Luigi Pizzaleo}\\{\footnotesize Docente presso Conservatorio Santa Cecilia di Roma}


\bigskip

Saranno esaminati gli aspetti più rilevanti della poetica di \textsc{Jean-Claude Risset }ed in particolare le connessioni profonde tra la musica e la scienza che il compositore ha esplorato. La Masterclass, articolata in ascolti e analisi, ospita il sassofonista \textsc{Enzo Filippetti} che eseguirà  il brano \textbf{\textit{Voilements}}.

\bigskip

\livel{Jean-Claude Risset}{Voilements}{14'}{per sassofono e nastro magnetico}{1987}
\medskip


\descrizione{Voilements}{Dedicato al sassofonista Daniel Kientzy che ha sviluppato le tecniche strumentali estese usate nella composizione. Il solista dialoga con l’elettronica su supporto. L’elettronica inizialmente fa da eco al solista, moltiplicandone il suono, ma alterando anche il modo di suonare, lo deforma, come  una ruota che non gira (la parola \textit{Voilements} allude ad un velo o ad una vela, ma significa anche  \textit{fibbia} o \textit{deformazione}).Il temperamento equabile viene eroso, la tensione cresce fino a un punto in cui le linee melodiche vengono trasformate in loop, quindi , come se ci fosse uno zoom all’indietro, in un tranquillo sfondo per i gesti del solista che usa varie tecniche di esecuzione. L’elettronica è stata realizzata digitalmente a Marsiglia (Faculté des Sciences de Luminy et Laboratoire de Mecanique et d'Acoustique du CNRS). I suoni registrati Daniel Kientzy sono stati trattati con il processore audio SYTER realizzato all’INA-GRM da Jean-Francois Allouis. Il nastro include anche suoni sintetizzati in tempo differito attraverso il software MusiC V, implementato su un micro-computer IBM-PC: le altezze di questi suoni, comunque, sono state impostate in tempo reale con una tastiera MIDI; il codice MIDI è stato trascritto in codice Music V. Queste tecnologie sono state sviluppate da Daniel Arfib, Frédéric Boyer, Pierre Dutilleux, Richard Kronland and Patrick Sanchez.}


\vfill

\end{flushleft}

\clearpage

\begin{flushright}

\large{
	\scshape{
	25 ottobre 2017 -- ore 18:00
	}}

\medskip
	
\small{Concerto Acusmatico
	\newline Il Suono di Piero [Aula Bianchini]}

\medskip


{\fontsize{20}{20} \svolk{\emph{Concerto Acusmatico III}}}

\normalsize

\medskip

regia del suono \textsc{Massimiliano Mascaro}

\bigskip

\livel{Lo\"{i}se Bulot}{Yami}{11'28}{}{2017}
\medskip

\livel{Sean Harold}{there is no image … there is no poetry (version II)}{6'58}{}{2014}
\medskip

\livel{Clements von Reusner}{Definierte Lastbedingung}{11'40}{}{2016}
\medskip

\livel{Nicola Rodriguez}{El viento será eterno}{8'30}{}{2017}
\medskip

\livel{Gilles Gobeil}{Sous l'écorce des pierres-promenade}{14'21}{}{2016-17}
\medskip

\vfill

\descrizione{Yami}{Ho composto questo brano ispira da un paesaggio sonoro, acqua e stelle. Con movimenti ascendenti e discendenti la composizione si sviluppa in due sezioni: una prima in cui ho assemblato frammenti sonori - da gocce d’acqua al suono della corrente- e una seconda in cui vengono evocati i riflessi del sole e della luna.}

\descrizione{there is no image … there is no poetry (version II)}{Originariamente scritto per sax soprano solo e nastro. Questa versione è per sola elettronica rifuggendo completamente l’elemento di iterazione tipico della versione dal vivo. Entrambe le versioni sono basate sul quartetto per Oboe K370 di Mozart. In questo brano gli elementi fondamentali del brano della composizione di Mozart sono usati con una nuova funzione e il quartetto originale è semi-dimenticato e ricordato solo da echi lontani.}


\descrizione{Definierte Lastbedingung}{(ingl. defined load condition) La composizione è basata sui suoni di campi elettromagnetici che crescono utilizzando apparecchi elettronici. Sono state realizzate registrazioni di paesaggi elettromagnetici presso l’\textit{Institute for Electrical Machines, Traction and Drives} (IMAB) dell’università tecnica di Braunschweig (Germania) con un apposito microfono. Questo materiale sonoro ha ben poco di ciò che caratterizza un suono \textit{musicale}: non ha né profondità né evoluzione organica.Questi suoni, nel loro rumore, sono statici anche se mossi internamente. Solitamente suonano aspri, disgustosi e spessi quasi ermetici  come un normale ronzio elettrico.\textit{Defined load condition}o ( un termine tecnico utilizzato nei test di macchinari elettrici) si basa su questi suoni che vengono esplorati nella loro struttura, rimodellati  e drammatizzati musicalmente attraverso gli strumenti di uno studio di musica elettronica. La frequenza della corrente elettrica in europa è di 50 Hz  quindi il 50 e i suoi multipli sono la base numerica su cui, in vari modi, si basa il brano. La composizione spazializzata ad 8 canali con un sistema ambisonic di terzo ordine. \textit{Definierte Lastbedingung} è il contributo tedesco al World New Music Days  che si terrà nel novembre 2017 a Vancouver.}

\descrizione{El viento será eterno}{Lavoro basato su registrazioni di suoni che ho scoperto durante un viaggio nell’Argentina settentrionale. Tra le diverse sonorità di questa regione le qualità sonore del vento mi hanno portato ha sviluppare una poetica in cui questo elemento non viene utilizzato solo come materiale sonoro ma anche come struttura formale della composizione. La composizione integra piccoli suoni corpuscolari e lo scorrere del vento rovente.}

\descrizione{Sous l'écorce des pierres-promenade}{ Dedicato a Folkmar Hein. Diversi dei suoni usati in questo brano mi sono stati procurati da Folkmar Hein che ha commissionato la composizione. La maggior parte di questi suoni, da lui registrati, possono essere considerati, in una certa misura, rumori privi di un significato preciso. Ciò che mi ha affascinato è stato il ragionare sulla possibilità di poterli integrare in un progetto più ampio che potesse evocare le lunghe passeggiate che il Sig. Heine fa regolarmente nella natura intorno Berlino. Mi ha raccontato che durante queste escursioni a un certo punto non solo la percezione del tempo ma anche quella del suono si trasformano. Forse la transizione a questo stato può facilitare l’esposizione a una sorta di dimensione onirica nascosta all’interno delle rocce.
Sous l’écorce des Pierres – promenade ha ricevuto una menzione d’onore al concorso internazionale Prix CIME 2017.}


\end{flushright}

\vfill

