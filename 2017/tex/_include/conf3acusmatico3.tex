% !TEX encoding = UTF-8 Unicode
% !TEX TS-program = XeLaTex
% !TEX root = ../EMU2017_booklet.tex

\begin{flushright}

\large{
	\scshape{
	25 ottobre 2017 -- ore 10:00 -- 13:30
	}}

\medskip
	
\small{Masterclass
	\newline Aula Bianchini}

\medskip

{\fontsize{18}{18} \svolk{\emph{Jean-Claude Risset. Metafore e archetipi del linguaggio elettroacustico}}}

\normalfont

\normalsize

\bigskip

Masterclass tenuta da \textsc{Luigi Pizzaleo}\\{\footnotesize Docente presso Conservatorio Santa Cecilia di Roma}


\bigskip

Saranno esaminati gli aspetti più rilevanti della poetica di \textsc{Jean-Claude Risset }ed in particolare le connessioni profonde tra la musica e la scienza che il compositore ha esplorato. La Masterclass, articolata in ascolti e analisi, ospita il sassofonista \textsc{Enzo Filippetti} che eseguirà  il brano \textbf{\textit{Voilements}}.

\bigskip

\livel{Jean-Claude Risset}{Voilements}{14'}{per sassofono e nastro magnetico}{1987}
\medskip


\descrizione{Voilements}{is dedicated to saxophonist Daniel Kientzy, who developed special performance technique used in the piece. The soloist dialogues with a tape. The tape first echoes the soloist, multiplying his sound, but it also alters its way of playing, it warps it, as a wheel which does not go round (the word \textit{Voilements} alludes to a veil or a sail, but is also means \textit{buckles} or \textit{warps}). The equal temperament tuning is eroded, the tension increases, up to a point where melodic lines get twisted into loops. Then, as if there were a zoom backwards, the pacific background for the gestures of the soloist, who uses various performance techniques. The tape was realized digitally in Marseille (Faculté des Sciences de Luminy et Laboratoire de Mecanique et d'Acoustique du CNRS). Sounds recorded by Daniel Kientzy have been transformed with the SYTER audioprocessor designed at INA-GRM by Jean-Francois Allouis. The tape also includes sounds synthesized in non real-time with the MUSIC V program, implemented on a IBM-PC compatible micro-computer: these sounds, however, have been specified in real time on MIDI keyboard; the MIDI code has been transcribed into MUSIC V code. These possibilities have been developed by Daniel Arfib, Frédéric Boyer, Pierre Dutilleux, Richard Kronland and Patrick Sanchez.}


\vfill

\end{flushright}

\clearpage

\begin{flushleft}

\large{
	\scshape{
	25 ottobre 2017 -- ore 18:00
	}}

\medskip
	
\small{Concerto Acusmatico
	\newline Il Suono di Piero [Aula Bianchini]}

\medskip


{\fontsize{20}{20} \svolk{\emph{Concerto Acusmatico III}}}

\normalsize

\medskip

regia del suono \textsc{Massimiliano Mascaro}

\bigskip

\livel{Lo\"{i}se Bulot}{Yami}{11'28}{}{2017}
\medskip

\livel{Sean Harold}{there is no image … there is no poetry (version II)}{6'58}{}{2014}
\medskip

\livel{Clements von Reusner}{Definierte Lastbedingung}{11'40}{}{2016}
\medskip

\livel{Nicola Rodriguez}{El viento será eterno}{8'30}{}{2017}
\medskip

\livel{Gilles Gobeil}{Sous l'écorce des pierres-promenade}{14'21}{}{2016-17}
\medskip

\vfill

\descrizione{Yami}{I composed this piece on the evocation of a nocturnal landscape, water and stars. On ascending and descending wave movements, the piece develops into a first part where I have assembled fragments - from the drop of water to the current - and in a second part evoking the reflections, from lunar light to sunlight.}

\descrizione{there is no image … there is no poetry (version II)}{was originally written for solo soprano saxophone with fixed media. This version of the work is for fixed media alone, eschewing the live element of the iteration version altogether. Both versions of the piece are based on Mozart’s Oboe Quartet, K. 370. Here, however, the supporting elements of Mozart’s quartet are remembered with new intent, while the driving force of the original quartet is half-forgotten and remembered only in echoes.}


\descrizione{Definierte Lastbedingung}{engl. defined load condition) is based upon the sounds of electromagnetic fields as they arise when using electric devices. Numerous recordings of electromagnetic landscapes were made at the \textit{Institute for Electrical Machines, Traction and Drives} (IMAB) of Technical University of Braunschweig (Germany) with a special microphone. This sound material has little of what a \textit{musical} sound is intrinsically. There is no depth and no momentum. In their noisiness these sounds are static, though moved inside. They usually seem bulky, harsh and repellent, even hermetic as the well known electrical hum. \textit{Defined load condition} (a technical term when testing electrical machines) is about with these sounds which are explored in their structure, reshaped and musically dramatized by the means of the electronic studio. The main frequency of electrical current in Europe is 50 hertz and hence 50 and its multiples is also the numerical key this composition is based upon in a variety of ways. spatialization: ambisonic 3rd order - 8 channel. \textit{Definierte Lastbedingung} is also this years German contribution to World New Music Days taking place november 2017 in Vancouver, Canada.}

\descrizione{El viento será eterno}{This work is built through the sounds that I discovered on a trip to the North of Argentina. Among the many sonorities of this region, the wind sound qualities led me to think of a poetic where the wind not only be used as sound material but as a formal part of the piece. This work integrates small particles and flow of the scorching wind.}

\descrizione{Sous l'écorce des pierres-promenade}{ To Folkmar Hein. Many of the sounds used in this piece were provided to me by Folkmar Hein, who commissioned the work. The majority of these sounds, which he himself recorded, can to some extent be considered as noises devoid of any precise meaning. What intrigued me was the question of whether it would be possible to integrate them into a broader project that could evoke the very long nature hikes that Mr. Hein regularly embarks upon in the surroundings of Berlin. He confided to me that after a certain time during these promenades, one’s perception of not only time but also of listening is gradually transformed. Perhaps the transition into this state might facilitate the exposure of a particular reverie concealed beneath the skin of the stones... Sous l’écorce des pierres – promenade received an Honorable Mention at the international competition Prix CIME 2017.}


\end{flushleft}

\vfill

