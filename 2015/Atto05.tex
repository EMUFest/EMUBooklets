% !TEX encoding = UTF-8 Unicode
% !TEX TS-program = XeLaTex
% !TEX root = EMU2015_booklet.tex

\livel{Luciano Berio}
{Sequenza III}{9'} 
{per voce femminile spazializzata}
{1965}

\livel{Cathy Berberian}
{Stripsody}{11'30''} %
{per voce sola spazializzata}
{1966}

\descrizione{Sequenza III}{La voce porta sempre con sé un eccesso di connotazioni. Dal rumore più insolente al canto più squisito, la voce significa sempre qualcosa, rimanda sempre ad altro da sé e crea una gamma molto vasta di associazioni. In Sequenza III ho cercato di assimilare musicalmente molti aspetti della vocalità quotidiana, anche quelli triviali, senza però per questo rinunciare ad alcuni aspetti intermedi ed al canto vero e proprio. Per controllare un insieme così vasto di comportamenti vocali era necessario frantumare il testo e in apparenza devastarlo, per poterne recuperare i frammenti su diversi piani espressivi e ricomporli in unità non più discorsive ma musicali. Era cioè necessario rendere il testo omogeneo e disponibile al progetto che consiste, nelle sue linee essenziali, nell’esorcizzare l’eccesso di connotazioni componendole in un’unità musicale. Ecco il breve testo «modulare» di Markus Kutter per Sequenza III: \\ «Give me	a few words for a woman \\ to sing a truth allowing us \\ to build a house without worrying before night comes» \\ In Sequenza III l’enfasi è posta sul simbolismo sonoro di gesti vocali e talvolta visivi, sulle “ombre di significato” che li accompagnano, sulle associazioni e sui conflitti che essi suggeriscono. Per questa ragione Sequenza III può anche essere considerata come un saggio di drammaturgia musicale la cui storia, in un certo senso, è il rapporto fra l’interprete e la sua stessa voce. Sequenza III è stata scritta nel 1965 per Cathy Berberian. \\ Luciano Berio}

% \descrizione{Sequenza III}{(author's note) The voice carries always an excess of connotations, whatever it is doing. From the grossest of noises to the most delicate of singing, the voice always means something, always refers beyond itself and creates a huge range of associations. In Sequenza III I tried to assimilate many aspects of everyday vocal life, including trivial ones, without losing intermediate levels or indeed normal singing. In order to control such a wide range of vocal behaviour, I felt I had to break up the text in an apparently devastating way, so as to be able to recuperate fragments from it on different expressive planes, and to reshape them into units that were not discursive but musical. The text had to be homogeneous, in order to lend itself to a project that consisted essentially of exorcising the excessive connotations and composing them into musical units. This is the “modular” text written by Markus Kutter for Sequenza III. \\ Give me	a few words	for a woman \\ to sing	a truth allowing us \\ to build a house	without worrying \\ before night comes \\ In Sequenza III the emphasis is given to the sound symbolism of vocal and sometimes visual gestures, with their accompanying “shadows of meaning”, and the associations and conflicts suggested by them. For this reason Sequenza III can also be considered as a dramatic essay whose story, so to speak, is the relationship between the soloist and her own voice. Sequenza III was written in 1965 for Cathy Berberian. Luciano Berio}

\descrizione{Stripsody}{«Cathy Berberian era alla ricerca di un testo per una delle sue performance musicali e contemporaneamente stava sviluppando una sorta di mondo sonoro usando solo le onomatopee codificate dalla lingua dei fumetti. Gradualmente si convinse che queste azioni musicali non avevano bisogno della musica scritta; così, mentre Cathy incominciava a \emph{cantare} questi suoni, Carmi procedeva a \emph{scrivere} la partitura. I due aspetti del lavoro sono nati insieme, e la voce di Cathy ha dato più di un suggerimento grafico mentre l'impaginazione di Carmi ha fornito più di una soluzione vocale». \\ Umberto Eco}

%\descrizione{Stripsody}{“Berberian was looking for a text for one of her musical performances, and thought of developing a kind of sound world using only the onomatopoeic inventions of the comic strips. Gradually the idea grew that this musical action had no need of music; thus, while Cathy began to sing these sounds, Carmi went on to write the score. The two aspects of the work were born together, and Cathy’s voice contributed more than one graphic suggestion while Carmi’s imagination produced more than one vocal solution.” E. Carmi, Stripsody – Interpretazione vocale di Cathy Berberian. Testo introduttivo di Umberto Eco, Arco d’Alibert – Kiko Galleries, Roma – Houston (Texas) 1966.}