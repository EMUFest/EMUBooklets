% !TEX encoding = UTF-8 Unicode
% !TEX TS-program = XeLaTex
% !TEX root = EMU2015_booklet.tex

{\fontsize{30}{30} \svolk{\emph{Passare all'atto}}}

% \svolk{\emph{Passer à l’acte)}

\begin{quote}
\begin{it}
	\svolk{Il programma EMUfest 2015 ricava dalle suggestioni del titolo il criterio
di selezione delle opere e  intende offrire al pubblico una panoramica
delle attuali esperienze compositive: quelle che nascono dalla ricerca,
dalla scoperta, dall’invenzione.

Nessuna pratica artistica è fine solo a se stessa e anche la musica,
astratta e immateriale, porta in se questa responsabilità. L’idea è ciò che
il compositore incarna nella musica ma è pure la conseguenza di una
interpretazione della realtà, una visione dei valori o delle derive della
nostra civiltà.

L’elemento essenziale che permette ad un’opera d’arte di compiersi e di
trasmettere i suoi contenuti  è la correlazione tra la necessità interiore
dell’artista e la scelta del modo di esprimerla. \emph{Passer à l’acte} è
dunque il gesto che definisce la pratica artistica ma, allo stesso tempo e
in modo indiretto, è anche l’esortazione per noi fruitori, a cogliere gli
stimoli offerti dall’opera e a rendere attiva la nostra riflessione.}

\hfill \emph{Michelangelo Lupone}

\end{it}
\end{quote}