% !TEX encoding = UTF-8 Unicode
% !TEX TS-program = XeLaTex
% !TEX root = EMU2015_booklet.tex

\noindent \textit{\textbf{TETRAREC}}

\smallskip

\noindent \emph{TETRAREC} è una tecnica microfonica spaziata che circonda il complesso dell'oggetto sonoro stumento-strumentista con quattro microfoni ai vertici di un tetraedro.

\noindent  \emph{TETRAREC} è ispirata alla tecnica \emph{A-Format} di  Michael Gerzon
(quattro microfoni coincidenti posizionati sulle facce di un tetraedro). Le differenze tra le due tecniche riguardano la distanza (coincidenti vs. spaziate) e lo scopo (riproduzione tridimensionale del complesso sonoro vs.  riproduzione di un oggetto acustico).

\noindent La tecnica \emph{TETRAREC} è stata sviluppata per i seguenti scopi:

\begin{enumerate}
\item Registrare la \emph{Forma Sonora} di strumenti acustici
\item Preservare il mascheramento causato dal corpo nella propagazione del suono
\item Preservare i movimenti del musicista durante l'esecuzione
\item Acquisire dati sulle \emph{Forme Sonore} di diversi strumenti acustici
\item Acquisire dati sulle \emph{Forme Sonore} di simili strumenti acustici suonati da diversi interpreti
\item Analizzare i dati e progettare un sistema di visualizzazione tridimensionale delle \emph{Forme Sonore}
\end{enumerate}

\vspace{1cm}

\noindent \textit{\textbf{Prossimi progetti}}

\smallskip

\noindent Il \emph{corpus} più esteso di musica contemporanea per strumento solo è rappresentata dalle \emph{Sequenze} di Luciano Berio. 
In questa ricerca sulla registrazione delle \emph{Forme Sonore} con la tecnica TETRAREC e sulla diffusione attraverso S.T.ONE, dove le esecuzioni sono descritte e mappate allo scopo di affinare le tecniche di diffusione, una \emph{Sequenza} può focalizzare l'attenzione dell'ascoltatore su ogni sfumatura della \emph{Forma Sonora}. Al tempo stesso riprodurre un sistema complesso di oggetti come un musicista che esegue una \emph{Sequenza} (che molto spesso implica  una serie complessa di azioni  fisiche) è l'obbiettivo di ricerca più grande per l'altoparlante.

\vfill
\noindent \textit{\textbf{acoustic S.T.ONE - Track list}}

\begin{itemize}
\item \emph{Density 21.5} - \textsc{Edgar Var\`ese}  - flauto  Elena D’Al\`o
\item \emph{improvvisazione } - bayan  - Alessandro Sbordoni
\item \emph{Sequenza IXb} -  \textsc{Luciano Berio} -  sassofono contralto  Danilo Perticaro
\item \emph{Suite for toy piano} - \textsc{John Cage} - toy piano Francesco Ziello
\end{itemize}
