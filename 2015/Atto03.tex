% !TEX encoding = UTF-8 Unicode
% !TEX TS-program = XeLaTex
% !TEX root = EMU2015_booklet.tex

\acusmatici{Jan Jacob Hofmann}
{Horizontal and Vertical Lines}{2005}{7'53''}

\acusmatici{Clemens Von Reusner}
{Topos Concrete}{2014}{9'18''}

\acusmatici{Damián Anache}
{Capturas del Único Camino}{2014-15}{14'29''}

\acusmatici{Jan Jacob Hofmann}
{Tensile Elements}{2000}{9'30''}

% \descrizione{Horizontal and Vertical Lines}{Two kinds of elements mark the context of the piece: Sounds of extremely long duration in the distance, shimmering and sharp like steel, stretched out in the timeline. Generated by patterns of chaotic oscillation and being non-linear, these horizontal elements refer to infinity. Their vertical organisation, the proportion of pitch among each other is organised by the harmonic ratio though. Much closer, several impulses: Sounds of infinite shortness, containing the whole spectra of frequencies in a mathematical sense, but having no extension in time. Being strictly vertical, they contradict with the long stretched horizontal elements. These two sonic elements, considered as horizontal and vertical, set the background structure of which matter is generated. The sound for this composition is derived from a non-linear algorithm for sound generation by physical modelling, as from granular synthesis. All the sounds have been created using Csound. The work was commissioned by the CRM for Musica Scienza 2005 where it got executed in June 2005.}

\descrizione{Horizontal and Vertical Lines}{Due elementi determinano la base del pezzo: suoni di eccezionale lunghezza dalla distanza, scintillante e tagliente come l'acciaio. Si estendevano lungo l'asse del tempo. Generati dal modello di oscillazione caotica e non-linearità, questi elementi orizzontali mostrano all'infinito. La loro organizzazione verticale, cioè il loro rapporto fra loro è pero disposte secondo le leggi dell'armonia. Molto più vicino: differenti suoni che si assomigliano le impulsi: suoni di durata infinitamente breve, tuttavia, contengono l'intero spettro di frequenze in senso matematico, ma non hanno estensione nel tempo. Strettamente allineati verticalmente, in contrasto con la lunga tesa elementi orizzontali. Entrambi questi elementi sonori, la forma orizzontale e verticale, creano la struttura basale di quella si sviluppano le forme. I suoni di questa composizione sono generati da un algoritmo non-lineare, così come la sintesi granulare. I suoni sono stati generati con il programma Csound. Il brano "Linee Orizzontale e Verticale" è stata composto come contributo a "Musica e Scienza" nel 2005, per conto di CRM, Roma.}

% \descrizione{Topos Concrete}{Concrete is a building material, a kind of unshaped dry powder made of sand, granulated stones and cement, dusty and chaotic. Mixed with water it becomes flexible and fluid and goes into a metamorphosis to become dry again, static and resistable and of any wanted shape. Aspects of working with native granularity, fluidness as well as stiffness and different kind of acoustic spaces were leading ideas of the composition.}

\descrizione{Topos Concrete}{Il calcestruzzo ("Concrete") è un materiale da costruzione, una sorta di polvere secca informe costituita di sabbia, pietre e cemento granulati, polveroso e caotico. Impastato con l'acqua diventa flessibile e fluido, quindi torna ad essere di nuovo asciutto, statico e resistente e di qualsivoglia forma. La composizione del brano è guidata da alcuni aspetti quali il lavorare con la granularità e la fluidità così come con la rigidità e con diversi tipi di spazi acustici.}

% \descrizione{Capturas del Único Camino}{"First Landscape" is a movement of “Capturas del Único Camino”, a piece which involves generative art ideas for offering an attractive object of passive contemplation. It is Ambisonics encoded and made with acoustic instruments samples recorded and performed by the composer. Then the samples are handled by a Pure Data patch according to the random events score of the piece. So that, the algorithm works as an electronic performer of the piece. info: http://conceptocero.com/capturasdelunicocamino}

\descrizione{Capturas del Único Camino}{"First Landscape" è un movimento di "Capturas del Único Camino", un brano che impiega le idee dell'arte generativa, per offrire un interessante oggetto di contemplazione passiva. Il brano è codificato Ambisonics e realizzato con campioni di strumenti acustici registrati e suonati dal compositore. I campioni sono gestiti da una patch di Pure Data, seguendo una partitura di eventi randomici. In questo modo l'algoritmo funziona come se fosse l'esecutore elettronico del pezzo. info: http://conceptocero.com/capturasdelunicocamino}

% \descrizione{Tensile Elements}{Subject of \textit{Tensile Elements} are special kinds of elements, characterised by their quality of sound, their movement and interaction. Elements pass through the space, defining it, having almost materiality but still being bodiless. Concentrations and disintegrations happen, the elements seem to have almost their own way of behaviour.}

\descrizione{Tensile Elements}{Tema del soggetto sono elementi che si caratterizzano per il loro determinato suono teso, così come per i loro movimenti e le loro interazioni. Tali elementi si muovono attraverso lo spazio, lo definiscono ed hanno quasi una materialità, nonostante ciò rimangono senza corpo. Addensamenti e dissolvenze si succedono, gli elementi sembrano avere una vita propria.}


%Sonic Architecture 

%Jan Jacob Hofmann's work is spatial electronic music, or, as he calls it,  “sonic architecture” – architecture made by sounds. Using the Ambisonic-Method makes it possible to assign a precise Position  in 3D space to every sound.  The sonic material of the composition generates a sonic space surrounding the listener. Sounding virtual materials do generate complex architectures made of sound, spaces in transformation do emerge.
%
%This is made possible by using a combination of the Ambisonic Method in combination with Csound, a computer-language for sound synthesis and digital signal processing. Jan Jacob Hofmann conceived and created a set of instruments that allow to assign a position in 3d space to every sound. Also the sound is placed in a sonic environment allowing distance perception by the creation of spatial reverb and early reflections, also using the ambisonic method.
%
%Embedded in an environment for composition, this code makes it possible to compose works of spatial music. This code is available open source on his site www.sonicarchitecture.de and is already used by other composers.
%
%Latest development is the extension of the spatial assignment of sounds to granular clouds so that spatial granular synthesis became possible allowing swarms and clouds of sound in space.







