CONCERTO 2
Concerto Trovalusci/Ruggeri 20.30 (Sala Accademica)

\livel{Simone Cardini}
{Potlach}{2014 - vincitore Premio Bucchi} %9'30''
{per flauto basso ed elettronica su supporto}
{flauto basso}{Gianni Trovalusci}

\descrizione{Potlach}{was a ritual ceremony performed among certain Native American tribes, during which the members of the community exchanged and squandered gifts, contributing to the cohesiveness of the social concord, whithin a totally inverted logic with respect to our free economy, following a reciprocity principle. The formal juxtaposition of the piece, disperses at once those elements that will constitute later on the concealed foundation that penetratrates the work.}

\descrizione{Potlach}{Il Potlach era una cerimonia rituale d’alcuni popoli di Nativi Americani, durante la quale i membri della tribù scambiavano e dilapidavano doni, in una logica completamente invertita rispetto alla nostra economia. La giustapposizione del brano \textit{dilapida} da subito quegli elementi che costituiranno il substrato che pervade l'opera. È la simbolizzazione dell'immagine che ne permette un'esegesi palesata; l'interprete realizza sé attraverso la sublimazione di gesti e momenti performativi, rappresentando la fenomenologia degli stessi. Da qui la necessità di superare il momento poietico ed estesico \textit{donando} la performance per un recupero del sistema di relazioni come tale.}


\livel{Dominique Schafer}
{Cendre}{2008} %10'00''
{per flauto basso e live electronics}
{flauto basso}{Gianni Trovalusci}

\descrizione{Cendre}{draws on the idea from the impermanence of materiality. The title of the piece gives further meaning referring to a fragile and delicate state, but also the potential of ashes as a fertilizer for something new. The bass flute being suspended within the electronics extends space and timbre, at times detaching itself within its own identity, but repeatedly gets reabsorbed into space. The piece was written for and premiered by Mario Caroli during his Fromm Residency at Harvard University.}

\descrizione{Cendre}{prende ispirazione dall’idea della precarietà di ciò che è materiale. Il titolo del brano acquista ulteriore significato riferendosi ad uno stato di fragilità e delicatezza, ma anche alle potenzialità fertilizzanti della cenere per qualcosa di nuovo. Grazie alla sospensione del flauto basso attraverso l’elettronica, il suo spazio ed il suo timbro vengono estesi, a tratti allontanandosi dalla propria identità per poi essere riassorbito all’interno dello spazio. Il brano è stato composto ed eseguito in prima assoluta da Mario Caroli durante la Fromm Residency presso la Harvard University.}


\livel{Maria Cristina De Amicis}
{[re-spì-ro]}{2015 - prima esecuzione assoluta} %8'00''
{per flauti e supporto digitale}
{flauti}{Gianni Trovalusci}

\descrizione{[re-spì-ro]}{Re-spi-ro (Breath) in this work, is the alternation of air movements which could produce real rhythmic and tonal structures. The sound of each breath adds and interacts itself with the previous one, drawing a music form which contracts and expands itself as well. Through the rhythmic articulation, inhalation and exhalation, the breath transforms and builds itself, removes and contradicts what has been just expressed. The digital support creates a constructive dialectic, supporting gently the concrete material. Breath transformation is described by the conduct of three different changing elements: localization, movement (direction and speed) and space size. In this way , the auditive perception of the space is transposed to the visual perception level. In this work I tried to realize the interaction between these changing elements on the acoustic level and associating with tones and pitches the breath expressive gestures of the performer.}

\descrizione{[re-spì-ro]}{Il re-spi-ro, in questa opera, è inteso come l’alternanza dei movimenti dell’aria in grado di generare vere e proprie strutture ritmiche e timbriche. Il suono di ogni respiro si aggiunge e interagisce con i precedenti, disegnando una forma musicale che si contrae e si espande.  Attraverso l’articolazione ritmica, l’inspirazione e l’espirazione, il respiro si trasforma e costruisce, dissolve o contraddice ciò che è stato appena espresso. Il supporto digitale crea una dialettica costruttiva sostenendo delicatamente il materiale concreto. La trasformazione del respiro è descritta attraverso il comportamento di tre variabili: localizzazione, movimento (direzione, velocità) e dimensione dell’ambiente. In questo modo la percezione uditiva dello spazio viene trasposta nel dominio della percezione visiva. In questo brano ho cercato di realizzare l’interazione tra queste variabili nel dominio acustico associando con timbri e altezze i gesti espressivi del respiro dell’esecutore.}


\acusmatico{Christian Eloy}
{La cicatrice d'Ulysse}{new version 2015} %13'00''
{acusmatico}

\descrizione{La cicatrice d'Ulysse}{This title, borrowed from Erich Auerbach, German writer and critic deceased in 1957, immediately sets the scene on a plane where realism is depicted in the aesthetics of Western music. Electroacoustic music has the ability, using “concrete” sounds (with all the ambiguity implied by the word), of giving us an immediate sensation of reality, which we can all situate in relation to our references and private objects of reference; these are the profound notions of sublimitas and humilitas, which merge and unite in musical expression. Every listener will be able to decipher his or her own images of a collective mental universe from the essence of this kind of artistic creation, just like the scar by which Ulysses was recognised by his former servant woman.}

\descrizione{La cicatrice d'Ulysse}{Il titolo, preso in prestito da Erich Auerbach, scrittore e critico tedesco morto nel 1957, ambienta immediatamente la scena all’interno di un aereo, tratteggiandola con un realismo tipico dell’estetica musicale occidentale. Attraverso l’uso di suoni concreti (con tutte le ambiguità insite nella parola stessa), la musica elettroacustica possiede l’abilità di restituirci una percezione immediata della realtà, che è possibile collocare in relazione ai propri riferimenti e agli oggetti segreti cui intende riferirsi; questi sono i significati profondi di sublimitas e humilitas, che si fondono e riuniscono nell’espressione musicale. Dall’essenza di questo tipo di creazione artistica ogni ascoltatore è in grado di decodificare la propria immagine di un universo mentale collettivo, proprio come la cicatrice che permise alla serva di riconoscere Ulisse.}


\livel{Giorgio Nottoli}
{7 Isole}{2015 - prima esecuzione assoluta}%15'
{per flauto, percussioni e live electronics}
{flauto}{Gianni Trovalusci}
percussioni -- \textsc{Gianluca Ruggeri}
\\

\descrizione{7 Isole}{I think the idea of the island is a fascinating metaphor, because I realize to think about many important things  as they are formed, in fact, by islands, which each contain a different world and a different feel, but together form a unitary context. In "7 Isole", each "island" is characterized by a particular combination of movement, pitches and colors of the sound. It consists of seven small pieces separated from one another, that can be performed in any order, however, strongly linked by way of forming the sound material and by a unitary constructive thinking. The instruments are used both with extended techniques that traditional, allowing different nuances both for the continuous sound as well as impulsive, both for what determined pitch and for what similar to the noise. The electronics supports and extends the sound of the instruments and, where necessary, becomes the instrument itself, completing the construction of the sound field. The dynamic distribution  in the listening space is obtained according to the principles of the method Ambisonic. Some of the processing and sound synthesis used have been developed by the author.}

\descrizione{7 Isole}{Quella dell'isola è per me una metafora affascinante, in quanto mi accorgo di pensare a molte cose importanti come fossero costituite, appunto, da isole, che contengono ciascuna un mondo diverso e quindi un diverso sentire, ma, insieme, costituiscono un contesto unitario. In 7Isole, ciascuna "isola" è caratterizzata da una particolare combinazione di movimento, altezze e colori del suono. Si tratta di sette piccoli pezzi fra loro separati, che si possono eseguire in un qualsiasi ordine, tuttavia fortemente legati dal modo di formare il materiale sonoro e da un pensiero costruttivo unitario. Gli strumenti sono utilizzati sia con tecniche estese che tradizionali, consentendo diverse sfumature sia per il suono continuo che per quello impulsivo, sia per quello ad altezza determinata e per quello simile al rumore. L'elettronica sostiene ed  estende il suono degli strumenti e, dove necessario, diviene strumento essa stessa, completando la costruzione del campo sonoro. La distribuzione dinamica nello spazio d'ascolto è ottenuta secondo i principi  del metodo Ambisonic. Alcune delle elaborazioni e sintesi del suono utilizzate sono state sviluppate dall'autore.}








