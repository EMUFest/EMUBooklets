% !TEX encoding = UTF-8 Unicode
% !TEX TS-program = XeLaTex
% !TEX root = EMU2015_booklet.tex

\biografia{James Andean}{Musicista e sound artist. È attivo sia come compositore che esecutore in vari campi, incluse composizioni elettroacustiche e performance, improvvisazioni, installazioni audio, e registrazioni. È membro fondatore del quartetto di improvvisazione e nuova musica Rank Ensemble, e fa parte del duo Plucié/DesAndes (audiovideo). Si è esibito per l'Europa e Nord America, e i suoi lavori sono stati presentati in tutto il mondo.}

\biografia{Christian Banasik}{(1963) ha studiato composizione alla Robert Schumann Academy of Music and Media di Dusseldorf e alla University of Music and Performing Arts di Francoforte. I suoi lavori strumentali ed elettronici sono stati eseguiti in concerti e programmi radio in tutta Europa, ma anche in America, Asia e Australia. È docente di Audio Visual Design presso la University for Applied Sciences e direttore artistico di Computer Music Studio "Studio 209" di Dusseldorf.}

\biografia{Francesco Bianco}{Laureato in Musicologia, frequenta il corso di Musica elettronica presso il Conservatorio di Roma Santa Cecilia. È stato visiting scholar al CRR de Boulogne-Billancourt (Parigi).  Musicista da sempre interessato alle profonde relazioni fra l'arte e la vita, ha esperienze musicali variegate di genere e stile, dalla composizione alla performance dal vivo, dall'azione scenica alla colonna sonora.}

\biografia{Giovanni Costantini}{(Corigliano d’Otranto - Lecce, 1965) Dal 1995 svolge attività di ricerca presso l'Università di Roma "Tor Vergata", dove è docente di Musica Elettronica. È direttore del Master in SONIC ARTS. Sue composizioni elettroacustiche sono state eseguite in numerosi concerti in Italia e all’estero e incise da Twilight Music (Roma) e IAEF (New York). La sua ricerca musicale è rivolta alla realizzazione della microstruttura e della macrostruttura del suono attraverso l’esplorazione e l’elaborazione in tempo reale di materiale acustico.}

\biografia{Elena D'Alò}{, flautista e ottavinista si laurea cum laude al biennio in Flauto, dopo un brillante diploma, presso il Conservatorio "Santa Cecilia" di Roma, con Deborah Kruzansky. Ha affiancato gli studi musicali con quelli scientifici, laureandosi in Fisica acustica presso "La Sapienza" con Paolo Camiz. Attualmente è iscritta al triennio di Musica Elettronica a Roma. Si esibisce in formazioni cameristiche e orchestrali, in un repertorio che va dal barocco al contemporaneo, per il quale ha suonato a festival come Nuova Consonanza, Atlante Sonoro XX secolo, ArteScienza ed EMUFest. Studia violoncello con Maurizio Massarelli.}

\biografia{Domenico De Simone}{Diplomato in Pianoforte, Jazz, Composizione e Musica Elettronica. Ha conseguito il diploma del corso di perfezionamento di Composizione presso l’Accademia Nazionale di Santa Cecilia e, con il massimo dei voti e la lode, il diploma accademico di II livello in Musica Elettronica. Sue composizioni sono state eseguite in Italia e all’estero (Cina, Lettonia, Canada, Cile, Argentina, Romania, Malta, ecc.) e trasmesse da Radio3.}

\biografia{Sara Ferrandino}{si è diplomata in pianoforte nel 2005 presso il Conservatorio di Perugia nella classe del Mº Tanganelli, conseguendo nel 2009, con votazione di 110 e Lode, la Laurea per il Biennio Specialistico. Nel 2012 ha ottenuto il diploma del Corso di Perfezionamento tenuto dal Mº Perticaroli, presso l’Accademia Nazionale di Santa Cecilia in Roma. Ha partecipato a numerosi concorsi nazionali e internazionali ottenendo sempre piazzamenti nelle prime posizioni. Si è esibita in molteplici concerti solistici e cameristici in prestigiose sale in Italia e all’estero. Collabora presso il Conservatorio di Perugia con le classi di corno, tromba, flauto, oboe e violino. È docente di pianoforte principale per i corsi pre-accademici presso il Conservatorio Santa Cecilia in Roma.}

\biografia{FREI}{FREI è un progetto di Paolo Gatti e Francesco Bianco, nato nel 2014. Si sono esibiti al Circolo Dal Verme (Studiolo Laps Showcase), al teatro Tor Bella Monaca (Slaps-pourri.1 anteprima). L'improvvisazione è alla base della poetica di Frei. La performance live è basata su elementi preordinati, i quali, durante lo spettacolo, vengono elaborati e sviluppati. La strumentazione è costituita da due laptop sui quali sono vi sono sistemi digitali programmati degli stessi componenti del duo.}

\biografia{Jorge García del Valle Méndez}{(1966) è cresciuto in spagna, dove ha studiato fagotto e composizione. Ora vive a Dresda (Germania) dove ha studiato composizione e musica elettronica. Le sue composizioni hanno avuto prime mondiali in tutto il mondo, commissioni da importanti istituzioni internazionali. Lavori di analisi digitale e "sound processing", applicati alla teoria e alla composizione. Premio di composizione Salvatore Martirano dellUniversità dell'Illinois, e premio di composizione Sächsischer Musikrat.}

\biografia{Paolo Gatti}{Laureato in ingegneria, consegue il master in ingegneria del suono presso l'università di Roma "Tor Vergata". Successivamente si laurea a pieni voti in musica elettronica presso il Conservatorio Santa Cecilia di Roma, sotto la guida di G.Nottoli,M.Lupone,N.Bernardini.Compositore, didatta e ricercatore, suoi lavori sono eseguiti in importanti manifestazioni e festival internazionali. Scrive musiche per spettacoli teatrali e rassegne poetiche. Nel 2015, la sua composizione Poltergeist, risulta fra i brani premiati al termine della finale nazionale del premio delle arti "Claudio Abbado".}

\biografia{Marco Marinoni}{ nasce a Monza nel 1974. Nel 2007 si diploma con il massimo dei voti e la lode in Musica Elettronica con Alvise Vidolin. Nel 2009 consegue il Diploma Accademico Sperimentale di Secondo Livello in Live Electronics e Regia del Suono con 110 e Lode e nel 2013 il Diploma Accademico Sperimentale di Secondo Livello in Composizione con 110 e Lode. Dal 1999 è attivo come compositore in ambito contemporaneo. Prix du Trivium nel 29e Concours International de Musique et d'Art Sonore Electroacoustiques - Bourges 2002. Finalista dell'International Gaudeamus Composition Prize 2002 e 2003. Vincitore della Seconda Call per Opere Elettroacustiche indetta dalla Federazione CEMAT. Primo Premio nel Primo Concorso di Composizione per Iperviolino - Genova 2007. Primo Premio nel VIII Concorso Internazionale di Composizione “Città di Udine”. Come musicologo partecipa ai convegni indetti dall'AIMI e dal GATM. È membro del SIMC - Società Italiana Musica Contemporanea. Le partiture dei suoi brani sono edite da ARSPUBLICA EDIZIONI MUSICALI e da TAUKAY. Nel 2015 esce il suo primo romanzo, La Confraternita di Ecate - Cauda Draconis (ed. Nerocromo). È professore di Esecuzione e Interpretazione della Musica Elettroacustica e coordinatore del Dipartimento di Musica Elettronica presso il Conservatorio "G. Verdi" di Como. Vive a Finale Ligure.}

\biografia{Mario Mary}{dottore in "Estetica, Scienza e Tecnologia delle Arti" (Università di Parigi VIII, Francia), Professore di Composizione di Musica Elettroacustica presso Academia Ranieri III di Monte-Carlo, e Direttore artistico di Monaco Electroacoustique - Incontri Internazionali di Musica Elettroacustica. Ha lavorato come ricercatore presso l'IRCAM e insegnato all'Università Parigi VIII, Ha vinto una ventina di premi in concorsi di composizione. Ha dato nomerosi conferenze e corsi in diversi paesi. http://ipt.univ-paris8.fr/mmary/}

\biografia{Davide Palmentiero}{Nasce a Salerno il 19 Maggio 1993. Sei anni dopo inizia a suonare la chitarra classica, per poi passare alla chitarra elettrica all’età di 13 anni, iniziando a suonare e registrare con varie band e artisti senza distinzioni di genere. A 19 anni inizia ad affacciarsi alla Musica Elettronica e un anno dopo si iscrive al Conservatorio di Napoli; qui mostra particolare interesse per l’improvvisazione radicale, sperimentando soprattutto applicazioni e tecniche riguardanti la chitarra. Costruisce e sviluppa continuamente il proprio strumento, con il quale si esibisce in vari festival, rassegne e altri contesti sia in solo che con con diverse formazioni e diversi artisti, tra i quali Bob Ostertag.}

\biografia{Alessandro Pirchio}{Studia presso il Conservatorio di Santa Cecilia con il M° Franz Albanese. Ha partecipato da solo o in formazioni cameristiche a la Rassegna “Musica a Roma per Roma”; il “Sutri Beethoven Festival; Stagione cameristica del Museo della ceramica di Viterbo. Ha suonato per lo spettacolo “La dodicesima notte” (Premio “Le maschere del teatro 2015” per le musiche originali del M° Piovani) in numerosi teatri italiani (Donizzetti di Bergamo, Ponchielli di Cremona, Verdi di Padova sono tra i più importanti). Attualmente ricopre la parte di Primo Flauto nella Banda della Gendarmeria Vaticana e dell’Ass. Nazionale Carabinieri.}

\biografia{Giuseppe Pisano}{Nato nel 1991. Inizia la sua attività musicale come batterista studiando da privatista con il M° Salvatore Tranchini. Da sempre interessato alle sonorità più estreme e rumorose, si concentra inizialmente sul black metal e sull'hardcore, per cambiare poi indirizzo in seguito alla sua permanenza in Norvegia dove si appassiona alla musica elettronica extra-colta ed inizia la militanza nel collettivo techno Stavanger Teknomune, alfieri della cultura rave che getta le sue basi nell'utilizzo di strumenti analogici e del vinile. Dopo due anni decide di proseguire i suoi studi musicali a livello accademico, tornando a Napoli e iscrivendosi al triennio di musica elettronica con il M° Elio Martusciello. Ad oggi è attivo nell' estetica del rumore che egli ricerca negli elementi della vita quotidiana. Attualmente suona la batteria con il gruppo La Bestia Carenne.}

\biografia{Matteo Rossi}{, percussionista, si diploma con il massimo dei voti presso il Conservatorio “S.Cecilia” di Roma con Gianluca Ruggeri. Segue il corso di perfezionamento presso l’Accademia Musicale Chigiana con Antonio Caggiano, e come membro del Chigiana Percussion Ensemble, si esibisce al CHIGIANA INTERNATIONAL FESTIVAL, RAVELLO FESTIVAL e MAXXI di Roma. Collabora con formazioni orchestrali e cameristiche quali PMCE, InDivenire Ensemble ed ensemble di percussioni quali Ars Ludi, Blow-Up Roma Percussion, Aere Silente con cui si esibisce in un repertorio percussionistico moderno e contemporaneo in diversi eventi quali Le esperienze del minimalismo, Le Forme del Suono, ArteScienza, EMUFest.}

\biografia{Massimo Varchione}{Nasce in Svizzera nel 1979. Diplomato in Composizione presso il Conservatorio Nicola Sala con il M° Luigi Turaccio. Studia Musica Elettronica presso il Conservatorio San Pietro a Majella di Napoli prima con il M° Agostino Di Scipio ed ora con il M° Elio Martusciello. Ha composto brani strumentali, elettroacustici e realizzato installazioni. Ha iniziato di recente un nuovo percorso dedicato all'improvvisazione radicale con il mezzo elettroacustico e con gli strumenti. In duo con il clarinettista Agostino Napolitano, nel 2014 è stato selezionato dal centro Tempo Reale di Firenze per partecipare al loro festival dedicato all'elettronica.}

\biografia{Benjamin D. Whiting}{si diploma in Composizione e prende un master in Teoria musicale e composizione presso la Florida State University, ed è attualmente dottorando presso la University of Illinois in Urbana-Champaign. Compone sia musica acustica che elettroacustica, e i suoi lavori sono stati eseguiti negli Stati Uniti e all'estero, ed editi dall'etichetta Experimental Music Studios della University of Illinois.}



