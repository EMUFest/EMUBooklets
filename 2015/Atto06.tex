% !TEX encoding = UTF-8 Unicode
% !TEX TS-program = XeLaTex
% !TEX root = EMU2015_booklet.tex

\livel{Elisabetta Capurso}
{Sezioni per organo spazializzato}{9'}
{per organo e live electronics}
{2015}

\livel{Cristiana Colaneri, Anna Terzaroli}
{Fantasia per due}{8'} 
{per sassofoni, percussioni ed elettronica}
{2015}

\brano{Demian Rudel Rey}
{Cenizas del Tiempo}{7'04''}
{acusmatico}{2015}\\

\livel{Gy\"orgy Ligeti}
{Volumina}{9'30''}
{per organo}
{1961-62, rivista 1966}

\livel{Alessandro Ratoci}
{Studio Mannaro}{11'30''}
{per sassofono baritono e live electronics}
{2013}

\esecutore{organo}{Antonella Barbarossa}
\esecutore{sassofoni}{Enzo Filippetti}
\esecutore{percussioni}{Valerio Cosmai}

% \descrizione{Sezioni per organo spazializzato}{Had many different changes in its general structure, the most important certainly the recent shift of the registers of the instrument. In the performance of the EMUFest 2015 the score has a further change for the presence of some electronic processing. For reasons of expression have been used some spatial sound systems, that fit perfectly to the sound of the acoustic instrument. Some microphones, amplifiers, some spatial sound's elaboration are amply present in the electronic score of the composition Sezioni.}

\descrizione{Sezioni per organo spazializzato}{Sezioni è opera che appartiene al periodo della prima attività compositiva. Dopo alcuni anni ha avuto alcuni cambiamenti nella struttura generale, un radicale cambiamento nella registrazione organistica.Il principio del mutamento di vista, simile a prisma ruotante è così la legenda della scrittura compositiva personale. La struttura generale ha tre sezioni, all'interno delle quali tre micro-sezioni hanno vita di improvvisazione. Nell'esecuzione di EMUFest2015 Sezioni ha un'ulteriore mutamento per la presenza di interventi elettronici. Per una  necessità di esigenze espressive sono stati  utilizzati alcuni sistemi di spazializzazione del suono  che si integrano perfettamente  con il suono originario dello strumento acustico. Microfonaggio, amplificazione, spazializzazione del suono sono gli interventi elettroacustici presenti nel processo di elaborazione sonora di Sezioni, suggeriti fondalmentalmente dalla raffinata registrazione organistica.}

% \descrizione{Fantasia per due}{“Fantasia per due” is among the EMUfest 2015 commissions destined to a couple of students of the Santa Cecilia Conservatory, one from the School of Composition, the other from the School of Electronic Music. The electronics uses the sound material produced by acoustic instruments, processing it and storing it on tape and it is built keeping in mind an equal and complementary relationship with the acoustic component. The latter is built starting from the indeterminate vibrations of a gong and from a single sound of the sax which, through gradual micro-intervals, conquers a wider frequency range interacting with percussions up to its full expansion. The conclusion is related to incipit.}

\descrizione{Fantasia per due}{Il brano è una commissione EMUfest, destinata ad una coppia di autori entrambi studenti del Conservatorio Santa Cecilia, uno proveniente dalla Scuola di Composizione, l'altro dalla Scuola di Musica Elettronica. L'elettronica utilizza il materiale sonoro prodotto dagli strumenti acustici, elaborandolo e fissandolo su tape ed è costruita nell'ottica di un rapporto paritetico e complementare con la composizione acustica. Questa è costruita partendo dalle vibrazioni indeterminate del gong e da un unico suono del sax, che gradualmente, anche con microintervalli, conquista un campo frequenziale più vasto, nell' interazione con le percussioni, fino alla libera espansione; la conclusione si ricollega all' incipit.}

% \descrizione{Cenizas del Tiempo}{Cenizas del Tiempo is an electroacoustic quadraphonic work, inspired by the idea that time ceases his ashes in our lives, gradually our being is consumed and the same thing happens with the materials and the sound objects. Also, expresses the experience of time in a city where everything occurs very quickly. In the work referential sounds of urban environments are perceived, moreover, samples from different ashtrays developed as more abstract sounds through processed and over-processed.}

\descrizione{Cenizas del Tiempo}{è un'opera quadrifonica elettroacustica, ispirata al tempo che porta le sue ceneri nella nostra vita, a poco a poco il nostro essere è consumato e la stessa cosa accade con i materiali e gli oggetti sonori. Inoltre, esprime l'esperienza del tempo in una città dove avviene tutto molto velocemente. In questo lavoro sono percepibili suoni riferiti ad ambienti urbani, e campioni di diversi posacenere che diventano suoni sempre più astratti attraverso continui processi.}

% \descrizione{Volumina}{eng}

\descrizione{Volumina}{Usa questo QR code dal tuo dispositivo mobile per visualizzare o scaricare la partitura completa e le note d’esecuzione}

\center{\begin{pspicture}(1in,1in)
	\psbarcode{https://goo.gl/JCU3Sc}{eclevel=L width=1 height=1}{qrcode}
\end{pspicture}}\footnote{https://goo.gl/JCU3Sc}

% \descrizione{Studio Mannaro}{ is a composition for baritone saxophone and live electronics. The saxophone is equipped with a microphone used both for amplification and sound pickup for processing by the computer. The amplified sound together with the processed sounds is then projected in the concert hall using a sound spatialization system based on virtual sources.}

\descrizione{Studio Mannaro}{ per saxofono baritono ed elettronica è stato presentato in prima esecuzione al festival Archipel di Ginevra nel 2012. L’idea poetica del brano è la ricerca dell’unità fra molteplici possibili dissociazioni: il registro estremo grave e estremo acuto del sassofono, le componenti fondamentali e armoniche dello spettro sonoro, la presenza tangibile dello strumento solista e la molteplicità virtuale dei campionamenti e delle sintesi elettroniche. Questo percorso di accettazione dell’identità complessa di un organismo “diverso” e peculiare si realizza attraverso il passaggio fra vari stadi sonori e conseguenti paesaggi immaginari ai quali non sono estranee certe categorie care alla musica di matrice popolare: il drone e il groove.}

