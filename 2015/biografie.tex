BIOGRAFIE

\biografia{James Andean}{Musician and sound artist. He is active as both a composer and a performer in a range of fields, including electroacoustic composition and performance, improvisation, sound installation, and sound recording. He is a founding member of improvisation and new music quartet Rank Ensemble, and one half of audiovisual performance art duo Plucié/DesAndes. He has performed throughout Europe and North America, and his works have been presented around the world.}

\biografia{James Andean}{Musicista e sound artist. È attivo sia come compositore che esecutore in vari campi, incluse composizioni elettroacustiche e performance, improvvisazioni, installazioni audio, e registrazioni. È membro fondatore del quartetto di improvvisazione e nuova musica Rank Ensemble, e fa parte del duo Plucié/DesAndes (audiovideo). Si è esibito per l'Europa e Nord America, e i suoi lavori sono stati presentati in tutto il mondo.}


\biografia{Damián Anache}{(1981, Bs As, ARG) Composer, Researcher and Teacher (CONICET, UNQ, FUC). His works have been played at: Roma (ITA), Morelia and DF (MEX), Quito (ECU), Córdoba, Rosario and Bs As (ARG). “Capturas del Único Camino” is his first album released by Inkilino Records and Concepto Cero labels (2014, ARG)

Website: damiananache.com.ar}

\biografia{Damián Anache}{(1981, Buonos Aires) Musicista e sound artist. è attivo come compositore e performer nell'improvvisazione, istallazione sonora, sound recording. è membro fondatore di "improvisation and new music quartet Rank Ensemble, e fa parte del duo audiovisual Plucié/DesAndes. Si è esibito in tutta l'Europa e Nord America i suoi lavori sono stati presentati in tutto il mondo.

Website: damiananache.com.ar}


\biografia{Alfredo Ardia}{Class 1989, he studied at LEMS - SPACE (Pesaro, Italy) and at CMT (Helsinki, Finland). He is interested in sound, its perception and how it relates with other media, exploring sound phenomena of elementary sound entities and its behaviors. He is inspired by the beauty of physics.

Web: http://alfredoardia.altervista.org/}

\biografia{Alfredo Ardia}{Classe 1989, ha studiato al LEMS (Pesaro, Italia) e al CMT (Helsinki, Finlandia). È interessato al suono, alla sua percezione e a come esso si relaziona con altri media, esplorando i fenomeni sonori di entità elementari ed i loro comportamenti. È ispirato dalla bellezza della fisica. 

Web: http://alfredoardia.altervista.org/}


\biografia{Luciano Azzigotti}{Composer based in Buenos Aires Argentina. His musical work encompasses different medias and listening, gesture, and writing interfaces. He studied composition at University of La Plata and in different scholarships and International seminars as active participant with Gerardo Gandini, Mauricio Kagel, George Aperghis, Chaya Czernowin and Rebecca Saunders. His music has been played in Argentina, Brasil, USA, Austria, Germany, France, Italy, by reputed ensambles and soloists. Founder of conDiT BsAs.}

\biografia{Luciano Azzigotti}{Compositore residente a Buones Aires, Argentina. La sua musica comprende differenti mezzi ed ascolti, gesti e interfacce di scrittura. Ha studiato composizione all'università di La Plata, in diversi master e partecipato a seminari internazionali con Gerardo Gandini, Mauricio Kagel, George Aperghis, Chaya Czernowin e Rebecca Saunders. La sua musica è stata eseguita in Argentina, Brasile, USA, Austria, Germania, Francia, Italia, da importanti ensemble e solisti. è fondatore del conDIT BsAs.}


\biografia{Christian Banasik}{(1963) studied composition at the Robert Schumann Academy of Music and Media in Dusseldorf and at the University of Music and Performing Arts in Frankfurt. His instrumental and electronic works have been featured in concerts and radio programs throughout Europe as well as in the Americas, Asia, and Australia. Banasik is lecturer for Audio Visual Design at the University for Applied Sciences and the artistic director of the Computer Music Studio "Studio 209" in Dusseldorf.}

\biografia{Christian Banasik}{(1963) ha studiato composizione alla Robert Schumann Academy of Music and Media di Dusseldorf e alla University of Music and Performing Arts di Francoforte. I suoi lavori strumentali ed elettronici sono stati eseguiti in concerti e programmi radio in tutta Europa, ma anche in America, Asia e Australia. È docente di Audio Visual Design presso la University for Applied Sciences e direttore artistico di Computer Music Studio "Studio 209" di Dusseldorf.}


\biografia{Carlo Barbagallo}{Carlo Barbagallo is a self-taught musician, composer, sound engineer, music producer, born in 1985 in Siracusa, Sicilia. He always recorded his music, experimenting the possibilities of home and studio creative recording.}

\biografia{Carlo Barbagallo}{(1985) musicista, compositore e sound engineer.Sin da piccolo ha registrato la propria musica, sperimentando sulle (non) possibilità creative della registrazione casalinga:i lavori sono in rete tramite l'etichetta Noja Recordings.Dal 2012 è musicista elettroacustico con ricerca concentrata su: spazializzazione di musica acusmatica, feedback nelle sue diverse forme, improvvisazione e programmazione-tramite-ascolto come tecniche compositive, sviluppo di strumenti algoritmici per generare partiture e suoni a partire dalla struttura di testi linguistici, estetica delle forme incomplete, la composizione collettiva.Nel2013 è co-fondatore del Collettivo di Musica Elettroacustica di Torino (CoMET).Tra i festival: Premio Phonologia, ICMC, PNDA, 57Festival Internazionale di Musica Contemporanea (Biennale di Venezia), Emufest, PIARS, XXCIM, Di_Stanze. http://nojarecordings.tumblr.com}


\biografia{Antonella Barbarossa}{eng}

\biografia{Antonella Barbarossa}{è nata e vive in Italia. Didatta del pianoforte, organista, direttrice d’orchestra, compositrice, filosofa e missionaria in Calabria per scelta dove assume la docenza di pianoforte al Conservatorio di stato di Cosenza nel 1976. Nel 1991 diviene direttore del Conservatorio di Vibo Valentia sino al 2013; nel 2003 fonda il politecnico internazionale “ scientia et ars” per la specializzazione in tecnologia del suono. In campo organistico ha eseguito in concerti pubblici l’opera integrale di Bach, Franck, Liszt e Messiaen, e in prima assoluta per la Rai e festivals internazionali, composizioni d’autori contemporanei. È vincitrice del primo premio al Concorso internazionale organistico di Roma nel 1981 e finalista al Concorso Internazionale di Lipsia per lo stesso strumento.}


\biografia{Francesco Bianco}{Graduated in Musicology, he studies Electronic Music at the Conservatory of Santa Cecilia in Rome. He has been visiting scholar at the CRR de Boulogne-Billancourt (Paris). Musician always been interested in the deep relationship between art and life, he has different genre and style musical experiences, from composition to live performance, from the scene to the soundtrack.}

\biografia{Francesco Bianco}{Laureato in Musicologia, frequenta il corso di Musica elettronica presso il Conservatorio di Roma Santa Cecilia. È stato visiting scholar al CRR de Boulogne-Billancourt (Parigi). 
Musicista da sempre interessato alle profonde relazioni fra l'arte e la vita, ha esperienze musicali variegate di genere e stile, dalla composizione alla performance dal vivo, dall'azione scenica alla colonna sonora.}


\biografia{Isobel Blank}{http://www.isobelblank.com}

\biografia{Isobel Blank}{Nasce a Pietrasanta in Toscana, si laurea con lode in filosofia estetica a Padova, vive e lavora a Torino. Le sue opere sono state esposte in numerose gallerie, musei e festivals, dagli Stati Uniti alla Cambogia, dal Messico alla Russia. Tra le esposizioni recenti, quella alla Triennale Fiberart International di Pittsburgh, al Museum of Modern Art di Mosca, a Palazzo Widmann di Venezia, alla Mumbai Art Room in India. Ha avuto diversi riconoscimenti tra cui il Primo Premio al Romaeuropa Webfactory nel 2009, sezione videoarte. http://www.isobelblank.com}


\biografia{Daniel Blinkhorn}{Daniel is an Australian composer. He is currently lecturing into the composition and music technology department at the Conservatorium of Music, University of Sidney. He is also an ardent location field recordist, where he has embarked upon a growing number of recording expeditions throughout Africa, Alaska, Amazon, West Indies, Northern Europe, Middle East, Australia and the North Pole. His creative works have received over 25 international and national composition citations. He is self-taught in electroacoustic’s, however has formally studied composition and the creative arts at a number of Australian universities including UOW where his doctoral degree in creative arts was recommended for special commendation. Other degrees include a BMus (hons), MMus, and a MA(r).}

\biografia{Daniel Blinkhorn}{Daniel è un compositore australiano
Attualmente impartisce lezione nel reparto di tecnologia, composizione e musica presso il Conservatorio dell' Università di Sidney. È anche un appassionato tecnico di field recording; ha intraprendeso spedizioni di registrazione in tutta l'Africa, Alaska, Amazon, Indie occidentali, Nord Europa, Medio Oriente, Australia e il Polo Nord. Le sue opere creative hanno ricevuto più di 25 citazioni di composizione internazionali e nazionali.È autodidatta in elettroacustica e ha studiato al tempo stesso composizione e arti creative in diverse università australiane, tra cui UOW, dove il suo dottorato ha ricevuto menzione speciale. Altre lauree: BMus (Hons ), MMus, e MA(r).}


\biografia{Elisabetta Braga}{eng}

\biografia{Elisabetta Braga}{Nata a Nardò, si diploma brillantemente in canto presso il Conservatorio di musica Santa Cecilia di Roma nel 2013. Comincia la sua attività concertistica esibendosi in vari teatri e sale da concerto Italia e all'estero, quali la Sala Accademica del Conservatorio Santa Cecilia di Roma, il Teatro Politeama Greco di Lecce, la Tchaikoskj Concert Hall di Mosca. Nel 2012 partecipa a Emufest per l'esecuzione di "Anabasi" di G.Baggiani, diretta da T. Battista. Debutta nel 2015 a Roma come Mimì ne "La Bohème" di G. Puccini e a Rieti nel 2015 come Gilda in "Rigoletto" di G. Verdi. Si è perfezionata partecipando come allieva effettiva alla masterclass del soprano Sumi Jo tenuta a Roma in aprile. Attualmente sta per conseguire il Diploma Accademico di secondo livello presso il Conservatorio Santa Cecilia di Roma.}


\biografia{Daniele Buccio}{Daniele Buccio graduated in piano at the “A. Casella” Conservatory in L’Aquila and in composition at the “G. Verdi” Conservatory in Turin and obtained his PhD in Musicology at the “Alma Mater Studiorum” in Bologna. He attended composition masterclasses at the Accademia Filarmonica of Bologna, at the “L. Perosi” Academy in Biella, and at the Chigiana Academy. He performed at the Teatro Regio in Parma, at the Auditorium Parco della Musica in Rome, at the Deptford Town Hall for the Liszt Society of London, and at the Congress Centre in Lugano.}

\biografia{Daniele Buccio}{Diplomato in pianoforte presso il Conservatorio “A. Casella” dell’Aquila ed in composizione presso il Conservatorio “G. Verdi” di Torino, è dottore di ricerca in musicologia e beni musicali. Ha seguito corsi di alto perfezionamento in composizione presso l’Accademia Filarmonica di Bologna, l’Accademia “L. Perosi” di Biella e l’Accademia Musicale Chigiana. Si è esibito come solista presso il Teatro Regio di Parma, l’Auditorium Parco della Musica di Roma, la Deptford Town Hall per la Liszt Society di Londra, il Palazzo dei Congressi di Lugano.}


\biografia{Sylvano Bussotti}{Born in Firenze on 10/01/1931. Starts studying on violin with Margherita Castellani before the age of five. He will study at the Florence’s conservatory “Luigi Cherubini” harmony and counterpoint with Roberto Lupi and the piano with Luigi Dallapiccola: studies that he will stop because of war, without achieving any official qualification. In Paris, from 1956 to 1958, he attended private course with Max Deutsch, he will meet Pierre Boulez and Heinz-Klaus Metzger, which will lead him to Darmstadt, where he’ll meet John Cage. In 1958, in Germany, he starts his public career, with the execution of his music by the pianist David Tudor, followed up by the presentation, in Paris, of tracks played by Cathy Berberian under the direction of Pierre Boulez.}

\biografia{Sylvano Bussotti}{Nato a Firenze il 1 ottobre 1931. Inizia lo studio del violino con Margherita Castellani ancora prima di compiere i cinque anni di età. Al Conservatorio "Luigi Cherubini" di Firenze studierà l'armonia e il contrappunto con Roberto Lupi e il pianoforte con Luigi Dallapiccola: studi che interromperà a causa della guerra, senza conseguire alcun titolo di studio. A Parigi, nel periodo che va da 1956 al 1958, frequenta i corsi privati di Max Deutsch, incontra Pierre Boulez e Heinz-Klaus Metzger, che lo condurrà a Darmstadt, dove conosce John Cage. Inizia in Germania, nel 1958, l'attività pubblica, con I'esecuzione delle sue musiche da parte del pianista David Tudor, seguita dalla presentazione a Parigi di brani eseguiti da Cathy Berberian sotto la direzione di Pierre Boulez.}


\biografia{Elisabetta Capurso}{is a pianist, composer, musicologist. She studied piano with Carlo Vidusso and Carlo Zecchi, composition with Domenico Guaccero and Brian Ferneyhough, orchestra with Daniele Paris, electronics music with Giorgio Nottoli. She obtained the degree in Letters Philosophy at the University “La Sapienza” of Rome, and in Electronics Music at the Conservatory S. Cecilia of Rome. She is well know on the international concert scene, and she has appeared, as a pianist, in the most prestigious halls invited by the principal Italian and foreign musical Institutions. Elisabetta Capurso is know in the world as well as modern and contemporary music performer, as well as composer. Performed at the Ferienkurses of Darmstadt and at some of the most important cultural Institutions like the ‘Foundation of contemporary Music Meeting’ her musical work is characterized by the elements of a learned counterpoint and of a remarkable capability of imparting its own knowledge. Her compositions have obtained a great success at some of the most important Italian musical Festivals, as well as on the Italian national networks RAI, as well as in many American countries. Among the others: Festival Nuova Consonanza, Rome; Antidogma Musica, Turin etc. Some works have been published by the Musical Editions: Zanibon (Peters) Padova ; Edi-Pan, Rome; AFM-Accord for Music SEDAM Rome. Piano Professor at ‘Rossini’ Conservatory in Pesaro, Piano Professor, Semiography and Laboratorio of contemporary music also at the ‘S. Cecilia’ Conservatory in Rome. She has received many prizes for her artistic values.}

\biografia{Elisabetta Capurso}{pianista compositrice musicologa ha completato la formazione pianistica al Mozarteum di Salisburgo, gli studi di composizione a Darmstadt. Della sua formazione musicale fanno parte gli studi di Direzione d’orchestra composizione elettronica gli studi umanistici; è laureata con lode in Lettere Filosofia all’Università La Sapienza. Recentemente ha conseguito con lode 
la seconda laurea in Musica elettronica al Conservatorio S. Cecilia. Concertista di livello internazionale, ha suonato in sale prestigiose per le maggiori associazioni italiane e straniere. Come compositrice ha scritto molti lavori di musica sinfonica cameristica elettronica, eseguite in teatri di rilievo in Italia e all’estero. Le sue opere sono state registrate da Rai Radio Tre, Radio Vaticana, pubblicate da Zanibon-Peters EdiPan AFM SEDAM. Diverse le commissioni di scrittura ricevute: Après alfa dai Solisti Aquilani Tre cose solamente dal Festival Internazionale Organo di Lecce. Ha un curriculum di Professore di Pianoforte nel Conservatorio G.Rossini di Pesaro, Professore di Pianoforte Semiografia della musica contemporanea Laboratorio contemporaneo nel Conservatorio S. Cecilia di Roma. Ha ricevuto premi e riconoscimenti per gli alti meriti artistici.}


\biografia{Simone Cardini}{He studies composition with F. Telli, piano with A. Torchiani; he participates in masterclasses held by S. Sciarrino, M. Andre, T. Tulev, M. Trojahn, P. Manoury. His compositions have been played in Europe and USA in eminent expositions and festivals like ArteScienza (2012), Contemporanea (2013), Nuova Consonanza (2013, 2014), Rondò (2014), NYCEMF (2015) by international ensembles like Divertimento Ensemble, PMCE and they have been awarde in competitions like AFAM (2013), Valentino Bucchi (2015).}

\biografia{Simone Cardini}{Studia composizione con F. Telli, pianoforte con A. Torchiani; partecipa a corsi di perfezionamento e seminari tenuti da S. Sciarrino, M. André, T. Tulev, M. Trojahn, P. Manoury. Le sue composizioni sono state eseguite in Europa e Stati Uniti in mostre eminenti e festival come ArteScienza (2012), Contemporanea (2013), Nuova Consonanza (2013, 2014), Rondò (2014), NYCEMF (2015) da ensemble internazionali come Divertimento Ensemble, PMCE e sono stati premiati in concorsi illustri come AFAM (2013), Valentino Bucchi ed 37 °. (2015), ecc Il suo scritto Musica e Architettura - Estetica e implicazioni sociologiche è stato pubblicato nel libro Musica & Architettura, Nuova Cultura Ed. (2012). La sua raccolta di lavori sarà pubblicata da Universal Edition (2015).}


\biografia{Antonio Carvallo}{A. Carvallo was born in Chile in 1972. Studied counterpoint and harmony with Rodolfo Norambuena. Then, he studies at University of Chile, where he got a Bachelor and Master degrees in Composition. Was professor at University of Chile from 2000 to 2002. After that he moved to Rome, studing Electronic Music with R. Bianchini and G. Nottoli at “Conservatorio Santa Cecilia” getting an Academic degree of First and Second level. Nowadays he teaches at University of Chile.}

\biografia{Antonio Carvallo}{A. Carvallo è nato in Cile nel 1972. Ha studiato contrappunto e armonia con Rodolfo Norambuena. Poi, presso l'Università del Cile, ha ottenuto il Diploma e il Master di Specializzazione in Composizione. Ha insegnato all' Università del Cile dal 2000 al 2002. Dopo di che, si è trasferito a Roma e ha studiato musica elettronica con R. Bianchini e G. Nottoli al Conservatorio "Santa Cecilia" ed ha ottenuto la Laurea di Primo e Secondo Livello. Attualmente insegna presso l'Università del Cile.}


\biografia{Pasquale Citera}{He studied piano with M°Gemma D'Alessio, Composition with M°Luciano Pelosi and M°Giovanni Piazza and Electronic Music with M°Giorgio Nottoli. He has been working with several theatre companies, movie studios, sculptors and photographers. He has composed music for classical and contemporary theater shows, like: Alcestis (Euripides), Lysistrata (Aristophanes), Amphitryon (Plautus), Locandiera (Goldoni), L’Avare (Molière), Da quale parte del vetro (Silvio Nanni), Il dito sulla bocca (Donatella Ferrara), Certe notti non accadono mai (Patrizia Masi). He wrote soundtracks for Nero-Film, he is Assistant Music in several schools of Rome and has been Professor of Music Technology. From the collaboration with the sculptor Arturo Ianniello are born different soundtracks of visual works collected in two exhibitions. He is currently Composer and Sound Designer for incidental music at Anfitrione Theatre and “Quercia Del Tasso” Amphiteatre.}

\biografia{Pasquale Citera}{(1981) Ha studiato pianoforte con il M° Gemma D’Alessio, Composizione con i M° Luciano Pelosi e Giovanni Piazza e Musica Elettronica con il M° Giorgio Nottoli. Da anni collabora con diverse compagnie teatrali e case di produzione cinematografica oltre che con scultori e fotografi. Ha composto musiche di scena per spettacoli classici e contemporanei. Tra gli altri: L’Alcesti di Euripide, Lisistrata di Aristofane, Anfitrione di Plauto, la Locandiera di Goldoni, l’Avaro di Molière, Da quale parte del vetro di Silvio Nanni, Il dito sulla bocca di Donatella Ferrara, Certe Notti non accadono mai di Patrizia Masi. Ha scritto colonne sonore per la Nero-Film, è Assistente Musicale in diverse scuole di Roma ed è stato docente di Tecnologie Musicali. Dalla collaborazione con lo scultore Arturo Ianniello sono nate diverse sonorizzazioni di opere visuali raccolte in due esposizioni. È attualmente Compositore e Sound Designer per musiche di scena al Teatro Anfitrione ed all’Anfiteatro della Quercia del Tasso.}


\biografia{Cristiana Colaneri}{Born in Rome, she got the medium fulfillment of the Composition course (under the guidance of M. Pasquale Lucia). She is now studying Composition at the Conservatory of Santa Cecilia in Rome in the class of M. Francesco Telli. She is currently preparing the final exam of the first level academic diploma. She has been a finalist in composition contests: Mea 2010 and Bucchi 2015.}

\biografia{Cristiana Colaneri}{Nata a Roma ha conseguito il compimento medio di composizione vecchio ordinamento (sotto la guida del M. Pasquale Lucia); studia Composizione presso il Conservatorio Santa Cecilia di Roma, nella classe del M. Francesco Telli. Deve sostenere la Prova finale del diploma accademico di I livello. Finalista ai concorsi Mea 2010 e Bucchi 2015.}


\biografia{Valerio Cosmai}{Born in Rome in 1983. He studied piano, performing several times as a solo pianist in the sala Baldini of Rome and in various embassies, including the US and Indonesian embassies, specializing in the Mozart repertoire. He obtained in 2008 a degree in Literature from the University "La Sapienza" of Rome. He is graduated in percussion with honors in 2014. As a percussionist, he works with the orchestra of the Conservatory also performing in important festivals of contemporary music. Since 2012 he works as a teacher of musical education in the school Pio IX in Rome.}

\biografia{Valerio Cosmai}{Nasce a Roma nel 1983. Studia pianoforte esibendosi più volte come pianista solista nella sala Baldini di Roma e in varie ambasciate, tra cui quella americana ed indonesiana, specializzandosi nel repertorio mozartiano. Consegue nel 2008 la laurea in Lettere presso l'università degli studi "La Sapienza" di Roma. Si diploma in percussioni con il massimo dei voti nel 2014. Come percussionista collabora con l'orchestra del Conservatorio esibendosi anche in importanti festival di musica contemporanea. Dal 2012 lavora come insegnante di educazione musicale nella scuola Pio IX di Roma.}


\biografia{Giovanni Costantini}{(Corigliano d’Otranto - Lecce, 1965) Since 1995, he do research at the Faculty of Engineering of University of Rome Tor Vergata, where he teaches courses in Sound Processing and Electronic Music. He is also associate researcher at the Institute of Acoustics "O. M. Corbino" of Rome. At the University Tor Vergata, he is Director of the Master in SONIC ARTS. His musical research is currently focused on the creation of the microstructure and macrostructure of sound through the exploration and real-time processing of acoustic material.}

\biografia{Giovanni Costantini}{(Corigliano d’Otranto - Lecce, 1965) Dal 1995 svolge attività di ricerca presso l'Università di Roma "Tor Vergata", dove è docente di Musica Elettronica. È direttore del Master in SONIC ARTS. Sue composizioni elettroacustiche sono state eseguite in numerosi concerti in Italia e all’estero e incise da Twilight Music (Roma) e IAEF (New York). La sua ricerca musicale è rivolta alla realizzazione della microstruttura e della macrostruttura del suono attraverso l’esplorazione e l’elaborazione in tempo reale di materiale acustico.}


\biografia{Elena D'Alò}{Elena D'Alò is a flutist and a piccolo player. She studied Flute (Diploma and Master's degree) in "Santa Cecilia" Conservatory in Roma, with teachers: Edda Silvestri, Bruno Paolo Lombardi, Deborah Kruzansky and Paolo Taballione. She also graduated in Acoustic Physics (Bachelor's degree) in "La Sapienza" University with Paolo Camiz as supervisor. Now she studies Electronic Music. She plays chamber music and in orchestral concerts too: from barocco to contemporary repertory. She also plays cello.}

\biografia{Elena D'Alò}{Elena D'Alò, flautista e ottavinista si laurea cum laude al biennio in Flauto, dopo un brillante diploma, presso il Conservatorio "Santa Cecilia" di Roma, con Deborah Kruzansky. Ha affiancato gli studi musicali con quelli scientifici, laureandosi in Fisica acustica presso "La Sapienza" con Paolo Camiz. Attualmente è iscritta al triennio di Musica Elettronica. Si esibisce in formazioni cameristiche e orchestrali, in un repertorio che va dal barocco al contemporaneo, per il quale ha suonato a festival come Nuova Consonanza, Atlante Sonoro XX secolo, Arte e Scienza ed EMUfest. Studia violoncello con Maurizio Massarelli.}


\biografia{Maria Cristina De Amicis}{(Avezzano, 1968) She studied Composition, Electronic Music, Organ and Organ Composition at Conservatorio “A.Casella” in L’Aquila. Her professional activity is focused on the most advanced trends of musical language. She founded, with other musicians and scientific researchers, “Istituto Gramma" where, from 1989, she realizes her artistic and scientific activity. Her works have been performed in remarkable contemporary music events in Italy and abroad (Lyon, Paris, Barcelona, Aveiro, Madrid, Budapest, Athens,Thessaloniki, Berlin, Frankfurt, Vienna). Since 2012 she’s Professor of Electronic Music at Conservatorio “A.Casella” in L’Aquila.}

\biografia{Maria Cristina De Amicis}{(Avezzano, 1968) Ha compiuto studi di Composizione, Musica Elettronica, Organo e Composizione Organistica, diplomandosi con il massimo dei voti presso il Conservatorio di Musica “A. Casella” de L'Aquila. Le sue opere sono state eseguite in importanti manifestazioni di musica contemporanea in Italia e all’estero tra cui (Lione, Parigi, Barcellona, Aveiro, Madrid, Budapest, Atene, Salonicco, Berlino, Francoforte, Vienna). Dal 2012 è docente di Musica Elettronica presso il Conservatorio di Musica “A.Casella” dell’Aquila.}


\biografia{Domenico De Simone}{Graduated in Piano, in Jazz and in Composition.
He obtained a diploma in the PhD course in composition at the Accademia Nazionale di Santa Cecilia with Azio Corghi in 2001. He has received a diploma in Electronic Music from the “Conservatorio di Santa Cecilia” with the maximum grade under the guidance of Giorgio Nottoli in 2004. He graduated cum laude in Electronic Music at the high course "Biennio Sperimentale di II Livello in Discipline Musicali” in 2006. He has also studied with Franco Donatoni, Ennio Morricone, Salvatore Sciarrino. His compositions have been performed in Italy and abroad (Canada, Argentina, Romania, China) and transmitted by Radio3.}

\biografia{Domenico De Simone}{Diplomato in Pianoforte, Jazz, Composizione e Musica Elettronica. Ha conseguito il diploma del corso di perfezionamento di Composizione presso l’Accademia Nazionale di Santa Cecilia e, con il massimo dei voti e la lode, il diploma accademico di II livello in Musica Elettronica. Sue composizioni sono state eseguite in Italia e all’estero (Cina, Lettonia, Canada, Cile, Argentina, Romania, Malta, ecc.) e trasmesse da Radio3.}


\biografia{James Dashow}{Has had commissions, awards and grants from the Bourges International Festival of Experimental Music (Prix Magistere), the Guggenheim, Fromm and Koussevitzky Foundations, Linz Ars Electronica, the Biennale di Venezia, the USA National Endowment for the Arts, RAI, the American Academy and Institute of Arts \& Letters, Prague Musica Nova et. al. In 2011 he was honored with the "CEMAT per la Musica" prize in recognition of his career of outstanding contributions to electronic music.}

\biografia{James Dashow}{Ha avuto commissioni, premi e borse di studio dal Bourges Festival Internazionale di Musica Sperimentale (Premio Magistere), il Guggenheim, Fromm e Fondazioni Koussevitzky, Linz Ars Electronica, la Biennale di Venezia, gli USA National Endowment for the Arts, RAI, l'American Academy and Institute of Arts \& Letters, Prague Musica Nova, etc. Nel 2011 gli è stato conferito del "CEMAT per la Musica" premio in riconoscimento della sua carriera di contributi eccezionali alla musica elettronica.}


\biografia{Gustavo Delgado}{Graduated in Electronic Music at the Conservatory of Music Santa Cecilia, in Rome, under tuition of composer Giorgio Nottoli. My main fields of interest are electroacoustic music composition, sound programming, sound design, sound engineering. I lecture in Conservatories of Music of Benevento and Latina.}

\biografia{Gustavo Delgado}{Buenos Aires (1976). Diploma di Secondo Livello specialistico in “Musica Elettronica” presso il Conservatorio di Musica “Santa Cecilia” di Roma sotto la guida del M° Giorgio Nottoli con il massimo dei voti. Laurea in “Composizione di Musica Elettroacustica” presso l’Università Nazionale di Quilmes (Buenos Aires, Argentina). Compositore di musica acousmatica, live electronics e di musica applicata, interessato allo studio delle tecniche di missaggio on the box e sound design.}


\biografia{Dennis Deovides A. Reyes III}{Studied music composition in his native Manila, Philippines. Dennis is currently pursuing his doctorate degree in music composition at the University of Illinois at Urbana-Champaign under the tutelage of Scott A. Wyatt. His compositions find inspiration in a wide range of subjects, from Asian music to modern art, and also incorporate elements of Philippine tradition. Dennis’ compositions have received numerous performances in Europe, Asia, and the United States.}

\biografia{Dennis Deovides A. Reyes III}{Ha studiato composizione musicale nella sua città nativa Manila, Filippine, prima di trasferirsi negli Stati uniti nel 2006. Attualmente Dennis sta facendo il dottorato in composizione musicale all'università dell'Illinois all'Urbana-Champaign con Scott A. Wyatt. Le sue composizioni trovano ispirazione da una vasta gamma di argomenti, dalla musica Asiatica all'arte moderna, e incorpora anche elementi della tradizione Filippina.}


\biografia{Christian Eloy}{Born in Amiens where he studied flute and composition at the conservatoire national of region and at the conservatoire national superior of Paris. Flutist in an orchestra, then director of a music school, before his meeting with Ivo Malec and the GRM at Radio France. Christian ELOY is the founder of Octandre (composers's association); he is in charge of the electroacoustic department of the Conservatoire National de Region in Bordeaux and of the workshop at the GRM, Groupe de Recherches Musicales /City of Paris. Christian ELOY is the artistic director of the SCRIME, research and creation studio in the university of Bordeaux I. Several awards : prize of the europeen community poetry and music - prize " François de Roubaix ". Composer of over fourty pieces, instrumental, electroacoustic, vocal and pedagogical. Published by Billaudot, Fuzeau, Lemoine, Combre, Notissimo and Jobert. Publications at PUF (France), Johnston Ed.(Irlande), MIT press (US), Le mensuel littéraire et poétique (Belgique). Confluences (France).}

\biografia{Christian Eloy}{Nato ad Amiens, ha studiato flauto e composizione al Conservatorio Nazionale della regione ed al Conservatorio Nazionale Superiore di Parigi. Prima del suo incontro con Ivo Malec e l’emittente GRM / Groupe de Recherches Musicales di Radio France, è stato flautista in orchestra e direttore di una scuola di musica. Christian Eloy ha fondato l’associazione di compositori Octandre, è a capo del dipartimento di elettroacustica del Conservatoire National de Region a Bordeaux e dei workshop per il GRM a Parigi ed è direttore artistico dello studio di ricerca e creatività SCRIME all’Università di Bordeaux I. Vincitore di numerosi premi, tra cui il premio europeo per poesia e musica " François de Roubaix ", ha composto oltre quaranta brani strumentali, elettroacustici, vocali e didattici. I suoi lavori sono pubblicati dalla Billaudot, Fuzeau, Lemoine, Combre, Notissimo e Jobert e i suoi articoli scientifici dal PUF (Francia), Johnston Ed. (Irlanda), MIT press (USA), Le mensuel littéraire et poétique (Belgio) e Confluences (Francia).}


\biografia{Sara Ferrandino}{graduated in piano in September 2005 at the Morlacchi Conservatory of Perugia, in the class of M° L. Tanganelli. At the same institution, in March 2009, she passed the level II Academical Diploma with top marks and special mention. In July 2011 she obtained the specialist diploma for the postgraduated course held by Mº S. Perticaroli at the Santa Cecilia Academy in Rome. She has partecipated in more than 30 national and international piano competitions, always reaching the top. She plays both as a piano soloist and in chamber ensembles with important musicians in prestigious classical concert halls in Italy and abroad. Sara Ferrandino is a piano teacher for the pre-academical courses at Santa Cecilia. She also collaborates with the Conservatory of Perugia for the courses of horn, trumpet, flute, violin, oboe and works as an artistic consultants in other important musical institutions in Rome, organizing masterclasses and pianistic competitions at international level.}

\biografia{Sara Ferrandino}{si è diplomata in pianoforte nel 2005 presso il Conservatorio di Perugia nella classe del Mº Tanganelli, conseguendo nel 2009, con votazione di 110 e Lode, la Laurea per il Biennio Specialistico. Nel 2012 ha ottenuto il diploma del Corso di Perfezionamento tenuto dal Mº Perticaroli, presso l’Accademia Nazionale di Santa Cecilia in Roma. Ha partecipato a numerosi concorsi nazionali e internazionali ottenendo sempre piazzamenti nelle prime posizioni. Si è esibita in molteplici concerti solistici e cameristici in prestigiose sale in Italia e all’estero. Collabora presso il Conservatorio di Perugia con le classi di corno, tromba, flauto, oboe e violino. È docente di pianoforte principale per i corsi pre-accademici presso il Conservatorio Santa Cecilia in Roma.}


\biografia{Enzo Filippetti}{is professor of Saxophone at Conservatorio “S. Cecilia” in Rom. In more than thirty years he gives concerts all over the world. He has performed at Biennale di Venezia, Mozarteum di Salisburgo, Rome, Milan, Paris, London, Berlin, Wien, Madrid, Bruxelles, Buenos Aires, Caracas, Riga, Birmingham, Köln, Lyon, St. Etienne (Francia), Principaute-Monaco-Monte-Carlo, Yeosu (Korea), Kawasaki, Adis Abeba, Chisnau, Taormina, Ravello. He has collaborated with Claude Delangle, Alda Caiello and Bruno Canino and many of the most important composers wrote for him more than a hundred works. As a soloist and with the Quartetto di Sassofoni Accademia he has recorded for the Nuova Era, Dynamic, Rai Trade and Cesmel. He has published studies for Riverberi Sonori and he direct a collection for the Sconfinarte editions.}

\biografia{Enzo Filippetti}{è professore di Sassofono al Conservatorio “S. Cecilia” di Roma e da più di trent’anni tiene concerti in tutto il mondo. Si è esibito alla Biennale di Venezia, al Mozarteum di Salisburgo, a Roma, Milano, Parigi, Londra, Berlino, Vienna, Madrid, Bruxelles, Buenos Aires, Caracas, Riga, Birmingham, Köln, Lyon, St. Etienne (Francia), Principato di Monaco, Yeosu (Korea), Kawasaki, Adis Abeba, Chisnau, Taormina, Ravello. Ha collaborato con Claude Delangle, Alda Caiello e Bruno Canino e molti tra i più importanti compositori hanno scritto per lui più di cento opere e gli hanno affidato numerose prime esecuzioni. Come solista e con il Quartetto di Sassofoni Accademia ha inciso per Nuova Era, Dynamic, Rai Trade e Cesmel. Ha pubblicato studi per Riverberi Sonori e cura una collana per le edizioni Sconfinarte.}


\biografia{Alessia Forganni}{Alessia Forganni (Brescia, 1982) si diploma in pianoforte presso il conservatorio Luca Marenzio, sotto la guida del M° M. Zana, e si laurea al D.A.M.S. - Indirizzo Cinema e Audiovisivi. Dal 2007 vive e insegna a Roma: negli ultimi anni all’attività classica ha affiancato un approccio moderno allo strumento: con il duo pianistico Duel, tra il 2009 al 2015 si è esibita in Europa, Russia, Libano e Sudafrica. Attualmente sta ultimando il triennio di Musica Elettronica presso il conservatorio Santa Cecilia: la sua ricerca compositiva è volta a una personale messa in relazione tra il background classico, l’esplorazione improvvisativa, l’utilizzo della voce e le istanze contemporanee.}

\biografia{Alessia Forganni}{Alessia Forganni (Brescia, 1982) si diploma in pianoforte presso il conservatorio Luca Marenzio, sotto la guida del M° M. Zana, e si laurea al D.A.M.S. - Indirizzo Cinema e Audiovisivi. Dal 2007 vive e insegna a Roma: negli ultimi anni all’attività classica ha affiancato un approccio moderno allo strumento: con il duo pianistico Duel, tra il 2009 al 2015 si è esibita in Europa, Russia, Libano e Sudafrica. Attualmente sta ultimando il triennio di Musica Elettronica presso il conservatorio Santa Cecilia: la sua ricerca compositiva è volta a una personale messa in relazione tra il background classico, l’esplorazione improvvisativa, l’utilizzo della voce e le istanze contemporanee.}


\biografia{FREI}{FREI is a project of Paolo Gatti and Francesco Bianco, created to investigate the aspects related to the live creation. Improvisation and experimentation are the starting points of their poetics: offering an experience of live electronics in an active way. The live performance is based on selected elements, which, during the show, are processed and developed. The instrumentation used is constituted by two laptops with digital systems made by the own composers.}

\biografia{FREI}{FREI è un progetto di Paolo Gatti e Francesco Bianco, nato nel 2014. Si sono esibiti al Circolo Dal Verme (Studiolo Laps Showcase), al teatro Tor Bella Monaca (Slaps-pourri.1 anteprima). L'improvvisazione è alla base della poetica di Frei. La performance live è basata su elementi preordinati, i quali, durante lo spettacolo, vengono elaborati e sviluppati. La strumentazione è costituita da due laptop sui quali sono vi sono sistemi digitali programmati degli stessi componenti del duo.}


\biografia{Javier Alejandro Garavaglia}{Composer/performer (viola & electronics). Associate Professor at the CASS Faculty, London Metropolitan University (UK). Compositions –performed in several countries of Europe, the Americas and Asia– include: acousmatic, audio-visual, solo/chamber/ensemble/orchestral works with or without the inclusion of electronic/interactive media. Some electroacoustic works commercially available CD releases (Germany, USA, Argentina and Denmark). Research topics: dramaturgy of music; full automation of live electronics; special sound diffusion systems. http://icem-www.folkwang-hochschule.de/~gara/ }

\biografia{Javier Alejandro Garavaglia}{Compositore/Performer (Viola ed elettronica). Professore associato alla facoltà di CASS, Università di Londra. Le Composizioni e le performance eseguite in molti paesi dell'Europa, Dell'America, e dell'Asia, includono: Acusmatici, audio-visual, lavori per solo/camera/ensemble e orchestra, con o senza l'inclusione di elettronica e media interattivi. Alcune opere elettroacustiche sono disponibili su CD (Germania, USA, Argentina e Danimarca). Settori di ricerca: drammaturgia musicale; automazioni del live Electronics; diffusione speciale di sistemi audio.}


\biografia{Jorge García del Valle Méndez}{Jorge García del Valle Méndez (1966*) grew up in Spain, where he studied bassoon and composition. Now he lives in Dresden (Germany) where he studied composition and electronic music. His compositions are worldwide premiered and broadcasted. Commissions from international institutions. Works on digital analysis and sound processing, applied to theory and composition. Salvatore Martirano Composition Award University of Illinois, Composition Award Sächsischer Musikrat.}

\biografia{Jorge García del Valle Méndez}{(1966) è cresciuto in spagna, dove ha studiato fagotto e composizione. Ora vive a Dresda (Germania) dove ha studiato composizione e musica elettronica. Le sue composizioni hanno avuto prime mondiali in tutto il mondo, commissioni da importanti istituzioni internazionali. Lavori di analisi digitale e "sound processing", applicati alla teoria e alla composizione. Premio di composizione Salvatore Martirano dellUniversità dell'Illinois, e premio di composizione Sächsischer Musikrat.}


\biografia{Paolo Gatti}{born in Rome in 1982.He took the B.Sc. degree in environmental engineering and a post graduate master in sound engineering at "Tor Vergata" University. Then he studied computer music at "Santa Cecilia" conservatory, taking the B.A. degree under the guidance of G.Nottoli, and the M.A. degree under the guidance of M.Lupone and N. Bernardini. He's a composer, teacher and researcher in the field of musical expressivity. His works are performed in important events and international festivals. He wrote music for theatre, poetry and dance performances. His work Poltergeist was one of the awarded compositions at the end of the national final of the "Claudio Abbado" prize.}

\biografia{Paolo Gatti}{Laureato in ingegneria, consegue il master in ingegneria del suono presso l'università di Roma "Tor Vergata". Successivamente si laurea a pieni voti in musica elettronica presso il Conservatorio Santa Cecilia di Roma, sotto la guida di G.Nottoli,M.Lupone,N.Bernardini.Compositore, didatta e ricercatore, suoi lavori sono eseguiti in importanti manifestazioni e festival internazionali. Scrive musiche per spettacoli teatrali e rassegne poetiche. Nel 2015, la sua composizione Poltergeist, risulta fra i brani premiati al termine della finale nazionale del premio delle arti "Claudio Abbado".}


\biografia{Núria Giménez-Comas}{She studied composition at the Escola Superior de Musica de Catalunya (ESMUC). After two years she continued her training at the Geneva Conservatory, where she studied composition with Luis Naon, and electroacoustic and instrumental with Michael Jarrell. She studied at Institut de Recherche et Coordination Acoustique/Musique (IRCAM) for two years, where she explored different type of synthesis and new spatialsiation system in 3D ambisonics. Núria has worked with musicians such as Harry Sparnaay, Klangforum's Wien trio, Ensemble Contrechamps, Brussels Philarmonic, and Diotima Quartet. She is a founding member of Ensemble Matka.}

\biografia{Núria Giménez-Comas}{Ha studiato composizione presso la Escola Superior de Musica de Catalunya (ESMUC). Dopo due anni ha continuato la sua formazione presso il Conservatorio di Ginevra, studiando composizione con Luis Naon ed elettroacustica con Michael Jarrell. Ha studiato presso l'Institut de Recherche et Coordination Acoustique / Musique (IRCAM) per due anni, dove ha esplorato diversi tipi di sintesi e il nuovo sistema di spazializzazione in Ambisonics 3D. Núria ha lavorato con musicisti come Harry Sparnaay, Wien trio di Klangforum, Ensemble Contrechamps, Bruxelles Philharmonic, e Diotima Quartetto. È membro fondatore di Ensemble Matka.}


\biografia{Virginia Guidi}{Graduated in Singing and Music Vocal Chamber Music at the Conservatory S. Cecilia where she specialized with honors in Music Vocal Chamber with S. Schiavoni with a thesis on the relationship between performer and composer in electroacoustic music. She has performed in Beijing– National Centre of the Performing Arts; Roma –Accademia Filarmonica, Tecnopolo, GNAM, Macro, MAXXI; Napoli – Arena Flegrea; Catania – Teatro Metropolitan, on national television (RAI 1, RAI 2, RAI 5, Telepace), and in important Festivals (EMUfest, ArteScienza). She sings with the Voxnova Italia “In the midst of things” Allora&Clazadilla’s with G. Coleman’s music at the Venice’s Biennale 2015.}

\biografia{Virginia Guidi}{Diplomata in Canto e in Musica Vocale da Camera al Conservatorio S. Cecilia, ivi specializzata con lode con S. Schiavoni con una tesi sperimentale sul rapporto tra interprete e compositore nella musica elettroacustica. Canta in Italia e all’estero (Pechino – National Centre of the Performing Arts; Roma –Accademia Filarmonica, Tecnopolo, GNAM, Macro, MAXXI; Napoli – Arena Flegrea; Catania – Teatro Metropolitan), su reti nazionali (RAI 1, RAI 2, RAI 5, Telepace) e in importanti Festival (EMUfest, ArteScienza). Con Voxnova Italia canta “In the Midst of Things” di Allora&Calzadilla, musica di G. Colemann, per la Biennale Arte 2015.}


\biografia{Jan Jacob Hofmann}{Born 1966. Diploma, branch of architecture at the University Of Applied Sciences, Frankfurt in 1995. 1997 second diploma at the class of Peter Cook and Enric Miralles at the Städelschule, Art School Frankfurt in 1995, a postgraduate class of conceptual design and architecture. Associate researcher at the "Signal Processing Applications Research Group", University of Derby, England. Nominated for the German Prize For Sonic Arts of the Museum for Sculpture "Glaskasten" of the City of Marl in 2006.}

\biografia{Jan Jacob Hofmann}{Diplomato in architettura alla Fachhochschule di Francoforte sul Meno nel 1995. Nello stesso anno viene ammesso al corso di Peter Cook ed Enric Miralles alla Städelschule, Scuola Superiore di Formazione Artistica di Francoforte per un corso post- universitario di architettura e progettazione concettuale. Diplomato nel 1997. Lavora come compositore, architetto e successivamente come fotografo. 
Dall' estate 2005 è "Associate Researcher" alla "Signal Processing Applications Research Group", Universita di Derby, Inghilterra. 
È venuto nominato al consiglio di amministrazione della Società Tesesca per la Musica Elettroacustica, DEGEM.}


\biografia{Sandro L'Abbate}{Class 1988. Graduated in photography at Fine Arts Academy in Italy. He is interested in audiovisual production, using interactive and electronics systems to observe physical phenomena. He is currently facing the sea. Web:http://sandrolabbate.altervista.org}

\biografia{Sandro L'Abbate}{Sandro L'Abbate, classe 1988. Diplomato in fotografia all'Accademia di Belle Arti di Rimini in Italia. È interessato alla produzione audio- visuale usando sistemi elettronici ed interattivi per osservare fenomeni fisici. Al momento si trova di fronte al mare. Web:http://sandrolabbate.altervista.org}


\biografia{Silvia Lanzalone}{composer (Salerno, 1970). Degree in Flute, Composition and Electronic Music at Conservatory of Salerno, L’Aquila and Rome. Her compositions are performed in Italy and abroad and are published by Ars Publica, Taukay, Suvini Zerboni. She works since 1997 at CRM - Centro Ricerche Musicali as musical assistant, researcher, composer. She is Professor of Electroaclustic Composition and Head of the Department of New Technology and Musical Languages of the Salerno Conservatory. (http://www.silvialanzalone.it/)}

\biografia{Silvia Lanzalone}{compositrice (Salerno 1970). Diploma di Flauto, Composizione e Musica Elettronica presso i Conservatori di Salerno, L’aquila e Roma. Sue composizioni sono edite da Ars Publica, Taukay e Suvini Zerboni e sono eseguite in festivals nazionali ed internazionali. Dal 1997 collabora con il CRM - Centro Ricerche Musicali di Roma. E’ Docente di Composizione Musicale Elettroacustica e Coordinatore del Dipartimento di Nuove Tecnologie e Linguaggi Musicali presso il Conservatorio “G. Martucci” di Salerno. (http://www.silvialanzalone.it/)}


\biografia{Jean-Francois Laporte}{http://www.jflaporte.com}

\biografia{Jean-Francois Laporte}{Compositore, esecutore ed inventore di strumenti musicali. Attivo sulla scena artistica contemporanea dalla fine degli anni ’90, l’artista canadese ha un approccio creativo ibrido, che unisce assieme sound art, composizione, interpretazione, performance, installazione sonora e arte digitale. Artista piuttosto intuitivo, ha appreso l’arte attraverso sperimentazioni concrete con la materia, basando il suo approccio alla composizione sull’ascolto attivo e sull’attenta osservazione della realtà di ciascun fenomeno. Nel corso degli anni ha dedicato una grande quantità di energie all’invenzione, lo sviluppo e la costruzione di nuovi strumenti musicali. È fondatore e direttore artistico delle produzioni Totem Contemporain di Montreal.
http://www.jflaporte.com}


\biografia{Gy\"orgy Ligeti}{eng}

\biografia{Gy\"orgy Ligeti}{ita}


\biografia{Jones Margarucci}{Electroacoustic music composition at the State Conservatory of Music “G. Martucci” and at KMH (Royal College of Music Stockholm). His music has been played in several festivals in Europe and North America, and has been selected for: Redshift Music - Postal Pieces (Vancouver – Canada – 2013); Vox Novus Fifteen Minutes of Fame - Yumi Suehiro (New York City – USA – 2014); Sonorities Festival 2015 (Belfast – North Ireland – 2015); SOUNDkitchen’s Earspace/Frontiers Festival 2015 (Birmingham – UK – 2015); Video Remakes - Call for Tape Music (La Fabbrica del Vedere) (Venice - Italy - 2015)}

\biografia{Jones Margarucci}{Ha studiato Composizione Elettroacustica presso il Conservatorio di Salerno e come exchange student presso la Royal Academy of Music (KMH) a Stoccolma. 
Sue musiche sono state eseguite in diversi festival in Europa e in Nord America e sono state selezionate per: Redshift Music - Postal Pieces (Vancouver – Canada – 2013); Vox Novus Fifteen Minutes of Fame - Yumi Suehiro (New York City – USA – 2014); Sonorities Festival 2015 (Belfast – North Ireland – 2015); SOUNDkitchen’s Earspace/Frontiers Festival 2015 (Birmingham – UK – 2015); Video Remakes - Call for Tape Music (La Fabbrica del Vedere) (Venice - Italy - 2015)}


\biografia{Marco Marinoni}{(10/April/1974) is a professor at the Conservatory of Music “G. Verdi” of Como (Italy) where he teaches Electroacoustic Performance Practice. He gained a M. Mus. Conservatory Degree in Computer Music (2007) 10/10 cum laude, a Master’s Degree in Sound Direction and Live-Electronics at the Conservatory of Music “B. Marcello” of Venice (2007), 110/110 cum laude with Alvise Vidolin and a Master’s Degree in Composition (2013), 110/110 cum laude.}

\biografia{Marco Marinoni}{(10/April/1974) è docente di Esecuzione ed interpretazione della musica elettroacustica presso il conservatorio G. Verdi di Como.}


\biografia{Raffaele Marsicano}{eng}

\biografia{Raffaele Marsicano}{ita}


\biografia{Francesc Martí}{Francesc Martí is a mathematician, computer scientist, composer, sound and digital media artist born in Barcelona and currently living in the UK. As composer and video artist, his works have been performed or exhibited all over the world, including international festivals, events and exhibitions. Currently, he combines his artistic and technology projects with his teaching Audio Technology and Image at Open University of Catalonia, and Music Technology at the De Montfort Uni. of Leicester.}

\biografia{Francesc Martí}{Francesc Martí è un matematico, informatico, compositore, sound and digital artist, nato a Barcellona e vive attualmente nel Regno Unito. Come compositore e digital media artist, i suoi lavori sono stati eseguiti o ha fatto esibizioni in tutto il mondo, inclusi feltival internazionali, eventi e manifestazioni. Attualmente, combina i suoi progetti artisti e tecnologici con l'insegnamento di Audio technology and Image alla open university della California, e Music Technology alla Montfort University di Leichester.}


\biografia{Mario Mary}{is a Doctor of “Aesthetic, Science and Technology of Arts” (University Paris VIII, France), actually he teaches "Electroacoustic Composition" at Academy Rainier III in Monaco, and is the artistic director of Monaco Electroacoustique – Electroacustic Music International Encounter. Between 1996 and 2010, he teaches at the University Paris VIII. He worked as a composer in research at the IRCAM. Teacher, researcher and composer, Mario MARY has been invited by numerous institutions to make compositions and to give conferences. His music has been distinguished in more than twenty composition competitions. http://ipt.univ-paris8.fr/mmary/}%foto

\biografia{Mario Mary}{Mario MARY dottor in "Estetica, Scienza e Tecnologia delle Arti" (Università di Parigi VIII, Francia), Professore di Composizione di Musica Elettroacustica presso Academia Ranieri III di Monte-Carlo, e Direttore artistico di Monaco Electroacoustique - Incontri Internazionali di Musica Elettroacustica. Ha lavorato come ricercatore presso l'IRCAM e insegnato all'Università Parigi VIII, Ha vinto una ventina di premi in concorsi di composizione. Ha dato nomerosos conferenze e corsi in diversi paesi.}%foto


\biografia{Massimiliano Mascaro}{Composer. He was born in Rome in 1986. He studies with M° Michelangelo Lupone and M° Nicola Bernardini. He studied at the Conservatory "A. Casella" in L'Aquila and He currently attends The Concervatory of "S. Cecilia " in Rome. He attends courses of Electroacoustic Composition and Classical Composition. The Electroacoustic music is the field in which he mainly carries out his musical activity.}

\biografia{Massimiliano Mascaro}{Compositore. Nato a Roma nel 1986. Allievo del M° Michelangelo Lupone e del M° Nicola Bernardini, si è formato presso il Conservatorio “A. Casella” di L'Aquila e successivamente presso il Conservatorio “S. Cecilia” di Roma affrontando gli studi della Composizione elettroacustica e della Composizione classica. La musica elettroacustica è il settore nel quale svolge la sua principale attività musicale.}


\biografia{Massimo Massimi}{Massimo Massimi, developed his music backgrounds at the Santa Cecilia Conservatory of Rome, graduated in lute and electroacoustic composition. After dealing with early music, he has dedicated himself to electronic music, with special attention to instrument and machine interaction.}

\biografia{Massimo Massimi}{Massimo Massimi si è formato musicalmente presso il Conservatorio Santa Cecilia di Roma, diplomato in liuto e musica elettronica, ha affrontando lo studio della musica antica e successivamente si è dedicato alla composizione elettroacustica con particolare attenzione all’interazione tra strumento e macchina.}


\biografia{Antonio Mazzotti}{Antonio Mazzotti graduated in Electronic Engineering at Polytechnic of Bari (Italy) and I received a degree of specialization in Signal Processing. Later, I continued in academic studies at Conservatory of Bari, where I graduated cum laude in Electronic Music, under the guidance of F. Scagliola. My compositions have been performed at festivals:Fimu Festival 2012,Silence Festival 2012, New York City E.M.Festival 2013-14,ICMC-SMC 2014,ICMC-SMC 2014,file.org.br 2015,uvm2015.unb.br,ICMC 2015.}

\biografia{Antonio Mazzotti}{ si è laureato in Ingegneria Elettronica presso il Politecnico di Bari e  specializzato in Signal Processing.  Ha proseguito gli studi accademici presso il Conservatorio di Bari, dove si è laureato con lode in Musica Elettronica, sotto la guida del M ° F. Scagliola. I suoi interessi spaziano sulla composizione, assistita da calcolatore, di lavori elettroacustici e audiovisivi. Alcune sue composizioni sono state eseguite in vari festival internazionali come  ‘FIMU Festival 2012’, ‘Silence Festival 2012’, ‘New York City EMFestival 2013/14’, ‘ICMC-SMC 2014’, ‘ file.org.br 2015’, uvm2015.unb.br,  ICMC 2015.}


\biografia{Ursula Meyer-König}{, lives in Zurich. After a career as a pediatrician, she undertook foundation and media art studies at the HGKZ in Zurich and the FH Aarau, Switzerland, followed by a continuation course in electro-acoustic composition at the Hochschule für Musik in Weimar, Germany under Prof. R. Minard. She is currently studying electroacoustic composition under Prof. G. Toro-Pérez at ZHdK and ICST, Zurich, Switzerland.}

\biografia{Ursula Meyer-König}{Vive a Zurigo. Dopo una carriera come pediatra, ha intrapreso gli studi base di arte e media al HGKZ di Zurigo e la FH di Aarau, in Svizzera, seguito da un corso di composizione elettroacustica al Hochschule für Musik in Weimar, Germania, con il Prof. R. Minard. Attualmente studia composizione elettroacustica con il Prof. G. Toro- Pérez a ZHdk e ICST, Zurigo, Svizzera.}


\biografia{Enrico Minaglia}{This is the third piece of the Physis series, a tribute to Giordano Bruno's philosophy. Physis III is focused on the "De gli eroici furori dialogue". It aims to depict in music the calm life of the wise man, and its sudden tranformation into the "furioso", a man who struggles to penetrate the misteries of Nature, and in the end will pay this knowledge with his life, like the mythical hunter Actaeon, and Giordano Bruno himself.}

\biografia{Enrico Minaglia}{Si tratta del terzo brano della serie "Physis", un tributo al pensiero di Giordano Bruno. Physis III è basata sul dialogo "De gli eroici furori". Ciò che la musica rappresenta è la vita tranquilla del saggio, e la sua improvvisa trasformazione nel "furioso", un uomo che lotta per penetrare i misteri della natura e che pagherà la sua conoscenza con la sua stessa vita, come la cacciatrice Actaeon, e Giordano bruno stesso.}


\biografia{Kenn Mouritzen}{Born in Copenhagen (DK) in 1972. Lives and works in Vienna (A) since 2007. He studied composition of electroacoustic music with Germán Toro-Perez and Martin Neukom at ZHdK in Zürich, Switzerland (until 2015). He also holds a Master's Degree in Comparative Literature and Philosophy (2004). Recently his music has been featured at EMU Festival, Musicacustica Beijing, Noisefloor Festival, Festival Archipel, RIME, NYCEMF. He was funded by the Danish Agency for Culture. Selection price at Bourges.}

\biografia{Kenn Mouritzen}{Nato a Copenaghen (Danimarca) nel 1972. Vive e lavora a Vienna (A ) dal 2007. Ha studiato composizione elettroacustica con Germán Toro - Perez e Martin Neukom a ZHdK a Zurigo, Svizzera (fino al 2015). Ha inoltre conseguito un Master in letteratura comparata e Filosofia (2004). Recentemente la sua musica è stata presentata ai Festival UEM, Musicacustica Pechino, Noisefloor Festival, Festival Archipel, RIME, NYCEMF. È stato finanziato dalla Agenzia danese per la Cultura. Premio selezione a Bourges.}


\biografia{Roberto Musanti}{Roberto Musanti, electronic musician and media artist. His works have participated to “Opera Nuda” Amsterdam, “Zeppelin” Barcelona, “UVM Symposium” Brasilia, “Kontakte” / “Music in touch” Cagliari, “Musica Viva” Lisboa, “Video Evening Photon Gallery” Lubjiana, ”MediaDepo” Lviv, “Electronicittà” Marseille, “Konsequenz” Napoli, “Decennale CEMAT”, “Saturazioni”, “EMUFest” Roma, “File Festival” Sao Paulo, “Simultan” Timisoara.}

\biografia{Roberto Musanti}{Roberto Musanti, musicista e media artist, autodidatta, diplomato in musica elettronica, insegna laboratorio di informatica e linguaggi di programmazione per la multimedialità. I suoi suoi lavori sono stati presentati, tra gli altri, ai festival “Zeppelin” Barcellona, “U.V.M.” Brasilia, “Kontakte” / “Music in touch” Cagliari, “Musica Viva” Lisbona, “Video Evening Photon Gallery” Lubjiana, ”MediaDepo” Lviv, “Electronicittà” Marseille, “Konsequenz” Napoli, “Decennale CEMAT”, “Saturazioni”, “EMUFest” Roma, “File Festival” Sao Paulo, “Simultan” Timisoara.}


\biografia{Giorgio Nottoli}{Giorgio Nottoli (composer, born 1945 in Cesena, Italy) he was Professor of Electronic Music at the Conservatory of Rome "Santa Cecilia" until 2013. He currently teaches electroacoustic composition at the University of Rome "Tor Vergata". The major part of his works are realized by means of electro-acoustic media both for synthesis and processing of sound. The objective is to make timbre the main musical parameter and a "construction unit" through the control of sound microstructure. In the works for instruments and live electronics, the aim of Giorgio Nottoli is to extend the sonority of the acoustic instruments by means of complex real time sound processing. He has designed both analog and digital musical systems in conjunction with various universities and research centers.}

\biografia{Giorgio Nottoli}{Giorgio Nottoli (compositore, nato a Cesena, Italia nel 1945) è stato docente di Musica Elettronica al Conservatorio di Roma “S.Cecilia” sino al 2013. Attualmente è docente di Composizione elettroacustica all’Università di Roma “Tor Vergata”. La maggior parte delle sue opere utilizza mezzi elettronici sia per la sintesi che per l'elaborazione del suono. Il centro della sua ricerca di musicista riguarda il timbro concepito quale parametro principale e "unità costruttiva" delle sue opere attraverso la composizione della microstruttura del suono. Nei suoi lavori per strumenti ed elettronica Giorgio Nottoli punta ad estendere la sonorità degli strumenti acustici mediante complesse elaborazioni del suono. Ha progettato vari sistemi elettronici per la musica utilizzando sia tecnologie analogiche che digitali in collaborazione con varie università e centri di ricerca.}


\biografia{Benjamin O'Brien}{composes, researches, and performs acoustic and electro-acoustic music that focuses on issues of translation and machine listening. He holds a Ph.D in Music at the University of Florida, a MA in Music Composition from Mills College, and a BA in Mathematics from the University of Virginia. His work is published by Oxford University Press, Taukay Edizioni Musicali, Canadian Electroacoustic Community, and SEAMUS. He lives in Marseille, France.}

\biografia{Benjamin O'Brien}{compone, ricerca, ed esegue musica acustica e elettroacustica che si concentra su questioni di trasformazione e l'ascolto delle macchine. Ha conseguito un dottorato in musica presso l'Università della Florida, un MA in composizione musicale al Mills College, e una laurea in Matematica presso l'Università della Virginia. La sua opera è pubblicata dalla Oxford University Press, Taukay Edizioni Musicali, canadese Elettroacustica Comunità, e Seamus. Vive a Marsiglia, in Francia.}


\biografia{João Pedro Oliveira}{João Pedro Oliveira completed a PhD in Music at the University of New York at Stony Brook. He has received numerous prizes and awards, including three Prizes at Bourges Electroacoustic Music Competition, the prestigious Magisterium Prize in the same competition, the Giga-Hertz Special Award, 1st Prize in Metamorphoses competition, etc.. He is Professor at Federal University of Minas Gerais (Brazil) and Aveiro University (Portugal).}

\biografia{João Pedro Oliveira}{João Pedro Oliveira ha completato il suo Dottorato di Ricerca in Musica all' Università Stony Brook di new York. ha ricevuto numerosi premi e riconoscimenti, inclusi tre premi alla Bourges Electroacoustic Music Competition, e il prestigioso Magisterium Prizes nella stessa competizione, il Giga-Hertz Special Award, il primo premio in Metamorphoses competition, etc.. He insegnante presso l'Università Federale di Minais Gerais (Brasile) e all'Università Aveiro (Portogallo).}


\biografia{Daniel Osorio}{Born in Santiago de Chile. In 1996 he begins Composition Studies with Prof. Pablo Aranda and Electroacoustic Music with Prof. Edgardo Cantón and Rolando Cori at the University of Chile. In 2005 he is granted a scholarship (Beca Presidente de la República - MIDEPLAN ) by the Government of Chile and moves to Saarbrücken/Germany where he starts his postgraduate studies in the field Composition with Prof. Theo Brandmüller, Dr. Prof. Stefan Litwin and Stefan Zintel at Hochschule für Musik Saar.}

\biografia{Daniel Osorio}{Nato a Santiago del Cile. Nel 1996 inizia gli studi di composizione con il Prof. Pablo Aranda e di musica elettroacustica con il Prof. Edgardo Canton e Rolando Cori presso l'Università del Cile. Nel 2005 gli viene concessa una borsa di studio (Beca Presidente de la República - MIDEPLAN) da parte del Governo del Cile e si trasferisce a Saarbrücken / Germania, dove inizia i suoi studi post-laurea in Composizione con il Prof. Theo Brandmüller, il Dr. Prof. Stefan Litwin e Stefan Zintel alla Hochschule für Musik Saar.}


\biografia{Davide Palmentiero}{Born in Salerno on May 19, 1993. Six years later he started playing classical guitar, before moving to the electric guitar. At the age of 13, he began playing and recording with various bands and artists of any kind. At 19 he began to work with electronic music and a year later he enrolled at the Conservatory of Naples; here shows particular interest in radical  improvisation, especially experiencing software and technical issues relating guitar. It builds and continuously develops his instrument, playing it in various festivals, exhibitions and other contexts both solo and with other formations and different artists, including Bob Ostertag.}

\biografia{Davide Palmentiero}{Nasce a Salerno il 19 Maggio 1993. Sei anni dopo inizia a suonare la chitarra classica, per poi passare alla chitarra elettrica all’età di 13 anni, iniziando a suonare e registrare con varie band e artisti senza distinzioni di genere. A 19 anni inizia ad affacciarsi alla Musica Elettronica e un anno dopo si iscrive al Conservatorio di Napoli; qui mostra particolare interesse per l’improvvisazione radicale, sperimentando soprattutto applicazioni e tecniche riguardanti la chitarra. Costruisce e sviluppa continuamente il proprio strumento, con il quale si esibisce in vari festival, rassegne e altri contesti sia in solo che con con diverse formazioni e diversi artisti, tra i quali Bob Ostertag.}


\biografia{Alessandro Pace}{Graduated in Flute with  M°Carlo Morena with score of "110 e lode" at the music academy of Santa Cecilia.
Now following his studies with flute and traditional composition in the same music academy. He did and does a conspicuous concerts activity of various genres.
He plays in the following ensembles: Orchestra Ars Ludi Romana (also as a soloist); Broadway Musical Orchestra (es. Festival di Todi); Indivenire Ensemble (contemporary repertoire). He played with the Panama's national orchestra in Panama City. He performed in the Contaminazioni festival both as flutist and composer. He joined the project of M° Antonio Di Pofi  about Silent film's music, both composing and playing for it. He played quite a lot of chamber music with various ensembles and continuously search for new experiences. First time (en)joining EMUFest.}

\biografia{Alessandro Pace}{si laurea in Flauto con il M°Carlo Morena con la votazione di 110 e lode presso il Conservatorio di Santa Cecilia di Roma. Prosegue gli studi in flauto, affiancati dagli studi in Composizione tradizionale nello stesso conservatorio. Ha fatto e continua a fare molti concerti nei vari generi. Suona con diversi ensemble: Orchestra Ars Ludi Romana (anche come solista); Broadway Musical Orchestra (es. Festival di Todi); Indivenire Ensemble (repertorio contemporaneo). Ha suonato nell'orchestra nazionale di Panama a Panama City. Ha preso parte al festival Contaminazioni sia come flautista che come compositore. Ha seguito il progetto del M° Antonio Di Pofi sulla musica dei film muti (anche qui sia come flautista che compositore). Suona molta musica da camera in diverse formazione ed è in continua ricerca di nuove esperienze. Per la prima volta si esibirà ad EMUFest.}


\biografia{Carlos D. Perales}{His works have been awarded at international competition contest likÈMiniaturas Electroacústicas' - Confluencias (Huelva, 2008), Laboratorio del Espacio LIEM-CDMC (Madrid, 2010), XXII Composition Contest SGAE (Madrid, 2011), Toy Piano World Summit (Luxembourg, 2012), Musica Nova (Prague, Czech republic, 2012), Luigi Russolo (France, 2012), Fundación Destellos (Argentine, 2013). PhD by Universidad Politécnica de Valencia. Since 2014 lectures Composition & Electroacoustic music at CSMCLM.}

\biografia{Carlos D. Perales}{Le sue opere sono state premiate ai concorsi internazionali 'Miniaturas Electroacústicas' - Confluencias (Huelva, 2008), Laboratorio del Espacio LIEM-CDMC (Madrid, 2010), XXII Concorso di Composizione SGAE (Madrid, 2011), Toy Piano Summit Mondiale (Lussemburgo, 2012), Musica Nova (Praga, Repubblica Ceca, 2012), Luigi Russolo (Francia, 2012), Fundación Destellos (Argentina, 2013). Ha conseguito il dottorato presso l'Universidad Politécnica di Valencia. Dal 2014 tiene lezioni di Composizione Elettroacustica al CSMCLM.}


\biografia{Alessandro Pirchio}{He studies at the Santa Cecilia Conservatory in Rome with M° Albanese. He has performed as soloist and in chamber music ensembles for: Musica a Roma per Roma”; “Sutri Beethoven Festival”; Chamber Music season of the Viterbo Ceramic Museum. Moreover, he played for the theatrical performance “Twelfth Night” (“Le maschere del teatro 2015” Award for original soundtrack by M° Piovani) in several Italian theaters (“Donizetti” in Bergamo, “Ponchielli” in Cremona, “Verdi” in Padova, Pistoia and Ravenna, among the most important). He is currently First Flute in the band of the Vatican Gendarmerie and Carabinieri National Association.}

\biografia{Alessandro Pirchio}{Studia presso il Conservatorio di Santa Cecilia con il M° Franz Albanese. Ha partecipato da solo o in formazioni cameristiche a la Rassegna “Musica a Roma per Roma”; il “Sutri Beethoven Festival; Stagione cameristica del Museo della ceramica di Viterbo. Ha suonato per lo spettacolo “La dodicesima notte” (Premio “Le maschere del teatro 2015” per le musiche originali del M° Piovani) in numerosi teatri italiani (Donizzetti di Bergamo, Ponchielli di Cremona, Verdi di Padova sono tra i più importanti). Attualmente ricopre la parte di Primo Flauto nella Banda della Gendarmeria Vaticana e dell’Ass. Nazionale Carabinieri.}


\biografia{Davide Palmentiero}{Born in 1991. He began his musical activity as a drummer, studying with Salvatore Tranchini. Always interested in  extreme and noisy music, was focused initially on black metal and hardcore. During his stay in Norway, he developes a passion for electronic music and starts a militancy in collective techno Stavanger Teknomune, apostles of rave culture that sets theirs roots in the use of analog instruments and vinyl. After two years he decided to continue his musical studies at an academic level, returning to Naples, and signing up to the course of Electronic Music with Elio Martusciello. Today his research is based on the aesthetic qualities of the noise, explored in the elements of everyday life. Currently he plays drums with the italian music group Bestia Carenne.}

\biografia{Davide Palmentiero}{Nato nel 1991. Inizia la sua attività musicale come batterista studiando da privatista con il M° Salvatore Tranchini. Da sempre interessato alle sonorità più estreme e rumorose, si concentra inizialmente sul black metal e sull'hardcore, per cambiare poi indirizzo in seguito alla sua permanenza in Norvegia dove si appassiona alla musica elettronica extra-colta ed inizia la militanza nel collettivo techno Stavanger Teknomune, alfieri della cultura rave che getta le sue basi nell'utilizzo di strumenti analogici e del vinile. Dopo due anni decide di proseguire i suoi studi musicali a livello accademico, tornando a Napoli e iscrivendosi al triennio di musica elettronica con il M° Elio Martusciello. Ad oggi è attivo nell' estetica del rumore che egli ricerca negli elementi della vita quotidiana. Attualmente suona la batteria con il gruppo La Bestia Carenne.}


\biografia{Maurizio Pisati}{Born in Milan in 1959, it’s present with his works in “Festival of Europe”, Australia, Usa, Japan, Latin America. His compositions have been awarded in national and international contests (between them: Bucchi ’83; Contilli ’83, Rass, B.Brecht ’85, Gaudeamus ’86, ICONS ’86, Petrassi ’89), they are published by Casa-Ricordi and transmitted by radio stations in the whole world. His works have been recorded on many different labels such as Ricordi-Fonit, Cetra, Edipan, BMG, CavalliRecordsBamberg, Victor, Limen, ArsPublica, SiltaClassics and LArecords, independent label founded by himself in 1997. His musical studies have been achieved at the Conservatorio di Milano, and integrated with summer camps in Darmstadt and in the Accademia di Città di Castello, he had his diploma with highest honours in composition with S. Sciarrino, A. Guarnieri and G. Manzoni. After that he got a diploma in Guitar, by playing concerts in Europe from 1983 to 1989 with the band Laboratorio Trio. At the conservatory of Bologna teaches Composition for Applied Music, Elements of Composition for educational purpose, Invention & Interpretation, and in the same place he founded CRS- Center for musical research, in 2014.}

\biografia{Maurizio Pisati}{Nato a Milano nel 1959, è presente con propri lavori in festival d’Europa, Australia, USA, Giappone, America Latina. Sue composizioni sono state premiate in concorsi nazionali e internazionali (tra cui: Bucchi’83; Contilli’83; Rass. B. Brecht’85; Gaudeamus’86; ICONS’86; Petrassi’89), sono pubblicate da Casa-Ricordi, trasmesse da emittenti radiofoniche europee ed extraeuropee, sono incise su CD Ricordi-Fonit Cetra, Edipan, BMG, CavalliRecordsBamberg, Victor, Limen, ArsPublica, SiltaClassics e LArecords, etichetta indipendente da lui fondata nel 1997. ha compiuto gli studi musicali al Conservatorio di Milano, oltre che ai corsi estivi di Darmstadt e all’Accademia di Città di Castello, diplomandosi con il massimo dei voti in Composizione con S.Sciarrino, A. Guarnieri e G.Manzoni, e in seguito anche in Chitarra svolgendo attività concertistica in Europa dal1983 al1989 col gruppo Laboratorio Trio. Al Conservatorio di Bologna insegna di Composizione per la Musica Applicata, Elementi di Composizione per la Didattica, Invenzione&Interpretazione, e nella stessa sede nel 2014 fonda CRS-Centro di ricerche musicali.}


\biografia{Karen Power}{Everyday environments and how we hear everyday sounds lies at the core of Karen’s practice with a continued interest in blurring the distinction between what most of us call ‘music’ and all other sound. She has found inspiration in the natural world and how we respond to spaces we occupy. She continually utilizes our inherent familiarity with such sounds and spaces as a means of engaging with audiences. Resulting works challenge the listeners’ memory of hearing. More @ www.karenpower.ie}

\biografia{Karen Power}{Ambienti quotidiani e rumori di ogni giorno si trovano al centro della pratica di Karen con un continuo interesse che porta ad offuscare la distinzione tra ciò che la maggior parte di noi chiama musica e il resto dei suoni. Ha trovato ispirazione nel mondo naturale e come noi rispondiamo agli spazi che occupiamo; utilizza continuamente la nostra familiarità inerente con tali suoni e spazi @ www.karenpower.ie}


\biografia{Federico Ripanti}{Born in Rome in 1987, Federico Ripanti is currently enrolled in Electronic Music at the “S. Cecilia” Conservatory of Music. In 2009 he graduated in Audio and Music Technology at the Saint Louis Music College. He attended private classes of piano, electric guitar and African percussions.}

\biografia{Federico Ripanti}{Nato a Roma nel 1987, studia Musica Elettronica presso il Conservatorio "S. Cecilia" di Roma. Nel 2009 si diploma in Fonia e Music Technology presso la Saint Louis Music College. Ha studiato privatamente pianoforte, chitarra elettrica e percussioni africane.}


\biografia{Alessandro Ratoci}{, composer and electronic music performer was born in 1980 in Tuscany. He obtained his master degree in composition at the Haute école de Musique de Genève with Michael Jarrel, Luis Naon and Eric Daubresse. Subsequently he moved to Paris to follow tuition at IRCAM cursus and at ManiFeste Academy. He is currently professor of computer music and sound engineer at the Haute école de Musique (HEMU) in Lausanne, Switzerland. His music has ben performed by ICTUS ensemble trio, Orchestre de Radio France, Barcelona Modern Ensemble and International Ensemble Modern Academy.}

\biografia{Alessandro Ratoci}{Studi musicali di composizione, pianoforte e musica elettronica presso il Conservatorio di Bologna, Master of Arts in composizione acustica ed elettronica alla HEM di Ginevra, perfezionamento al cursus IRCAM 2014-2015 in Parigi. Compositore, interprete di musica elettronica e didatta, insegna alla HEMU di Losanna e al Conservatorio G.B.Martini di Bologna. Le sue musiche sono state eseguite dal Ictus Ensemble Trio, Modern Ensemble Academy, Orchestra de la HEM di Ginevra, Orchestra di Radio France}


\biografia{Matteo Rossi}{eng}

\biografia{Matteo Rossi}{, percussionista, si diploma con il massimo dei voti presso il Conservatorio “S.Cecilia” di Roma con Gianluca Ruggeri. Segue il corso di perfezionamento presso l’Accademia Musicale Chigiana con Antonio Caggiano, e come membro del Chigiana Percussion Ensemble, si esibisce al CHIGIANA INTERNATIONAL FESTIVAL, RAVELLO FESTIVAL e MAXXI di Roma. Collabora con formazioni orchestrali e cameristiche quali PMCE, InDivenire Ensemble ed ensemble di percussioni quali Ars Ludi, Blow-Up Roma Percussion, Aere Silente con cui si esibisce in un repertorio percussionistico moderno e contemporaneo in diversi eventi quali Le esperienze del minimalismo, Le Forme del Suono, ArteScienza, EMUFest.}


\biografia{Demian Rudel Rey}{(Argentina - October 24, 1987) Composer. He has graduated at Conservatory of Music “Piazzolla” and at National University of Arts. He was awarded in TRINAC, TRIME, FINM, BIENALBahíaBlanca, SADAIC, conDiT, etc. He has been selected in MUSLAB 2014 (Mexico), Interensemble2015 (Italy) and SIRGA Festival 2015 (Spain). He has participated as LiveSamplingPlayer in Les Chants de l'Amour by Grisey in Usina del Arte and in Das Mädchen mit den Schwefelhölzern by Lachenmann in Teatro Colón.}

\biografia{Demian Rudel Rey}{(Argentina - October 24, 1987) Compositore. Si è diplomato al conservatorio di Musica "Piazzolla" e presso L'università Nazionale delle Arti. è stato premiato al TRINAC, Trime, FINM, BIENALbahìaBlanca, SADAIC,CONDIT, ecc. È stato selezionato per il MUSLAB 2014 (Messsico), Interensemble2015 (Italia) e SIRGA Festival 2015 (Spagna). Ha partecipato come "LiveSamplingPlayer" a Les Chants de l'Amor di Grisey, Usina del arte e in Das Mädchen mit den Schwefelhölzern di Lachenmann al Teatro Colón.}


\biografia{Gianluca Ruggeri}{Performer, director, composer and teacher. He graduated in percussion instruments and choir conducting. After an eraly career as a percussionist in symphonic orchestras of Rome, he focused his work on the solo repertoire and chamber contemporary research, focusing on electroacoustic music ( K . Stockhausen, B. Truax, Y. Taira, M. Lupone) and performance (J. Cage, G. Battistelli, L. Hiller, L. Berio). In 1987 he founded with Antonio Caggiano, ARS LUDI, a modular ensemble with which he has performed all over the world. He has conducted many works by F. Evangelisti, K. Stockhausen, M. Betta, C. Crivelli, M. Fischione, L. Cinque, C. Candrew, L. Berio, S. Reich. Is teacher of percussion instruments at the Conservatory of Music "Santa Cecilia" in Rome.}

\biografia{Gianluca Ruggeri}{Performer, direttore, autore e didatta. Diplomato in Strumenti a percussione e Direzione di Coro. Dopo gli esordi come percussionista nelle orchestre lirico-sinfoniche di Roma, ha incentrato il suo lavoro sul repertorio solistico e cameristico contemporaneo concentrandosi sulla ricerca elettro-acustica (K. Stockhausen, B. Truax, Y. Taira, M. Lupone) e sulla “performance” (J. Cage, G. Battistelli, L. Hiller, L. Berio) Nel 1987 ha fondato con Antonio Caggiano, ARS LUDI, un ensemble modulare con cui si è esibito in tutto il mondo. In veste di direttore ha diretto opere di F. Evangelisti, K. Stockhausen, M. Betta, C. Crivelli, M. Fischione, L. Cinque, C. Cardew, L. Berio, S. Reich, B. Sorensen, De Machaut e I. Stravinsky. Attualmente si dedica in vari modi all’approfondimento dell’opera di S. Reich. E’ docente di Strumenti a Percussione presso il Conservatorio di Musica “S.Cecilia” di Roma.}


\biografia{Dimitrios Savva}{Born in Cyprus, 1987. He received his Bachelor degree (distinction) in music composition from the Ionian University of Corfu and his Master degree (distinction) in Electroacoustic composition from the University of Manchester. In January 2015 he started his PhD in Sheffield University under the supervision of Adrian Moore. His compositions have been performed in Greece, Cyprus, United Kingdom, Germany, Belgium, France, Italy, Portugal, Brazil and USA.}

\biografia{Dimitrios Savva}{Nato a Cipro, 1987. È diplomato con lode in Composizione presso la Ionian University di Corfu e laureato (con lode) in Composizione Elettroacustica presso l'Università di Manchester. Attualmente è dottorando alla Scheffield University sotto la supervisione di Adrian Moore. Le sue composizioni sono state suonate in Grecia, Cipro, Regno Unito, Germania, Belgio, Francia, Italia, Portogallo, Brazile e Usa.}


\biografia{Dominique Schafer}{A native of Fribourg, Switzerland, Dominique Schafer is a composer whose breadth of musical expression encompasses both, acoustic instrumentation and electroacoustic media. His music has been performed by ensembles and performers such as the Arditti String Quartet, Dinosaur Annex Ensemble, Ensemble Fa, Boston Modern Orchestra Project (BMOP), Talea Ensemble, Frances Marie Uitti, Alarm will Sound, and the California EAR Unit, at festivals such as Musica Nova Finland, June in Buffalo, and others.}

\biografia{Dominique Schafer}{Nato in Svizzera a Friburgo, è un compositore la cui espressività musicale spazia da lavori per strumenti acustici a lavori multimediali elettroacustici. Le sue composizioni sono state eseguite dall’Arditti String Quartet, Dinosaur Annex Ensemble, Ensemble Fa, Boston Modern Orchestra Project (BMOP), Talea Ensemble, Frances Marie Uitti, Alarm will Sound, California EAR Unit, nel Musica Nova Finland Festival e nel June di Buffalo, tra i tanti.}


\biografia{Claudia Jane Scroccaro}{Claudia Jane Scroccaro graduated in Musicology at “Tor Vergata” University of Rome and studied Orchestral Conducting with Piero Bellugi. While pursuing a Phd in Music Theory at McGill University in Montreal, she decided to return to Italy to study composition with Luigi Verdi at “S. Cecilia” Conservatory of Music. Her music has been performed at the British School of Rome, at the M.K. Ciurlionis School of Arts in Lithuania, at the “Ennio Morricone” Auditorium of Rome, and at the London College of Music. She composed the soundtrack for the film-documentary “I’m coming home” awarded at the Sidney Film Festival and at the International Filmmaker Festival of World Cinema in Milan; she was Composer in Residence “DAR 2015” for the Lithuanian Composer’s Union.}

\biografia{Claudia Jane Scroccaro}{Claudia Jane Scroccaro graduated in Musicology at “Tor Vergata” University of Rome and studied Orchestral Conducting with Piero Bellugi. While pursuing a Phd in Music Theory at McGill University in Montreal, she decided to return to Italy to study composition with Luigi Verdi at “S. Cecilia” Conservatory of Music. Her music has been performed at the British School of Rome, at the M.K. Ciurlionis School of Arts in Lithuania, at the “Ennio Morricone” Auditorium of Rome, and at the London College of Music. She composed the soundtrack for the film-documentary “I’m coming home” awarded at the Sidney Film Festival and at the International Filmmaker Festival of World Cinema in Milan; she was Composer in Residence “DAR 2015” for the Lithuanian Composer’s Union.}


\biografia{Arturo Tallini}{Arturo Tallini played all over Europe, in United States, Tunisia, Algeria, Egypt. He is Guitar Professor at Santa Cecilia Conservatory, in Rome and he teaches in other Italian conservatories and abroad univiersities. In more than 30 years of career, he became a reference point for contemporary music, with many pieces written the contemporary music ensemble \textit{Modus Novus} the Accademia Nazionale di Santa Cecilia Choir, Carlo Morena, and Enzo Filippetti. He is coordinator and professor at the M.A. Master in Contemporary Music at Santa Cecilia Conservatory in Rome.}

\biografia{Arturo Tallini}{ha suonato in tutta Europa, negli Stati  Uniti, in Egitto, Algeria e Tunisia. È docente al Conservatorio di Santa Cecilia in Roma e tiene regolarmente masterclass nei conservatori e italiani e università straniere. Considerato all'unanimità un riferimento per il repertorio contemporaneo, collabora con artisti di fama internazionale, Michiko Hirayama, il gruppo di musica contemporanea \textit{Modus Novus} di Madrid, il Coro dell’Accademia Nazionale di Santa il flautista Carlo Morena. È coordinatore del Master Annuale di II Livello in Interpretazione della Musica Contemporanea del Conservatorio di Santa Cecilia in cui è anche docente di chitarra.}


\biografia{Anna Terzaroli}{She holds a Bachelor's degree in Electronic Music from the Santa Cecilia Conservatory in Rome, where she is currently completing a Master's degree in Electronic Music. As a composer she is dedicated to contemporary acoustic and electroacoustic music. Her musical works are selected and presented in many concerts and festivals in Italy and abroad. Since 2009 she collaborates at the EMUfest festival. She is a member of the AIMI (Italian Computer Music Association) board.}

\biografia{Anna Terzaroli}{Laureata in Musica Elettronica presso il Conservatorio Santa Cecilia di Roma, attualmente sta concludendo il Biennio specialistico presso lo stesso Conservatorio. Come compositrice si dedica alla musica contemporanea acustica ed elettroacustica, suoi lavori sono selezionati e presentati in vari concerti e festival, in Italia e all'estero. Dal 2009 collabora a EMUfest, è membro del Consiglio Direttivo dell'AIMI (Associazione Informatica Musicale Italiana).}


\biografia{Gianni Trovalusci}{Contemporary repertoire with Pierre-Yves Artaud in Paris and Performing Practices of Early Music with Jesper Christensen and Traversiere with Oskar Peter at the Schola Cantorum Basileensis.
He performed in the field of contemporary and ancient music, in music theatre and avant-garde performance at very important Festivals as NYCEMF New York City Electroacoustic Music Festival; Munich Biennale; Nuova Consonanza, Museo Casa Scelsi, Musica e Scienza, EMUFest Rome; M.A.N.C.A. Festival, Nice; GAS Festival, Goteborg; Udine Jazz Festival; REC Reggio Emilia Contemporanea, British Film Institute London, Nancy Opera, Flanders Opera, Ars Electronica - BrucknerHaus Linz, Neue Alte Musik Cologne, CCA Glasgow, Stockholm New Music, Nits de Musica Mirò Foundation Barcelona, etc.}

\biografia{Gianni Trovalusci}{Diplomato in flauto al Conservatorio S. Cecilia, ha approfondito il repertorio contemporaneo con P.-Y. Artaud a Parigi e la prassi esecutiva della musica antica con J. Christensen e O. Peter presso la Schola Cantorum di Basilea. Dagli anni settanta è attivo nel campo della musica contemporanea, antica, nel teatro musicale e performance d'avanguardia; ha lavorato con importanti artisti e si è esibito nei più importanti festival europei e nazionali. È Segretario Artistico della Federazione Cemat.}


\biografia{Giovanni Ubertini}{After the diploma in piano, brilliantly achieved (9.25/10) at the Conservatory "O. Respighi" of Latina, following courses with the American Charles Rosen and with M° Donella D'Alessio. Under the guidance of Luigi Sacco, in 2009 he graduated (10 cum laude) in organ and organ comp. at the Conservatory of Latina and in May 2014, under the guidance of M° Alessandro Licata, he follows graduating in organ and organ comp. (110 cum laude and honorable mention) al the Conservatory “S. Cecilia” in Rome. He currently follows the musical course in Choir conducting and choir comp. with M° Mauro Bacherini at the Conservatory of Latina. Finally, the music titles alongside a degree in Law from "La Sapienza" of Rome, and two years of law practice.}

\biografia{Giovanni Ubertini}{A seguito del diploma in pianoforte, conseguito brillantemente (9.25/10) presso il Conservatorio "O. Respighi" di Latina, segue corsi di perfezionamento con lo statunitense Charles Rosen e con il M° Donella D'Alessio. Sotto la guida del M° Luigi Sacco, nel 2009 si diploma (10 cum laude) in organo e comp. organistica presso il Conservatorio di Latina e nel maggio 2014, sotto la guida del M° Alessandro Licata, conclude il biennio in organo e comp. organistica (110 cum laude e menzione d’onore) presso il Conservatorio “S. Cecilia”. Attualmente è triennalista nel corso di Direzione di coro e comp. corale nella classe del M° Mauro Bacherini presso il Conservatorio di Latina. Ai titoli musicali affianca la laurea in Giurisprudenza.}


\biografia{Kyle Vanderburg}{Kyle Vanderburg composes eclectically polystylistic music fueled by rhythmic drive and melodic infatuation. In addition to composing, he is an active computer programmer, writing code for interactive performances, utilities related to composer workflow automation, and unusual controllers.}

\biografia{Kyle Vanderburg}{Kyle Vanderburg compone ecletticamente musica polistilistica, alimentata da unità ritmica e infatuazione melodica. Oltre ad essere compositore, è anche attivo programmatore di computer, scrive codici per performance interattive, servizi di pubblica utilità relativi alla automazione del workflow, e di controllori inusuali.}


\biografia{Daniele Vantaggio}{Daniele Vantaggio (Rome, 1987) is a producer, sound designer and sound engineer. He graduated in HD recording at Saint Louis College of Rome in 2006, under the guidance of M° L. Spagnoletti. He also obtained a degree in Sound Engineering thanks to V. Nocenzi and L. Pozzi. Since 2009 he has studied Electronic Music at Santa Cecilia Conservatory. He always developed a big interest in underground musical scene and he performed in Europe and South America. Currently he takes care of production, post-production, soundtracks for cinema and theatre. He teaches and also directs a national radio program.}

\biografia{Daniele Vantaggio}{Daniele Vantaggio (Rome, 1987) is a producer, sound designer and sound engineer. He graduated in HD recording at Saint Louis College of Rome in 2006, under the guidance of M° L. Spagnoletti. He also obtained a degree in Sound Engineering thanks to V. Nocenzi and L. Pozzi. Since 2009 he has studied Electronic Music at Santa Cecilia Conservatory. He always developed a big interest in underground musical scene and he performed in Europe and South America. Currently he takes care of production, post-production, soundtracks for cinema and theatre. He teaches and also directs a national radio program.}


\biografia{Massimo Varchione}{Born in Switzerland in 1979. He graduated in composition at the Conservatory Nicola Sala with Luigi Turaccio. He study Electronic Music at the Conservatory San Pietro a Majella in Naples before with Agostino Di Scipio and now with Elio Martusciello. He composed instrumental pieces, electroacoustic and created installations. He has recently started a new journey dedicated to performing radical improvisation with electroacoustic and acoustic instruments. In duo with clarinetist Agostino Napolitano, in 2014 was selected by the center for electronic music Tempo Reale in Florence, to participate to the homonymous festival.}

\biografia{Massimo Varchione}{Nasce in Svizzera nel 1979. Diplomato in Composizione presso il Conservatorio Nicola Sala con il M° Luigi Turaccio. Studia Musica Elettronica presso il Conservatorio San Pietro a Majella di Napoli prima con il M° Agostino Di Scipio ed ora con il M° Elio Martusciello. Ha composto brani strumentali, elettroacustici e realizzato installazioni. Ha iniziato di recente un nuovo percorso dedicato all'improvvisazione radicale con il mezzo elettroacustico e con gli strumenti. In duo con il clarinettista Agostino Napolitano, nel 2014 è stato selezionato dal centro Tempo Reale di Firenze per partecipare al loro festival dedicato all'elettronica.}


\biografia{Clemens Von Reusner}{Clemens von Reusner is a composer and soundartist based in Germany, who is focused on acousmatic music. International broadcasts and performances of his compositions. www.cvr-net.de}

\biografia{Clemens Von Reusner}{è un compositore e soundartist residente in Germania, il cui lavoro si concentra sulla musica acusmatica. Le sue composizioni hanno ottenuto trasmissioni ed esecuzioni a livello internazionale.}


\biografia{Benjamin D. Whiting}{ received his BM in Music Composition and his MM in Music Theory and Composition from Florida State University, and is now pursuing his DMA at the University of Illinois at Urbana-Champaign. He is an active composer of both acoustic and electroacoustic music, and has had his works performed in the United States and abroad, and released on the ABLAZE and University of Illinois Experimental Music Studios labels.}

\biografia{Benjamin D. Whiting}{si diploma in Composizione e prende un master in Teoria musicale e composizione presso la Florida State University, ed è attualmente dottorando presso la University of Illinois in Urbana-Champaign. Compone sia musica acustica che elettroacustica, e i suoi lavori sono stati eseguiti negli Stati Uniti e all'estero, ed editi dall'etichetta Experimental Music Studios della University of Illinois.}


\biografia{Giuseppe Zampetti}{Composer. He was born in Rome in 1992. He studies with M° Francesco Telli. He attends Contemporary Composition at the Conservatory of "S.Cecilia" in Rome.}

\biografia{Giuseppe Zampetti}{Compositore. Nato a Roma nel 1992. Allievo del M° Francesco Telli, studente di Composizione indirizzo Contemporaneo presso il Conservatorio “S. Cecilia” di Roma.}


\biografia{Francesco Ziello}{eng}

\biografia{Francesco Ziello}{ita}
