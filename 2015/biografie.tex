% !TEX encoding = UTF-8 Unicode
% !TEX TS-program = XeLaTex
% !TEX root = EMU2015_booklet.tex

\biografia{James Andean}{Musicista e sound artist. È attivo sia come compositore che esecutore in vari campi, incluse composizioni elettroacustiche e performance, improvvisazioni, installazioni audio, e registrazioni. È membro fondatore del quartetto di improvvisazione e nuova musica Rank Ensemble, e fa parte del duo Plucié/DesAndes (audiovideo). Si è esibito per l'Europa e Nord America, e i suoi lavori sono stati presentati in tutto il mondo.}

\biografia{Damián Anache}{(1981, Buonos Aires) Musicista e sound artist. è attivo come compositore e performer nell'improvvisazione, istallazione sonora, sound recording. è membro fondatore di "improvisation and new music quartet Rank Ensemble, e fa parte del duo audiovisual Plucié/DesAndes. Si è esibito in tutta l'Europa e Nord America i suoi lavori sono stati presentati in tutto il mondo. Website: damiananache.com.ar}

\biografia{Alfredo Ardia}{Classe 1989, ha studiato al LEMS (Pesaro, Italia) e al CMT (Helsinki, Finlandia). È interessato al suono, alla sua percezione e a come esso si relaziona con altri media, esplorando i fenomeni sonori di entità elementari ed i loro comportamenti. È ispirato dalla bellezza della fisica.  Web: http://alfredoardia.altervista.org/}

\biografia{Luciano Azzigotti}{Compositore di Buones Aires, Argentina. La sua musica comprende differenti mezzi ed ascolti, gesti e interfacce di scrittura. Ha studiato composizione all'università di La Plata, in diversi master e partecipato a seminari internazionali con Gerardo Gandini, Mauricio Kagel, George Aperghis, Chaya Czernowin e Rebecca Saunders. La sua musica è stata eseguita in Argentina, Brasile, USA, Austria, Germania, Francia, Italia, da importanti ensemble e solisti. è fondatore del conDIT BsAs.}

\biografia{Christian Banasik}{(1963) ha studiato composizione alla Robert Schumann Academy of Music and Media di Dusseldorf e alla University of Music and Performing Arts di Francoforte. I suoi lavori strumentali ed elettronici sono stati eseguiti in concerti e programmi radio in tutta Europa, ma anche in America, Asia e Australia. È docente di Audio Visual Design presso la University for Applied Sciences e direttore artistico di Computer Music Studio "Studio 209" di Dusseldorf.}

\biografia{Carlo Barbagallo}{(1985) musicista, compositore e sound engineer.Sin da piccolo ha registrato la propria musica, sperimentando sulle (non) possibilità creative della registrazione casalinga:i lavori sono in rete tramite l'etichetta Noja Recordings.Dal 2012 è musicista elettroacustico con ricerca concentrata su: spazializzazione di musica acusmatica, feedback nelle sue diverse forme, improvvisazione e programmazione-tramite-ascolto come tecniche compositive, sviluppo di strumenti algoritmici per generare partiture e suoni a partire dalla struttura di testi linguistici, estetica delle forme incomplete, la composizione collettiva.Nel2013 è co-fondatore del Collettivo di Musica Elettroacustica di Torino (CoMET).Tra i festival: Premio Phonologia, ICMC, PNDA, 57Festival Internazionale di Musica Contemporanea (Biennale di Venezia), Emufest, PIARS, XXCIM, Di\textunderscore Stanze. http://nojarecordings.tumblr.com}

\biografia{Antonella Barbarossa}{è nata e vive in Italia. Didatta del pianoforte, organista, direttrice d’orchestra, compositrice, filosofa e missionaria in Calabria per scelta dove assume la docenza di pianoforte al Conservatorio di stato di Cosenza nel 1976. Nel 1991 diviene direttore del Conservatorio di Vibo Valentia sino al 2013; nel 2003 fonda il politecnico internazionale “ scientia et ars” per la specializzazione in tecnologia del suono. In campo organistico ha eseguito in concerti pubblici l’opera integrale di Bach, Franck, Liszt e Messiaen, e in prima assoluta per la Rai e festivals internazionali, composizioni d’autori contemporanei. È vincitrice del primo premio al Concorso internazionale organistico di Roma nel 1981 e finalista al Concorso Internazionale di Lipsia per lo stesso strumento.}

\biografia{Cathy Berberian}{Mezzosoprano e compositrice statunitense d'origine armena (Attleboro 1925 - Roma 1983). Esordì a Napoli nel 1957 e si dedicò prevalentemente all'attività concertistica rivelandosi come una delle sostenitrici più valide della "nuova vocalità". Considerata una delle maggiori interpreti della musica contemporanea, ha coltivato anche il genere folcloristico, la musica da cabaret ed è stata apprezzata interprete di Monteverdi e di musica del periodo barocco. Hanno scritto per lei I. Stravinskij, L. Berio, S. Bussotti e J. Cage. Ha composto musica vocale e strumentale; ha pubblicato La nuova vocalità nell'opera contemporanea (1966).}

\biografia{Luciano Berio}{Allievo di G. F. Ghedini e G. C. Paribeni a Milano e di L. Dallapiccola a Tanglewood, nel 1954 ha fondato con B. Maderna lo Studio di fonologia musicale alla RAI di Milano. Esponente fra i più agguerriti e significativi dell'avanguardia musicale contemporanea, si è dedicato tra i primi all'esperienza elettronica. Ha scritto musiche di scena, pezzi orchestrali, da camera e vocali in cui ha alternato l'uso di strumenti tradizionali a quelli derivati dalla tecnica elettronica. Presidente e sovrintendente dell'Accademia di Santa Cecilia dal 2000, B. ha svolto una intensa attività didattica a Darmstadt, Colonia e in varie università degli USA; nel 2002 ha proposto un nuovo finale di Turandot di G. Puccini in sostituzione di quello realizzato da F. Alfano.}

\biografia{Francesco Bianco}{Laureato in Musicologia, frequenta il corso di Musica elettronica presso il Conservatorio di Roma Santa Cecilia. È stato visiting scholar al CRR de Boulogne-Billancourt (Parigi).  Musicista da sempre interessato alle profonde relazioni fra l'arte e la vita, ha esperienze musicali variegate di genere e stile, dalla composizione alla performance dal vivo, dall'azione scenica alla colonna sonora.}

\biografia{Isobel Blank}{Nasce a Pietrasanta in Toscana, si laurea con lode in filosofia estetica a Padova, vive e lavora a Torino. Le sue opere sono state esposte in numerose gallerie, musei e festivals, dagli Stati Uniti alla Cambogia, dal Messico alla Russia. Tra le esposizioni recenti, quella alla Triennale Fiberart International di Pittsburgh, al Museum of Modern Art di Mosca, a Palazzo Widmann di Venezia, alla Mumbai Art Room in India. Ha avuto diversi riconoscimenti tra cui il Primo Premio al Romaeuropa Webfactory nel 2009, sezione videoarte. http://www.isobelblank.com}

\biografia{Daniel Blinkhorn}{ è un compositore australiano. Attualmente impartisce lezione nel reparto di tecnologia, composizione e musica presso il Conservatorio dell' Università di Sidney. È anche un appassionato tecnico di field recording; ha intraprendeso spedizioni di registrazione in tutta l'Africa, Alaska, Amazon, Indie occidentali, Nord Europa, Medio Oriente, Australia e il Polo Nord. Le sue opere creative hanno ricevuto più di 25 citazioni di composizione internazionali e nazionali.È autodidatta in elettroacustica e ha studiato al tempo stesso composizione e arti creative in diverse università australiane, tra cui UOW, dove il suo dottorato ha ricevuto menzione speciale. Altre lauree: BMus (Hons ), MMus, e MA(r).}

\biografia{Elisabetta Braga}{Nata a Nardò, si diploma brillantemente in canto presso il Conservatorio di musica Santa Cecilia di Roma nel 2013. Comincia la sua attività concertistica esibendosi in vari teatri e sale da concerto Italia e all'estero, quali la Sala Accademica del Conservatorio Santa Cecilia di Roma, il Teatro Politeama Greco di Lecce, la Tchaikoskj Concert Hall di Mosca. Nel 2012 partecipa a Emufest per l'esecuzione di "Anabasi" di G.Baggiani, diretta da T. Battista. Debutta nel 2015 a Roma come Mimì ne "La Bohème" di G. Puccini e a Rieti nel 2015 come Gilda in "Rigoletto" di G. Verdi. Si è perfezionata partecipando come allieva effettiva alla masterclass del soprano Sumi Jo tenuta a Roma in aprile. Attualmente sta per conseguire il Diploma Accademico di secondo livello presso il Conservatorio Santa Cecilia di Roma.}

\biografia{Daniele Buccio}{Diplomato in pianoforte presso il Conservatorio “A. Casella” dell’Aquila ed in composizione presso il Conservatorio “G. Verdi” di Torino, è dottore di ricerca in musicologia e beni musicali. Ha seguito corsi di alto perfezionamento in composizione presso l’Accademia Filarmonica di Bologna, l’Accademia “L. Perosi” di Biella e l’Accademia Musicale Chigiana. Si è esibito come solista presso il Teatro Regio di Parma, l’Auditorium Parco della Musica di Roma, la Deptford Town Hall per la Liszt Society di Londra, il Palazzo dei Congressi di Lugano.}

\biografia{Sylvano Bussotti}{Nato a Firenze il 1 ottobre 1931. Inizia lo studio del violino con Margherita Castellani ancora prima di compiere i cinque anni di età. Al Conservatorio "Luigi Cherubini" di Firenze studierà l'armonia e il contrappunto con Roberto Lupi e il pianoforte con Luigi Dallapiccola: studi che interromperà a causa della guerra, senza conseguire alcun titolo di studio. A Parigi, nel periodo che va da 1956 al 1958, frequenta i corsi privati di Max Deutsch, incontra Pierre Boulez e Heinz-Klaus Metzger, che lo condurrà a Darmstadt, dove conosce John Cage. Inizia in Germania, nel 1958, l'attività pubblica, con I'esecuzione delle sue musiche da parte del pianista David Tudor, seguita dalla presentazione a Parigi di brani eseguiti da Cathy Berberian sotto la direzione di Pierre Boulez.}

\biografia{Elisabetta Capurso}{pianista compositrice musicologa ha completato la formazione pianistica al Mozarteum di Salisburgo, gli studi di composizione a Darmstadt. Della sua formazione musicale fanno parte gli studi di Direzione d’orchestra composizione elettronica gli studi umanistici; è laureata con lode in Lettere Filosofia all’Università La Sapienza. Recentemente ha conseguito con lode  la seconda laurea in Musica elettronica al Conservatorio S. Cecilia. Concertista di livello internazionale, ha suonato in sale prestigiose per le maggiori associazioni italiane e straniere. Come compositrice ha scritto molti lavori di musica sinfonica cameristica elettronica, eseguite in teatri di rilievo in Italia e all’estero. Le sue opere sono state registrate da Rai Radio Tre, Radio Vaticana, pubblicate da Zanibon-Peters EdiPan AFM SEDAM. Diverse le commissioni di scrittura ricevute: Après alfa dai Solisti Aquilani Tre cose solamente dal Festival Internazionale Organo di Lecce. Ha un curriculum di Professore di Pianoforte nel Conservatorio G.Rossini di Pesaro, Professore di Pianoforte Semiografia della musica contemporanea Laboratorio contemporaneo nel Conservatorio S. Cecilia di Roma. Ha ricevuto premi e riconoscimenti per gli alti meriti artistici.}

\biografia{Simone Cardini}{Studia composizione con F. Telli, pianoforte con A. Torchiani; partecipa a corsi di perfezionamento e seminari tenuti da S. Sciarrino, M. André, T. Tulev, M. Trojahn, P. Manoury. Le sue composizioni sono state eseguite in Europa e Stati Uniti in mostre eminenti e festival come ArteScienza (2012), Contemporanea (2013), Nuova Consonanza (2013, 2014), Rondò (2014), NYCEMF (2015) da ensemble internazionali come Divertimento Ensemble, PMCE e sono stati premiati in concorsi illustri come AFAM (2013), Valentino Bucchi ed 37 °. (2015), ecc Il suo scritto Musica e Architettura - Estetica e implicazioni sociologiche è stato pubblicato nel libro Musica \& Architettura, Nuova Cultura Ed. (2012). La sua raccolta di lavori sarà pubblicata da Universal Edition (2015).}

\biografia{Antonio Carvallo}{A. Carvallo è nato in Cile nel 1972. Ha studiato contrappunto e armonia con Rodolfo Norambuena. Poi, presso l'Università del Cile, ha ottenuto il Diploma e il Master di Specializzazione in Composizione. Ha insegnato all' Università del Cile dal 2000 al 2002. Dopo di che, si è trasferito a Roma e ha studiato musica elettronica con R. Bianchini e G. Nottoli al Conservatorio "Santa Cecilia" ed ha ottenuto la Laurea di Primo e Secondo Livello. Attualmente insegna presso l'Università del Cile.}

\biografia{Pasquale Citera}{(1981) Ha studiato pianoforte con il M° Gemma D’Alessio, Composizione con i M° Luciano Pelosi e Giovanni Piazza e Musica Elettronica con il M° Giorgio Nottoli. Da anni collabora con diverse compagnie teatrali e case di produzione cinematografica oltre che con scultori e fotografi. Ha composto musiche di scena per spettacoli classici e contemporanei. Tra gli altri: L’Alcesti di Euripide, Lisistrata di Aristofane, Anfitrione di Plauto, la Locandiera di Goldoni, l’Avaro di Molière, Da quale parte del vetro di Silvio Nanni, Il dito sulla bocca di Donatella Ferrara, Certe Notti non accadono mai di Patrizia Masi. Ha scritto colonne sonore per la Nero-Film, è Assistente Musicale in diverse scuole di Roma ed è stato docente di Tecnologie Musicali. Dalla collaborazione con lo scultore Arturo Ianniello sono nate diverse sonorizzazioni di opere visuali raccolte in due esposizioni. È attualmente Compositore e Sound Designer per musiche di scena al Teatro Anfitrione ed all’Anfiteatro della Quercia del Tasso.}

\biografia{Cristiana Colaneri}{Nata a Roma ha conseguito il compimento medio di composizione vecchio ordinamento (sotto la guida del M. Pasquale Lucia); studia Composizione presso il Conservatorio Santa Cecilia di Roma, nella classe del M. Francesco Telli. Deve sostenere la Prova finale del diploma accademico di I livello. Finalista ai concorsi Mea 2010 e Bucchi 2015.}

\biografia{Valerio Cosmai}{Nasce a Roma nel 1983. Studia pianoforte esibendosi più volte come pianista solista nella sala Baldini di Roma e in varie ambasciate, tra cui quella americana ed indonesiana, specializzandosi nel repertorio mozartiano. Consegue nel 2008 la laurea in Lettere presso l'università degli studi "La Sapienza" di Roma. Si diploma in percussioni con il massimo dei voti nel 2014. Come percussionista collabora con l'orchestra del Conservatorio esibendosi anche in importanti festival di musica contemporanea. Dal 2012 lavora come insegnante di educazione musicale nella scuola Pio IX di Roma.}

\biografia{Giovanni Costantini}{(Corigliano d’Otranto - Lecce, 1965) Dal 1995 svolge attività di ricerca presso l'Università di Roma "Tor Vergata", dove è docente di Musica Elettronica. È direttore del Master in SONIC ARTS. Sue composizioni elettroacustiche sono state eseguite in numerosi concerti in Italia e all’estero e incise da Twilight Music (Roma) e IAEF (New York). La sua ricerca musicale è rivolta alla realizzazione della microstruttura e della macrostruttura del suono attraverso l’esplorazione e l’elaborazione in tempo reale di materiale acustico.}

\biografia{Elena D'Alò}{, flautista e ottavinista si laurea cum laude al biennio in Flauto, dopo un brillante diploma, presso il Conservatorio "Santa Cecilia" di Roma, con Deborah Kruzansky. Ha affiancato gli studi musicali con quelli scientifici, laureandosi in Fisica acustica presso "La Sapienza" con Paolo Camiz. Attualmente è iscritta al triennio di Musica Elettronica a Roma. Si esibisce in formazioni cameristiche e orchestrali, in un repertorio che va dal barocco al contemporaneo, per il quale ha suonato a festival come Nuova Consonanza, Atlante Sonoro XX secolo, ArteScienza ed EMUfest. Studia violoncello con Maurizio Massarelli.}

\biografia{Maria Cristina De Amicis}{(Avezzano, 1968) Ha compiuto studi di Composizione, Musica Elettronica, Organo e Composizione Organistica, diplomandosi con il massimo dei voti presso il Conservatorio di Musica “A. Casella” de L'Aquila. Le sue opere sono state eseguite in importanti manifestazioni di musica contemporanea in Italia e all’estero tra cui (Lione, Parigi, Barcellona, Aveiro, Madrid, Budapest, Atene, Salonicco, Berlino, Francoforte, Vienna). Dal 2012 è docente di Musica Elettronica presso il Conservatorio di Musica “A.Casella” dell’Aquila.}

\biografia{Vittoriana De Amicis}{(L’Aquila, 1992) a 15 anni intraprende lo studio del canto lirico presso il Conservatorio A.Casella sotto la guida di Antonella Cesari. Ha seguito numerosi corsi di perfezionamento in Italia e all’estero, nel 2013 viene selezionata dal Mozarteum di Salisburgo per prendere parte all’accademia estiva con Horiana Branisteanu. Sempre nel 2013 è assegnataria di una borsa di studio Erasmus e frequenta la classe di Anton Scharinger all’Universit\"at f\"ur Musik di Vienna. Nel 2014 si è diplomata a L’Aquila con il massimo dei voti e la lode e attualmente studia a Roma con Elizabeth Norberg-Schultz.}

\biografia{Domenico De Simone}{Diplomato in Pianoforte, Jazz, Composizione e Musica Elettronica. Ha conseguito il diploma del corso di perfezionamento di Composizione presso l’Accademia Nazionale di Santa Cecilia e, con il massimo dei voti e la lode, il diploma accademico di II livello in Musica Elettronica. Sue composizioni sono state eseguite in Italia e all’estero (Cina, Lettonia, Canada, Cile, Argentina, Romania, Malta, ecc.) e trasmesse da Radio3.}

\biografia{James Dashow}{Ha avuto commissioni, premi e borse di studio dal Bourges Festival Internazionale di Musica Sperimentale (Premio Magistere), il Guggenheim, Fromm e Fondazioni Koussevitzky, Linz Ars Electronica, la Biennale di Venezia, gli USA National Endowment for the Arts, RAI, l'American Academy and Institute of Arts \& Letters, Prague Musica Nova, etc. Nel 2011 gli è stato conferito del "CEMAT per la Musica" premio in riconoscimento della sua carriera di contributi eccezionali alla musica elettronica.}

\biografia{Gustavo Delgado}{Buenos Aires (1976). Diploma di Secondo Livello specialistico in “Musica Elettronica” presso il Conservatorio di Musica “Santa Cecilia” di Roma sotto la guida del M° Giorgio Nottoli con il massimo dei voti. Laurea in “Composizione di Musica Elettroacustica” presso l’Università Nazionale di Quilmes (Buenos Aires, Argentina). Compositore di musica acousmatica, live electronics e di musica applicata, interessato allo studio delle tecniche di missaggio on the box e sound design.}

\biografia{Dennis Deovides A. Reyes III}{Ha studiato composizione musicale nella sua città nativa Manila, Filippine, prima di trasferirsi negli Stati uniti nel 2006. Attualmente Dennis sta facendo il dottorato in composizione musicale all'università dell'Illinois all'Urbana-Champaign con Scott A. Wyatt. Le sue composizioni trovano ispirazione da una vasta gamma di argomenti, dalla musica Asiatica all'arte moderna, e incorpora anche elementi della tradizione Filippina.}

\biografia{Christian Eloy}{Nato ad Amiens, ha studiato flauto e composizione al Conservatorio Nazionale della regione ed al Conservatorio Nazionale Superiore di Parigi. Prima del suo incontro con Ivo Malec e l’emittente GRM / Groupe de Recherches Musicales di Radio France, è stato flautista in orchestra e direttore di una scuola di musica. Christian Eloy ha fondato l’associazione di compositori Octandre, è a capo del dipartimento di elettroacustica del Conservatoire National de Region a Bordeaux e dei workshop per il GRM a Parigi ed è direttore artistico dello studio di ricerca e creatività SCRIME all’Università di Bordeaux I. Vincitore di numerosi premi, tra cui il premio europeo per poesia e musica " François de Roubaix ", ha composto oltre quaranta brani strumentali, elettroacustici, vocali e didattici. I suoi lavori sono pubblicati dalla Billaudot, Fuzeau, Lemoine, Combre, Notissimo e Jobert e i suoi articoli scientifici dal PUF (Francia), Johnston Ed. (Irlanda), MIT press (USA), Le mensuel littéraire et poétique (Belgio) e Confluences (Francia).}

\biografia{Sara Ferrandino}{si è diplomata in pianoforte nel 2005 presso il Conservatorio di Perugia nella classe del Mº Tanganelli, conseguendo nel 2009, con votazione di 110 e Lode, la Laurea per il Biennio Specialistico. Nel 2012 ha ottenuto il diploma del Corso di Perfezionamento tenuto dal Mº Perticaroli, presso l’Accademia Nazionale di Santa Cecilia in Roma. Ha partecipato a numerosi concorsi nazionali e internazionali ottenendo sempre piazzamenti nelle prime posizioni. Si è esibita in molteplici concerti solistici e cameristici in prestigiose sale in Italia e all’estero. Collabora presso il Conservatorio di Perugia con le classi di corno, tromba, flauto, oboe e violino. È docente di pianoforte principale per i corsi pre-accademici presso il Conservatorio Santa Cecilia in Roma.}

\biografia{Enzo Filippetti}{è professore di Sassofono al Conservatorio “S. Cecilia” di Roma e da più di trent’anni tiene concerti in tutto il mondo. Si è esibito alla Biennale di Venezia, al Mozarteum di Salisburgo, a Roma, Milano, Parigi, Londra, Berlino, Vienna, Madrid, Bruxelles, Buenos Aires, Caracas, Riga, Birmingham, Köln, Lyon, St. Etienne (Francia), Principato di Monaco, Yeosu (Korea), Kawasaki, Adis Abeba, Chisnau, Taormina, Ravello. Ha collaborato con Claude Delangle, Alda Caiello e Bruno Canino e molti tra i più importanti compositori hanno scritto per lui più di cento opere e gli hanno affidato numerose prime esecuzioni. Come solista e con il Quartetto di Sassofoni Accademia ha inciso per Nuova Era, Dynamic, Rai Trade e Cesmel. Ha pubblicato studi per Riverberi Sonori e cura una collana per le edizioni Sconfinarte.}

\biografia{Alessia Forganni}{ (Brescia, 1982) si diploma in pianoforte presso il conservatorio Luca Marenzio, sotto la guida del M° M. Zana, e si laurea al D.A.M.S. - Indirizzo Cinema e Audiovisivi. Dal 2007 vive e insegna a Roma: negli ultimi anni all’attività classica ha affiancato un approccio moderno allo strumento: con il duo pianistico Duel, tra il 2009 al 2015 si è esibita in Europa, Russia, Libano e Sudafrica. Attualmente sta ultimando il triennio di Musica Elettronica presso il conservatorio Santa Cecilia: la sua ricerca compositiva è volta a una personale messa in relazione tra il background classico, l’esplorazione improvvisativa, l’utilizzo della voce e le istanze contemporanee.}

\biografia{FREI}{FREI è un progetto di Paolo Gatti e Francesco Bianco, nato nel 2014. Si sono esibiti al Circolo Dal Verme (Studiolo Laps Showcase), al teatro Tor Bella Monaca (Slaps-pourri.1 anteprima). L'improvvisazione è alla base della poetica di Frei. La performance live è basata su elementi preordinati, i quali, durante lo spettacolo, vengono elaborati e sviluppati. La strumentazione è costituita da due laptop sui quali sono vi sono sistemi digitali programmati degli stessi componenti del duo.}

\biografia{Javier Alejandro Garavaglia}{Compositore/Performer (Viola ed elettronica). Professore associato alla facoltà di CASS, Università di Londra. Le Composizioni e le performance eseguite in molti paesi dell'Europa, Dell'America, e dell'Asia, includono: Acusmatici, audio-visual, lavori per solo/camera/ensemble e orchestra, con o senza l'inclusione di elettronica e media interattivi. Alcune opere elettroacustiche sono disponibili su CD (Germania, USA, Argentina e Danimarca). Settori di ricerca: drammaturgia musicale; automazioni del live Electronics; diffusione speciale di sistemi audio.}

\biografia{Jorge García del Valle Méndez}{(1966) è cresciuto in spagna, dove ha studiato fagotto e composizione. Ora vive a Dresda (Germania) dove ha studiato composizione e musica elettronica. Le sue composizioni hanno avuto prime mondiali in tutto il mondo, commissioni da importanti istituzioni internazionali. Lavori di analisi digitale e "sound processing", applicati alla teoria e alla composizione. Premio di composizione Salvatore Martirano dellUniversità dell'Illinois, e premio di composizione Sächsischer Musikrat.}

\biografia{Paolo Gatti}{Laureato in ingegneria, consegue il master in ingegneria del suono presso l'università di Roma "Tor Vergata". Successivamente si laurea a pieni voti in musica elettronica presso il Conservatorio Santa Cecilia di Roma, sotto la guida di G.Nottoli,M.Lupone,N.Bernardini.Compositore, didatta e ricercatore, suoi lavori sono eseguiti in importanti manifestazioni e festival internazionali. Scrive musiche per spettacoli teatrali e rassegne poetiche. Nel 2015, la sua composizione Poltergeist, risulta fra i brani premiati al termine della finale nazionale del premio delle arti "Claudio Abbado".}

\biografia{Núria Giménez-Comas}{Ha studiato composizione presso la Escola Superior de Musica de Catalunya (ESMUC). Dopo due anni ha continuato la sua formazione presso il Conservatorio di Ginevra, studiando composizione con Luis Naon ed elettroacustica con Michael Jarrell. Ha studiato presso l'Institut de Recherche et Coordination Acoustique / Musique (IRCAM) per due anni, dove ha esplorato diversi tipi di sintesi e il nuovo sistema di spazializzazione in Ambisonics 3D. Núria ha lavorato con musicisti come Harry Sparnaay, Wien trio di Klangforum, Ensemble Contrechamps, Bruxelles Philharmonic, e Diotima Quartetto. È membro fondatore di Ensemble Matka.}

\biografia{Virginia Guidi}{Diplomata in Canto e in Musica Vocale da Camera al Conservatorio S. Cecilia, ivi specializzata con lode con S. Schiavoni con una tesi sperimentale sul rapporto tra interprete e compositore nella musica elettroacustica. Canta in Italia e all’estero (Pechino – National Centre of the Performing Arts; Roma –Accademia Filarmonica, Tecnopolo, GNAM, Macro, MAXXI; Napoli – Arena Flegrea; Catania – Teatro Metropolitan), su reti nazionali (RAI 1, RAI 2, RAI 5, Telepace) e in importanti Festival (EMUfest, ArteScienza). Con Voxnova Italia canta “In the Midst of Things” di Allora \& Calzadilla, musica di G. Colemann, per la Biennale Arte 2015.}

\biografia{Jan Jacob Hofmann}{Diplomato in architettura alla Fachhochschule di Francoforte sul Meno nel 1995. Nello stesso anno viene ammesso al corso di Peter Cook ed Enric Miralles alla Städelschule, Scuola Superiore di Formazione Artistica di Francoforte per un corso post- universitario di architettura e progettazione concettuale. Diplomato nel 1997. Lavora come compositore, architetto e successivamente come fotografo. Dall' estate 2005 è "Associate Researcher" alla "Signal Processing Applications Research Group", Universita di Derby, Inghilterra. È stato nominato al consiglio di amministrazione della Società Tesesca per la Musica Elettroacustica, DEGEM.}

\biografia{Sandro L'Abbate}{, classe 1988. Diplomato in fotografia all'Accademia di Belle Arti di Rimini in Italia. È interessato alla produzione audio- visuale usando sistemi elettronici ed interattivi per osservare fenomeni fisici. Al momento si trova di fronte al mare. Web:http://sandrolabbate.altervista.org}

\biografia{Silvia Lanzalone}{compositrice (Salerno 1970). Diploma di Flauto, Composizione e Musica Elettronica presso i Conservatori di Salerno, L’aquila e Roma. Sue composizioni sono edite da Ars Publica, Taukay e Suvini Zerboni e sono eseguite in festivals nazionali ed internazionali. Dal 1997 collabora con il CRM - Centro Ricerche Musicali di Roma. È Docente di Composizione Musicale Elettroacustica e Coordinatore del Dipartimento di Nuove Tecnologie e Linguaggi Musicali presso il Conservatorio “G. Martucci” di Salerno. (http://www.silvialanzalone.it/)}

\biografia{Jean-Francois Laporte}{Compositore, esecutore ed inventore di strumenti musicali. Attivo sulla scena artistica contemporanea dalla fine degli anni ’90, l’artista canadese ha un approccio creativo ibrido, che unisce assieme sound art, composizione, interpretazione, performance, installazione sonora e arte digitale. Artista piuttosto intuitivo, ha appreso l’arte attraverso sperimentazioni concrete con la materia, basando il suo approccio alla composizione sull’ascolto attivo e sull’attenta osservazione della realtà di ciascun fenomeno. Nel corso degli anni ha dedicato una grande quantità di energie all’invenzione, lo sviluppo e la costruzione di nuovi strumenti musicali. È fondatore e direttore artistico delle produzioni Totem Contemporain di Montreal. http://www.jflaporte.com}

\biografia{Gy\"orgy Ligeti}{Musicista di origine ungherese. Dopo aver preso la cittadinanza austriaca nel 1967, dal 1973 è insegnante di Composizione alla Hochschule für Musik di Amburgo ed è spesso invitato a tenere conferenze e seminari in importanti centri musicali di Europa e degli Stati Uniti. Nell'attività creativa, specialmente a partire dagli anni Settanta, ha continuato a sviluppare l'attitudine a una raffinata e inquieta manipolazione di tutti i parametri del suono, secondo un atteggiamento espressivo che non rifugge da ironiche bizzarrie e da violente accensioni timbriche, spesso legate a una ricerca esplicita di teatralità. Perciò le sue composizioni più recenti tendono ad allontanarsi dalle sottili vibrazioni materiche e dal senso di stupite e illusorie sonorità che caratterizzavano i suoi lavori nati nel clima di Darmstadt.}

\biografia{Jones Margarucci}{Ha studiato Composizione Elettroacustica presso il Conservatorio di Salerno e come exchange student presso la Royal Academy of Music (KMH) a Stoccolma. Sue musiche sono state eseguite in diversi festival in Europa e in Nord America e sono state selezionate per: Redshift Music - Postal Pieces (Vancouver – Canada – 2013); Vox Novus Fifteen Minutes of Fame - Yumi Suehiro (New York City – USA – 2014); Sonorities Festival 2015 (Belfast – North Ireland – 2015); SOUNDkitchen’s Earspace/Frontiers Festival 2015 (Birmingham – UK – 2015); Video Remakes - Call for Tape Music (La Fabbrica del Vedere) (Venice - Italy - 2015)}

\biografia{Marco Marinoni}{ nasce a Monza nel 1974. Nel 2007 si diploma con il massimo dei voti e la lode in Musica Elettronica con Alvise Vidolin. Nel 2009 consegue il Diploma Accademico Sperimentale di Secondo Livello in Live Electronics e Regia del Suono con 110 e Lode e nel 2013 il Diploma Accademico Sperimentale di Secondo Livello in Composizione con 110 e Lode. Dal 1999 è attivo come compositore in ambito contemporaneo. Prix du Trivium nel 29e Concours International de Musique et d'Art Sonore Electroacoustiques - Bourges 2002. Finalista dell'International Gaudeamus Composition Prize 2002 e 2003. Vincitore della Seconda Call per Opere Elettroacustiche indetta dalla Federazione CEMAT. Primo Premio nel Primo Concorso di Composizione per Iperviolino - Genova 2007. Primo Premio nel VIII Concorso Internazionale di Composizione “Città di Udine”. Come musicologo partecipa ai convegni indetti dall'AIMI e dal GATM. È membro del SIMC - Società Italiana Musica Contemporanea. Le partiture dei suoi brani sono edite da ARSPUBLICA EDIZIONI MUSICALI e da TAUKAY. Nel 2015 esce il suo primo romanzo, La Confraternita di Ecate - Cauda Draconis (ed. Nerocromo). È professore di Esecuzione e Interpretazione della Musica Elettroacustica e coordinatore del Dipartimento di Musica Elettronica presso il Conservatorio "G. Verdi" di Como. Vive a Finale Ligure.}

\biografia{Raffaele Marsicano}{ diplomato in trombone nel 2006 al conservatorio di Salerno, continua i suoi studi musicali diplomandosi anche in strumentazione per banda nel 2011 e composizione nel 2015 presso il conservatorio di Milano, dove attualmente frequenta il biennio specialistico di composizione. Considerata la sua duplice natura da trombonista e compositore, incentra le sue ricerche sulla sperimentazione di nuovi suoni degli ottoni applicata alla composizione.}

\biografia{Francesc Martí}{ è un matematico, informatico, compositore, sound and digital artist, nato a Barcellona e vive attualmente nel Regno Unito. Come compositore e digital media artist, i suoi lavori sono stati eseguiti o ha fatto esibizioni in tutto il mondo, inclusi feltival internazionali, eventi e manifestazioni. Attualmente, combina i suoi progetti artisti e tecnologici con l'insegnamento di Audio technology and Image alla open university della California, e Music Technology alla Montfort University di Leichester.}

\biografia{Mario Mary}{Mario MARY dottor in "Estetica, Scienza e Tecnologia delle Arti" (Università di Parigi VIII, Francia), Professore di Composizione di Musica Elettroacustica presso Academia Ranieri III di Monte-Carlo, e Direttore artistico di Monaco Electroacoustique - Incontri Internazionali di Musica Elettroacustica. Ha lavorato come ricercatore presso l'IRCAM e insegnato all'Università Parigi VIII, Ha vinto una ventina di premi in concorsi di composizione. Ha dato nomerosos conferenze e corsi in diversi paesi. http://ipt.univ-paris8.fr/mmary/}

\biografia{Massimiliano Mascaro}{Compositore. Nato a Roma nel 1986. Allievo del M° Michelangelo Lupone e del M° Nicola Bernardini, si è formato presso il Conservatorio “A. Casella” di L'Aquila e successivamente presso il Conservatorio “S. Cecilia” di Roma affrontando gli studi della Composizione elettroacustica e della Composizione classica. La musica elettroacustica è il settore nel quale svolge la sua principale attività musicale.}

\biografia{Massimo Massimi}{ si è formato musicalmente presso il Conservatorio Santa Cecilia di Roma, diplomato in liuto e musica elettronica, ha affrontando lo studio della musica antica e successivamente si è dedicato alla composizione elettroacustica con particolare attenzione all’interazione tra strumento e macchina.}

\biografia{Antonio Mazzotti}{ si è laureato in Ingegneria Elettronica presso il Politecnico di Bari e  specializzato in Signal Processing.  Ha proseguito gli studi accademici presso il Conservatorio di Bari, dove si è laureato con lode in Musica Elettronica, sotto la guida del M° F. Scagliola. I suoi interessi spaziano sulla composizione, assistita da calcolatore, di lavori elettroacustici e audiovisivi. Alcune sue composizioni sono state eseguite in vari festival internazionali come  ‘FIMU Festival 2012’, ‘Silence Festival 2012’, ‘New York City EMFestival 2013/14’, ‘ICMC-SMC 2014’, ‘ file.org.br 2015’, uvm2015.unb.br,  ICMC 2015.}

\biografia{Ursula Meyer-König}{Vive a Zurigo. Dopo una carriera come pediatra, ha intrapreso gli studi base di arte e media al HGKZ di Zurigo e la FH di Aarau, in Svizzera, seguito da un corso di composizione elettroacustica al Hochschule für Musik in Weimar, Germania, con il Prof. R. Minard. Attualmente studia composizione elettroacustica con il Prof. G. Toro- Pérez a ZHdk e ICST, Zurigo, Svizzera.}

\biografia{Enrico Minaglia}{Nasce a Bologna il 26 Dicembre del 1980. Studia composizione al conservatorio dell'Aquila con Alessandro Sbordoni, e al conservatorio di Milano con Fabio Vacchi e  Alessandro Solbiati.  Sempre a L'aquila ha svolto il ruolo di assistente del maestro Michelangelo Lupone nella classe di musica elettronica. In seguito si diploma in direzione d'orchestra. Ha arrangiato e condotto brani per film/tv/teatro musicale, As.Li.Co e Casa Ricordi.}

\biografia{Kenn Mouritzen}{Nato a Copenaghen (Danimarca) nel 1972. Vive e lavora a Vienna (A ) dal 2007. Ha studiato composizione elettroacustica con Germán Toro - Perez e Martin Neukom a ZHdK a Zurigo, Svizzera (fino al 2015). Ha inoltre conseguito un Master in letteratura comparata e Filosofia (2004). Recentemente la sua musica è stata presentata ai Festival UEM, Musicacustica Pechino, Noisefloor Festival, Festival Archipel, RIME, NYCEMF. È stato finanziato dalla Agenzia danese per la Cultura. Premio selezione a Bourges.}

\biografia{Roberto Musanti}{Roberto Musanti, musicista e media artist, autodidatta, diplomato in musica elettronica, insegna laboratorio di informatica e linguaggi di programmazione per la multimedialità. I suoi suoi lavori sono stati presentati, tra gli altri, ai festival “Zeppelin” Barcellona, “U.V.M.” Brasilia, “Kontakte” / “Music in touch” Cagliari, “Musica Viva” Lisbona, “Video Evening Photon Gallery” Lubjiana, ”MediaDepo” Lviv, “Electronicittà” Marseille, “Konsequenz” Napoli, “Decennale CEMAT”, “Saturazioni”, “EMUFest” Roma, “File Festival” Sao Paulo, “Simultan” Timisoara.}

\biografia{Giorgio Nottoli}{Giorgio Nottoli (compositore, nato a Cesena, Italia nel 1945) è stato docente di Musica Elettronica al Conservatorio di Roma “S.Cecilia” sino al 2013. Attualmente è docente di Composizione elettroacustica all’Università di Roma “Tor Vergata”. La maggior parte delle sue opere utilizza mezzi elettronici sia per la sintesi che per l'elaborazione del suono. Il centro della sua ricerca di musicista riguarda il timbro concepito quale parametro principale e "unità costruttiva" delle sue opere attraverso la composizione della microstruttura del suono. Nei suoi lavori per strumenti ed elettronica Giorgio Nottoli punta ad estendere la sonorità degli strumenti acustici mediante complesse elaborazioni del suono. Ha progettato vari sistemi elettronici per la musica utilizzando sia tecnologie analogiche che digitali in collaborazione con varie università e centri di ricerca.}

\biografia{Benjamin O'Brien}{compone, ricerca, ed esegue musica acustica e elettroacustica che si concentra su questioni di trasformazione e l'ascolto delle macchine. Ha conseguito un dottorato in musica presso l'Università della Florida, un MA in composizione musicale al Mills College, e una laurea in Matematica presso l'Università della Virginia. La sua opera è pubblicata dalla Oxford University Press, Taukay Edizioni Musicali, canadese Elettroacustica Comunità, e Seamus. Vive a Marsiglia, in Francia.}

\biografia{João Pedro Oliveira}{ ha completato il suo Dottorato di Ricerca in Musica all' Università Stony Brook di new York. ha ricevuto numerosi premi e riconoscimenti, inclusi tre premi alla Bourges Electroacoustic Music Competition, e il prestigioso Magisterium Prizes nella stessa competizione, il Giga-Hertz Special Award, il primo premio in Metamorphoses competition, etc.. He insegnante presso l'Università Federale di Minais Gerais (Brasile) e all'Università Aveiro (Portogallo).}

\biografia{Daniel Osorio}{Nato a Santiago del Cile. Nel 1996 inizia gli studi di composizione con il Prof. Pablo Aranda e di musica elettroacustica con il Prof. Edgardo Canton e Rolando Cori presso l'Università del Cile. Nel 2005 gli viene concessa una borsa di studio (Beca Presidente de la República - MIDEPLAN) da parte del Governo del Cile e si trasferisce a Saarbrücken / Germania, dove inizia i suoi studi post-laurea in Composizione con il Prof. Theo Brandmüller, il Dr. Prof. Stefan Litwin e Stefan Zintel alla Hochschule für Musik Saar.}

\biografia{Davide Palmentiero}{Nasce a Salerno il 19 Maggio 1993. Sei anni dopo inizia a suonare la chitarra classica, per poi passare alla chitarra elettrica all’età di 13 anni, iniziando a suonare e registrare con varie band e artisti senza distinzioni di genere. A 19 anni inizia ad affacciarsi alla Musica Elettronica e un anno dopo si iscrive al Conservatorio di Napoli; qui mostra particolare interesse per l’improvvisazione radicale, sperimentando soprattutto applicazioni e tecniche riguardanti la chitarra. Costruisce e sviluppa continuamente il proprio strumento, con il quale si esibisce in vari festival, rassegne e altri contesti sia in solo che con con diverse formazioni e diversi artisti, tra i quali Bob Ostertag.}

\biografia{Alessandro Pace}{laureato in Flauto con il M°Carlo Morena con la votazione di 110 e lode presso il Conservatorio di Santa Cecilia di Roma. Prosegue gli studi in flauto, affiancati dagli studi in Composizione tradizionale nello stesso conservatorio. Ha fatto e continua a fare molti concerti nei vari generi. Suona con diversi ensemble: Orchestra Ars Ludi Romana (anche come solista); Broadway Musical Orchestra (es. Festival di Todi); Indivenire Ensemble (repertorio contemporaneo). Ha suonato nell'orchestra nazionale di Panama a Panama City. Ha preso parte al festival Contaminazioni sia come flautista che come compositore. Ha seguito il progetto del M° Antonio Di Pofi sulla musica dei film muti (anche qui sia come flautista che compositore). Suona molta musica da camera in diverse formazione ed è in continua ricerca di nuove esperienze. Per la prima volta si esibirà ad EMUFest.}

\biografia{Carlos D. Perales}{Le sue opere sono state premiate ai concorsi internazionali 'Miniaturas Electroacústicas' - Confluencias (Huelva, 2008), Laboratorio del Espacio LIEM-CDMC (Madrid, 2010), XXII Concorso di Composizione SGAE (Madrid, 2011), Toy Piano Summit Mondiale (Lussemburgo, 2012), Musica Nova (Praga, Repubblica Ceca, 2012), Luigi Russolo (Francia, 2012), Fundación Destellos (Argentina, 2013). Ha conseguito il dottorato presso l'Universidad Politécnica di Valencia. Dal 2014 tiene lezioni di Composizione Elettroacustica al CSMCLM.}

\biografia{Alessandro Pirchio}{Studia presso il Conservatorio di Santa Cecilia con il M° Franz Albanese. Ha partecipato da solo o in formazioni cameristiche a la Rassegna “Musica a Roma per Roma”; il “Sutri Beethoven Festival; Stagione cameristica del Museo della ceramica di Viterbo. Ha suonato per lo spettacolo “La dodicesima notte” (Premio “Le maschere del teatro 2015” per le musiche originali del M° Piovani) in numerosi teatri italiani (Donizzetti di Bergamo, Ponchielli di Cremona, Verdi di Padova sono tra i più importanti). Attualmente ricopre la parte di Primo Flauto nella Banda della Gendarmeria Vaticana e dell’Ass. Nazionale Carabinieri.}

\biografia{Davide Palmentiero}{Nato nel 1991. Inizia la sua attività musicale come batterista studiando da privatista con il M° Salvatore Tranchini. Da sempre interessato alle sonorità più estreme e rumorose, si concentra inizialmente sul black metal e sull'hardcore, per cambiare poi indirizzo in seguito alla sua permanenza in Norvegia dove si appassiona alla musica elettronica extra-colta ed inizia la militanza nel collettivo techno Stavanger Teknomune, alfieri della cultura rave che getta le sue basi nell'utilizzo di strumenti analogici e del vinile. Dopo due anni decide di proseguire i suoi studi musicali a livello accademico, tornando a Napoli e iscrivendosi al triennio di musica elettronica con il M° Elio Martusciello. Ad oggi è attivo nell' estetica del rumore che egli ricerca negli elementi della vita quotidiana. Attualmente suona la batteria con il gruppo La Bestia Carenne.}

\biografia{Maurizio Pisati}{Nato a Milano nel 1959, è presente con propri lavori in festival d’Europa, Australia, USA, Giappone, America Latina. Sue composizioni sono state premiate in concorsi nazionali e internazionali (tra cui: Bucchi’83; Contilli’83; Rass. B. Brecht’85; Gaudeamus’86; ICONS’86; Petrassi’89), sono pubblicate da Casa-Ricordi, trasmesse da emittenti radiofoniche europee ed extraeuropee, sono incise su CD Ricordi-Fonit Cetra, Edipan, BMG, CavalliRecordsBamberg, Victor, Limen, ArsPublica, SiltaClassics e LArecords, etichetta indipendente da lui fondata nel 1997. ha compiuto gli studi musicali al Conservatorio di Milano, oltre che ai corsi estivi di Darmstadt e all’Accademia di Città di Castello, diplomandosi con il massimo dei voti in Composizione con S.Sciarrino, A. Guarnieri e G.Manzoni, e in seguito anche in Chitarra svolgendo attività concertistica in Europa dal1983 al1989 col gruppo Laboratorio Trio. Al Conservatorio di Bologna insegna di Composizione per la Musica Applicata, Elementi di Composizione per la Didattica, Invenzione \& Interpretazione, e nella stessa sede nel 2014 fonda CRS - Centro di ricerche musicali.}

\biografia{Karen Power}{Ambienti quotidiani e rumori di ogni giorno si trovano al centro della pratica di Karen con un continuo interesse che porta ad offuscare la distinzione tra ciò che la maggior parte di noi chiama musica e il resto dei suoni. Ha trovato ispirazione nel mondo naturale e come noi rispondiamo agli spazi che occupiamo; utilizza continuamente la nostra familiarità inerente con tali suoni e spazi. www.karenpower.ie}

\biografia{Federico Ripanti}{Nato a Roma nel 1987, studia Musica Elettronica presso il Conservatorio "S. Cecilia" di Roma. Nel 2009 si diploma in Fonia e Music Technology presso la Saint Louis Music College. Ha studiato privatamente pianoforte, chitarra elettrica e percussioni africane.}

\biografia{Alessandro Ratoci}{Studi musicali di composizione, pianoforte e musica elettronica presso il Conservatorio di Bologna, Master of Arts in composizione acustica ed elettronica alla HEM di Ginevra, perfezionamento al cursus IRCAM 2014-2015 in Parigi. Compositore, interprete di musica elettronica e didatta, insegna alla HEMU di Losanna e al Conservatorio G.B.Martini di Bologna. Le sue musiche sono state eseguite dal Ictus Ensemble Trio, Modern Ensemble Academy, Orchestra de la HEM di Ginevra, Orchestra di Radio France}

\biografia{Matteo Rossi}{, percussionista, si diploma con il massimo dei voti presso il Conservatorio “S.Cecilia” di Roma con Gianluca Ruggeri. Segue il corso di perfezionamento presso l’Accademia Musicale Chigiana con Antonio Caggiano, e come membro del Chigiana Percussion Ensemble, si esibisce al CHIGIANA INTERNATIONAL FESTIVAL, RAVELLO FESTIVAL e MAXXI di Roma. Collabora con formazioni orchestrali e cameristiche quali PMCE, InDivenire Ensemble ed ensemble di percussioni quali Ars Ludi, Blow-Up Roma Percussion, Aere Silente con cui si esibisce in un repertorio percussionistico moderno e contemporaneo in diversi eventi quali Le esperienze del minimalismo, Le Forme del Suono, ArteScienza, EMUFest.}

\biografia{Demian Rudel Rey}{(Argentina - October 24, 1987) Compositore. Si è diplomato al conservatorio di Musica "Piazzolla" e presso L'università Nazionale delle Arti. è stato premiato al TRINAC, Trime, FINM, BIENALbahìaBlanca, SADAIC,CONDIT, ecc. È stato selezionato per il MUSLAB 2014 (Messsico), Interensemble2015 (Italia) e SIRGA Festival 2015 (Spagna). Ha partecipato come "LiveSamplingPlayer" a Les Chants de l'Amor di Grisey, Usina del arte e in Das Mädchen mit den Schwefelhölzern di Lachenmann al Teatro Colón.}

\biografia{Gianluca Ruggeri}{Performer, direttore, autore e didatta. Diplomato in Strumenti a percussione e Direzione di Coro. Dopo gli esordi come percussionista nelle orchestre lirico-sinfoniche di Roma, ha incentrato il suo lavoro sul repertorio solistico e cameristico contemporaneo concentrandosi sulla ricerca elettro-acustica (K. Stockhausen, B. Truax, Y. Taira, M. Lupone) e sulla “performance” (J. Cage, G. Battistelli, L. Hiller, L. Berio) Nel 1987 ha fondato con Antonio Caggiano, ARS LUDI, un ensemble modulare con cui si è esibito in tutto il mondo. In veste di direttore ha diretto opere di F. Evangelisti, K. Stockhausen, M. Betta, C. Crivelli, M. Fischione, L. Cinque, C. Cardew, L. Berio, S. Reich, B. Sorensen, De Machaut e I. Stravinsky. Attualmente si dedica in vari modi all’approfondimento dell’opera di S. Reich. È docente di Strumenti a Percussione presso il Conservatorio di Musica “S.Cecilia” di Roma.}

\biografia{Dimitrios Savva}{Nato a Cipro, 1987. È diplomato con lode in Composizione presso la Ionian University di Corfu e laureato (con lode) in Composizione Elettroacustica presso l'Università di Manchester. Attualmente è dottorando alla Scheffield University sotto la supervisione di Adrian Moore. Le sue composizioni sono state suonate in Grecia, Cipro, Regno Unito, Germania, Belgio, Francia, Italia, Portogallo, Brazile e Usa.}

\biografia{Dominique Schafer}{Nato in Svizzera a Friburgo, è un compositore la cui espressività musicale spazia da lavori per strumenti acustici a lavori multimediali elettroacustici. Le sue composizioni sono state eseguite dall’Arditti String Quartet, Dinosaur Annex Ensemble, Ensemble Fa, Boston Modern Orchestra Project (BMOP), Talea Ensemble, Frances Marie Uitti, Alarm will Sound, California EAR Unit, nel Musica Nova Finland Festival e nel June di Buffalo, tra i tanti.}

\biografia{Claudia Jane Scroccaro}{Laureata in musicologia all’Università “Tor Vergata” di Roma, ha studiato direzione d’orchestra con Piero Bellugi. Durante il dottorato in Teoria e Analisi Musicale alla McGill University di Montreal, decide di tornare in Italia per studiare composizione con Luigi Verdi al conservatorio di musica “S. Cecilia”. I suoi lavori sono stati eseguiti presso la British School of Rome, la M.K. Ciurlionis School of Arts in Lituania, l’Auditorium “Ennio Morricone” di Tor Vergata e il London College of Music. Ha composto le musiche per il film-documentario “I’m coming home” premiato al Sidney Film Festival e all’International Filmmaker Festival of World Cinema di Milano; è stata Composer in Residence “DAR 2015” per la Lithuanian Composer’s Union.}

\biografia{Giuseppe Silvi}{[Tivoli (RM) - 1981] Studia Saxofono e Musica Elettronica presso il Conservatorio "S. Cecilia" di Roma. Si diploma in Musica Elettronica nel 2013 con Giorgio Nottoli. Sue musiche vengono eseguite in diversi Festival di musica elettroacustica, nel 2013 il brano \textit{A. SAX.} (per sax e live electronics) viene eseguito al festival Internazionale "Monaco Electroacoustique" ed è finalista al Concorso Franco Evangelisti con il brano PS: \textit{Song \#04} (per mezzosoprano, percussioni ed elettronica). È Tecnico del Suono specializzato in registrazioni surround, incide per edizioni Tactus, Naxos, Brilliant Classic e Sony.}

\biografia{Arturo Tallini}{ ha suonato in tutta Europa, negli Stati  Uniti, in Egitto, Algeria e Tunisia. È docente al Conservatorio di Santa Cecilia in Roma e tiene regolarmente masterclass nei conservatori e italiani e università straniere. Considerato all'unanimità un riferimento per il repertorio contemporaneo, collabora con artisti di fama internazionale, Michiko Hirayama, il gruppo di musica contemporanea \textit{Modus Novus} di Madrid, il Coro dell’Accademia Nazionale di Santa il flautista Carlo Morena. È coordinatore del Master Annuale di II Livello in Interpretazione della Musica Contemporanea del Conservatorio di Santa Cecilia in cui è anche docente di chitarra.}

\biografia{Anna Terzaroli}{Laureata in Musica Elettronica presso il Conservatorio Santa Cecilia di Roma, attualmente sta concludendo il Biennio specialistico presso lo stesso Conservatorio. Come compositrice si dedica alla musica contemporanea acustica ed elettroacustica, suoi lavori sono selezionati e presentati in vari concerti e festival, in Italia e all'estero. Dal 2009 collabora a EMUfest, è membro del Consiglio Direttivo dell'AIMI (Associazione Informatica Musicale Italiana).}

\biografia{Gianni Trovalusci}{Diplomato in flauto al Conservatorio S. Cecilia, ha approfondito il repertorio contemporaneo con P.-Y. Artaud a Parigi e la prassi esecutiva della musica antica con J. Christensen e O. Peter presso la Schola Cantorum di Basilea. Dagli anni settanta è attivo nel campo della musica contemporanea, antica, nel teatro musicale e performance d'avanguardia; ha lavorato con importanti artisti e si è esibito nei più importanti festival europei e nazionali. È Segretario Artistico della Federazione Cemat.}

\biografia{Giovanni Ubertini}{A seguito del diploma in pianoforte, conseguito brillantemente (9.25/10) presso il Conservatorio "O. Respighi" di Latina, segue corsi di perfezionamento con lo statunitense Charles Rosen e con il M° Donella D'Alessio. Sotto la guida del M° Luigi Sacco, nel 2009 si diploma (10 cum laude) in organo e comp. organistica presso il Conservatorio di Latina e nel maggio 2014, sotto la guida del M° Alessandro Licata, conclude il biennio in organo e comp. organistica (110 cum laude e menzione d’onore) presso il Conservatorio “S. Cecilia”. Attualmente è triennalista nel corso di Direzione di coro e comp. corale nella classe del M° Mauro Bacherini presso il Conservatorio di Latina. Ai titoli musicali affianca la laurea in Giurisprudenza.}

\biografia{Kyle Vanderburg}{Kyle Vanderburg compone ecletticamente musica polistilistica, alimentata da unità ritmica e infatuazione melodica. Oltre ad essere compositore, è anche attivo programmatore di computer, scrive codici per performance interattive, servizi di pubblica utilità relativi alla automazione del workflow, e di controllori inusuali.}

\biografia{Daniele Vantaggio}{ (Roma, 1987) è un produttore, sound designer e sound engineer. Si diploma alla St.Louis di Roma nel 2006 con il M° Luca Spagnoletti, in sintesi del suono e HD Recording e ottiene attestato di Sound Engineering con Vittorio Nocenzi e Lorenzo Pozzi. Dal 2009 studia presso il dipartimento di Musica Elettronica del Conservatorio Santa Cecilia. Da sempre il suo interesse è rivolto al suono della scena underground. Si esibisce in Europa e Sud America. Attualmente si occupa di produzioni discografiche, post-produzione, produzioni cinematografiche e teatrali, tiene corsi di formazione e conduce un programma radiofonico nazionale.}

\biografia{Massimo Varchione}{Nasce in Svizzera nel 1979. Diplomato in Composizione presso il Conservatorio Nicola Sala con il M° Luigi Turaccio. Studia Musica Elettronica presso il Conservatorio San Pietro a Majella di Napoli prima con il M° Agostino Di Scipio ed ora con il M° Elio Martusciello. Ha composto brani strumentali, elettroacustici e realizzato installazioni. Ha iniziato di recente un nuovo percorso dedicato all'improvvisazione radicale con il mezzo elettroacustico e con gli strumenti. In duo con il clarinettista Agostino Napolitano, nel 2014 è stato selezionato dal centro Tempo Reale di Firenze per partecipare al loro festival dedicato all'elettronica.}

\biografia{Clemens Von Reusner}{è un compositore e soundartist residente in Germania, il cui lavoro si concentra sulla musica acusmatica. Le sue composizioni hanno ottenuto trasmissioni ed esecuzioni a livello internazionale.}

\biografia{Benjamin D. Whiting}{si diploma in Composizione e prende un master in Teoria musicale e composizione presso la Florida State University, ed è attualmente dottorando presso la University of Illinois in Urbana-Champaign. Compone sia musica acustica che elettroacustica, e i suoi lavori sono stati eseguiti negli Stati Uniti e all'estero, ed editi dall'etichetta Experimental Music Studios della University of Illinois.}

\biografia{Giuseppe Zampetti}{Compositore. Nato a Roma nel 1992. Allievo del M° Francesco Telli, studente di Composizione indirizzo Contemporaneo presso il Conservatorio “S. Cecilia” di Roma.}

\biografia{Francesco Ziello}{Si è laureato nel 2012 in Musica Elettronica al Conservatorio di Roma “S.Cecilia”, dove parallelamente ha studiato Pianoforte e Composizione. Polistrumentista attivo in diverse formazioni, partecipa come esecutore e performer in vari festival legati al mondo della musica contemporanea. Come compositore ha collaborato con l’Accademia Nazionale di Danza per uno spettacolo presentato al Centro di Ricerche Musicali nel Luglio 2015. Attualmente iscritto al Biennio di Musica Elettronica sotto la guida di Michelangelo Lupone.}
