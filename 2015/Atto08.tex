CONCERTO 8
Concerto 20.30 (Sala Accademica)

\livel{Enrico Minaglia}
{Physis III - Metal}%7'10''
{per trombone e live electronics}
{trombone}{Raffaele Marsicano}

\descrizione{Physis III - Metal}{This is the third piece of the Physis series, a tribute to Giordano Bruno's philosophy. Physis III  is focused on the "De gli eroici furori dialogue". It aims to depict in music the calm life of the wise man, and its sudden tranformation into the "furioso", a man who struggles to penetrate the misteries of Nature, and in the end will pay this knowledge with his life, like the mythical hunter Actaeon, and Giordano Bruno himself.}

\descrizione{Physis III - Metal}{Si tratta del terzo brano della serie "Physis", un tributo al pensiero di Giordano Bruno. Physis III è basata sul dialogo "De gli eroici furori". Ciò che la musica rappresenta è la vita tranquilla del saggio, e la sua improvvisa trasformazione nel "furioso", un uomo che lotta per penetrare i misteri della natura e che pagherà la sua conoscenza con la sua stessa vita, come la cacciatrice Actaeon, e Giordano bruno stesso.}


\livel{Núria Giménez-Comas}
{No More Words}{2015} %6'27''
{per soprano e live electronics}
{soprano}{Virginia Guidi}

\descrizione{No More Words}{Different concepts and ideas has given some inputs to the global conception of the piece: as information overload and non-useless information or text-speechs. To work with this ideas, I have used an open software of “speech to text” (Mary) of some “historical apologies” (found on the net) with different “artificial voices”. After I have granulated this texts to create voice masses in the space around us, (using a library of Open Music called Prisma). To recreate the idea of different spaces ‘saturation’ I have also introduced to the beginning some fragments of noisy landscapes with spoken voices and resynthesis layers of this soundscapes. This sound waves and masses are combined with a voice line that appears gradually and it’s given by the poem of Edgard Allan Poe. In “A dream within a dream” the poet is dealing in what we perceive as a reality and non-reality.}%aggiungere testo Poe

\descrizione{No More Words}{Idee e concetti differenti hanno contribuito alla concezione globale del brano: il sovraccarico di informazioni, i dati non inutili, il testo parlato. Per lavorare con queste idee, ho usato un software libero "speech to text" (Mary) con alcune "scuse storiche" (trovate in rete), con diverse voci artificiali. Dopo di che ho granulato questi testi per creare masse vocali nello spazio tutt' attorno, (utilizzando una libreria di Open Music chiamata Prisma). Per ricreare l'idea di "saturazione ' dei diversi spazi ho inoltre introdotto all' inizio del brano alcuni frammenti di paesaggi rumorosi con voci recitanti e strati di resintesi di questi paesaggi sonori. Le onde sonore e le masse sono state poi unite ad una linea di voce che appare gradualmente dalla poesia di Edgar Allan Poe. In \textit{Un sogno dentro un sogno}", il poeta si occupa di ciò che percepiamo come realtà e non realtà.}%aggiungere testo Poe


\acusmatico{Silvia Lanzalone}
{eRose}{2013} %9'54''
{acusmatico}

\descrizione{eRose}{The word ‘eRose’ is composed by the words ‘electronic’ and ‘rose and was created for this piece to indicate the contrast between the natural and the virtual. The eRose piece is referred to the woman’s world: women’s bodies are modified and transfigured by digital communication, like ‘eroded roses’. Real sounds are totally transformed by computer to express the feeling of unease, but also the feeling of new discovery. The sounds are repeatedly clipped, ‘eroded’ or, on the contrary, sometimes carefully smoothed and refined.}

\descrizione{eRose}{La parola ‘eRose’ è stata coniata per questo brano attraverso la contrazione delle parole inglesi 'electronic' e 'rose', ed è utilizzata per i suoi eterogenei significati è anche per la struttura del fenomeno esogeno che descrive. Il brano è rivolto ad un universo femminile, di cui viene messo in evidenza il processo di decontestualizzazione che la comunicazione digitale ha intrapreso sull'immagine della donna. La parola 'eRose' Il materiale sonoro è infatti, continuamente eroso, corroso, abraso, oppure a volte, al contrario, anche meticolosamente limato, levigato, patinato. }

\livel{Luciano Azzigotti}
{Vilanos}{2014} %10''
{per due flauti amplificati}
{flauto}{Alessandro Pace, Elena D'Alò}

\descrizione{Vilanos}{When a flower dies, and it’s petals wither, a perfect sphere formes inside. Like the subway, the veins, the umbrellas, like the earthworm’s path... arboreal systems grow regardless of the materials that comprise them, displaying the same geometry. They really come to fill the grooves that already exist in the space. The flute sound is transformed into a swirling turbulence, air that is fragmented and divided into a spiral. Hence the re- feeding of a listening technique transforms holes into resonators, the duo as symmetrical exchange point.}

\descrizione{Vilanos}{Quando un fiore muore, e i suoi petali appassiscono, al suo interno si forma una sfera perfetta. Come la metropolitana, le vene, gli ombrelli, come il percorso sotterraneo di un lombrico... I sistemi boreali crescono senza riguardo verso i materiali che li comprimono, percorrendo la stessa geometria. Arrivano a ricoprire le scanalature che già esistono nello spazio. Il suono del flauto è trasformato in una vorticosa turbolenza, l'aria viene frammentata e divisa in una spirale. Pertanto una tecnica d'ascolto di rialimentazione reciproca trasforma i fori in risuonatori, il duo come un punto di simmetria centrale.}


\livel{Karen Power}
{Deafening silence}{2014} %5'54"
{pianoforte e live electronics}
{pianoforte}{Francesco Ziello}

\descrizione{Deafening silence}{is simply about fusing an acoustic instrument, in this case the piano, with a series of cricket recordings from Australia, Japan, Crete, Laos, Cambodia, Italy and Canada. Field recording has become a large part of my artistic practice and I hope that this is the first work in a series for solo acoustic instrument and natural environments from around the world. All recordings have been made by me while attentively listening to each space.}

\descrizione{Deafening silence}{Il silenzio assordante è semplicemente la fusione di uno strumento acustico, in questo caso il pianoforte, con una serie di registrazioni di grilli dell'Australia, Giappone, Creta, Laos, Cambogia, Italy e Canada. Il Field recording ha preso una larga parte della mia pratica artistica e spero che questo sia il primo lavoro di una serie per solo strumento acustico e ambienti naturali del mondo. Tutte le registrazioni sono fatte da me, mentre ascoltavo attentamente ogni spazio.}


\livel{Pasquale Citera}
{Musica per organi caldi}{2015} %6'
{intonazione e fuga elettroacustica per organo e live elctronics}
{organo}{Giovanni Ubertini}

\descrizione{Musica per organi caldi}{Usa questo QR code dal tuo dispositivo mobile per visualizzare o scaricare la partitura completa e le note d’esecuzione} %QR code partitura - link

\descrizione{Musica per organi caldi}{You can scan this QR code with your mobile device to view or download the complete score and performance notes}%QR code partitura - link


