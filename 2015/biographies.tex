\biografia{James Andean}{Musician and sound artist. He is active as both a composer and a performer in a range of fields, including electroacoustic composition and performance, improvisation, sound installation, and sound recording. He is a founding member of improvisation and new music quartet Rank Ensemble, and one half of audiovisual performance art duo Plucié/DesAndes. He has performed throughout Europe and North America, and his works have been presented around the world.}


\biografia{Damián Anache}{(1981, Bs As, ARG) Composer, Researcher and Teacher (CONICET, UNQ, FUC). His works have been played at: Roma (ITA), Morelia and DF (MEX), Quito (ECU), Córdoba, Rosario and Bs As (ARG). “Capturas del Único Camino” is his first album released by Inkilino Records and Concepto Cero labels (2014, ARG)

Website: damiananache.com.ar}


\biografia{Alfredo Ardia}{Class 1989, he studied at LEMS - SPACE (Pesaro, Italy) and at CMT (Helsinki, Finland). He is interested in sound, its perception and how it relates with other media, exploring sound phenomena of elementary sound entities and its behaviors. He is inspired by the beauty of physics.

Web: http://alfredoardia.altervista.org/}


\biografia{Luciano Azzigotti}{Composer based in Buenos Aires Argentina. His musical work encompasses different medias and listening, gesture, and writing interfaces. He studied composition at University of La Plata and in different scholarships and International seminars as active participant with Gerardo Gandini, Mauricio Kagel, George Aperghis, Chaya Czernowin and Rebecca Saunders. His music has been played in Argentina, Brasil, USA, Austria, Germany, France, Italy, by reputed ensambles and soloists. Founder of conDiT BsAs.}


\biografia{Christian Banasik}{(1963) studied composition at the Robert Schumann Academy of Music and Media in Dusseldorf and at the University of Music and Performing Arts in Frankfurt. His instrumental and electronic works have been featured in concerts and radio programs throughout Europe as well as in the Americas, Asia, and Australia. Banasik is lecturer for Audio Visual Design at the University for Applied Sciences and the artistic director of the Computer Music Studio "Studio 209" in Dusseldorf.}


\biografia{Carlo Barbagallo}{Carlo Barbagallo is a self-taught musician, composer, sound engineer, music producer, born in 1985 in Siracusa, Sicilia. He always recorded his music, experimenting the possibilities of home and studio creative recording.}


\biografia{Antonella Barbarossa}{eng}

%\biografia{Antonella Barbarossa}{è nata e vive in Italia. Didatta del pianoforte, organista, direttrice d’orchestra, compositrice, filosofa e missionaria in Calabria per scelta dove assume la docenza di pianoforte al Conservatorio di stato di Cosenza nel 1976. Nel 1991 diviene direttore del Conservatorio di Vibo Valentia sino al 2013; nel 2003 fonda il politecnico internazionale “ scientia et ars” per la specializzazione in tecnologia del suono. In campo organistico ha eseguito in concerti pubblici l’opera integrale di Bach, Franck, Liszt e Messiaen, e in prima assoluta per la Rai e festivals internazionali, composizioni d’autori contemporanei. È vincitrice del primo premio al Concorso internazionale organistico di Roma nel 1981 e finalista al Concorso Internazionale di Lipsia per lo stesso strumento.}


\biografia{Cathy Berberian}{eng}


\biografia{Luciano Berio}{eng}


\biografia{Francesco Bianco}{Graduated in Musicology, he studies Electronic Music at the Conservatory of Santa Cecilia in Rome. He has been visiting scholar at the CRR de Boulogne-Billancourt (Paris). Musician always been interested in the deep relationship between art and life, he has different genre and style musical experiences, from composition to live performance, from the scene to the soundtrack.}


\biografia{Isobel Blank}{http://www.isobelblank.com}

%\biografia{Isobel Blank}{Nasce a Pietrasanta in Toscana, si laurea con lode in filosofia estetica a Padova, vive e lavora a Torino. Le sue opere sono state esposte in numerose gallerie, musei e festivals, dagli Stati Uniti alla Cambogia, dal Messico alla Russia. Tra le esposizioni recenti, quella alla Triennale Fiberart International di Pittsburgh, al Museum of Modern Art di Mosca, a Palazzo Widmann di Venezia, alla Mumbai Art Room in India. Ha avuto diversi riconoscimenti tra cui il Primo Premio al Romaeuropa Webfactory nel 2009, sezione videoarte. http://www.isobelblank.com}


\biografia{Daniel Blinkhorn}{ is an Australian composer. He is currently lecturing into the composition and music technology department at the Conservatorium of Music, University of Sidney. He is also an ardent location field recordist, where he has embarked upon a growing number of recording expeditions throughout Africa, Alaska, Amazon, West Indies, Northern Europe, Middle East, Australia and the North Pole. His creative works have received over 25 international and national composition citations. He is self-taught in electroacoustic’s, however has formally studied composition and the creative arts at a number of Australian universities including UOW where his doctoral degree in creative arts was recommended for special commendation. Other degrees include a BMus (hons), MMus, and a MA(r).}


\biografia{Elisabetta Braga}{Born in Nardò (Le), she gradueted at Conservatorio "Santa Cecilia" in Rome in 2013. She performed in Italy and abroad, singing at Sala Accademica in Conservatorio "Santa Cecilia", "Teatro Politeama Greco" in Lecce, "Tchaikovskj Concert Hall" in Moscow. For Emufest, in 2012 she  performed "Anabasi" by G. Baggiani conducted by T. Battista. She debutted tre role of Mimì in Puccini's "La Bohème" in Rome and the role of Gilda in Verdi's "Rigoletto".  In Rome,  she attempted Sumi Jo's masterclass in april. She is going ti take "Diploma Accademico di Secondo Livello" at Conservatorio "Santa Cecilia".}


\biografia{Daniele Buccio}{Graduated in piano at the “A. Casella” Conservatory in L’Aquila and in composition at the “G. Verdi” Conservatory in Turin and obtained his PhD in Musicology at the “Alma Mater Studiorum” in Bologna. He attended composition masterclasses at the Accademia Filarmonica of Bologna, at the “L. Perosi” Academy in Biella, and at the Chigiana Academy. He performed at the Teatro Regio in Parma, at the Auditorium Parco della Musica in Rome, at the Deptford Town Hall for the Liszt Society of London, and at the Congress Centre in Lugano.}


\biografia{Sylvano Bussotti}{Born in Firenze on 10/01/1931. Starts studying on violin with Margherita Castellani before the age of five. He will study at the Florence’s conservatory “Luigi Cherubini” harmony and counterpoint with Roberto Lupi and the piano with Luigi Dallapiccola: studies that he will stop because of war, without achieving any official qualification. In Paris, from 1956 to 1958, he attended private course with Max Deutsch, he will meet Pierre Boulez and Heinz-Klaus Metzger, which will lead him to Darmstadt, where he’ll meet John Cage. In 1958, in Germany, he starts his public career, with the execution of his music by the pianist David Tudor, followed up by the presentation, in Paris, of tracks played by Cathy Berberian under the direction of Pierre Boulez.}


\biografia{Elisabetta Capurso}{is a pianist, composer, musicologist. She studied piano with Carlo Vidusso and Carlo Zecchi, composition with Domenico Guaccero and Brian Ferneyhough, orchestra with Daniele Paris, electronics music with Giorgio Nottoli. She obtained the degree in Letters Philosophy at the University “La Sapienza” of Rome, and in Electronics Music at the Conservatory S. Cecilia of Rome. She is well know on the international concert scene, and she has appeared, as a pianist, in the most prestigious halls invited by the principal Italian and foreign musical Institutions. Elisabetta Capurso is know in the world as well as modern and contemporary music performer, as well as composer. Performed at the Ferienkurses of Darmstadt and at some of the most important cultural Institutions like the ‘Foundation of contemporary Music Meeting’ her musical work is characterized by the elements of a learned counterpoint and of a remarkable capability of imparting its own knowledge. Her compositions have obtained a great success at some of the most important Italian musical Festivals, as well as on the Italian national networks RAI, as well as in many American countries. Among the others: Festival Nuova Consonanza, Rome; Antidogma Musica, Turin etc. Some works have been published by the Musical Editions: Zanibon (Peters) Padova ; Edi-Pan, Rome; AFM-Accord for Music SEDAM Rome. Piano Professor at ‘Rossini’ Conservatory in Pesaro, Piano Professor, Semiography and Laboratorio of contemporary music also at the ‘S. Cecilia’ Conservatory in Rome. She has received many prizes for her artistic values.}


\biografia{Simone Cardini}{He studies composition with F. Telli, piano with A. Torchiani; he participates in masterclasses held by S. Sciarrino, M. Andre, T. Tulev, M. Trojahn, P. Manoury. His compositions have been played in Europe and USA in eminent expositions and festivals like ArteScienza (2012), Contemporanea (2013), Nuova Consonanza (2013, 2014), Rondò (2014), NYCEMF (2015) by international ensembles like Divertimento Ensemble, PMCE and they have been awarde in competitions like AFAM (2013), Valentino Bucchi (2015).}


\biografia{Antonio Carvallo}{A. Carvallo was born in Chile in 1972. Studied counterpoint and harmony with Rodolfo Norambuena. Then, he studies at University of Chile, where he got a Bachelor and Master degrees in Composition. Was professor at University of Chile from 2000 to 2002. After that he moved to Rome, studing Electronic Music with R. Bianchini and G. Nottoli at “Conservatorio Santa Cecilia” getting an Academic degree of First and Second level. Nowadays he teaches at University of Chile.}


\biografia{Pasquale Citera}{He studied piano with M°Gemma D'Alessio, Composition with M°Luciano Pelosi and M°Giovanni Piazza and Electronic Music with M°Giorgio Nottoli. He has been working with several theatre companies, movie studios, sculptors and photographers. He has composed music for classical and contemporary theater shows, like: Alcestis (Euripides), Lysistrata (Aristophanes), Amphitryon (Plautus), Locandiera (Goldoni), L’Avare (Molière), Da quale parte del vetro (Silvio Nanni), Il dito sulla bocca (Donatella Ferrara), Certe notti non accadono mai (Patrizia Masi). He wrote soundtracks for Nero-Film, he is Assistant Music in several schools of Rome and has been Professor of Music Technology. From the collaboration with the sculptor Arturo Ianniello are born different soundtracks of visual works collected in two exhibitions. He is currently Composer and Sound Designer for incidental music at Anfitrione Theatre and “Quercia Del Tasso” Amphiteatre.}


\biografia{Cristiana Colaneri}{Born in Rome, she got the medium fulfillment of the Composition course (under the guidance of M. Pasquale Lucia). She is now studying Composition at the Conservatory of Santa Cecilia in Rome in the class of M. Francesco Telli. She is currently preparing the final exam of the first level academic diploma. She has been a finalist in composition contests: Mea 2010 and Bucchi 2015.}


\biografia{Valerio Cosmai}{Born in Rome in 1983. He studied piano, performing several times as a solo pianist in the sala Baldini of Rome and in various embassies, including the US and Indonesian embassies, specializing in the Mozart repertoire. He obtained in 2008 a degree in Literature from the University "La Sapienza" of Rome. He is graduated in percussion with honors in 2014. As a percussionist, he works with the orchestra of the Conservatory also performing in important festivals of contemporary music. Since 2012 he works as a teacher of musical education in the school Pio IX in Rome.}


\biografia{Giovanni Costantini}{(Corigliano d’Otranto - Lecce, 1965) Since 1995, he do research at the Faculty of Engineering of University of Rome Tor Vergata, where he teaches courses in Sound Processing and Electronic Music. He is also associate researcher at the Institute of Acoustics "O. M. Corbino" of Rome. At the University Tor Vergata, he is Director of the Master in SONIC ARTS. His musical research is currently focused on the creation of the microstructure and macrostructure of sound through the exploration and real-time processing of acoustic material.}


\biografia{Elena D'Alò}{ is a flutist and a piccolo player. She studied Flute (Diploma and Master's degree) in Conservatorio "Santa Cecilia" in Roma, with teachers: Edda Silvestri, Bruno Paolo Lombardi, Deborah Kruzansky, Paolo Taballione e Gianni Trovalusci. She also graduated in Acoustic Physics (Bachelor's degree) in "La Sapienza" University with Paolo Camiz as supervisor. Now she studies Electronic Music. She plays chamber music and in orchestral concerts too: from barocco to contemporary repertory. She also plays cello.}


\biografia{Maria Cristina De Amicis}{(Avezzano, 1968) She studied Composition, Electronic Music, Organ and Organ Composition at Conservatorio “A.Casella” in L’Aquila. Her professional activity is focused on the most advanced trends of musical language. She founded, with other musicians and scientific researchers, “Istituto Gramma" where, from 1989, she realizes her artistic and scientific activity. Her works have been performed in remarkable contemporary music events in Italy and abroad (Lyon, Paris, Barcelona, Aveiro, Madrid, Budapest, Athens,Thessaloniki, Berlin, Frankfurt, Vienna). Since 2012 she’s Professor of Electronic Music at Conservatorio “A.Casella” in L’Aquila.}


\biografia{Vittoriana De Amicis}{eng}


\biografia{Domenico De Simone}{Graduated in Piano, in Jazz and in Composition.
He obtained a diploma in the PhD course in composition at the Accademia Nazionale di Santa Cecilia with Azio Corghi in 2001. He has received a diploma in Electronic Music from the “Conservatorio di Santa Cecilia” with the maximum grade under the guidance of Giorgio Nottoli in 2004. He graduated cum laude in Electronic Music at the high course "Biennio Sperimentale di II Livello in Discipline Musicali” in 2006. He has also studied with Franco Donatoni, Ennio Morricone, Salvatore Sciarrino. His compositions have been performed in Italy and abroad (Canada, Argentina, Romania, China) and transmitted by Radio3.}


\biografia{James Dashow}{Has had commissions, awards and grants from the Bourges International Festival of Experimental Music (Prix Magistere), the Guggenheim, Fromm and Koussevitzky Foundations, Linz Ars Electronica, the Biennale di Venezia, the USA National Endowment for the Arts, RAI, the American Academy and Institute of Arts \& Letters, Prague Musica Nova et. al. In 2011 he was honored with the "CEMAT per la Musica" prize in recognition of his career of outstanding contributions to electronic music.}


\biografia{Gustavo Delgado}{Graduated in Electronic Music at the Conservatory of Music Santa Cecilia, in Rome, under tuition of composer Giorgio Nottoli. My main fields of interest are electroacoustic music composition, sound programming, sound design, sound engineering. I lecture in Conservatories of Music of Benevento and Latina.}


\biografia{Dennis Deovides A. Reyes III}{Studied music composition in his native Manila, Philippines. Dennis is currently pursuing his doctorate degree in music composition at the University of Illinois at Urbana-Champaign under the tutelage of Scott A. Wyatt. His compositions find inspiration in a wide range of subjects, from Asian music to modern art, and also incorporate elements of Philippine tradition. Dennis’ compositions have received numerous performances in Europe, Asia, and the United States.}


\biografia{Christian Eloy}{Born in Amiens where he studied flute and composition at the conservatoire national of region and at the conservatoire national superior of Paris. Flutist in an orchestra, then director of a music school, before his meeting with Ivo Malec and the GRM at Radio France. Christian ELOY is the founder of Octandre (composers's association); he is in charge of the electroacoustic department of the Conservatoire National de Region in Bordeaux and of the workshop at the GRM, Groupe de Recherches Musicales /City of Paris. Christian ELOY is the artistic director of the SCRIME, research and creation studio in the university of Bordeaux I. Several awards : prize of the europeen community poetry and music - prize " François de Roubaix ". Composer of over fourty pieces, instrumental, electroacoustic, vocal and pedagogical. Published by Billaudot, Fuzeau, Lemoine, Combre, Notissimo and Jobert. Publications at PUF (France), Johnston Ed.(Irlande), MIT press (US), Le mensuel littéraire et poétique (Belgique). Confluences (France).}


\biografia{Sara Ferrandino}{graduated in piano in September 2005 at the Morlacchi Conservatory of Perugia, in the class of M° L. Tanganelli. At the same institution, in March 2009, she passed the level II Academical Diploma with top marks and special mention. In July 2011 she obtained the specialist diploma for the postgraduated course held by Mº S. Perticaroli at the Santa Cecilia Academy in Rome. She has partecipated in more than 30 national and international piano competitions, always reaching the top. She plays both as a piano soloist and in chamber ensembles with important musicians in prestigious classical concert halls in Italy and abroad. Sara Ferrandino is a piano teacher for the pre-academical courses at Santa Cecilia. She also collaborates with the Conservatory of Perugia for the courses of horn, trumpet, flute, violin, oboe and works as an artistic consultants in other important musical institutions in Rome, organizing masterclasses and pianistic competitions at international level.}


\biografia{Enzo Filippetti}{is professor of Saxophone at Conservatorio “S. Cecilia” in Rom. In more than thirty years he gives concerts all over the world. He has performed at Biennale di Venezia, Mozarteum di Salisburgo, Rome, Milan, Paris, London, Berlin, Wien, Madrid, Bruxelles, Buenos Aires, Caracas, Riga, Birmingham, Köln, Lyon, St. Etienne (Francia), Principaute-Monaco-Monte-Carlo, Yeosu (Korea), Kawasaki, Adis Abeba, Chisnau, Taormina, Ravello. He has collaborated with Claude Delangle, Alda Caiello and Bruno Canino and many of the most important composers wrote for him more than a hundred works. As a soloist and with the Quartetto di Sassofoni Accademia he has recorded for the Nuova Era, Dynamic, Rai Trade and Cesmel. He has published studies for Riverberi Sonori and he direct a collection for the Sconfinarte editions.}


\biografia{Alessia Forganni}{ (Brescia, 1982) graduated in Piano at Luca Marenzio Conservatory under the guidance of M° M. Zana. She also graduated at University in D.A.M.S. Since 2007 she has lived and taught in Rome: in the last years she combined classical piano with a modern approach: playing in the piano duo named Duel she had the chance to perform in Europe, Russia, Lebanon and South Africa between 2009 and 2015. Now she is finishing bachelor’s degree in Electronic Music at Santa Cecilia Conservatory: her research aspires to find a personal compromise between her classical background, extemporary exploration, use of the voice and contemporary influence.}


\biografia{FREI}{FREI is a project of Paolo Gatti and Francesco Bianco, created to investigate the aspects related to the live creation. Improvisation and experimentation are the starting points of their poetics: offering an experience of live electronics in an active way. The live performance is based on selected elements, which, during the show, are processed and developed. The instrumentation used is constituted by two laptops with digital systems made by the own composers.}


\biografia{Javier Alejandro Garavaglia}{Composer/performer (viola & electronics). Associate Professor at the CASS Faculty, London Metropolitan University (UK). Compositions –performed in several countries of Europe, the Americas and Asia– include: acousmatic, audio-visual, solo/chamber/ensemble/orchestral works with or without the inclusion of electronic/interactive media. Some electroacoustic works commercially available CD releases (Germany, USA, Argentina and Denmark). Research topics: dramaturgy of music; full automation of live electronics; special sound diffusion systems. 

http://icem-www.folkwang-hochschule.de/~gara/ }


\biografia{Jorge García del Valle Méndez}{(1966) grew up in Spain, where he studied bassoon and composition. Now he lives in Dresden (Germany) where he studied composition and electronic music. His compositions are worldwide premiered and broadcasted. Commissions from international institutions. Works on digital analysis and sound processing, applied to theory and composition. Salvatore Martirano Composition Award University of Illinois, Composition Award Sächsischer Musikrat.}


\biografia{Paolo Gatti}{born in Rome in 1982. He took the B.Sc. degree in environmental engineering and a post graduate master in sound engineering at "Tor Vergata" University. Then he studied computer music at "Santa Cecilia" conservatory, taking the B.A. degree under the guidance of G.Nottoli, and the M.A. degree under the guidance of M.Lupone and N. Bernardini. He's a composer, teacher and researcher in the field of musical expressivity. His works are performed in important events and international festivals. He wrote music for theatre, poetry and dance performances. His work Poltergeist was one of the awarded compositions at the end of the national final of the "Claudio Abbado" prize.}


\biografia{Núria Giménez-Comas}{She studied composition at the Escola Superior de Musica de Catalunya (ESMUC). After two years she continued her training at the Geneva Conservatory, where she studied composition with Luis Naon, and electroacoustic and instrumental with Michael Jarrell. She studied at Institut de Recherche et Coordination Acoustique/Musique (IRCAM) for two years, where she explored different type of synthesis and new spatialsiation system in 3D ambisonics. Núria has worked with musicians such as Harry Sparnaay, Klangforum's Wien trio, Ensemble Contrechamps, Brussels Philarmonic, and Diotima Quartet. She is a founding member of Ensemble Matka.}


\biografia{Virginia Guidi}{Graduated in Singing and Music Vocal Chamber Music at the Conservatory S. Cecilia where she specialized with honors in Music Vocal Chamber with S. Schiavoni with a thesis on the relationship between performer and composer in electroacoustic music. She has performed in Beijing– National Centre of the Performing Arts; Roma –Accademia Filarmonica, Tecnopolo, GNAM, Macro, MAXXI; Napoli – Arena Flegrea; Catania – Teatro Metropolitan, on national television (RAI 1, RAI 2, RAI 5, Telepace), and in important Festivals (EMUfest, ArteScienza). She sings with the Voxnova Italia “In the midst of things” Allora&Clazadilla’s with G. Coleman’s music at the Venice’s Biennale 2015.}


\biografia{Jan Jacob Hofmann}{Born 1966. Diploma, branch of architecture at the University Of Applied Sciences, Frankfurt in 1995. 1997 second diploma at the class of Peter Cook and Enric Miralles at the Städelschule, Art School Frankfurt in 1995, a postgraduate class of conceptual design and architecture. Associate researcher at the "Signal Processing Applications Research Group", University of Derby, England. Nominated for the German Prize For Sonic Arts of the Museum for Sculpture "Glaskasten" of the City of Marl in 2006.}


\biografia{Sandro L'Abbate}{Class 1988. Graduated in photography at Fine Arts Academy in Italy. He is interested in audiovisual production, using interactive and electronics systems to observe physical phenomena. He is currently facing the sea. Web:http://sandrolabbate.altervista.org}


\biografia{Silvia Lanzalone}{composer (Salerno, 1970). Degree in Flute, Composition and Electronic Music at Conservatory of Salerno, L’Aquila and Rome. Her compositions are performed in Italy and abroad and are published by Ars Publica, Taukay, Suvini Zerboni. She works since 1997 at CRM - Centro Ricerche Musicali as musical assistant, researcher, composer. She is Professor of Electroaclustic Composition and Head of the Department of New Technology and Musical Languages of the Salerno Conservatory. (http://www.silvialanzalone.it/)}


\biografia{Jean-Francois Laporte}{http://www.jflaporte.com}

%\biografia{Jean-Francois Laporte}{Compositore, esecutore ed inventore di strumenti musicali. Attivo sulla scena artistica contemporanea dalla fine degli anni ’90, l’artista canadese ha un approccio creativo ibrido, che unisce assieme sound art, composizione, interpretazione, performance, installazione sonora e arte digitale. Artista piuttosto intuitivo, ha appreso l’arte attraverso sperimentazioni concrete con la materia, basando il suo approccio alla composizione sull’ascolto attivo e sull’attenta osservazione della realtà di ciascun fenomeno. Nel corso degli anni ha dedicato una grande quantità di energie all’invenzione, lo sviluppo e la costruzione di nuovi strumenti musicali. È fondatore e direttore artistico delle produzioni Totem Contemporain di Montreal.
http://www.jflaporte.com}


\biografia{Gy\"orgy Ligeti}{eng}


\biografia{Jones Margarucci}{Electroacoustic music composition at the State Conservatory of Music “G. Martucci” and at KMH (Royal College of Music Stockholm). His music has been played in several festivals in Europe and North America, and has been selected for: Redshift Music - Postal Pieces (Vancouver – Canada – 2013); Vox Novus Fifteen Minutes of Fame - Yumi Suehiro (New York City – USA – 2014); Sonorities Festival 2015 (Belfast – North Ireland – 2015); SOUNDkitchen’s Earspace/Frontiers Festival 2015 (Birmingham – UK – 2015); Video Remakes - Call for Tape Music (La Fabbrica del Vedere) (Venice - Italy - 2015)}


\biografia{Marco Marinoni}{(10/April/1974) is a professor at the Conservatory of Music “G. Verdi” of Como (Italy) where he teaches Electroacoustic Performance Practice. He gained a M. Mus. Conservatory Degree in Computer Music (2007) 10/10 cum laude, a Master’s Degree in Sound Direction and Live-Electronics at the Conservatory of Music “B. Marcello” of Venice (2007), 110/110 cum laude with Alvise Vidolin and a Master’s Degree in Composition (2013), 110/110 cum laude.}


\biografia{Raffaele Marsicano}{ graduated in Trombone in 2006 at the Salerno Conservatory. He also graduated in Wind Band Instrumentation (2011) and Composition (2015) at the Milano Conservatory, where hÈs still studying Composition in the upper course. Due to his double nature of composer and trombonist, he focused his researches on the study of new sounds of brass instruments applied to contemporary music.}


\biografia{Francesc Martí}{ is a mathematician, computer scientist, composer, sound and digital media artist born in Barcelona and currently living in the UK. As composer and video artist, his works have been performed or exhibited all over the world, including international festivals, events and exhibitions. Currently, he combines his artistic and technology projects with his teaching Audio Technology and Image at Open University of Catalonia, and Music Technology at the De Montfort Uni. of Leicester.}


\biografia{Mario Mary}{is a Doctor of “Aesthetic, Science and Technology of Arts” (University Paris VIII, France), actually he teaches "Electroacoustic Composition" at Academy Rainier III in Monaco, and is the artistic director of Monaco Electroacoustique – Electroacustic Music International Encounter. Between 1996 and 2010, he teaches at the University Paris VIII. He worked as a composer in research at the IRCAM. Teacher, researcher and composer, Mario MARY has been invited by numerous institutions to make compositions and to give conferences. His music has been distinguished in more than twenty composition competitions. 

http://ipt.univ-paris8.fr/mmary/}%foto


\biografia{Massimiliano Mascaro}{Composer. He was born in Rome in 1986. He studies with M° Michelangelo Lupone and M° Nicola Bernardini. He studied at the Conservatory "A. Casella" in L'Aquila and He currently attends The Concervatory of "S. Cecilia " in Rome. He attends courses of Electroacoustic Composition and Classical Composition. The Electroacoustic music is the field in which he mainly carries out his musical activity.}


\biografia{Massimo Massimi}{, developed his music backgrounds at the Santa Cecilia Conservatory of Rome, graduated in lute and electroacoustic composition. After dealing with early music, he has dedicated himself to electronic music, with special attention to instrument and machine interaction.}


\biografia{Antonio Mazzotti}{ graduated in Electronic Engineering at Polytechnic of Bari (Italy) and I received a degree of specialization in Signal Processing. Later, I continued in academic studies at Conservatory of Bari, where I graduated cum laude in Electronic Music, under the guidance of F. Scagliola. My compositions have been performed at festivals:Fimu Festival 2012,Silence Festival 2012, New York City E.M.Festival 2013-14,ICMC-SMC 2014,ICMC-SMC 2014,file.org.br 2015,uvm2015.unb.br,ICMC 2015.}


\biografia{Ursula Meyer-König}{ lives in Zurich. After a career as a pediatrician, she undertook foundation and media art studies at the HGKZ in Zurich and the FH Aarau, Switzerland, followed by a continuation course in electro-acoustic composition at the Hochschule für Musik in Weimar, Germany under Prof. R. Minard. She is currently studying electroacoustic composition under Prof. G. Toro-Pérez at ZHdK and ICST, Zurich, Switzerland.}


\biografia{Enrico Minaglia}{Born in Bologna on the 12/26/1980. I graduated in Composition at L'Aquila Conservatory with Alessandro Sbordoni, and at Milano Conservatory with Fabio Vacchi and Alessandro Solbiati. I've been attending for a year as unofficial pupil Michelangelo Lupone's Electronic music class at L'Aquila Conservatory. I'm about to graduate in Conducting at L'Aquila Conservatory. I work as arranger, orchestrator and conductor for movies/TV/theatre music productions, As.Li.Co and Casa Ricordi.}


\biografia{Kenn Mouritzen}{Born in Copenhagen (DK) in 1972. Lives and works in Vienna (A) since 2007. He studied composition of electroacoustic music with Germán Toro-Perez and Martin Neukom at ZHdK in Zürich, Switzerland (until 2015). He also holds a Master's Degree in Comparative Literature and Philosophy (2004). Recently his music has been featured at EMU Festival, Musicacustica Beijing, Noisefloor Festival, Festival Archipel, RIME, NYCEMF. He was funded by the Danish Agency for Culture. Selection price at Bourges.}


\biografia{Roberto Musanti}{Roberto Musanti, electronic musician and media artist. His works have participated to “Opera Nuda” Amsterdam, “Zeppelin” Barcelona, “UVM Symposium” Brasilia, “Kontakte” / “Music in touch” Cagliari, “Musica Viva” Lisboa, “Video Evening Photon Gallery” Lubjiana, ”MediaDepo” Lviv, “Electronicittà” Marseille, “Konsequenz” Napoli, “Decennale CEMAT”, “Saturazioni”, “EMUFest” Roma, “File Festival” Sao Paulo, “Simultan” Timisoara.}


\biografia{Giorgio Nottoli}{Giorgio Nottoli (composer, born 1945 in Cesena, Italy) he was Professor of Electronic Music at the Conservatory of Rome "Santa Cecilia" until 2013. He currently teaches electroacoustic composition at the University of Rome "Tor Vergata". The major part of his works are realized by means of electro-acoustic media both for synthesis and processing of sound. The objective is to make timbre the main musical parameter and a "construction unit" through the control of sound microstructure. In the works for instruments and live electronics, the aim of Giorgio Nottoli is to extend the sonority of the acoustic instruments by means of complex real time sound processing. He has designed both analog and digital musical systems in conjunction with various universities and research centers.}


\biografia{Benjamin O'Brien}{composes, researches, and performs acoustic and electro-acoustic music that focuses on issues of translation and machine listening. He holds a Ph.D in Music at the University of Florida, a MA in Music Composition from Mills College, and a BA in Mathematics from the University of Virginia. His work is published by Oxford University Press, Taukay Edizioni Musicali, Canadian Electroacoustic Community, and SEAMUS. He lives in Marseille, France.}


\biografia{João Pedro Oliveira}{ completed a PhD in Music at the University of New York at Stony Brook. He has received numerous prizes and awards, including three Prizes at Bourges Electroacoustic Music Competition, the prestigious Magisterium Prize in the same competition, the Giga-Hertz Special Award, 1st Prize in Metamorphoses competition, etc.. He is Professor at Federal University of Minas Gerais (Brazil) and Aveiro University (Portugal).}


\biografia{Daniel Osorio}{Born in Santiago de Chile. In 1996 he begins Composition Studies with Prof. Pablo Aranda and Electroacoustic Music with Prof. Edgardo Cantón and Rolando Cori at the University of Chile. In 2005 he is granted a scholarship (Beca Presidente de la República - MIDEPLAN ) by the Government of Chile and moves to Saarbrücken/Germany where he starts his postgraduate studies in the field Composition with Prof. Theo Brandmüller, Dr. Prof. Stefan Litwin and Stefan Zintel at Hochschule für Musik Saar.}


\biografia{Davide Palmentiero}{Born in Salerno on May 19, 1993. Six years later he started playing classical guitar, before moving to the electric guitar. At the age of 13, he began playing and recording with various bands and artists of any kind. At 19 he began to work with electronic music and a year later he enrolled at the Conservatory of Naples; here shows particular interest in radical  improvisation, especially experiencing software and technical issues relating guitar. It builds and continuously develops his instrument, playing it in various festivals, exhibitions and other contexts both solo and with other formations and different artists, including Bob Ostertag.}


\biografia{Alessandro Pace}{Graduated in Flute with  M°Carlo Morena with score of "110 e lode" at the music academy of Santa Cecilia. Now following his studies with flute and traditional composition in the same music academy. He did and does a conspicuous concerts activity of various genres. He plays in the following ensembles: Orchestra Ars Ludi Romana (also as a soloist); Broadway Musical Orchestra (es. Festival di Todi); Indivenire Ensemble (contemporary repertoire). He played with the Panama's national orchestra in Panama City. He performed in the Contaminazioni festival both as flutist and composer. He joined the project of M° Antonio Di Pofi  about Silent film's music, both composing and playing for it. He played quite a lot of chamber music with various ensembles and continuously search for new experiences. First time (en)joining EMUFest.}


\biografia{Carlos D. Perales}{His works have been awarded at international competition contest likÈMiniaturas Electroacústicas' - Confluencias (Huelva, 2008), Laboratorio del Espacio LIEM-CDMC (Madrid, 2010), XXII Composition Contest SGAE (Madrid, 2011), Toy Piano World Summit (Luxembourg, 2012), Musica Nova (Prague, Czech republic, 2012), Luigi Russolo (France, 2012), Fundación Destellos (Argentine, 2013). PhD by Universidad Politécnica de Valencia. Since 2014 lectures Composition & Electroacoustic music at CSMCLM.}


\biografia{Alessandro Pirchio}{He studies at the Santa Cecilia Conservatory in Rome with M° Albanese. He has performed as soloist and in chamber music ensembles for: Musica a Roma per Roma”; “Sutri Beethoven Festival”; Chamber Music season of the Viterbo Ceramic Museum. Moreover, he played for the theatrical performance “Twelfth Night” (“Le maschere del teatro 2015” Award for original soundtrack by M° Piovani) in several Italian theaters (“Donizetti” in Bergamo, “Ponchielli” in Cremona, “Verdi” in Padova, Pistoia and Ravenna, among the most important). He is currently First Flute in the band of the Vatican Gendarmerie and Carabinieri National Association.}


\biografia{Davide Palmentiero}{Born in 1991. He began his musical activity as a drummer, studying with Salvatore Tranchini. Always interested in  extreme and noisy music, was focused initially on black metal and hardcore. During his stay in Norway, he developes a passion for electronic music and starts a militancy in collective techno Stavanger Teknomune, apostles of rave culture that sets theirs roots in the use of analog instruments and vinyl. After two years he decided to continue his musical studies at an academic level, returning to Naples, and signing up to the course of Electronic Music with Elio Martusciello. Today his research is based on the aesthetic qualities of the noise, explored in the elements of everyday life. Currently he plays drums with the italian music group Bestia Carenne.}


\biografia{Maurizio Pisati}{Born in Milan in 1959, it’s present with his works in “Festival of Europe”, Australia, Usa, Japan, Latin America. His compositions have been awarded in national and international contests (between them: Bucchi ’83; Contilli ’83, Rass, B.Brecht ’85, Gaudeamus ’86, ICONS ’86, Petrassi ’89), they are published by Casa-Ricordi and transmitted by radio stations in the whole world. His works have been recorded on many different labels such as Ricordi-Fonit, Cetra, Edipan, BMG, CavalliRecordsBamberg, Victor, Limen, ArsPublica, SiltaClassics and LArecords, independent label founded by himself in 1997. His musical studies have been achieved at the Conservatorio di Milano, and integrated with summer camps in Darmstadt and in the Accademia di Città di Castello, he had his diploma with highest honours in composition with S. Sciarrino, A. Guarnieri and G. Manzoni. After that he got a diploma in Guitar, by playing concerts in Europe from 1983 to 1989 with the band Laboratorio Trio. At the conservatory of Bologna teaches Composition for Applied Music, Elements of Composition for educational purpose, Invention & Interpretation, and in the same place he founded CRS- Center for musical research, in 2014.}


\biografia{Karen Power}{Everyday environments and how we hear everyday sounds lies at the core of Karen’s practice with a continued interest in blurring the distinction between what most of us call ‘music’ and all other sound. She has found inspiration in the natural world and how we respond to spaces we occupy. She continually utilizes our inherent familiarity with such sounds and spaces as a means of engaging with audiences. Resulting works challenge the listeners’ memory of hearing. 

More @ www.karenpower.ie}


\biografia{Federico Ripanti}{Born in Rome in 1987, Federico Ripanti is currently enrolled in Electronic Music at the “S. Cecilia” Conservatory of Music. In 2009 he graduated in Audio and Music Technology at the Saint Louis Music College. He attended private classes of piano, electric guitar and African percussions.}


\biografia{Alessandro Ratoci}{, composer and electronic music performer was born in 1980 in Tuscany. He obtained his master degree in composition at the Haute école de Musique de Genève with Michael Jarrel, Luis Naon and Eric Daubresse. Subsequently he moved to Paris to follow tuition at IRCAM cursus and at ManiFeste Academy. He is currently professor of computer music and sound engineer at the Haute école de Musique (HEMU) in Lausanne, Switzerland. His music has ben performed by ICTUS ensemble trio, Orchestre de Radio France, Barcelona Modern Ensemble and International Ensemble Modern Academy.}


\biografia{Matteo Rossi}{percussionist. He studied percussions in “S.Cecilia” Conservatory  with Gianluca Ruggeri. He is attending a perfectioning course with the Chigiana Musical Accademy directed by Antonio Caggiano and performs as a member of Chigiana Percussion Ensemble at Chigiana International Festival, Ravello Festival and MAXXI in Rome. Also collaborates with orchestras and chamber ensembles such PMCE, InDivenire Ensemble and percussion ensembles which Ars Ludi, Blow-Up Roma Percussion, Aere Silente with whom he performs in a modern and contemporary percussionistic repertory during the events such as Le esperienze del minimalismo, Le Forme del Suono, Artescienza, Emufest.}


\biografia{Demian Rudel Rey}{(Argentina - October 24, 1987) Composer. He has graduated at Conservatory of Music “Piazzolla” and at National University of Arts. He was awarded in TRINAC, TRIME, FINM, BIENALBahíaBlanca, SADAIC, conDiT, etc. He has been selected in MUSLAB 2014 (Mexico), Interensemble2015 (Italy) and SIRGA Festival 2015 (Spain). He has participated as LiveSamplingPlayer in Les Chants de l'Amour by Grisey in Usina del Arte and in Das Mädchen mit den Schwefelhölzern by Lachenmann in Teatro Colón.}


\biografia{Gianluca Ruggeri}{Performer, director, composer and teacher. He graduated in percussion instruments and choir conducting. After an eraly career as a percussionist in symphonic orchestras of Rome, he focused his work on the solo repertoire and chamber contemporary research, focusing on electroacoustic music ( K . Stockhausen, B. Truax, Y. Taira, M. Lupone) and performance (J. Cage, G. Battistelli, L. Hiller, L. Berio). In 1987 he founded with Antonio Caggiano, ARS LUDI, a modular ensemble with which he has performed all over the world. He has conducted many works by F. Evangelisti, K. Stockhausen, M. Betta, C. Crivelli, M. Fischione, L. Cinque, C. Candrew, L. Berio, S. Reich. Is teacher of percussion instruments at the Conservatory of Music "Santa Cecilia" in Rome.}


\biografia{Dimitrios Savva}{Born in Cyprus, 1987. He received his Bachelor degree (distinction) in music composition from the Ionian University of Corfu and his Master degree (distinction) in Electroacoustic composition from the University of Manchester. In January 2015 he started his PhD in Sheffield University under the supervision of Adrian Moore. His compositions have been performed in Greece, Cyprus, United Kingdom, Germany, Belgium, France, Italy, Portugal, Brazil and USA.}


\biografia{Dominique Schafer}{A native of Fribourg, Switzerland, Dominique Schafer is a composer whose breadth of musical expression encompasses both, acoustic instrumentation and electroacoustic media. His music has been performed by ensembles and performers such as the Arditti String Quartet, Dinosaur Annex Ensemble, Ensemble Fa, Boston Modern Orchestra Project (BMOP), Talea Ensemble, Frances Marie Uitti, Alarm will Sound, and the California EAR Unit, at festivals such as Musica Nova Finland, June in Buffalo, and others.}


\biografia{Claudia Jane Scroccaro}{Graduated in Musicology at “Tor Vergata” University of Rome and studied Orchestral Conducting with Piero Bellugi. While pursuing a Phd in Music Theory at McGill University in Montreal, she decided to return to Italy to study composition with Luigi Verdi at “S. Cecilia” Conservatory of Music. Her music has been performed at the British School of Rome, at the M.K. Ciurlionis School of Arts in Lithuania, at the “Ennio Morricone” Auditorium of Rome, and at the London College of Music. She composed the soundtrack for the film-documentary “I’m coming home” awarded at the Sidney Film Festival and at the International Filmmaker Festival of World Cinema in Milan; she was Composer in Residence “DAR 2015” for the Lithuanian Composer’s Union.}


\biografia{Giuseppe Silvi}{eng}


\biografia{Arturo Tallini}{ played all over Europe, in United States, Tunisia, Algeria, Egypt. He is Guitar Professor at Santa Cecilia Conservatory, in Rome and he teaches in other Italian conservatories and abroad univiersities. In more than 30 years of career, he became a reference point for contemporary music, with many pieces written the contemporary music ensemble \textit{Modus Novus} the Accademia Nazionale di Santa Cecilia Choir, Carlo Morena, and Enzo Filippetti. He is coordinator and professor at the M.A. Master in Contemporary Music at Santa Cecilia Conservatory in Rome.}


\biografia{Anna Terzaroli}{She holds a Bachelor's degree in Electronic Music from the Santa Cecilia Conservatory in Rome, where she is currently completing a Master's degree in Electronic Music. As a composer she is dedicated to contemporary acoustic and electroacoustic music. Her musical works are selected and presented in many concerts and festivals in Italy and abroad. Since 2009 she collaborates at the EMUfest festival. She is a member of the AIMI (Italian Computer Music Association) board.}


\biografia{Gianni Trovalusci}{Contemporary repertoire with Pierre-Yves Artaud in Paris and Performing Practices of Early Music with Jesper Christensen and Traversiere with Oskar Peter at the Schola Cantorum Basileensis. He performed in the field of contemporary and ancient music, in music theatre and avant-garde performance at very important Festivals as NYCEMF New York City Electroacoustic Music Festival; Munich Biennale; Nuova Consonanza, Museo Casa Scelsi, Musica e Scienza, EMUFest Rome; M.A.N.C.A. Festival, Nice; GAS Festival, Goteborg; Udine Jazz Festival; REC Reggio Emilia Contemporanea, British Film Institute London, Nancy Opera, Flanders Opera, Ars Electronica - BrucknerHaus Linz, Neue Alte Musik Cologne, CCA Glasgow, Stockholm New Music, Nits de Musica Mirò Foundation Barcelona, etc.}


\biografia{Giovanni Ubertini}{After the diploma in piano, brilliantly achieved (9.25/10) at the Conservatory "O. Respighi" of Latina, following courses with the American Charles Rosen and with M° Donella D'Alessio. Under the guidance of Luigi Sacco, in 2009 he graduated (10 cum laude) in organ and organ comp. at the Conservatory of Latina and in May 2014, under the guidance of M° Alessandro Licata, he follows graduating in organ and organ comp. (110 cum laude and honorable mention) al the Conservatory “S. Cecilia” in Rome. He currently follows the musical course in Choir conducting and choir comp. with M° Mauro Bacherini at the Conservatory of Latina. Finally, the music titles alongside a degree in Law from "La Sapienza" of Rome, and two years of law practice.}


\biografia{Kyle Vanderburg}{Kyle Vanderburg composes eclectically polystylistic music fueled by rhythmic drive and melodic infatuation. In addition to composing, he is an active computer programmer, writing code for interactive performances, utilities related to composer workflow automation, and unusual controllers.}


\biografia{Daniele Vantaggio}{ (Rome, 1987) is a producer, sound designer and sound engineer. He graduated in HD recording at Saint Louis College of Rome in 2006, under the guidance of M° L. Spagnoletti. He also obtained a degree in Sound Engineering thanks to V. Nocenzi and L. Pozzi. Since 2009 he has studied Electronic Music at Santa Cecilia Conservatory. He always developed a big interest in underground musical scene and he performed in Europe and South America. Currently he takes care of production, post-production, soundtracks for cinema and theatre. He teaches and also directs a national radio program.}


\biografia{Massimo Varchione}{Born in Switzerland in 1979. He graduated in composition at the Conservatory Nicola Sala with Luigi Turaccio. He study Electronic Music at the Conservatory San Pietro a Majella in Naples before with Agostino Di Scipio and now with Elio Martusciello. He composed instrumental pieces, electroacoustic and created installations. He has recently started a new journey dedicated to performing radical improvisation with electroacoustic and acoustic instruments. In duo with clarinetist Agostino Napolitano, in 2014 was selected by the center for electronic music Tempo Reale in Florence, to participate to the homonymous festival.}


\biografia{Clemens Von Reusner}{Clemens von Reusner is a composer and soundartist based in Germany, who is focused on acousmatic music. International broadcasts and performances of his compositions. www.cvr-net.de}


\biografia{Benjamin D. Whiting}{ received his BM in Music Composition and his MM in Music Theory and Composition from Florida State University, and is now pursuing his DMA at the University of Illinois at Urbana-Champaign. He is an active composer of both acoustic and electroacoustic music, and has had his works performed in the United States and abroad, and released on the ABLAZE and University of Illinois Experimental Music Studios labels.}


\biografia{Giuseppe Zampetti}{Composer. He was born in Rome in 1992. He studies with M° Francesco Telli. He attends Contemporary Composition at the Conservatory of "S.Cecilia" in Rome.}


\biografia{Francesco Ziello}{eng}
