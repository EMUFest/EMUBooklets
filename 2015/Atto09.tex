CONCERTO 9
Concerto 17.00 ACUSMATICI (aula Bianchini)
Presentazione e Regia del Suono Luana Lunetta

\acusmatici{Daniel Osorio}
{Spiegelung}{2013} %9'38''

\descrizione{Spiegelung}{"Spiegelung" (Image on the Mirror) is a work commissioned for the DVD project "Long Sguardo". The work is based on the Stefano AmoresÈs text "Specchiatura" and later became part of the video by visual artist Horkay Istvàn. "Spiegelung", an acousmatic piece for 5.1 surround, was composed from recordings by soprano Franziska Erdmann and narrator Jens Alles, to which was added the digital processing of different sound sources, also recorded for the work. The complexity and length of the text allowed generating different sound perspectives through different forms of sound synthesis, which however, are subordinate to rhythm and sound density of spoken text (German).}

\descrizione{Spiegelung}{"Spiegelung" (immagine a specchio) è un lavoro commissionato per il progetto DVD "Long Sguardo". Il lavoro si basa sul testo di Stefano Amorese "Specchiatura" e, in seguito, è entrato a far parte del video dell'artista visivo Horkay Istvàn. "Spiegelung", un brano acusmatico per surround 5.1, è stato composto con registrazioni del soprano Franziska Erdmann e del narratore Jens Alles, a cui è stata aggiunta l'elaborazione digitale delle diverse sorgenti sonore, anch'esse registrate per il lavoro compositivo. La complessità e la lunghezza del testo hanno permesso la generazione di prospettive sonore differenti attraverso diverse forme di sintesi del suono, che sono comunque subordinate al ritmo e alla densità del suono del testo parlato (in Tedesco).}

%\descrizione{Spiegelung}{„Spiegelung„ es una obra encargada para el proyecto en DVD „Largo Sguardo“ y se basó en el texto de Stefano Amorese („Specchiatura“). Es una pieza compuesta para surround 5.1 a partir de las grabaciones Franziska Erdmann y Jens Alles, a lo que se agregó el procesamiento digital de diferentes fuentes sonoras, también grabadas para la obra. La complejidad y extensión del texto permitió generar diferentes perspectivas sonoras a través de diferentes formas de síntesis de sonido.}


\acusmatici{Daniel Blinkhorn}
{frostbYte - wildflower}{2014} %13'00''

\descrizione{frostbYte - wildflower}{frostbYte is the last in a cycle of works using field recordings from the high artic region of Svalbard. What was most discernible when recording fragments of glacial ice floating in fjords were the many and varied sonorous ecosystems emanating from underwater, each with its own distinctive personality. In every instance the ice fragments reacted differently to temperature, pressure and other observable phenomena, producing similar, yet unique sonorities. Throughout the work I wanted to capture some of the delicate complexity, as well as the unified symmetries produced through the charismatic, audible ecosystems indelibly linked to each of the naturally formed ice sculptures.} % In order to transcribe, then sculpt these natural carvings into gestures and phrases within the piece I chose to de-construct a number of hydrophone recordings into discrete elements, often organised into families of sound shapes. These typomorphologies were then re-constructed into a variety of gestures, phrases and forms, each of which contained its own attendant ecosystem of sound, much like the original field recordings. From a broader perspective, the resultant phrases are intended to mimic the idea of something that is carved. To my mind the final geometries and patterns sculptured became like those of the short-lived wildflowers that grow in the region, each populating its own unique ecosystem and all subject to the natural forces at play around them.


\descrizione{frostbYte - wildflower}{frostbYte è l'ultimo di un ciclo di opere che utilizzano registrazioni ambientali realizzate nell'alta regione artica di Svalbard. Ciò che è risultato più evidente durante le registrazioni dei frammenti di ghiaccio galleggianti nei fiordi, sono stati i molti e vari ecosistemi sonori sott'acqua, ognuno con la propria distinta personalità. I frammenti di ghiaccio hanno reagito in modo diverso a seconda della temperatura, pressione e altri fenomeni osservabili, producendo sonorità simili, ma uniche.In questo lavoro ho voluto catturare un po ' della delicata complessità e contemporaneamente le simmetrie unificate, prodotte attraverso i carismatici ecosistemi sonori indelebilmente legati a ciascuna delle sculture di ghiaccio formatesi naturalmente.} %Al fine di trascrivere, poi scolpire queste sculture naturali in gesti e frasi all'interno il pezzo, ho scelto di de-costruire una serie di registrazioni con idrofoni in elementi discreti, spesso organizzati in famiglie di forme sonore. Queste tipomorfologie sono state poi ri-costruite in una varietà di gesti, frasi e forme, ognuna delle quali conteneva un proprio ecosistema sonoro, proprio come i field recordings originali.Da una prospettiva più ampia, le frasi risultanti sono destinate ad imitare l'idea di qualcosa che viene scolpito. A mio avviso, le geometrie finali e i modelli scolpiti sono diventati come quelli dei fiori selvatici dalla breve vita che crescono nella regione, i quali popolano il proprio unico ecosistema e sono tutti soggetti alle forze naturali che agiscono intorno ad essi.

\acusmatici{Antonio Carvallo}
{Vri}{2013} %6'59''

\descrizione{Vri}{The piece is made from the deployment of numerical proportions in charge of controlling the parameters that regulate the synthesis and processing of sound, proportions that somehow guarantee a certain sonority and organization that tends to achieve unity. The piece is made from synthesis subtractive (white noise filtered) and convolution. The resulting sounds are analyzed and their formants with more energy are filtered; finally, an analysis and resynthesis process generates frequency bands that are being shipped, differentially, to the four channels.}

\descrizione{Vri}{Il pezzo è realizzato utilizzando il dispiegamento di proporzioni numeriche incaricate di controllare i parametri che regolano la sintesi e l'elaborazione del suono, proporzioni che in qualche modo garantiscono una certa sonorità e organizzazione, in modo da raggiungere l'unità. Il pezzo è realizzato in sintesi sottrattiva (rumore bianco filtrato) e convoluzione. I suoni risultanti sono analizzati e le formanti con più energia vengono filtrate; infine, un processo di analisi e risintesi genera bande di frequenza che vengono spedite, in modo differenziato, ai quattro canali.}


\acusmatici{Kenn Mouritzen}
{Cat-back}{2015} %8'39''

\descrizione{Cat-back}{Cat-back is a micro composition on basis of bass clarinet. The recordings were extensively treated in order to give other character while at the same time keeping the powerfull mechanical effect this instrument has.}

\descrizione{Cat-back}{Cat-back è una micro composizione basata su clarinetto basso. Le registrazioni sono state ampiamente trattate in modo da dare un altro carattere al suono, ma al tempo stesso mantenere il potente effetto meccanico dello strumento.}


\acusmatici{Kyle Vanderburg}
{Reverie of Solitude}{2014} %10'00''

\descrizione{Reverie of Solitude}{The piece serves as both an exploration of and a invitation to reverie, providing a space wherein the listener is asked to reconsider their idea of what it means to daydream. At once immersed in a familiar crowd hum, lost among the multitude; it is easy to believe that this daydream is not an expression of solitude, but rather a longing for solitude. And so the piece suggests the pattern of a day dream: the crowd noise giving way to a train, a lazy lawn sprinkler, a contemplative rain storm, a frothing river which becomes a bucolic afternoon on the lake. Each vignette is a self-contained narrative wherein to consider solitude in a natural context. The metaphor of water and the alternating themes of movement and respite invite the listener to reflect on the purpose of a daydream: to escape, to pacify, or to enrich a perfect moment. After having their attention turned to the daydream they themselves have been lulled into, the listener is returned to the crowd hum having established a personal sense of solitude within the piece and within the audience.

Program Note by Walter Jordan}

\descrizione{Reverie of Solitude}{Il brano può essere considerato sia come un'esplorazione che un invito al fantasticare, fornendo uno spazio in cui l'ascoltatore è invitato a riconsiderare l' idea di ciò che significa sognare ad occhi aperti. All'inizio l'ascoltatore è immerso in un familiare ambiente con ronzio di folla, perso tra la moltitudine; è facile credere che questo sogno ad occhi aperti non sia espressione della solitudine, ma piuttosto un desiderio di solitudine. E così la composizione suggerisce il modello di una fantasticheria: il rumore della folla cede il posto a quello di un treno, un pigro sistema d'irrigazione per prati, una contemplativa tempesta di pioggia, la schiuma di un fiume che diventa un pomeriggio bucolico sul lago. Ogni vignetta è una narrazione autonoma che prende in considerazione la solitudine in un contesto naturale. La metafora dell'acqua e l'alternarsi di  tematiche di movimento e di tregua invitano l'ascoltatore a riflettere sullo scopo del sogno: la fuga, per pacificare, o per arricchire un momento perfetto. Dopo essere stato cullato dal sogno ad occhi aperti, l'ascoltatore viene restituito al ronzio della folla dopo aver stabilito un personale senso di solitudine all'interno del brano e nel pubblico.

Note di Walter Jordan}