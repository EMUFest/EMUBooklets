CONCERTO 4


\livel{Alessia Forganni, Daniele Vantaggio}
{KOMA - Kontrolled Organic Movement Act}{2015} %10'00''
{Kontrolled Organic Movement Act}
{voce, pianoforte,  kinekt, e live electronics}{Alessia Forganni, Daniele Vantaggio}

\descrizione{KOMA - Kontrolled Organic Movement Act}{Our entire life is made by acts. Life is articulated day by day through repetition. Millions of automatisms assure us a cognitive and energetic saving. So life seems to be an automated system that is difficult to avoid or control, without a real consciousness of our actions. Maybe we should see our actions as a ritual gesture, instead of a simple repetition of the same act. Sometimes we should give time, meaning and attention to our acts to really control them. The performance includes tape, live piano, voices, live electronics and sound controlling by body movement through Kinect and Leap Motion system.}%abbiamo foto

\descrizione{KOMA - Kontrolled Organic Movement Act}{Tutta la nostra vita è fatta di gesti: grazie alla ripetizione di milioni di essi la vita è scandita ogni giorno. Forme di automatismi permettono un risparmio cognitivo ed energetico e alla maggior parte non si fa caso. La vita appare così un sistema automatizzato dal quale è difficile sottrarsi e avere reale controllo, uno stato di assenza di reale coscienza e consapevolezza. Per rendere i nostri gesti un atto consapevole bisogna vederli nella loro “ritualità” piuttosto che “ripetitività”, conferendovi significato e tempo,  focalizzandovi al massimo la nostra attenzione. La performance include nastro, piano e voci dal vivo, live electronics e controllo del suono tramite movimento con i sistemi di sensoristica Kinect e Leap Motion.}%abbiamo foto


\acusmatico{Massimo Massimi}
{Kronos in Fabula I}{2015} %10'00''
{acusmatico}

\descrizione{Kronos in Fabula I}{The acousmatic piece “Kronos in Fabula 1” is part of a series of compositions dedicated to the speculation about time. The material’s acoustic density represents the time trend; the possibilities of the listening points’ collocation on the time axis are connected to a feasible fruition “directionality” linked to the sequence and reiteration of sounds events.}

\descrizione{Kronos in Fabula I}{“Kronos in Fabula I” è l’acusmatico di un ciclo di composizioni dedicate alla speculazione sul tempo. La densità acustica del materiale rappresenta l’andamento del movimento temporale, le possibilità di collocazione di punti di ascolto sull’asse del tempo vengono associate a una possibile “direzionalità” dell’ascolto legata al concatenarsi e al reiterarsi degli eventi sonori.}


\livel{Claudia Jane Scroccaro, Federico Ripanti}
{On-y r\^eve}{2015} %7'40''
{per pianoforte e nastro}
{pianoforte}{Daniele Buccio}

\descrizione{On-y r\^eve}{The work is a “nocturnal representation” for piano and tape, suffused with suspended atmospheres and textures typical of oneiric visions. The piano part is mostly rhapsodic, while the acousmatic elements enhance and filter the resonances produced through the tonal pedal, conveying granulated processes of prepared piano. A contrast between darkness and light alludes to a chiaroscuro counterpoint, where each tile is devised manipulating and recomposing elements derived from Alexander Scriabin’s \textit{Poème-Nocturne} op. 61, in a tribute to commemorate the centenary celebration of his passing.}

\descrizione{On-y r\^eve}{The work is a “nocturnal representation” for piano and tape, suffused with suspended atmospheres and textures typical of oneiric visions. The piano part is mostly rhapsodic, while the acousmatic elements enhance and filter the resonances produced through the tonal pedal, conveying granulated processes of prepared piano. A contrast between darkness and light alludes to a chiaroscuro counterpoint, where each tile is devised manipulating and recomposing elements derived from Alexander Scriabin’s \textit{Poème-Nocturne} op. 61, in a tribute to commemorate the centenary celebration of his passing.}


\livel{Carlos D. Perales}
{17 haiku}{2012} %10'
{flauto e live electronics}
{flauto}{Alessandro Pace}

\descrizione{17 haiku}{Heirs of haikai, comic-themed and outgoing figuralisms in Japanese poetry, haiku was taking its own identity in the complex art of expression of consciousness. In them, senses are suggested, not the meanings. Haiku does not mean, because it says nothing. Indeed, haiku does not have an informative function. But, perhaps, to the expressive function, a suggestive function should be added.} %aggiungi pdf

\descrizione{17 Haiku} {Eredi degli haikai, fumetti a tema e figuralismi uscenti della poesia giapponese, gli haiku si sono guadagnati una propria identità nella complessa arte di espressione della coscienza. In essi si suggeriscono i sensi, non i significati. Haiku non significa, perché non dice nulla. Infatti, gli haiku non hanno una funzione informativa. Ma, forse, alla funzione espressiva bisogna aggiungere una funzione suggestiva.} %aggiungi pdf


\livel{Massimiliano Mascaro, Giuseppe Zampetti}
{L'abile medico}{2015} %10'00''
{per soprano, mezzosoprano e live electronics}
{soprano}{Elisabetta Braga}
mezzosoprano -- \textsc{Virginia Guidi}
\\

\descrizione{L'abile medico}{The piece is a prayer. Trying to go further, beyond the suffering it is a difficult task; and sometimes it causes reconsideration, but eventually the courage leads to victory. Il Sutra del Loto is one of the most important texts of the literature of Mahayana Buddhism. In the first section the breath, the primary form of meditation predisposes to concentration becoming  a prayer, the supplication of  the Mystic Law. Later, through meditation we realize the complexity of existence: an illusory chaos. Finally the resolute soul emerges, confirming definitively the depth of his Essence. To increase the listening ability and participation we recommend to close the eyes.}%aggiungere pdf

\descrizione{L'abile medico}{Il brano è una preghiera. Cercare di andare oltre, al di là della sofferenza è un compito difficile, e talvolta porta dei ripensamenti, ma alla fine il coraggio conduce alla vittoria. Il Sutra del Loto è uno dei testi più importanti della letteratura del Buddismo \textit{Mah y n a}. Nella prima sezione il respiro, forma primaria di raccoglimento, predispone alla concentrazione trasformandosi in preghiera, invocazione della Legge Mistica. In seguito, attraverso la meditazione si realizza la complessità dell'esistenza: è il caos apparente. Finalmente l'anima risoluta riemerge, affermando definitivamente la profondità della sua Essenza. Per aumentare la capacità di ascolto e di partecipazione si consiglia di seguire ad occhi chiusi.}%aggiungere pdf

