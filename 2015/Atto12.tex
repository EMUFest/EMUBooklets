% !TEX encoding = UTF-8 Unicode
% !TEX TS-program = XeLaTex
% !TEX root = EMU2015_booklet.tex

\livel{Jorge García del Valle Méndez}
{no sun, no moon}{2012} %9'20''
{per flauto basso ed elettronica su supporto}
{flauto basso}{Alessandro Pirchio}

\livel{Giovanni Costantini}
{Traccia sospesa}{2015} %7'35''
{pianoforte ed elettronica su supporto}
{pianoforte}{Sara Ferrandino}


\acusmatico{James Andean}
{Déchirure}{2013} %9'58''
{acusmatico}

\livel{Mario Mary}
{Rock}{2013} %8'30''
{per pianoforte ed elettronica su supporto}
{pianoforte}{Sara Ferrandino}

\acusmatico{Inhorep}
{Inhorep@emufest}{2015} %15'00''
{laptop e autocostruiti}
{chitarra preparata e laptop}{Davide Palmentiero}, strumenti autocostruiti e laptop -- \textsc{Giuseppe Pisano}, laptop -- \textsc{Massimo Varchione}
\\

% \descrizione{no sun, no moon}{no sun, no moon immerses into a world without reference points, where the reality is considered by two different points of view, and we do not know which is the real and which is not. In no sun, no moon, the bass flute and the electronics are the two worlds, reality and parallel reality. Both are the same thing and simultaneously its opposite, interacting and reacting one another. The raw materials for the electronics are exclusively bass flute samples. Honorary Mention at the CICEM 2014.}

\descrizione{no sun, no moon}{Con \textit{no sun, no moon} siamo immersi in un mondo senza punti di riferimento, in cui la realtà è considerata da due diversi punti di vista, e non sappiamo quale sia la reale e quale non lo sia. Il flauto basso e l'elettronica sono i due  mondi; realtà e realtà parallela. Entrambi sono la stessa cosa e contemporaneamente il loro contrario, interagiscono e reagiscono tra di loro. Le materie prime dell'elettronica sono esclusivamente campioni di flauto basso. Mensione d'onore al Cicem 2014.}

\descrizione{Traccia sospesa}{The piece evokes events, places, sounds and feelings related to the First World War. The piano is played in a "non-classical" way, exploring new sonorities useful to convey in the listener pain, dismay, disbelief. The electronics consists in a textures created through elaborations of piano sounds: an alter ego with which the piano can dialogue, in an atmosphere suspended between memories and uncertainties.}

\descrizione{Traccia sospesa}{Il brano vuole evocare avvenimenti, luoghi, suoni e sentimenti legati alla prima guerra mondiale. Il pianoforte diventa strumento utile a trasmettere dolore, sgomento, incredulità, mediante un utilizzo “non classico” che ne esplora sonorità nuove. La parte elettronica è costituita da una trama di tracce sonore realizzate al computer attraverso elaborazioni di suoni di pianoforte: un alter ego con cui dialogare, ricercando e ricordando, in un’atmosfera sospesa e di continua incertezza.}

% \descrizione{Déchirure}{Déchirure: a tearing, a painful separation... This piece involves a number of 'déchirures', both musical as well as figurative, although the only literal 'tearing' is saved for the final phrase. This work was composed for Presque Rien 2013, in which it received a Special Mention. For this project, sounds from Luc Ferrari's archives were made available to composers for the composition of new works; all sound materials used in the piece are sourced and developed beginning from these recordings.}

\descrizione{Déchirure}{Déchirure: uno strappo, una dolorosa separazione…Questo brano comporta una serie di "déchirure" (strappi), sia musicali, così come figurativi, anche se l'unico letterale "strappo" viene lasciato per la frase finale. Questo lavoro è stato composto per Presque Rien 2013, in cui ha ricevuto una Menzione Speciale. Questo progetto, i suoni sono stati resi disponibili da archivi di Luc Ferrari ai compositori per la composizione di nuove opere; tutti i materiali sonori utilizzati nel brano sonori sono di provenienza e sviluppati a partire da queste registrazioni.}

% \descrizione{Rock}{It is a kind of homage to progressive rock, also called symphonic rock, which appeared in the 70s, and continues to accompany me in my inner world, though my compositions are always distinctly contemporary aesthetics. In my teens, groups like Yes, Pink Floyd, Emerson Lake & Palmer, injected me the virus of electronic sound, before than I discovered the existence of contemporary and electroacoustic music. "Rock" is not a rock, but is inspired by the energy and character of the music of my youth. Moreover, this work is based on a virtuous dialogue between the instrumentalist and the electroacoustic part.}

\descrizione{Rock}{Si tratta di una sorta di omaggio al rock progressivo, chiamato anche sinfonico, che è apparso negli anni '70, e continua ad accompagnarmi nel mio mondo interiore, anche se le mie composizioni sono sempre un'estetica decisamente contemporanea. Nei miei adolescenza, gruppi come Yes, Pink Floyd, Emerson Lake \& Palmer, ho iniettato il virus del suono elettronico, prima di scoprire l'esistenza di musica contemporanea ed elettroacustica. "Rock" non è una roccia, ma si ispira l'energia e il carattere.}

% \descrizione{Inhorep@emufest}{\textit{INHOREP} is a project of radical electroacoustic improvisation. The trio (first active as Improvviso) was formed in the course of Electronic Music of the  Conservatory of Naples under the impulse of  Elio Martusciello. The members immediately focused themselves on live exhibition, participating in several Italian festivals. Their performance is based on the interplay between the three musicians (guests are always welcome), strengthened through rehearsals and on the the attitude of listening to each other. The musicians improve theirs instruments with a DIY (Do it Yourself ) philosophy, recovering instruments and other sound objects, using the circuit bending and other techniques.}

\descrizione{Inhorep@emufest}{\textit{INHOREP} è un progetto di improvvisazione radicale elettroacustica. Il trio (prima attivo come Improvviso) si è formato nella classe di Musica Elettronica del Conservatorio di Napoli sotto l'impulso del M° Elio Martusciello. Fin da subito il gruppo ha deciso di puntare sull'esibizione dal vivo, partecipando a diversi festival italiani. Le loro performance si basano sull'interazione tra i tre musicisti (e gli eventuali ospiti, sempre assai graditi), affinata attraverso le prove e l'attitudine all'ascolto reciproco. Individualmente, gli strumentisti costruiscono e migliorano costantemente i loro strumenti attraverso la filosofia D.I.Y. (Do it Yourself), recuperando strumenti e altri oggetti sonori, ricorrendo alle possibilità del circuit bending ed altre tecniche.}
