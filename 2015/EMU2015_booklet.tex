% !TEX encoding = UTF-8 Unicode
% !TEX TS-program = XeLaTex


\documentclass[8pt, twoside, a5paper]{extreport}
\usepackage{lipsum} % ----------------------------------------------------------- cancellabile
\usepackage{blindtext}	% dummy text

\usepackage{latexsym}
\usepackage[polutonikogreek, italian, english]{babel}
\usepackage[T1]{fontenc}
\usepackage[utf8]{inputenc}

\usepackage[margin=1cm]{geometry}

\usepackage{multicol}
\usepackage[hang, small,labelfont=bf,up,textfont=it,up]{caption} % Custom captions under/above floats in tables or figures
\usepackage{booktabs} % Horizontal rules in tables
\usepackage{float} % they need to be placed in specific locations with the [H] (e.g. \begin{table}[H])

\usepackage{graphicx}
\usepackage{amssymb}
\usepackage{epstopdf}

\usepackage{hyperref} % For hyperlinks in the PDF

\usepackage{titlesec} % Allows customization of titles
\renewcommand\thesection{\Roman{section}} % Roman numerals for the sections
\renewcommand\thesubsection{\Roman{subsection}} % Roman numerals for subsections
\titleformat{\section}[block]{\large\scshape\centering}{\thesection.}{1em}{} % Change the look of the section titles
\titleformat{\subsection}[block]{\large}{\thesubsection.}{1em}{} % Change the look of the section titles

% \usepackage{fancyhdr} % Headers and footers
% \pagestyle{fancy} % All pages have headers and footers
% \fancyhead{} % Blank out the default header
% \fancyfoot{} % Blank out the default footer
% \fancyhead{}%[C]{Sei volte EMUfest $\bullet$ Novembre 2013 }%$\bullet$ Vol. XXI, No. 1} % Custom header text
% \fancyfoot[RO,LE]{\thepage} % Custom footer text

\renewcommand{\thefootnote}{\textasteriskcentered}



\usepackage{mwe,tikz}
\usepackage{tcolorbox}

\usepackage{color}

\usepackage{fontspec,xltxtra,xunicode}
\defaultfontfeatures{Mapping=tex-text}
\setromanfont[Mapping=tex-text]{Source Sans Pro}
\setsansfont[Scale=MatchLowercase,Mapping=tex-text]{Source Sans Pro}
\setmonofont[]{Source Code Pro}

\newfontfamily{\svolk}{Volkhov}

\linespread{1.2}

\usepackage{lipsum}

\definecolor{supercolor}{RGB}{3,39,36}

%\topmargin -2.66in

\newtcbox{\mybox}{nobeforeafter,colframe=supercolor,colback=supercolor,boxrule=0.5pt,arc=8pt,
  boxsep=0pt,left=2pt,right=2pt,top=6pt,bottom=2pt,tcbox raise base}



%----------------------------------------------------------------------------------------
%	NEW COMMANDS
%----------------------------------------------------------------------------------------

\newcommand{\greco}[1]{%
\begin{otherlanguage*}{greek}#1\end{otherlanguage*}}

\newcommand{\brano}[6]{%
\noindent \textsc{#1}\\ %
\noindent \textbf{\textit{#2}} -- #3\\%
\noindent #4\\ %
\noindent #5 -- \textsc{#6}%
\\
}%

\newcommand{\installazione}[4]{%
\noindent \textsc{#1}\\ %
\noindent \textbf{\textit{#2}} -- #3\\%
\noindent #4%
\\
}%

%: #1 autore, #2 titolo, #3 anno ed esecuzione, #4 descrizione, #5 strumenti, #6 esecutore

\newcommand{\descrizione}[2]{%
\noindent \textbf{\textit{#1}} %
#2 %
\\
}%

\newcommand{\acusmatico}[3]{%
\noindent \textsc{#1}\\ %
\noindent \textbf{\textit{#2}} -- #3%
\\
}%

\newcommand{\biografia}[2]{%
\noindent \textsc{#1} %
#2 %
\medskip
}%

%----------------------------------------------------------------------------------------
%	TITLE SECTION
%----------------------------------------------------------------------------------------

\title{
	\svolk{CONSERVATORIO DI MUSICA S. CECILIA} \\
	%\vspace{-15mm}
	\fontsize{50}{50}
	\svolk{
		\emph{
			EMUFest 2015
			}
		}
	} % Article title

\author{
	\textsc{Conservatorio di Musica Santa Cecilia} \\
	5 -- 10 ottobre 2015 \\
	\textsc{Università di Roma Tor Vergata} \\
	13 e 14 ottobre 2015 \\
	Roma
}

\vfill

\date{}

%\dedica{.2\textwidth}{\small A Monica:\\
%per tutto ciò che mi hai insegnato\\
%e per tutto ciò che ancora avresti dovuto insegnarmi.}

%----------------------------------------------------------------------------------------

\begin{document}
\pagestyle{empty}
\maketitle 



%----------------------------------------------------------------------------------------

\section*{20 ottobre 2014 -- ore 20:30}

\subsection*{
	{\small CONCERTO 1\footnote{ Il concerto verrà trasmesso in diretta streaming da Radio Cemat}} \\
	\textsf{Sala Accademica}}

{\fontsize{30}{30} \svolk{\emph{ATTO I}}}

\subsection*{\textsf{A cura di Laura Bianchini}}

I brani proposti in questo concerto considerano lo spazio come parametro privilegiato per la creazione musicale.
Spazio di risonanza, spazio dell’immaginario, spazio determinato, spazio d’azione, spazio virtuale.
L’innesco (trigger), la dinamicità dell’azione nel processo creativo, trovano senso in questo concerto che intende stimolare il pubblico ad una cosciente fruizione.

\bigskip

\begin{multicols}{2}

\brano{Silvia Lanzalone}
{Èleghos}{2014}
{per flauto aumentato, tubo risonante ed elettronica}
{flauto}{Gianni Trovalusci}

\installazione{Marco Ferrazza}
{Radiale}{2014 (prima esecuzione italiana)}
{acusmatico}

\brano{Agostino Di Scipio}
{Studio sul rumore di fondo, nel tratto vocale (Ecosistemico udibile n.3b)}{2004/2005 (prima esecuzione assoluta)}
{per tre voci femminili}
{voci}{Concetta Cucchiarelli, \\Virginia Guidi, Angelina Yershova}

\brano{Feed-Band} %(Valentino De Luca, Francesco Quercia, Sara Disanto, Antonio Scarcia, Giuliano Scarola, Francesco Scagliola)}
{A Sad Song}{2013}
{feedback, indeterminazione e forma aperta di una performance}
{esecutori}{Valentino De Luca, Francesco Quercia, Sara Disanto, Giuliano Scarola, Antonio Scarcia, Francesco Scagliola}

%\hline

\vspace{.5cm}

\descrizione{Èleghos}{Il titolo evidenzia gli aspetti evocativi derivanti dal riferimento alla cultura ellenistica, in sintonia con l’antico termine greco \greco{ἐλεγεῖα} (eleghéia), o elegìa, e la parola \greco{έλεγος} (èleghos), il cui significato può essere indicato come "canto accompagnato dal flauto". Èleghos è stato scritto per il flautista Gianni Trovalusci.
Il flauto aumentato può essere considerato come una reinterpretazione del flauto moderno, le cui caratteristiche risultano rinnovate relativamente ad aspetti timbrici, di diffusione acustica e di controllo. Lo strumento, denominato Reso-Flute, è prodotto applicando al corpo del flauto tradizionale alcuni microfoni e sensori, allo scopo di rilevare dettagli sonori e gestuali. Il suono viene diffuso da un tubo risonante accordato sulle stesse frequenze di risonanza individuate dai microfoni interni allo strumento.
Reso-Flute è ideato e realizzato da Silvia Lanzalone in collaborazione con Antonio Marra (\href{http://www.resocap.it/}{www.resocap.it}) ed utilizza un flauto Vivaldi (\href{http://www.flautivivaldi.com/}{www.flautivivaldi.com}).}

%in caso di cambiamento grafico mettere in nota: Il flauto aumentato Reso-Flute utilizza un flauto Vivaldi

\descrizione{Radiale}{In \textit{Radiale} c'è un nucleo formato da pochi elementi sonori sviluppati lungo i vari corsi. Il pezzo si organizza in diverse direzioni derivate da quella originaria materia, assume diversi aspetti allo scopo di dimostrare diverse possibilità di espressione, ma, soprattutto, incoraggia a pensare le differenti modalità di arrangiare i segnali acustici a mano.}


\descrizione{Studio sul rumore di fondo, nel tratto vocale (Ecosistemico udibile n.3b)}{I brani e le installazioni del progetto \textit{Ecosistemico udibile} sono reti di interazioni sonore operanti in accoppiamento strutturale con l'ambiente che ospita l’esecuzione, nell’unità di tempo e spazio dell’esperienza. Istante dopo istante, il processo esecutivo crea le condizioni per la propria sopravvivenza (identità sistemica) e per le mutazioni di forma che ne emergono (differenziazione), trovandone le risorse nello spazio che accoglie l'esecuzione e l'ascolto. Lo \textit{Studio sul rumore di fondo} (commissionde del DAAD di Berlino, 2005) prova a far emergere qualcosa di musicale a partire da ciò che viene solitamente escluso dall'ascolto: si parte con un niente-di-suono e si prova a farne qualcosa nel corso dell’esecuzione, con l’ausilio di alcuni microfoni, alcuni altoparlanti, un computer. Nello \textit{Studio sul rumore di fondo, nel tratto vocale} questo stesso processo viene reso operativo in una nicchia più piccola dello spazio esecutivo, ma anche più flessibile: la bocca e il tratto vocale del corpo umano. Un microfono in miniatura sonda la cavità orale rilevandone residui sonori non-intenzionali, piccoli e involontari movimenti muscolari. Eventualmente, se l’equilibrio, incerto e variabile, tra rumore-di-fondo-interno (bocca) e rumore-di-fondo-intorno (sala) sfugge al controllo, l'esecutore deve ricorrere a misure di sicurezza e, con un minimo di violenza, reintegrare temporaneamente un migliore equilibrio dinamico. La versione presentata oggi è polifonica, e introduce un’ulteriore sistemico livello, relativo alla dinamica di interazione e di equilibrio tra i partecipanti all’esecuzione. I quali riescono ad agire in "ensemble" solo se, durante la performance, riescono ad ascoltare ed a procedere secondo intenzioni e decisioni comuni. }


\descrizione{A Sad Song}{è una composizione tripartita per chitarre, speakers, nastro e live electronics.
\textit{A Sad Song} è il desiderio di approdare a differenti traguardi, di percorrere e varcare spazi ancora inesplorati. Alla ricerca di modalità espressive nuove si serve di strumenti tradizionali utilizzando di essi tuttavia caratteristiche non consuete, ponendo al centro della realizzazione un elemento onnipresente nei vari parametri della musica senza il quale l'orecchio umano non potrebbe percepire e dunque trasformare in sensazioni e emozioni quello che ascolta, il feedback.}


\descrizione{A contemporary Dawn: a wor(l)d soundscape}{Al pari di altre categorie dell'arte, \textit{accusata} di essere astratta e lontana dalla realtà: la musica adopera l'illusione per affrontare l'illusione stessa. Ma questo non significa allontanarsi dal mondo, bensì amarlo disperatamente. La musica è un'occasione per adottare l'incertezza come principio di funzionamento, mettendo in discussione la cultura fino a portarla al suo estremo.
\textit{A contemporary Dawn} è un paesaggio sonoro che gioca con le dimensioni. Questa performance interagisce con i suoi ascoltatori attraverso l’uso di Twitter: ogni tweet che include l’hashtag \textit{\#dawnscape} può modificare lo spazio e i movimenti di un viaggio musicale che attraversa il pubblico, investigando un linguaggio musicale che evolve e involve dal secolo scorso.
Fluttuante nello spazio l'Atlante - così come lo abbiamo conosciuto finora - cerca nella profondità del tempo e nel dedalo di vecchie e nuove reti: una storia differente. Un’alba nuova.}

\end{multicols}

\clearpage


%----------------------------------------------------------------------------------------

% ---------------------------------------------------------------------------------------------
\section*{ }

\subsection*{\textsf{Autori ed Esecutori}\\}

{\fontsize{30}{30} \svolk{\emph{Biografie}}}

\bigskip

\begin{multicols}{2}

\biografia{Natasha Barrett}{Compositrice freelance che lavora con la musica, la ricerca e l'uso creativo del suono. Ha conseguito in Inghilterra master e dottorato in composizione elettroacustica, dopo di che, nel 1999, si trasferisce in Norvegia, dove ha vissuto. Si concentrò sulla musica acusmatica e sulla musica strumentale con live electronics. Dal 1999 il suo lavoro con il suono si è ampliato fino a comprendere sound-art, installazioni sonore-architettoniche, tecniche interattive, la collaborazione con i progettisti e scienziati sperimentali e performance live e l'improvvisazione. Esempi recenti di questo includono l'uso di dati scientifici e dei processi geologici in sound-art, composizione spaziale per altoparlante a matrice emisferica e uno speciale interesse nell’HOA, e il suo terzo progetto d’installazione con il gruppo Oceano Design Research Association.

Le sue opere sono eseguite e commissionate in tutto il mondo e ha ricevuto numerosi riconoscimenti , in particolare il Music Prize Consiglio nordico (Norden / Scandinavia, 2006), Giga - Hertz Award (Germania, 2008), Edvard Prize (2004, Norvegia), Noroit- Leonce Petitot (Arras, Francia, 2002 e 1998), Bourges International Electroacoustic Music Awards (Francia del 2001, 1998 e 1995), Musica Nova (2001), IV CIMESP 2001 Concours Scrime, (Francia 2000), International Electroacoustic Creation Competition of Ciberart (Italia 2000), Concours Luigi Russolo (Italia 1995 e 1998), Prix Ars Electronica (Linz , Austria 1998), 9th International Rostrum for electoacoustic music (2002). Le sue installazioni includono un lavoro importante per la Commissione dello Stato Norvegese per l'arte negli spazi pubblici}


\end{multicols}

\clearpage

\section*{ }

%\subsection*{\textsf{}\\}

\hyphenation{Michele Andreotti Guido Capotosto Federico Coderoni Gianmarco Costa
Marco De Martino Simone Giudice Matteo Ilardo Leonardo Mammozzetti
Danilo Marro Massimiliano Mascaro Alessandro Pacetta
Federico Paganelli Ivo Papadopoulos Ivano Pecorini Susanna Rimondotto Federico Ripanti}

\begin{center}

{\fontsize{30}{30} \svolk{\emph{Organizzazione}}}
\medskip

{\fontsize{12}{12} \textsf{Dipartimento di Musica Elettronica}}\\
\vspace{.5cm}

\textbf{\textit{Comitato artistico EMUfest}}\\
Nicola Bernardini, Michelangelo Lupone, Alfredo Santoloci, Franco Sbacco
\medskip

\textbf{\textit{Comitato organizzatore}}\\
Francesco Bianco, Elena D'Alò, Paolo Gatti, Marco Giordano, Virginia Guidi, Luana Lunetta, Massimo Massimi, 	Luigi Pizzaleo, Federico Scalas, Giuseppe Silvi, Anna Terzaroli, Francesco Ziello
\medskip

\textbf{\textit{Supervisione tecnica}}\\
Federico Scalas
\medskip

\textbf{\textit{Responsabili di palco}}\\
Luana Lunetta, Massimo Massimi
\medskip

\textbf{\textit{Regia del suono}}\\
Giuseppe Silvi
\medskip

\textbf{\textit{Regia Conferenze}}\\
Anna Terzaroli
\medskip

\textbf{\textit{Tecnico di registrazione}}\\
Federico Coderoni
\medskip

\textbf{\textit{Tecnici luci}}\\
Massimiliano Mascaro, Simone Giudice
\medskip

\textbf{\textit{Ufficio Stampa}}\\
Francesco Bianco, Paolo Gatti
\medskip

\textbf{\textit{Staff esteso}}\\
Michele Andreotti, Guido Capotosto, Federico Coderoni, Gianmarco Costa,
Marco~De~Martino, Simone Giudice, Matteo Ilardo, Leonardo Mammozzetti,
Danilo Marro, Massimiliano Mascaro, Alessandro Pacetta,
Federico Paganelli, Ivo Papadopoulos, Ivano~Pecorini, Susanna Rimondotto, Federico Ripanti
\end{center}


%\begin{figure}[!h]
%\centering
%\includegraphics[width=6.4cm]{loghi.jpg}
%%\caption{}
%%\label{fig2_42}
%\end{figure}

\end{document}
