% !TEX encoding = UTF-8 Unicode
% !TEX TS-program = XeLaTex


\documentclass[8pt, twoside, a5paper]{extreport}
\usepackage{lipsum} % ----------------------------------------------------------- cancellabile
\usepackage{blindtext}	% dummy text

\usepackage{latexsym}
\usepackage[polutonikogreek, italian, english]{babel}
\usepackage[T1]{fontenc}
%\usepackage[utf8]{inputenc}

\usepackage[margin=1cm]{geometry}

\usepackage{multicol}
\usepackage[hang, small,labelfont=bf,up,textfont=it,up]{caption} % Custom captions under/above floats in tables or figures
\usepackage{booktabs} % Horizontal rules in tables
\usepackage{float} % they need to be placed in specific locations with the [H] (e.g. \begin{table}[H])

\usepackage{graphicx}
\usepackage{amssymb}
\usepackage{epstopdf}

\usepackage{hyperref} % For hyperlinks in the PDF

\usepackage{titlesec} % Allows customization of titles
\renewcommand\thesection{\Roman{section}} % Roman numerals for the sections
\renewcommand\thesubsection{\Roman{subsection}} % Roman numerals for subsections
\titleformat{\section}[block]{\large\scshape\centering}{\thesection.}{1em}{} % Change the look of the section titles
\titleformat{\subsection}[block]{\large}{\thesubsection.}{1em}{} % Change the look of the section titles

% \usepackage{fancyhdr} % Headers and footers
% \pagestyle{fancy} % All pages have headers and footers
% \fancyhead{} % Blank out the default header
% \fancyfoot{} % Blank out the default footer
% \fancyhead{}%[C]{Sei volte EMUfest $\bullet$ Novembre 2013 }%$\bullet$ Vol. XXI, No. 1} % Custom header text
% \fancyfoot[RO,LE]{\thepage} % Custom footer text

\renewcommand{\thefootnote}{\textasteriskcentered}

\usepackage{mwe,tikz}
\usepackage{tcolorbox}

\usepackage{color}

\usepackage{fontspec,xltxtra,xunicode}
\defaultfontfeatures{Mapping=tex-text}
\setromanfont[Mapping=tex-text]{Source Sans Pro}
\setsansfont[Scale=MatchLowercase,Mapping=tex-text]{Source Sans Pro}
\setmonofont[]{Source Code Pro}

\newfontfamily{\svolk}{Volkhov}

\linespread{1.1}

\usepackage{lipsum}

\definecolor{supercolor}{RGB}{3,39,36}

%\topmargin -2.66in

\newtcbox{\mybox}{nobeforeafter,colframe=supercolor,colback=supercolor,boxrule=0.5pt,arc=8pt,
  boxsep=0pt,left=2pt,right=2pt,top=6pt,bottom=2pt,tcbox raise base}

\input{comandi2015.tex}

%----------------------------------------------------------------------------------------
%	TITLE SECTION
%----------------------------------------------------------------------------------------

\title{
	\svolk{CONSERVATORIO DI MUSICA S. CECILIA} \\
	%\vspace{-15mm}
	\fontsize{50}{50}
	\svolk{
		\emph{
			EMUFest 2015
			}
		}
	} % Article title

\author{
	\textsc{Conservatorio di Musica Santa Cecilia} \\
	5 -- 10 ottobre 2015 \\
	\textsc{Università di Roma Tor Vergata} \\
	13 e 14 ottobre 2015 \\
	Roma
}

\vfill

\date{}

%\dedica{.2\textwidth}{\small A Monica:\\
%per tutto ciò che mi hai insegnato\\
%e per tutto ciò che ancora avresti dovuto insegnarmi.}

%----------------------------------------------------------------------------------------

\begin{document}
\pagestyle{empty}
\maketitle 

\clearpage

testo michelangelo

\clearpage

%----------------------------------------------------------------------------------------

\section*{5 ottobre 2015 -- ore 17:00}

\subsection*{{\small Concerto Acusmatico in Cupola Ambisonic} \\
	\textsf{Aula Bianchini}}

{\fontsize{30}{30} \svolk{\emph{ATTO I}}}

\subsection*{\textsf{regia del suono di Francesco Ziello}}

\begin{quote}
{\svolk \small
In uno dei suoi più noti aforismi Victor Hugo scrisse che la musica è rumore che pensa.
Un pensiero destinato ad evolversi nel tempo, che passa per il movimento nello spazio e nelle tessiture sonore coerenti ma allo stesso tempo contraddittorie di cinque compositori provenienti da diverse parti del mondo,
riuniti in un’unica esperienza d’ascolto.}

\emph{F. Ziello}
\end{quote}

%\bigskip

\begin{multicols}{2}

% !TEX encoding = UTF-8 Unicode
% !TEX TS-program = XeLaTex
% !TEX root = EMU2015_booklet.tex

\acusmatici{Ursula Meyer-K\"onig}
{Allears}{2012-13} %8' 

\acusmatici{Benjamin O'Brien}
{Along the eaves}{2012-13} %8'20'' 

\acusmatici{Dennis Deovides A. Reyes III}
{Bolgia}{2014} %7'31''

\acusmatici{Dimitrios Savva}
{Balloon Theories}{2012-13} %14'30'' 

\acusmatici{Jones Margarucci}
{Inhabitated PlacesPart II}{2012-13} %5'52"


%\descrizione{Allears}{The inspiration for this work originally came from a series of intensive discussions with people who are deaf or have hearing impairments. We talked about the pros and cons of technical apparatuses such as hearing aids or cochlea implants, the different ethical and emotional responses people have to them, and the identity issues they raise. Wearing hearing aids also changes how sounds are perceived, sometimes causing interference, distortions, diminished spatial perception and noise overflow.}

\descrizione{Allears}{Originariamente l'ispirazione per questo lavoro  proveniva da una serie di intense discussioni con persone non udenti o che hanno problemi d'udito. Abbiamo parlato dei pro e dei contro di apparati tecnici, come apparecchi acustici o impianti cocleari, le diverse risposte etiche ed emotive che le persone sentono, e i problemi d'identità che sollevano. 
Indossare apparecchi acustici cambia anche come i suoni vengono percepiti, a volte causando interferenze, distorsioni,percezione spaziale ridotta e troppo pieno di rumore.}


%\descrizione{Along the Eaves}{takes its name from the following line in Franz Kafka’s “A Crossbreed” “On the moonlight nights its favorite promenade is along the eaves.” To compose the work, I developed custom software and used these programs in different ways to process and sequence my source materials, which, in this case, included audio recordings of water, babies, and string instruments. My interest is to create sonic coincidences that suggest relationships between sounds and the illusions they foster.}

\descrizione{Along the Eaves}{ prende il nome dalla riga che di Franz Kafka "Incrocio". "Sulle notti di luna la sua passeggiata preferita è lungo la grondaia" Per comporre l'opera, ho sviluppato software personalizzato e utilizzato questi programmi in modi diversi per elaborare e sequenziare i miei materiali di base, che, in questo caso, include registrazioni audio d’acqua, bambini, e di strumenti a corda. Il mio interesse è quello di creare coincidenze sonore che suggeriscono i rapporti tra i suoni e le illusioni che promuovono.}


%\descrizione{Bolgia}{Bolgia is an Italian word that means pocket or trench.  This term has been used by Dante Alighieri in his notable literary work Inferno.  Bolgia is a stereo fixed media electroacoustic composition, which depicts Alighieri’s journey to the eighth circle of hell, and his experiences to its horrific environment.  The musical gestures and sonic events of the piece evoke the different sounds and emotions of hell.}

\descrizione{Bolgia}{è una parola italiana che significa "fossa" e "luogo chiassoso in cui regna la confusione". Questo termine è stato usato da Dante Alighieri nel suo noto lavoro letterario "Inferno". Bolgia è un brano stereofonico fisso per composizioni elettroacustiche, che illustra il viaggio di Alighieri nell'ottavo girone dell'inferno, e la sua esperienza in questo posto terribile. I gesti musicali e l'evento sonoro del pezzo evocano i diversi suoni e le diverse emozioni dell'inferno.}


%\descrizione{Balloon Theories}{«I was always enjoying squeezing balloons, pressing them with my fingers until they pop… It has not been up until now that I realized why…»}

\descrizione{Balloon Theories}{«Ho sempre trovato divertente strizzare palloncini, premerli con le dita fino allo scoppio... Non mi è mai interessato fino a quando non ho capito perché...»}


%\descrizione{Inhabitated Places Part II}{Inhabitated Places part II is based on the concept of algorithmic composition. Although the general shape of this piece has been determined in a conventional way, every sound that one can hear are selected in real time by different algorithms written in SuperCollider. These algorithms choose randomly audio files from different folders and play them at different speeds and in different moments. It is as if we had placed several different objects in several boxes (that represent our shape), but every time we open one of these boxes the objects placed inside are positioned differently from how we had left them previously. The piece was also rendered in B-format Ambisonic.}

\descrizione{Inhabitated Places part II}{ è una composizione elettroacustica basata sul concetto di musica algoritmica. Sebbene la forma generale del brano sia stata determinata apriori e in modo convenzionale, tutti i suoni che ascoltiamo vengono scelti in tempo reale da vari algoritmi scritti in SuperCollider. Questi algoritmi selezionano in modo pseudocasuale dei samples da diverse cartelle e li riproducono a velocità diverse e in diversi momenti.  
È come se avessimo sistemato in una scatola (che in questo caso rappresenta la struttura dell’opera) degli oggetti in un dato ordine, ma ogni qual volta apriamo la scatola li troviamo disposti in modo differente da come li avevamo lasciati.}


\end{multicols}

\clearpage
%----------------------------------------------------------------------------------------

\section*{5 ottobre 2015 -- ore 20:30}

\subsection*{{\small Concerto Acusmatico \& Live Electronics\footnote{ Il concerto verrà trasmesso in diretta streaming da Radio Cemat}} \\
	\textsf{Sala Accademica}}

{\fontsize{30}{30} \svolk{\emph{ATTO II}}}

\subsection*{\textsf{Regia del suono di Federico Paganelli}}

\begin{quote}
{\svolk \small
Il flauto è forse uno degli strumenti con il più ampio repertorio contemporaneo; la serata inaugurale dell’Emufest sarà caratterizzata proprio dall’esecuzione di brani per flauto e live electronics.    
Tra le composizioni in programma le prime esecuzioni di: \emph{[re-sp\`{\i}-ro]}, opera di Maria Cristina De Amicis, e  \emph{7 Isole}, nuovo lavoro di Giorgio Nottoli, ideatore di questo festival.}

\emph{F. Paganelli}
\end{quote}

%\bigskip

\begin{multicols}{2}

% !TEX encoding = UTF-8 Unicode
% !TEX TS-program = XeLaTex
% !TEX root = EMU2015_booklet.tex

\livel{Simone Cardini}
{Potlach}{2014 - vincitore Premio Bucchi} %9'30''
{per flauto basso ed elettronica su supporto}
{flauto basso}{Gianni Trovalusci}

\livel{Dominique Schafer}
{Cendre}{2008} %10'00''
{per flauto basso e live electronics}
{flauto basso}{Gianni Trovalusci}

\livel{Maria Cristina De Amicis}
{[re-spì-ro]}{2015 - prima esecuzione assoluta} %8'00''
{per flauti e supporto digitale}
{flauti}{Gianni Trovalusci}

\acusmatico{Christian Eloy}
{La cicatrice d'Ulysse}{new version 2015} %13'00''
{acusmatico}

\livel{Giorgio Nottoli}
{7 Isole}{2015 - prima esecuzione assoluta}%15'
{per flauto, percussioni e live electronics}
{flauto}{Gianni Trovalusci}
percussioni -- \textsc{Gianluca Ruggeri}
\\



%\descrizione{Potlach}{was a ritual ceremony performed among certain Native American tribes, during which the members of the community exchanged and squandered gifts, contributing to the cohesiveness of the social concord, whithin a totally inverted logic with respect to our free economy, following a reciprocity principle. The formal juxtaposition of the piece, disperses at once those elements that will constitute later on the concealed foundation that penetratrates the work.}

\descrizione{Potlach}{Il Potlach era una cerimonia rituale d’alcuni popoli di Nativi Americani, durante la quale i membri della tribù scambiavano e dilapidavano doni, in una logica completamente invertita rispetto alla nostra economia. La giustapposizione del brano \textit{dilapida} da subito quegli elementi che costituiranno il substrato che pervade l'opera. È la simbolizzazione dell'immagine che ne permette un'esegesi palesata; l'interprete realizza sé attraverso la sublimazione di gesti e momenti performativi, rappresentando la fenomenologia degli stessi. Da qui la necessità di superare il momento poietico ed estesico \textit{donando} la performance per un recupero del sistema di relazioni come tale.}

%\descrizione{Cendre}{draws on the idea from the impermanence of materiality. The title of the piece gives further meaning referring to a fragile and delicate state, but also the potential of ashes as a fertilizer for something new. The bass flute being suspended within the electronics extends space and timbre, at times detaching itself within its own identity, but repeatedly gets reabsorbed into space. The piece was written for and premiered by Mario Caroli during his Fromm Residency at Harvard University.}

\descrizione{Cendre}{prende ispirazione dall’idea della precarietà di ciò che è materiale. Il titolo del brano acquista ulteriore significato riferendosi ad uno stato di fragilità e delicatezza, ma anche alle potenzialità fertilizzanti della cenere per qualcosa di nuovo. Grazie alla sospensione del flauto basso attraverso l’elettronica, il suo spazio ed il suo timbro vengono estesi, a tratti allontanandosi dalla propria identità per poi essere riassorbito all’interno dello spazio. Il brano è stato composto ed eseguito in prima assoluta da Mario Caroli durante la Fromm Residency presso la Harvard University.}

%\descrizione{[re-spì-ro]}{Re-spi-ro (Breath) in this work, is the alternation of air movements which could produce real rhythmic and tonal structures. The sound of each breath adds and interacts itself with the previous one, drawing a music form which contracts and expands itself as well. Through the rhythmic articulation, inhalation and exhalation, the breath transforms and builds itself, removes and contradicts what has been just expressed. The digital support creates a constructive dialectic, supporting gently the concrete material. Breath transformation is described by the conduct of three different changing elements: localization, movement (direction and speed) and space size. In this way , the auditive perception of the space is transposed to the visual perception level. In this work I tried to realize the interaction between these changing elements on the acoustic level and associating with tones and pitches the breath expressive gestures of the performer.}

\descrizione{[re-spì-ro]}{Il re-spi-ro, in questa opera, è inteso come l’alternanza dei movimenti dell’aria in grado di generare vere e proprie strutture ritmiche e timbriche. Il suono di ogni respiro si aggiunge e interagisce con i precedenti, disegnando una forma musicale che si contrae e si espande.  Attraverso l’articolazione ritmica, l’inspirazione e l’espirazione, il respiro si trasforma e costruisce, dissolve o contraddice ciò che è stato appena espresso. Il supporto digitale crea una dialettica costruttiva sostenendo delicatamente il materiale concreto. La trasformazione del respiro è descritta attraverso il comportamento di tre variabili: localizzazione, movimento (direzione, velocità) e dimensione dell’ambiente. In questo modo la percezione uditiva dello spazio viene trasposta nel dominio della percezione visiva. In questo brano ho cercato di realizzare l’interazione tra queste variabili nel dominio acustico associando con timbri e altezze i gesti espressivi del respiro dell’esecutore.}

%\descrizione{La cicatrice d'Ulysse}{This title, borrowed from Erich Auerbach, German writer and critic deceased in 1957, immediately sets the scene on a plane where realism is depicted in the aesthetics of Western music. Electroacoustic music has the ability, using “concrete” sounds (with all the ambiguity implied by the word), of giving us an immediate sensation of reality, which we can all situate in relation to our references and private objects of reference; these are the profound notions of sublimitas and humilitas, which merge and unite in musical expression. Every listener will be able to decipher his or her own images of a collective mental universe from the essence of this kind of artistic creation, just like the scar by which Ulysses was recognised by his former servant woman.}

\descrizione{La cicatrice d'Ulysse}{Il titolo, preso in prestito da Erich Auerbach, scrittore e critico tedesco morto nel 1957, ambienta immediatamente la scena all’interno di un aereo, tratteggiandola con un realismo tipico dell’estetica musicale occidentale. Attraverso l’uso di suoni concreti (con tutte le ambiguità insite nella parola stessa), la musica elettroacustica possiede l’abilità di restituirci una percezione immediata della realtà, che è possibile collocare in relazione ai propri riferimenti e agli oggetti segreti cui intende riferirsi; questi sono i significati profondi di sublimitas e humilitas, che si fondono e riuniscono nell’espressione musicale. Dall’essenza di questo tipo di creazione artistica ogni ascoltatore è in grado di decodificare la propria immagine di un universo mentale collettivo, proprio come la cicatrice che permise alla serva di riconoscere Ulisse.}

%\descrizione{7 Isole}{I think the idea of the island is a fascinating metaphor, because I realize to think about many important things  as they are formed, in fact, by islands, which each contain a different world and a different feel, but together form a unitary context. In "7 Isole", each "island" is characterized by a particular combination of movement, pitches and colors of the sound. It consists of seven small pieces separated from one another, that can be performed in any order, however, strongly linked by way of forming the sound material and by a unitary constructive thinking. The instruments are used both with extended techniques that traditional, allowing different nuances both for the continuous sound as well as impulsive, both for what determined pitch and for what similar to the noise. The electronics supports and extends the sound of the instruments and, where necessary, becomes the instrument itself, completing the construction of the sound field. The dynamic distribution  in the listening space is obtained according to the principles of the method Ambisonic. Some of the processing and sound synthesis used have been developed by the author.}

\descrizione{7 Isole}{Quella dell'isola è per me una metafora affascinante, in quanto mi accorgo di pensare a molte cose importanti come fossero costituite, appunto, da isole, che contengono ciascuna un mondo diverso e quindi un diverso sentire, ma, insieme, costituiscono un contesto unitario. In 7Isole, ciascuna "isola" è caratterizzata da una particolare combinazione di movimento, altezze e colori del suono. Si tratta di sette piccoli pezzi fra loro separati, che si possono eseguire in un qualsiasi ordine, tuttavia fortemente legati dal modo di formare il materiale sonoro e da un pensiero costruttivo unitario. Gli strumenti sono utilizzati sia con tecniche estese che tradizionali, consentendo diverse sfumature sia per il suono continuo che per quello impulsivo, sia per quello ad altezza determinata e per quello simile al rumore. L'elettronica sostiene ed  estende il suono degli strumenti e, dove necessario, diviene strumento essa stessa, completando la costruzione del campo sonoro. La distribuzione dinamica nello spazio d'ascolto è ottenuta secondo i principi  del metodo Ambisonic. Alcune delle elaborazioni e sintesi del suono utilizzate sono state sviluppate dall'autore.}











\end{multicols}

\clearpage

%----------------------------------------------------------------------------------------

\section*{6 ottobre 2015 -- ore 17:00}

\subsection*{{\small Concerto Acusmatico in High Order Ambisonic Domain} \\
	\textsf{Aula Bianchini}}

{\fontsize{30}{30} \svolk{\emph{ATTO III}}}

\subsection*{\textsf{regia del suono di Federico Paganelli}}

\bigskip

\begin{multicols}{2}

% !TEX encoding = UTF-8 Unicode
% !TEX TS-program = XeLaTex
% !TEX root = EMU2015_booklet.tex

\acusmatici{Jan Jacob Hofmann}
{Horizontal and Vertical Lines}{2005}{7'53''}

\acusmatici{Clemens Von Reusner}
{Topos Concrete}{2014}{9'18''}

\acusmatici{Damián Anache}
{Capturas del Único Camino}{2014-15}{14'29''}

\acusmatici{Jan Jacob Hofmann}
{Tensile Elements}{2000}{9'30''}

% \descrizione{Horizontal and Vertical Lines}{Two kinds of elements mark the context of the piece: Sounds of extremely long duration in the distance, shimmering and sharp like steel, stretched out in the timeline. Generated by patterns of chaotic oscillation and being non-linear, these horizontal elements refer to infinity. Their vertical organisation, the proportion of pitch among each other is organised by the harmonic ratio though. Much closer, several impulses: Sounds of infinite shortness, containing the whole spectra of frequencies in a mathematical sense, but having no extension in time. Being strictly vertical, they contradict with the long stretched horizontal elements. These two sonic elements, considered as horizontal and vertical, set the background structure of which matter is generated. The sound for this composition is derived from a non-linear algorithm for sound generation by physical modelling, as from granular synthesis. All the sounds have been created using Csound. The work was commissioned by the CRM for Musica Scienza 2005 where it got executed in June 2005.}

\descrizione{Horizontal and Vertical Lines}{Due elementi determinano la base del pezzo: suoni di eccezionale lunghezza dalla distanza, scintillante e tagliente come l'acciaio. Si estendevano lungo l'asse del tempo. Generati dal modello di oscillazione caotica e non-linearità, questi elementi orizzontali mostrano all'infinito. La loro organizzazione verticale, cioè il loro rapporto fra loro è pero disposte secondo le leggi dell'armonia. Molto più vicino: differenti suoni che si assomigliano le impulsi: suoni di durata infinitamente breve, tuttavia, contengono l'intero spettro di frequenze in senso matematico, ma non hanno estensione nel tempo. Strettamente allineati verticalmente, in contrasto con la lunga tesa elementi orizzontali. Entrambi questi elementi sonori, la forma orizzontale e verticale, creano la struttura basale di quella si sviluppano le forme. I suoni di questa composizione sono generati da un algoritmo non-lineare, così come la sintesi granulare. I suoni sono stati generati con il programma Csound. Il brano "Linee Orizzontale e Verticale" è stata composto come contributo a "Musica e Scienza" nel 2005, per conto di CRM, Roma.}

% \descrizione{Topos Concrete}{Concrete is a building material, a kind of unshaped dry powder made of sand, granulated stones and cement, dusty and chaotic. Mixed with water it becomes flexible and fluid and goes into a metamorphosis to become dry again, static and resistable and of any wanted shape. Aspects of working with native granularity, fluidness as well as stiffness and different kind of acoustic spaces were leading ideas of the composition.}

\descrizione{Topos Concrete}{Il calcestruzzo ("Concrete") è un materiale da costruzione, una sorta di polvere secca informe costituita di sabbia, pietre e cemento granulati, polveroso e caotico. Impastato con l'acqua diventa flessibile e fluido, quindi torna ad essere di nuovo asciutto, statico e resistente e di qualsivoglia forma. La composizione del brano è guidata da alcuni aspetti quali il lavorare con la granularità e la fluidità così come con la rigidità e con diversi tipi di spazi acustici.}

% \descrizione{Capturas del Único Camino}{"First Landscape" is a movement of “Capturas del Único Camino”, a piece which involves generative art ideas for offering an attractive object of passive contemplation. It is Ambisonics encoded and made with acoustic instruments samples recorded and performed by the composer. Then the samples are handled by a Pure Data patch according to the random events score of the piece. So that, the algorithm works as an electronic performer of the piece. info: http://conceptocero.com/capturasdelunicocamino}

\descrizione{Capturas del Único Camino}{"First Landscape" è un movimento di "Capturas del Único Camino", un brano che impiega le idee dell'arte generativa, per offrire un interessante oggetto di contemplazione passiva. Il brano è codificato Ambisonics e realizzato con campioni di strumenti acustici registrati e suonati dal compositore. I campioni sono gestiti da una patch di Pure Data, seguendo una partitura di eventi randomici. In questo modo l'algoritmo funziona come se fosse l'esecutore elettronico del pezzo. info: http://conceptocero.com/capturasdelunicocamino}

% \descrizione{Tensile Elements}{Subject of \textit{Tensile Elements} are special kinds of elements, characterised by their quality of sound, their movement and interaction. Elements pass through the space, defining it, having almost materiality but still being bodiless. Concentrations and disintegrations happen, the elements seem to have almost their own way of behaviour.}

\descrizione{Tensile Elements}{Tema del soggetto sono elementi che si caratterizzano per il loro determinato suono teso, così come per i loro movimenti e le loro interazioni. Tali elementi si muovono attraverso lo spazio, lo definiscono ed hanno quasi una materialità, nonostante ciò rimangono senza corpo. Addensamenti e dissolvenze si succedono, gli elementi sembrano avere una vita propria.}


%Sonic Architecture 

%Jan Jacob Hofmann's work is spatial electronic music, or, as he calls it,  “sonic architecture” – architecture made by sounds. Using the Ambisonic-Method makes it possible to assign a precise Position  in 3D space to every sound.  The sonic material of the composition generates a sonic space surrounding the listener. Sounding virtual materials do generate complex architectures made of sound, spaces in transformation do emerge.
%
%This is made possible by using a combination of the Ambisonic Method in combination with Csound, a computer-language for sound synthesis and digital signal processing. Jan Jacob Hofmann conceived and created a set of instruments that allow to assign a position in 3d space to every sound. Also the sound is placed in a sonic environment allowing distance perception by the creation of spatial reverb and early reflections, also using the ambisonic method.
%
%Embedded in an environment for composition, this code makes it possible to compose works of spatial music. This code is available open source on his site www.sonicarchitecture.de and is already used by other composers.
%
%Latest development is the extension of the spatial assignment of sounds to granular clouds so that spatial granular synthesis became possible allowing swarms and clouds of sound in space.










\end{multicols}

\clearpage
%----------------------------------------------------------------------------------------

\section*{6 ottobre 2015 -- ore 20:30}

\subsection*{{\small Concerto Acusmatico \& Live Electronics\footnote{ Il concerto verrà trasmesso in diretta streaming da Radio Cemat}} \\
	\textsf{Sala Accademica}}

{\fontsize{30}{30} \svolk{\emph{ATTO IV}}}

\subsection*{\textsf{Regia del suono di Luana Lunetta}}

Il movimento nello spazio fisico per l'attuazione di processi sonori.
Il movimento del suono nello spazio per mettere in risalto differenti tematiche compositive.
Il testo poetico che si tramuta in suono.

Il concerto Atto IV comprende due brani nati dalla collaborazione tra allievi delle classi di Composizione e Musica Elettronica, un progetto musicale live-electronics che utilizza sensori di movimento per il trattramento del suono, un brano per flauto ed elettronica su supporto e un brano acusmatico che usufruisce della diffusione sonora spazializzata.     
*Luana Lunetta*

%\bigskip

\begin{multicols}{2}

% !TEX encoding = UTF-8 Unicode
% !TEX TS-program = XeLaTex
% !TEX root = EMU2015_booklet.tex

\livel{Alessia Forganni, Daniele Vantaggio}
{KOMA - Kontrolled Organic Movement Act}{10'}
{voce, pianoforte,  kinekt, e live electronics}
{2015}

\brano{Massimo Massimi}
{Kronos in Fabula I}{10'}
{acusmatico}
{2015}\\

\livel{Claudia Jane Scroccaro, Federico Ripanti}
{On-y r\^eve}{7'40''}
{per pianoforte e nastro}
{2015}

\livel{Carlos D. Perales}
{17 haiku}{10'}
{flauto e live electronics}
{2012}

\livel{Massimiliano Mascaro, Giuseppe Zampetti}
{L'abile medico}{10'}
{per soprano, mezzosoprano e live electronics}
{2015}


\esecutore{voce, pianoforte}{Alessia Forganni}
\esecutore{pianoforte}{Daniele Buccio}
\esecutore{flauto}{Alessandro Pace}
\esecutore{soprano}{Elisabetta Braga}
\esecutore{mezzosoprano}{Virginia Guidi}

\vspace{-5mm}

% \descrizione{KOMA - Kontrolled Organic Movement Act}{Our entire life is made by acts. Life is articulated day by day through repetition. Millions of automatisms assure us a cognitive and energetic saving. So life seems to be an automated system that is difficult to avoid or control, without a real consciousness of our actions. Maybe we should see our actions as a ritual gesture, instead of a simple repetition of the same act. Sometimes we should give time, meaning and attention to our acts to really control them. The performance includes tape, live piano, voices, live electronics and sound controlling by body movement through Kinect and Leap Motion system.}%abbiamo foto

\descrizione{KOMA - Kontrolled Organic Movement Act}{Tutta la nostra vita è fatta di gesti: grazie alla ripetizione di milioni di essi la vita è scandita ogni giorno. Forme di automatismi permettono un risparmio cognitivo ed energetico e alla maggior parte non si fa caso. La vita appare così un sistema automatizzato dal quale è difficile sottrarsi e avere reale controllo, uno stato di assenza di reale coscienza e consapevolezza. Per rendere i nostri gesti un atto consapevole bisogna vederli nella loro “ritualità” piuttosto che “ripetitività”, conferendovi significato e tempo,  focalizzandovi al massimo la nostra attenzione. La performance include nastro, piano e voci dal vivo, live electronics e controllo del suono tramite movimento con i sistemi di sensoristica Kinect e Leap Motion.}%abbiamo foto

% \descrizione{Kronos in Fabula I}{The acousmatic piece “Kronos in Fabula 1” is part of a series of compositions dedicated to the speculation about time. The material’s acoustic density represents the time trend; the possibilities of the listening points’ collocation on the time axis are connected to a feasible fruition “directionality” linked to the sequence and reiteration of sounds events.}

\descrizione{Kronos in Fabula I}{“Kronos in Fabula I” è l’acusmatico di un ciclo di composizioni dedicate alla speculazione sul tempo. La densità acustica del materiale rappresenta l’andamento del movimento temporale, le possibilità di collocazione di punti di ascolto sull’asse del tempo vengono associate a una possibile “direzionalità” dell’ascolto legata al concatenarsi e al reiterarsi degli eventi sonori.}

% \descrizione{On-y r\^eve}{The work is a “nocturnal representation” for piano and tape, suffused with suspended atmospheres and textures typical of oneiric visions. The piano part is mostly rhapsodic, while the acousmatic elements enhance and filter the resonances produced through the tonal pedal, conveying granulated processes of prepared piano. A contrast between darkness and light alludes to a chiaroscuro counterpoint, where each tile is devised manipulating and recomposing elements derived from Alexander Scriabin’s \textit{Poème-Nocturne} op. 61, in a tribute to commemorate the centenary celebration of his passing.}

\descrizione{On-y r\^eve}{The work is a “nocturnal representation” for piano and tape, suffused with suspended atmospheres and textures typical of oneiric visions. The piano part is mostly rhapsodic, while the acousmatic elements enhance and filter the resonances produced through the tonal pedal, conveying granulated processes of prepared piano. A contrast between darkness and light alludes to a chiaroscuro counterpoint, where each tile is devised manipulating and recomposing elements derived from Alexander Scriabin’s \textit{Poème-Nocturne} op. 61, in a tribute to commemorate the centenary celebration of his passing.}

% \descrizione{17 haiku}{Heirs of haikai, comic-themed and outgoing figuralisms in Japanese poetry, haiku was taking its own identity in the complex art of expression of consciousness. In them, senses are suggested, not the meanings. Haiku does not mean, because it says nothing. Indeed, haiku does not have an informative function. But, perhaps, to the expressive function, a suggestive function should be added.} %aggiungi pdf

\descrizione{17 Haiku} {Eredi degli haikai, fumetti a tema e figuralismi uscenti della poesia giapponese, gli haiku si sono guadagnati una propria identità nella complessa arte di espressione della coscienza. In essi si suggeriscono i sensi, non i significati. Haiku non significa, perché non dice nulla. Infatti, gli haiku non hanno una funzione informativa. Ma, forse, alla funzione espressiva bisogna aggiungere una funzione suggestiva.} %aggiungi pdf

% \descrizione{L'abile medico}{The piece is a prayer. Trying to go further, beyond the suffering it is a difficult task; and sometimes it causes reconsideration, but eventually the courage leads to victory. Il Sutra del Loto is one of the most important texts of the literature of Mahayana Buddhism. In the first section the breath, the primary form of meditation predisposes to concentration becoming  a prayer, the supplication of  the Mystic Law. Later, through meditation we realize the complexity of existence: an illusory chaos. Finally the resolute soul emerges, confirming definitively the depth of his Essence. To increase the listening ability and participation we recommend to close the eyes.}%aggiungere pdf

\descrizione{L'abile medico}{Il brano è una preghiera. Cercare di andare oltre, al di là della sofferenza è un compito difficile, e talvolta porta dei ripensamenti, ma alla fine il coraggio conduce alla vittoria. Il Sutra del Loto è uno dei testi più importanti della letteratura del Buddismo \textit{Mah y na}. Nella prima sezione il respiro, forma primaria di raccoglimento, predispone alla concentrazione trasformandosi in preghiera, invocazione della Legge Mistica. In seguito, attraverso la meditazione si realizza la complessità dell'esistenza: è il caos apparente. Finalmente l'anima risoluta riemerge, affermando definitivamente la profondità della sua Essenza. Per aumentare la capacità di ascolto e di partecipazione si consiglia di seguire ad occhi chiusi.}%aggiungere pdf




\end{multicols}

\clearpage

%----------------------------------------------------------------------------------------

\section*{7 ottobre 2015 -- ore 12:30}

\subsection*{{\small Concerto con Spazializzazione Interattiva in HOA} \\
	\textsf{Aula Bianchini}}

{\fontsize{30}{30} \svolk{\emph{ATTO V}}}

\subsection*{\textsf{regia del suono di Balandino Di Donato}}

\bigskip

VoicErutseG è una performance interattiva, nella quale la voce del soprano Vittoriana De Amicis viene spazializzata utilizzando il sistema gSPAT. La performance prevede l’esecuzione della Sequenza III di Luciano Berio e poi di Stryspody di Cathy Berberian.

VoicErutseG is an interactive performance, which shows the early results of the research conducted by Balandino Di Donato at the Integra Lab.
In VoicErutseG the voice of the Soprano Vittoriana De Amicis is spatialised using the gSPAT, while she will be performing Sequenza III by Luciano Berio and Stryspody by Cathy Berberian.

\begin{multicols}{2}

% !TEX encoding = UTF-8 Unicode
% !TEX TS-program = XeLaTex
% !TEX root = EMU2015_booklet.tex

\livel{Luciano Berio}
{Sequenza III}{1965} %9'00''
{per voce sola spazializzata}
{voce}{Vittoriana De Amicis}

\livel{Cathy Berberian}
{Stripsody}{1966} %11'30''
{per voce sola spazializzata}
{voce}{Vittoriana De Amicis}

\descrizione{Sequenza III}{}

\descrizione{Stripsody}{}



\end{multicols}

\clearpage
%----------------------------------------------------------------------------------------

\section*{7 ottobre 2015 -- ore 20:30}

\subsection*{{\small Concerto Acusmatico \& Live Electronics\footnote{ Il concerto verrà trasmesso in diretta streaming da Radio Cemat}} \\
	\textsf{Sala Accademica}}

{\fontsize{30}{30} \svolk{\emph{ATTO VI}}}

\subsection*{\textsf{Regia del suono di Marco De Martino}}

Un concerto che narra le modulazioni dell’aria attraverso forme diverse d’insufflazione. I toni, i ritmi e i timbri che si diffondono dalle canne d’organo e del sax, incontrano, dialogano e interagiscono con l’elettronica e la multi percussione fino a diventare un’unica massa cangiante, intima o dirompente, animata o pacata, sempre espressiva.

%A concert that narrates the modulations of the air through different forms of  insufflation. Tones, rhythms and timbres that spread from the organ pipes and the sax, meet, talk and interact with electronics and multi percussions to become a changing, disruptive or intimate, lively or calm, always expressive unique mass.

%\bigskip

\begin{multicols}{2}

% !TEX encoding = UTF-8 Unicode
% !TEX TS-program = XeLaTex
% !TEX root = EMU2015_booklet.tex

\livel{Elisabetta Capurso}
{Sezioni per organo spazializzato}{9'}
{per organo e live electronics}
{2015}

\livel{Cristiana Colaneri, Anna Terzaroli}
{Fantasia per due}{8'} 
{per sassofoni, percussioni ed elettronica}
{2015}

\brano{Demian Rudel Rey}
{Cenizas del Tiempo}{7'04''}
{acusmatico}{2015}\\

\livel{Gy\"orgy Ligeti}
{Volumina}{9'30''}
{per organo}
{1961-62, rivista 1966}

\livel{Alessandro Ratoci}
{Studio Mannaro}{11'30''}
{per sassofono baritono e live electronics}
{2013}

\esecutore{organo}{Antonella Barbarossa}
\esecutore{sassofoni}{Enzo Filippetti}
\esecutore{percussioni}{Valerio Cosmai}

% \descrizione{Sezioni per organo spazializzato}{Had many different changes in its general structure, the most important certainly the recent shift of the registers of the instrument. In the performance of the EMUFest 2015 the score has a further change for the presence of some electronic processing. For reasons of expression have been used some spatial sound systems, that fit perfectly to the sound of the acoustic instrument. Some microphones, amplifiers, some spatial sound's elaboration are amply present in the electronic score of the composition Sezioni.}

\descrizione{Sezioni per organo spazializzato}{Sezioni è opera che appartiene al periodo della prima attività compositiva. Dopo alcuni anni ha avuto alcuni cambiamenti nella struttura generale, un radicale cambiamento nella registrazione organistica.Il principio del mutamento di vista, simile a prisma ruotante è così la legenda della scrittura compositiva personale. La struttura generale ha tre sezioni, all'interno delle quali tre micro-sezioni hanno vita di improvvisazione. Nell'esecuzione di EMUFest2015 Sezioni ha un'ulteriore mutamento per la presenza di interventi elettronici. Per una  necessità di esigenze espressive sono stati  utilizzati alcuni sistemi di spazializzazione del suono  che si integrano perfettamente  con il suono originario dello strumento acustico. Microfonaggio, amplificazione, spazializzazione del suono sono gli interventi elettroacustici presenti nel processo di elaborazione sonora di Sezioni, suggeriti fondalmentalmente dalla raffinata registrazione organistica.}

% \descrizione{Fantasia per due}{“Fantasia per due” is among the EMUfest 2015 commissions destined to a couple of students of the Santa Cecilia Conservatory, one from the School of Composition, the other from the School of Electronic Music. The electronics uses the sound material produced by acoustic instruments, processing it and storing it on tape and it is built keeping in mind an equal and complementary relationship with the acoustic component. The latter is built starting from the indeterminate vibrations of a gong and from a single sound of the sax which, through gradual micro-intervals, conquers a wider frequency range interacting with percussions up to its full expansion. The conclusion is related to incipit.}

\descrizione{Fantasia per due}{Il brano è una commissione EMUfest, destinata ad una coppia di autori entrambi studenti del Conservatorio Santa Cecilia, uno proveniente dalla Scuola di Composizione, l'altro dalla Scuola di Musica Elettronica. L'elettronica utilizza il materiale sonoro prodotto dagli strumenti acustici, elaborandolo e fissandolo su tape ed è costruita nell'ottica di un rapporto paritetico e complementare con la composizione acustica. Questa è costruita partendo dalle vibrazioni indeterminate del gong e da un unico suono del sax, che gradualmente, anche con microintervalli, conquista un campo frequenziale più vasto, nell' interazione con le percussioni, fino alla libera espansione; la conclusione si ricollega all' incipit.}

% \descrizione{Cenizas del Tiempo}{Cenizas del Tiempo is an electroacoustic quadraphonic work, inspired by the idea that time ceases his ashes in our lives, gradually our being is consumed and the same thing happens with the materials and the sound objects. Also, expresses the experience of time in a city where everything occurs very quickly. In the work referential sounds of urban environments are perceived, moreover, samples from different ashtrays developed as more abstract sounds through processed and over-processed.}

\descrizione{Cenizas del Tiempo}{è un'opera quadrifonica elettroacustica, ispirata al tempo che porta le sue ceneri nella nostra vita, a poco a poco il nostro essere è consumato e la stessa cosa accade con i materiali e gli oggetti sonori. Inoltre, esprime l'esperienza del tempo in una città dove avviene tutto molto velocemente. In questo lavoro sono percepibili suoni riferiti ad ambienti urbani, e campioni di diversi posacenere che diventano suoni sempre più astratti attraverso continui processi.}

% \descrizione{Volumina}{eng}

\descrizione{Volumina}{Usa questo QR code dal tuo dispositivo mobile per visualizzare o scaricare la partitura completa e le note d’esecuzione}

\center{\begin{pspicture}(1in,1in)
	\psbarcode{https://goo.gl/JCU3Sc}{eclevel=L width=1 height=1}{qrcode}
\end{pspicture}}\footnote{https://goo.gl/JCU3Sc}

% \descrizione{Studio Mannaro}{ is a composition for baritone saxophone and live electronics. The saxophone is equipped with a microphone used both for amplification and sound pickup for processing by the computer. The amplified sound together with the processed sounds is then projected in the concert hall using a sound spatialization system based on virtual sources.}

\descrizione{Studio Mannaro}{ per saxofono baritono ed elettronica è stato presentato in prima esecuzione al festival Archipel di Ginevra nel 2012. L’idea poetica del brano è la ricerca dell’unità fra molteplici possibili dissociazioni: il registro estremo grave e estremo acuto del sassofono, le componenti fondamentali e armoniche dello spettro sonoro, la presenza tangibile dello strumento solista e la molteplicità virtuale dei campionamenti e delle sintesi elettroniche. Questo percorso di accettazione dell’identità complessa di un organismo “diverso” e peculiare si realizza attraverso il passaggio fra vari stadi sonori e conseguenti paesaggi immaginari ai quali non sono estranee certe categorie care alla musica di matrice popolare: il drone e il groove.}




\end{multicols}

\clearpage

%----------------------------------------------------------------------------------------

\section*{8 ottobre 2015 -- ore 17:00}

\subsection*{{\small Concerto Audiovisuale} \\
	\textsf{Sala Medaglioni}}

{\fontsize{30}{30} \svolk{\emph{ATTO VII}}}

\subsection*{\textsf{regia del suono di Leonardo Mammozzetti}}

Un attenta riflessione sui brani presentati in questo 7° concerto lascia intendere che non tutto è dato per scontato e che l'arte Audio-Visiva è quasi spesso frutto di di un compromesso stabilito dai compositori. Rappresentazioni astratte per mezzo di processi generativi, studi sulla sintesi granulare e sulla frammentazione delle immagini, percorsi pseudo-narrativi e altre tecniche, sono più o meno importanti, o che ruolo hanno rispetto a ciò che poi sentiamo? 
Dunque nasce prima un immagine o un suono o una semplice idea?

%Careful reflection on the pieces presented in this 7th concert suggests that all is not taken for granted and the Audio-Visual Art is almost often the result of a compromise established by the composers. Abstract representations through generative processes, studies on granular synthesis and the fragmentation of images, pseudo-narrative paths and other techniques are more or less important, or what role than what we hear then?
%So it was born before a picture or a sound or a simple idea?

%\bigskip

\begin{multicols}{2}

% !TEX encoding = UTF-8 Unicode
% !TEX TS-program = XeLaTex
% !TEX root = EMU2015_booklet.tex

\acusmatici{Antonio Mazzotti}
{I haven't seen you on the jumbotrons at TimeSquare}{2014}{10'44''}

\acusmatici{Francesc Martí}
{Speech 2}{2015}{7'53''}

\acusmatici{Alfredo Ardia, Sandro L'Abbate}
{Studio N.1}{2015}{2'}

\acusmatici{Carlo Barbagallo}
{Drowning}{2015}{8'00''}

\acusmatici{Roberto Musanti}
{Rotational Chaos}{2014}{3'50''}

\acusmatici{João Pedro Oliveira}
{Hydatos}{2012}{13'00''}

% \descrizione{I haven't seen you on the jumbotrons at TimeSquare}{'I have not seen you on the jumbotrons at TimeSSquare' was realized with the ComputerAidedAlgorithmComposition.It was conceived as a study for the computational models to produce musically meaningful results.The model which I am designing has been adopted as a tool of composition, investigating on the deep connection between sound and emotional meaning.It was implemented in “Mathematica”, “Csound” and Kyma, that uses the Pacarana as audio accelerator. "Processing" for the rendering video.}

\descrizione{I haven't seen you on the jumbotrons at TimeSquare}{Il modello che ho progettato indaga a fondo il legame tra le composizioni multimediali audio/video ed il significato emotivo. Il video e  l'audio in sincronia sono generati utilizzando lo stessa algoritmo. La composizione si sviluppa basandosi su pratiche di tipo algoritmico per progettazione assistita dal computer. L’ insieme di simboli gestiti da una serie di regole inserite nell’algoritmo producono espressioni complesse. Gli errori, imperfezioni e limitazioni dei particolari mezzi compositivi sono elementi costitutivi del pezzo. L’implementazione per il tramite di "Mathematica", "Csound" e’ Kyma’, che utilizza Pacarana come acceleratore audio. "Processing" per il video rendering.}

% \descrizione{Speech 2}{Speech 2 is an experimental audiovisual piece created from a series of old clips from the public affairs interview program The Open Mind. Technically, in this piece, the author has been experimenting how granular sound synthesis techniques, and pseudo-random number generators can be used for audiovisual creative works. All the piece sounds and images come from that series of clip, in other words, no other sound samples or images have been used to create the final result.}

\descrizione{Speech 2}{è un brano audio-video sperimentale ricavato da una serie di vecchi video presi da interviste del programma di affari pubblici \emph{The Open Mind}. Tecnicamente, in questo brano, l'autore ha sperimentato in quali modi le tecniche di sintesi granulare e i generatori di numeri pseudorandomici possono essere utilizzati in lavori creativi audiovisuali. Tutti i frammenti di suono e le immagini provengono da queste serie di video, in altre parole, nessun altro campione di suono o di immagini sono state utilizzate per ricreare il risultato finale.}

% \descrizione{Studio N.1}{A disoriented individual interacts with a virtual space. The body becomes an instrument, therefore able to respond to stimuli according to a specific sound logic. As any computer program, it is a system where the defects, bugs, disruptions and any kind of interferences, lead to a fragmented pace of the narrative itself. This work was born with the intention of creating a connection between what we see and what we hear, it's a study about connections and interactions of these processes. The simple sound material, sine waves and glitches, presents complex internal movements based on beats phenomenon linked to the dynamics of the video.}

\descrizione{Studio N.1}{Un individuo interagisce all’interno di uno spazio virtuale. Il corpo diviene strumento in grado di rispondere agli stimoli, secondo precise logiche sonore. Come ogni programma informatico, esso è un sistema in parte difettoso dove interferenze e interruzioni di vario genere determinano un andatura frammentata della narrazione. Il lavoro è nato con l’intenzione di creare una relazione tra ciò che si osserva e ciò che si ascolta, uno studio sul legame e sull’interazione di questi due aspetti. Il materiale sonoro utilizzato, sinusoidi e glitch, apparentemente semplice e di carattere minimalista, presenta in realtà complessi movimenti interni, dovuti al fenomeno dei battimenti, che seguono le dinamiche del video secondo precisi criteri compositivi.}

% \descrizione{Drowning}{Drowning is a three movements piece written by Carlo Barbagallo for Jean Francois Laporte's Table de Babel. Video directed & performed by Isobel Blank. Drowning: I: Surprise / Involuntary Breath Holding II: Unconsciousness III: Hypoxic Convulsions / Clinical Death}

\descrizione{Drowning}{I: Surprise/Involuntary Breath Holding II: Unconsciousness III: Hypoxic Convulsions/Clinical Death - Drowning, un invito ad annegare, è un brano in tre movimenti composto per la Babel Table, strumento musicale che utilizza l’aria compressa come sorgente di energia, d’invenzione di Jean- Francois Laporte. Ispirata alla fasi dell’annegamento, la composizione del brano è stata commissionata da Productions Totem Contemporain in collaborazione con la Scuola di Musica Elettronica del Conservatorio di Torino. Il video è stato diretto e realizzato dall’artista Isobel Blank. Il brano, nella sua versione dal vivo, è stato già eseguito a Giugno 2015 al Conservatorio G.Verdi di Torino, a Settembre alla Fountain School of Performing Arts della Dalhousie University (Halifax, Canada) e verrà eseguito a Novembre all’università di Huddersfield (UK). Musica: Carlo Barbagallo Video: Isobel Blank Babel Table: Jean-Francois Laporte}


% \descrizione{Rotational Chaos}{"Chaos rotation" is an audiovisual work that explores the relationship between images and sounds. Although the composition is abstract, because it is based primarily on the relationship between the forms, and between them and the sounds, the assembly of the material has the effect of a kind of narrative that we can define Geo/Math fiction.}

\descrizione{Rotational Chaos}{è un lavoro audiovisuale che esplora il rapporto tra suono e immagine. In primo piano, solidi di rotazione, il cui profilo è influenzato dal contenuto armonico di suoni da generatori caotici,  evidenziano il contrasto tra le simmetrie delle loro forme e il timbro sonoro. In secondo piano un sistema particellare funge da "sfondo" alla composizione. Benché si tratti di un lavoro astratto, i materiali grafici scelti e il loro montaggio, anche in rapporto al suono, invitano a una lettura “narrativa”, una sorta di geo/math-fiction.}

% \descrizione{Hydatos}{Hydatos is a greek word that means “water”. This piece is inspired on the first verses of the Old Testament (Genesis Chapter 1:2) “And the Spirit of God moved upon the face of the waters.” The audio part of this piece was commissioned by Gulbenkian Foundation, and was composed in the composer’s personal studio and at the NOVARS Center in Manchester. The video part was done at the composer’s personal studio.}

\descrizione{Hydatos}{Hydatos è una parola greca che significa “acqua”. Questo brano trae ispirazione dai primi versi del Vecchio Testamento (Genesi Capitolo 1:2) “E lo spirito di Dio aleggiava sulla superficie delle acque.” La parte sonora di questo video è stata commissionata dalla Fondazione Gulbenkian, ed è stata composta nello studio personale del compositore e al NOVARS Center a Manchester. La parte video è stata realizzata nello stesso studio personale del compositore.}




\end{multicols}

\clearpage
%----------------------------------------------------------------------------------------

\section*{8 ottobre 2015 -- ore 20:30}

\subsection*{{\small Concerto Acusmatico \& Live Electronics\footnote{ Il concerto verrà trasmesso in diretta streaming da Radio Cemat}} \\
	\textsf{Sala Accademica}}

{\fontsize{30}{30} \svolk{\emph{ATTO VIII}}}

\subsection*{\textsf{Regia del suono di Marco De Martino}}

\bigskip

\begin{multicols}{2}

% !TEX encoding = UTF-8 Unicode
% !TEX TS-program = XeLaTex
% !TEX root = EMU2015_booklet.tex

\livel{Enrico Minaglia}
{Physis III - Metal}{7'10''}
{per trombone e live electronics}
{2014}

\livel{N\'uria Gim\'enez-Comas}
{No More Words}{6'27''}
{per soprano e live electronics}
{2015}

\brano{Silvia Lanzalone}
{eRose}{9'54''}
{acusmatico}{2013}\\

\livel{Luciano Azzigotti}
{Vilanos}{10'}
{per due flauti amplificati}
{2014}

\livel{Karen Power}
{Deafening silence}{6'}
{pianoforte e live electronics}
{2014}

\livel{Pasquale Citera}
{Musica per organi caldi}{6'}
{intonazione e fuga elettroacustica per organo e live elctronics}
{2015}

\esecutore{trombone}{Raffaele Marsicano}
\esecutore{soprano}{Virginia Guidi}
\esecutore{flauto}{Alessandro Pace, Elena D'Alò}
\esecutore{pianoforte}{Francesco Ziello}
\esecutore{organo}{Giovanni Ubertini}

% \descrizione{Physis III - Metal}{This is the third piece of the Physis series, a tribute to Giordano Bruno's philosophy. Physis III  is focused on the "De gli eroici furori dialogue". It aims to depict in music the calm life of the wise man, and its sudden tranformation into the "furioso", a man who struggles to penetrate the misteries of Nature, and in the end will pay this knowledge with his life, like the mythical hunter Actaeon, and Giordano Bruno himself.}

\descrizione{Physis III - Metal}{Si tratta del terzo brano della serie "Physis", un tributo al pensiero di Giordano Bruno. Physis III è basata sul dialogo "De gli eroici furori". Ciò che la musica rappresenta è la vita tranquilla del saggio, e la sua improvvisa trasformazione nel "furioso", un uomo che lotta per penetrare i misteri della natura e che pagherà la sua conoscenza con la sua stessa vita, come la cacciatrice Actaeon, e Giordano bruno stesso.}

% \descrizione{No More Words}{Different concepts and ideas has given some inputs to the global conception of the piece: as information overload and non-useless information or text-speechs. To work with this ideas, I have used an open software of “speech to text” (Mary) of some “historical apologies” (found on the net) with different “artificial voices”. After I have granulated this texts to create voice masses in the space around us, (using a library of Open Music called Prisma). To recreate the idea of different spaces ‘saturation’ I have also introduced to the beginning some fragments of noisy landscapes with spoken voices and resynthesis layers of this soundscapes. This sound waves and masses are combined with a voice line that appears gradually and it’s given by the poem of Edgard Allan Poe. In “A dream within a dream” the poet is dealing in what we perceive as a reality and non-reality.}%aggiungere testo Poe

\descrizione{No More Words}{Idee e concetti differenti hanno contribuito alla concezione globale del brano: il sovraccarico di informazioni, i dati non inutili, il testo parlato. Per lavorare con queste idee, ho usato un software libero "speech to text" (Mary) con alcune "scuse storiche" (trovate in rete), con diverse voci artificiali. Dopo di che ho granulato questi testi per creare masse vocali nello spazio tutt' attorno, (utilizzando una libreria di Open Music chiamata Prisma). Per ricreare l'idea di "saturazione ' dei diversi spazi ho inoltre introdotto all' inizio del brano alcuni frammenti di paesaggi rumorosi con voci recitanti e strati di resintesi di questi paesaggi sonori. Le onde sonore e le masse sono state poi unite ad una linea di voce che appare gradualmente dalla poesia di Edgar Allan Poe. In \textit{Un sogno dentro un sogno}", il poeta si occupa di ciò che percepiamo come realtà e non realtà.}%aggiungere testo Poe

% \descrizione{eRose}{The word ‘eRose’ is composed by the words ‘electronic’ and ‘rose and was created for this piece to indicate the contrast between the natural and the virtual. The eRose piece is referred to the woman’s world: women’s bodies are modified and transfigured by digital communication, like ‘eroded roses’. Real sounds are totally transformed by computer to express the feeling of unease, but also the feeling of new discovery. The sounds are repeatedly clipped, ‘eroded’ or, on the contrary, sometimes carefully smoothed and refined.}

\descrizione{eRose}{La parola ‘eRose’ è stata coniata per questo brano attraverso la contrazione delle parole inglesi 'electronic' e 'rose', ed è utilizzata per i suoi eterogenei significati è anche per la struttura del fenomeno esogeno che descrive. Il brano è rivolto ad un universo femminile, di cui viene messo in evidenza il processo di decontestualizzazione che la comunicazione digitale ha intrapreso sull'immagine della donna. La parola 'eRose' Il materiale sonoro è infatti, continuamente eroso, corroso, abraso, oppure a volte, al contrario, anche meticolosamente limato, levigato, patinato. }

% \descrizione{Vilanos}{When a flower dies, and it’s petals wither, a perfect sphere formes inside. Like the subway, the veins, the umbrellas, like the earthworm’s path... arboreal systems grow regardless of the materials that comprise them, displaying the same geometry. They really come to fill the grooves that already exist in the space. The flute sound is transformed into a swirling turbulence, air that is fragmented and divided into a spiral. Hence the re- feeding of a listening technique transforms holes into resonators, the duo as symmetrical exchange point.}

\descrizione{Vilanos}{Quando un fiore muore, e i suoi petali appassiscono, al suo interno si forma una sfera perfetta. Come la metropolitana, le vene, gli ombrelli, come il percorso sotterraneo di un lombrico... I sistemi boreali crescono senza riguardo verso i materiali che li comprimono, percorrendo la stessa geometria. Arrivano a ricoprire le scanalature che già esistono nello spazio. Il suono del flauto è trasformato in una vorticosa turbolenza, l'aria viene frammentata e divisa in una spirale. Pertanto una tecnica d'ascolto di rialimentazione reciproca trasforma i fori in risuonatori, il duo come un punto di simmetria centrale.}

% \descrizione{Deafening silence}{is simply about fusing an acoustic instrument, in this case the piano, with a series of cricket recordings from Australia, Japan, Crete, Laos, Cambodia, Italy and Canada. Field recording has become a large part of my artistic practice and I hope that this is the first work in a series for solo acoustic instrument and natural environments from around the world. All recordings have been made by me while attentively listening to each space.}

\descrizione{Deafening silence}{Il silenzio assordante è semplicemente la fusione di uno strumento acustico, in questo caso il pianoforte, con una serie di registrazioni di grilli dell'Australia, Giappone, Creta, Laos, Cambogia, Italy e Canada. Il Field recording ha preso una larga parte della mia pratica artistica e spero che questo sia il primo lavoro di una serie per solo strumento acustico e ambienti naturali del mondo. Tutte le registrazioni sono fatte da me, mentre ascoltavo attentamente ogni spazio.}

%\descrizione{Musica per organi caldi}{You can scan this QR code with your mobile device to view or download the complete score and performance notes}%QR code partitura - link

\descrizione{Musica per organi caldi}{Usa questo QR code dal tuo dispositivo mobile per visualizzare o scaricare la partitura completa e le note d’esecuzione} %QR code partitura - link

\begin{pspicture}(1in,1in)
	\psbarcode{https://goo.gl/Jy2Esa}{eclevel=L width=1 height=1}{qrcode}
\end{pspicture}\footnote{https://goo.gl/Jy2Esa}




\end{multicols}

\clearpage

%----------------------------------------------------------------------------------------

\section*{9 ottobre 2015 -- ore 17:00}

\subsection*{{\small Concerto Acusmatico in Cupola Ambisonic} \\
	\textsf{Aula Bianchini}}

{\fontsize{30}{30} \svolk{\emph{ATTO IX}}}

\subsection*{\textsf{regia del suono di Luana Lunetta}}

Cinque viaggi all'interno del suono.
Da quello registrato a quello recitato, dai rumori concreti ai grani resintetizzati.
Le drammaturgie musicali si evolvono nello spazio 3D ricreato nella cupola sonora; l'ascoltatore ne resta così coinvolto e può piacevolmente farsi circondare dalle traettorie sonore.

%\bigskip

\begin{multicols}{2}

% !TEX encoding = UTF-8 Unicode
% !TEX TS-program = XeLaTex
% !TEX root = EMU2015_booklet.tex

\acusmatici{Daniel Osorio}
{Spiegelung}{2013}{9'38''}

\acusmatici{Daniel Blinkhorn}
{frostbYte - wildflower}{2014}{13'00''}

\acusmatici{Antonio Carvallo}
{Vri}{2013}{6'59''}

\acusmatici{Kenn Mouritzen}
{Cat-back}{2015}{8'39''}

\acusmatici{Kyle Vanderburg}
{Reverie of Solitude}{2014}{10'00''}

% \descrizione{Spiegelung}{"Spiegelung" (Image on the Mirror) is a work commissioned for the DVD project "Long Sguardo". The work is based on the Stefano AmoresÈs text "Specchiatura" and later became part of the video by visual artist Horkay Istvàn. "Spiegelung", an acousmatic piece for 5.1 surround, was composed from recordings by soprano Franziska Erdmann and narrator Jens Alles, to which was added the digital processing of different sound sources, also recorded for the work. The complexity and length of the text allowed generating different sound perspectives through different forms of sound synthesis, which however, are subordinate to rhythm and sound density of spoken text (German).}

\descrizione{Spiegelung}{"Spiegelung" (immagine a specchio) è un lavoro commissionato per il progetto DVD "Long Sguardo". Il lavoro si basa sul testo di Stefano Amorese "Specchiatura" e, in seguito, è entrato a far parte del video dell'artista visivo Horkay Istvàn. "Spiegelung", un brano acusmatico per surround 5.1, è stato composto con registrazioni del soprano Franziska Erdmann e del narratore Jens Alles, a cui è stata aggiunta l'elaborazione digitale delle diverse sorgenti sonore, anch'esse registrate per il lavoro compositivo. La complessità e la lunghezza del testo hanno permesso la generazione di prospettive sonore differenti attraverso diverse forme di sintesi del suono, che sono comunque subordinate al ritmo e alla densità del suono del testo parlato (in Tedesco).}

%\descrizione{Spiegelung}{„Spiegelung„ es una obra encargada para el proyecto en DVD „Largo Sguardo“ y se basó en el texto de Stefano Amorese („Specchiatura“). Es una pieza compuesta para surround 5.1 a partir de las grabaciones Franziska Erdmann y Jens Alles, a lo que se agregó el procesamiento digital de diferentes fuentes sonoras, también grabadas para la obra. La complejidad y extensión del texto permitió generar diferentes perspectivas sonoras a través de diferentes formas de síntesis de sonido.}

%\descrizione{frostbYte - wildflower}{frostbYte is the last in a cycle of works using field recordings from the high artic region of Svalbard. What was most discernible when recording fragments of glacial ice floating in fjords were the many and varied sonorous ecosystems emanating from underwater, each with its own distinctive personality. In every instance the ice fragments reacted differently to temperature, pressure and other observable phenomena, producing similar, yet unique sonorities. Throughout the work I wanted to capture some of the delicate complexity, as well as the unified symmetries produced through the charismatic, audible ecosystems indelibly linked to each of the naturally formed ice sculptures.} % In order to transcribe, then sculpt these natural carvings into gestures and phrases within the piece I chose to de-construct a number of hydrophone recordings into discrete elements, often organised into families of sound shapes. These typomorphologies were then re-constructed into a variety of gestures, phrases and forms, each of which contained its own attendant ecosystem of sound, much like the original field recordings. From a broader perspective, the resultant phrases are intended to mimic the idea of something that is carved. To my mind the final geometries and patterns sculptured became like those of the short-lived wildflowers that grow in the region, each populating its own unique ecosystem and all subject to the natural forces at play around them.

\descrizione{frostbYte - wildflower}{frostbYte è l'ultimo di un ciclo di opere che utilizzano registrazioni ambientali realizzate nell'alta regione artica di Svalbard. Ciò che è risultato più evidente durante le registrazioni dei frammenti di ghiaccio galleggianti nei fiordi, sono stati i molti e vari ecosistemi sonori sott'acqua, ognuno con la propria distinta personalità. I frammenti di ghiaccio hanno reagito in modo diverso a seconda della temperatura, pressione e altri fenomeni osservabili, producendo sonorità simili, ma uniche.In questo lavoro ho voluto catturare un po ' della delicata complessità e contemporaneamente le simmetrie unificate, prodotte attraverso i carismatici ecosistemi sonori indelebilmente legati a ciascuna delle sculture di ghiaccio formatesi naturalmente.} %Al fine di trascrivere, poi scolpire queste sculture naturali in gesti e frasi all'interno il pezzo, ho scelto di de-costruire una serie di registrazioni con idrofoni in elementi discreti, spesso organizzati in famiglie di forme sonore. Queste tipomorfologie sono state poi ri-costruite in una varietà di gesti, frasi e forme, ognuna delle quali conteneva un proprio ecosistema sonoro, proprio come i field recordings originali.Da una prospettiva più ampia, le frasi risultanti sono destinate ad imitare l'idea di qualcosa che viene scolpito. A mio avviso, le geometrie finali e i modelli scolpiti sono diventati come quelli dei fiori selvatici dalla breve vita che crescono nella regione, i quali popolano il proprio unico ecosistema e sono tutti soggetti alle forze naturali che agiscono intorno ad essi.

% \descrizione{Vri}{The piece is made from the deployment of numerical proportions in charge of controlling the parameters that regulate the synthesis and processing of sound, proportions that somehow guarantee a certain sonority and organization that tends to achieve unity. The piece is made from synthesis subtractive (white noise filtered) and convolution. The resulting sounds are analyzed and their formants with more energy are filtered; finally, an analysis and resynthesis process generates frequency bands that are being shipped, differentially, to the four channels.}

\descrizione{Vri}{Il pezzo è realizzato utilizzando il dispiegamento di proporzioni numeriche incaricate di controllare i parametri che regolano la sintesi e l'elaborazione del suono, proporzioni che in qualche modo garantiscono una certa sonorità e organizzazione, in modo da raggiungere l'unità. Il pezzo è realizzato in sintesi sottrattiva (rumore bianco filtrato) e convoluzione. I suoni risultanti sono analizzati e le formanti con più energia vengono filtrate; infine, un processo di analisi e risintesi genera bande di frequenza che vengono spedite, in modo differenziato, ai quattro canali.}

% \descrizione{Cat-back}{Cat-back is a micro composition on basis of bass clarinet. The recordings were extensively treated in order to give other character while at the same time keeping the powerfull mechanical effect this instrument has.}

\descrizione{Cat-back}{Cat-back è una micro composizione basata su clarinetto basso. Le registrazioni sono state ampiamente trattate in modo da dare un altro carattere al suono, ma al tempo stesso mantenere il potente effetto meccanico dello strumento.}

% \descrizione{Reverie of Solitude}{The piece serves as both an exploration of and a invitation to reverie, providing a space wherein the listener is asked to reconsider their idea of what it means to daydream. At once immersed in a familiar crowd hum, lost among the multitude; it is easy to believe that this daydream is not an expression of solitude, but rather a longing for solitude. And so the piece suggests the pattern of a day dream: the crowd noise giving way to a train, a lazy lawn sprinkler, a contemplative rain storm, a frothing river which becomes a bucolic afternoon on the lake. Each vignette is a self-contained narrative wherein to consider solitude in a natural context. The metaphor of water and the alternating themes of movement and respite invite the listener to reflect on the purpose of a daydream: to escape, to pacify, or to enrich a perfect moment. After having their attention turned to the daydream they themselves have been lulled into, the listener is returned to the crowd hum having established a personal sense of solitude within the piece and within the audience. Program Note by Walter Jordan}

\descrizione{Reverie of Solitude}{Il brano può essere considerato sia come un'esplorazione che un invito al fantasticare, fornendo uno spazio in cui l'ascoltatore è invitato a riconsiderare l' idea di ciò che significa sognare ad occhi aperti. All'inizio l'ascoltatore è immerso in un familiare ambiente con ronzio di folla, perso tra la moltitudine; è facile credere che questo sogno ad occhi aperti non sia espressione della solitudine, ma piuttosto un desiderio di solitudine. E così la composizione suggerisce il modello di una fantasticheria: il rumore della folla cede il posto a quello di un treno, un pigro sistema d'irrigazione per prati, una contemplativa tempesta di pioggia, la schiuma di un fiume che diventa un pomeriggio bucolico sul lago. Ogni vignetta è una narrazione autonoma che prende in considerazione la solitudine in un contesto naturale. La metafora dell'acqua e l'alternarsi di  tematiche di movimento e di tregua invitano l'ascoltatore a riflettere sullo scopo del sogno: la fuga, per pacificare, o per arricchire un momento perfetto. Dopo essere stato cullato dal sogno ad occhi aperti, l'ascoltatore viene restituito al ronzio della folla dopo aver stabilito un personale senso di solitudine all'interno del brano e nel pubblico. Note di Walter Jordan}


\end{multicols}

\clearpage
%----------------------------------------------------------------------------------------

\section*{9 ottobre 2015 -- ore 20:30}

\subsection*{{\small Concerto Acusmatico \& Live Electronics\footnote{ Il concerto verrà trasmesso in diretta streaming da Radio Cemat}} \\
	\textsf{Sala Accademica}}

{\fontsize{30}{30} \svolk{\emph{ATTO X}}}

\subsection*{\textsf{Regia del suono di Francesco Ziello}}

\bigskip

\begin{multicols}{2}

CONCERTO 10
Concerto 20.30 (Sala Accademica)
MASTER CONTEMPORANEA Battista


\livel{Javier Alejandro Garavaglia}
{Hoquetus}{2005-6} %9'
{for Soprano saxophone and multi-track electronics}
{sassofono soprano}{Enzo Filippetti}

\descrizione{Hoquetus}{The first version of the piece was commissioned by Prof. Esther Lamneck (NYU), who also premiered it at the 14th Florida Electroacoustic Music Festival on the Tárogató (April 2005). The current version is a further development, with changes in the instrumental part (including its adaptation for Saxophone) as well as in the electronics, which have been completely reprogrammed since the world premiere.Basically, the work tries to recreate in a different context the medieval technique of the Hocket. Being this technique polyphonic, the Tárogató part delivers a sort of “hiccup” technique with itself by the use of different registers. Moreover, the computer makes a Hocket-like response to the Tárogató at certain moments.}

\descrizione{Hoquetus}{La prima versione di questo brano fu commissionata dal prof. Esther Lamneck (NYU), e fu premiato alla quattordicesima edizione del Florida Electroacoustic Music Festival (Aprile 2015). La versione revisionata ha avuto diversi sviluppi, con il cambio da parte degli strumenti (compreso il suo adattamento per sassofono) non che nell'elettronica, dove è stata completamente riprogrammata rispetto alla prima mondiale. Il lavoro ha cercato di ricreare in un differente contesto la tecnica medievale dell' Hoquetus. Essendo una tecnica polifonica, le parti di Tárogató fornisce una sorta di tecnica del "singhiozzo" mediante l uso di diversi registri. Inoltre, in certi momenti  il computer esegue una risposta simil-hoquetus al Tárogató.}


\acusmatico{Gustavo Delgado}
{Permanente e transitorio}{2015} %7'20''
{acusmatico}

\descrizione{Permanente e transitorio}{The composition deals with two seemingly opposite concepts in a such closer relationship: permanent and temporary. By the combination of different mixing and editing techniques very often used in the cinema and video games sfx, there were created very complex "impact sounds" classified by their dynamic structure as short (temporaries). Resonance frequencies have been isolated and emphasized from those materials in order to obtain stable sounds (permanents) to be used as starting points or bridges to the original impact sounds and viceversa.}

\descrizione{Permanente e transitorio}{Mediante il montaggio e la combinazione di diverse tecniche di missaggio e sound design utilizzate al cinema, ai videogiochi e alla musica elettronica, sono stati creati numerosi materiali di tipo “impact sounds” (materiale di tipo transitorio) da cui sono estratte delle risonanze (materiale di tipo permanente) dinamicamente modulate in frequenza (FM) su una patch in MaxMSP creata dall’autore.}


\livel{Sylvano Bussotti}
{Ultima RARA}{versione 2015} %8'00''
{per un chitarrista recitante}
{chitarra}{Arturo Tallini}

\descrizione{Ultima Rara}{eng}

\descrizione{Ultima Rara}{Ultima Rara è pensata “per voce e da una a tre chitarre” la versione di questa sera riunisce le 4 figure e infatti viene proposta come versione “per un chitarrista recitante” così come proposta da Arturo Tallini con l’approvazione del compositore. Il testo e conseguentemente la musica hanno un carattere multiforme, fortemente teso all’espressione intima, seppure non disgiunta da un’ironia che qua e là fa capolino. Il brano ci racconta la fine di una storia passando dal dolore, al delirio alla malinconia. Il tipo di recitazione non è indicata in alcun modo dunque la versione di stasera è stata interamente elaborata dall’interprete.}


\livel{James Dashow}
{Soundings in Pure Duration n.7}{2015} %10'02"
{per sassofono contralto e live electronics}
{sassofono}{Enzo Filippetti}

\descrizione{Soundings in Pure Duration n.7}{The seventh in the series of Soundings in Pure Duration provides a wealth of timbres and trajectories in octophonic space designed to place the alto saxophone soloist in the middle of varying degrees of transformative motion, musical and  physical.  No, the saxophonist doesn't fly around the hall, but the three-dimensional temporally structured spatialization of the constantly evolving sounds creates the impression of everything, including the soloist, moving in contrappuntal complexities.}

\descrizione{Soundings in Pure Duration n.7}{Il settimo brano della serie Soundings in Pure Duration propone un'abbondanza di timbri e tragitti nello spazio ottofonico, creati per collocare il sassofono contralto al centro 

di una varietà di movimenti trasformativi, musicali e fisici.  No, il sassofonista non vola attraverso la sala, ma la spazializzazione dei suoni in evoluzione è strutturata 

temporalmente nelle tre dimensioni così da creare l'impressione che tutto, incluso il solista, si muova nelle complessità contrappuntistiche.}


\livel{Maurizio Pisati}
{SENTI!}{} %15'00''
{chitarra, percussione, orchestra d’archi, e live electronics}
{chitarra}{Arturo Tallini}

\descrizione{SENTI!}{In SENTI! (listen up!) the sounds and gestures born in Senti? (can you hear?) and Sì. Cos’è? (Yes. What is this?) came to an end, closing a triptyque made in the two previous operas. Now the final disposition of the compositions is transformed into an essential dialogue: «Senti? Sì, cos’è? SENTI!». The triptyque is prompted by a short story by August Strindberg “Stora grusharpan” here adapted at the composer will. This leads us into the sea bottom, and from here it tells of sounds never heard before: a Piano falls into water, a storm deforms it and the fishes explore it lightly touching it’s string, while on the shore two guys are listening. The title of the first track –Senti?– it’s the question that one of the two asks to the other after hearing that sounds. In the second track the friend answers, asking too “Sì. Cos’è?”.  In the latter part, after wondering briefly on the nature of the sounds, no one asks anything or waits for an answer, until one of the two states: “sì, ma senti!”. From the brief dialogue of the two guys, still able to appreciate the marvel, is born the third score, which owes his shape to the imaginary breathe of things and sounds sinked in unreachable places.}

\descrizione{SENTI!}{In SENTI! giungono alla conclusione i gesti e alcune sonorità nate in Senti? e Sì. Cos’è?, i due primi pezzi di un trittico. Ora quindi l’ordine finale delle tre composizioni è diventato anche un dialogo essenziale: «Senti? Sì, cos’è? Senti!» Il trittico prende le mosse dalla novella di August Strindberg “Stora grusharpan” qui liberamente adattata. Ci conduce su un fondale marino e da lì racconta di sonorità mai sentite: un Pianoforte cade in acqua, una tempesta lo deforma e i pesci lo esplorano sfiorando le corde, mentre sulla riva due ragazzi ascoltano. Il titolo del primo brano –Senti?– è la domanda che uno rivolge all’altro dopo aver udito quei suoni. Nel secondo pezzo l’amico risponde, a sua volta chiedendo: “Sì, cos’è?”. Infine qui, nel terzo brano, dopo un breve scambio di supposizioni sulla natura di quei suoni, nessuno più domanda o attende risposta, ma uno dei due formula la frase forse più importante, l’esortazione ad ascoltare: “sì, ma senti!”. Dal breve dialogo dei due ragazzi che ancora sanno godere della meraviglia, nasce questa terza partitura, che deve la sua forma al respiro immaginario delle cose e dei suoni sprofondati in un luogo di solito inaccessibile.}


\end{multicols}

\clearpage
%----------------------------------------------------------------------------------------

\section*{12 ottobre 2015 -- ore 18:00}

\subsection*{{\small Concerto Acusmatico \& Live Electronics\footnote{ Il concerto verrà trasmesso in diretta streaming da Radio Cemat}} \\
	\textsf{Auditorium Ennio Morricone}}

{\fontsize{30}{30} \svolk{\emph{ATTO XI}}}

\subsection*{\textsf{regia del suono di Massimiliano Mascaro}}

“Vieni, disse la Musa,
cantami un canto che nessun poeta ha mai cantato,
cantami l'universale.
Nella nostra vasta terra,
in mezzo alla volgarità smisurata e alla feccia,
racchiuso e sicuro nel suo cuore più intimo,
si nasconde il seme della perfezione.”
(Walt Whitman - Canto dell'universale)

“La metafora è probabilmente la forza più feconda che l'uomo possieda.”
(Jos\'e Ortega y  Gasset - La disumanizzazione dell'arte)

L'esibizione Atto XI racchiude cinque brani che inglobano le varie forme della musica elettronica, passando da un live-electronics a un acustico e tape, finendo con un acusmatico. I titoli di questi lavori hanno come filo conduttore il rapporto tra l'ambiente e la natura umana che li integra e li unisce.

%\bigskip

\begin{multicols}{2}

% !TEX encoding = UTF-8 Unicode
% !TEX TS-program = XeLaTex
% !TEX root = EMU2015_booklet.tex

\livel{Frei}
{Poetics}{2015} %16'50''
{per due laptop}
{due laptop}{Paolo Gatti, Francesco Bianco}

\livel{Christian Banasik}
{Ik ́}{2013} %9'03''
{per flauto e live electronics}
{flauto}{Elena D'Alò}

\livel{Marco Marinoni}
{Finita è la terra}{2015} %4'16''
{pianoforte e live electronics}
{pianoforte}{Sara Ferrandino}

\livel{Domenico De Simone}
{A[LIVE]}{2015} %5'30''
{pianoforte, vibrafono  ed elettronica su supporto}
{pianoforte}{Sara Ferrandino}
vibrafono -- \textsc{Matteo Rossi}
\\

\acusmatico{Benjamin D. Whiting}
{Illumina! Arabidopsis thaliana}{2015} %9'18''
{acusmatico}

% \descrizione{Poetics}{Work based on the concepts of entropy, redundancy, and noise and how these play a role on messages and information in a musical discourse. From an aesthetic point of view, everything is resolved, here, in an alternation between punctiform moments, almost static, and growing dynamic more or less sudden.}

\descrizione{Poetics}{Lavoro basato sull'improvvisazione giocata sull'alternanza fra momenti puntiformi, quasi statici, e dinamiche di crescendo più o meno improvvise. Altro tema trattato è il rapporto fra comunicabilità e incomunicabilità tramite l'indagare delle modalità di trasmissione dei messaggi musicali e vocali e i concetti di entropia, ridondanza e di rumore.}

% \descrizione{Ik ́}{\textit{Ik ́} is the name of the 2nd day in the ritual calendar of the Maya associated with breath and wind.The liturgical year consisted of twenty cycles and their glyphs, each of them thirteen days long, and had 260 days in all. For the tone material I use a contemporary folk song from Central America and a virtually generated original song of the Maya.The flute score consists of virtuoso abstracted variations mixed with the graphical-audio analysis of this particular glyph. It is divided in 20 parts.}

\descrizione{Ik ́}{\textit{Ik ́} è il nome del secondo giorno del calendario rituale dei Maya, associato al respiro e al vento. L'anno liturgico era composto da venti cicli e i loro glifi, ognuno dei quali durava tredici giorni, avendo in totale 260 giorni. Per il materiale tonale ho usato una canzone popolare contemporanea dell'America Centrale, e una canzone dei Maya generata virtualmente. La parte del flauto consiste in variazioni astratte miste all'analisi grafica-audio di questo particolare glifo. È diviso in 20 parti.}

% \descrizione{Finita è la terra}{is a small formal experiment that makes use of three piano sound objects, variants of a single musical image, which remain identical to themselves during the entire piece, arranged in a formal course partly iterative and partly structured through microvariations. These objects interact along the time axis with material fixed on support, agglomerations deriving from their possible electroacoustic processing – a live-electronics frozen, or even more than a live-electronics, an image of it, captured and distilled through an alchemical process which aims to turn something that was only movement, dispersion, probability (grain clouds applied to the three sound events of origin) in its \greco{εἴδωλον}, playing this way with the perception and memory in a continuum consisting of appearance / disappearance / memory / expectation / desire.}

\descrizione{Finita è la terra}{è un piccolo esperimento formale che fa uso di tre oggetti sonori pianistici, varianti di un'unica immagine musicale, che permangono identici a se stessi durante tutto il brano, disposti secondo un decorso formale in parte iterativo e in parte microvariato. Tali oggetti interagiscono lungo l’asse del tempo con materiali fissati su supporto, agglomerazioni derivanti da una loro possibile processazione elettroacustica – un live-electronics ghiacciato, o ancora, più che un live-electronics, una sua immagine, catturata e distillata, mediante un processo alchemico volto a trasformare qualcosa che era solo movimento, dispersione, probabilità (le nubi di granulazioni applicate ai tre eventi sonori di partenza) nel suo \greco{εἴδωλον}, giocando in questo modo con la percezione e la memoria in un continuum fatto di apparizione/sparizione/ricordo/attesa / desiderio.}

% \descrizione{A[LIVE]}{I AM NOT THERE \\ Do not stand at my grave and weep. \\ I am not there. I do not sleep. \\ I am a thousand winds that blow. \\ I am the diamond glints on snow. \\ I am the sunlight on ripened grain.  \\ I am the gentle autumn rain. \\ When you awaken in the morning’s hush \\ I am the swift uplifting rush \\ Of quiet birds in circled flight. \\ I am the soft stars that shine at night. \\ Do not stand at my grave and cry; \\ I am not there. I did not die. \\ Mary Elizabeth Frye}

\descrizione{A[LIVE]}{IO NON SONO LÌ \\ Non piangere sulla mia tomba.  \\ Io non sono lì. Io non sto dormendo.  \\ Io sono mille venti che soffiano. \\ Sono lo scintillio del diamante sulla neve. \\ Sono il sole che brilla sul grano maturo. \\ Sono la lieve pioggia d'autunno.  \\ Quando tu ti svegli nel silenzio del mattino, \\ io sono il rapido fruscio \\  degli uccelli che volano in cerchio. \\ Io sono la piccola stella che brilla di notte. \\ Non piangere sulla mia tomba;  \\ io non sono lì, la mia anima non è morta.   \\ (libera traduzione della poesia I AM NOT THERE di Mary Elizabeth Frye)}

% \descrizione{Illumina! Arabidopsis thaliana}{This piece represents the ongoing artistic and scientific collaboration between genomic biologist Aleel K. Grennan and myself. Grennan is studying the rate of photosynthesis between a natural wild type of Arabidopsis thaliana leaf and three genetically engineered mutants with different sizes of chloroplasts. I designed the majority of the sonic material in DISSCO and KYMA, incorporating Grennan’s data into several parameters, thus creating a wealth of stylized sounds.}

\descrizione{Illumina! Arabidopsis thaliana}{Questo brano rappresenta l'attiva collaborazione artistica e scientifica tra il biologo genetico Aleel K. Grennan. Grennan sta studiando l'indice di fotosintesi tra un tipo naturale selvatico di foglia Arabidopsis thaliana e tre diverse mutazioni geneticamente modificate con diverse grandezze di cloroplasti. La maggior parte dei materiali sonori sono stati progettati con DISSCO and KYMA, incorporando i dati di Grennans attraverso alcuni parametri, sono stati creati così un'abbondanza di suoni stilizzati.}




\end{multicols}

\clearpage
%----------------------------------------------------------------------------------------

\section*{13 ottobre 2015 -- ore 18:00}

\subsection*{{\small Concerto Acusmatico \& Live Electronics\footnote{ Il concerto verrà trasmesso in diretta streaming da Radio Cemat}} \\
	\textsf{Auditorium Ennio Morricone}}

{\fontsize{30}{30} \svolk{\emph{ATTO XII}}}

\subsection*{\textsf{Regia del suono di Francesco Bianco}}

\bigskip

\begin{multicols}{2}

% !TEX encoding = UTF-8 Unicode
% !TEX TS-program = XeLaTex
% !TEX root = EMU2015_booklet.tex

\livel{Jorge García del Valle Méndez}
{No sun, no moon}{9'20''}
{per flauto basso ed elettronica su supporto}
{2012}

\livel{Giovanni Costantini}
{Traccia sospesa}{7'35''} 
{pianoforte ed elettronica su supporto}
{2015}


\brano{James Andean}
{Déchirure}{9'58''}
{acusmatico}{2013}\\

\livel{Mario Mary}
{Rock}{8'30''}
{per pianoforte ed elettronica su supporto}
{2013}

\livel{Inhorep}
{Inhorep@emufest}{15'}
{laptop e autocostruiti}
{2015}
\\

\esecutore{flauto basso}{Alessandro Pirchio}
\esecutore{pianoforte}{Sara Ferrandino}

\noindent \textsc{Inhorep}:\\
\esecutore{chitarra preparata e laptop}{Davide Palmentiero}
\esecutore{strumenti autocostruiti e laptop}{Giuseppe Pisano}
\esecutore{laptop}{Massimo Varchione}

%\vspace{12mm}

% \descrizione{no sun, no moon}{no sun, no moon immerses into a world without reference points, where the reality is considered by two different points of view, and we do not know which is the real and which is not. In no sun, no moon, the bass flute and the electronics are the two worlds, reality and parallel reality. Both are the same thing and simultaneously its opposite, interacting and reacting one another. The raw materials for the electronics are exclusively bass flute samples. Honorary Mention at the CICEM 2014.}

\descrizione{No sun, no moon}{Con \textit{no sun, no moon} siamo immersi in un mondo senza punti di riferimento, in cui la realtà è considerata da due diversi punti di vista, e non sappiamo quale sia la reale e quale non lo sia. Il flauto basso e l'elettronica sono i due  mondi; realtà e realtà parallela. Entrambi sono la stessa cosa e contemporaneamente il loro contrario, interagiscono e reagiscono tra di loro. Le materie prime dell'elettronica sono esclusivamente campioni di flauto basso. Mensione d'onore al Cicem 2014.}

%\descrizione{Traccia sospesa}{The piece evokes events, places, sounds and feelings related to the First World War. The piano is played in a "non-classical" way, exploring new sonorities useful to convey in the listener pain, dismay, disbelief. The electronics consists in a textures created through elaborations of piano sounds: an alter ego with which the piano can dialogue, in an atmosphere suspended between memories and uncertainties.}

\descrizione{Traccia sospesa}{Il brano vuole evocare avvenimenti, luoghi, suoni e sentimenti legati alla prima guerra mondiale. Il pianoforte diventa strumento utile a trasmettere dolore, sgomento, incredulità, mediante un utilizzo “non classico” che ne esplora sonorità nuove. La parte elettronica è costituita da una trama di tracce sonore realizzate al computer attraverso elaborazioni di suoni di pianoforte: un alter ego con cui dialogare, ricercando e ricordando, in un’atmosfera sospesa e di continua incertezza.}

% \descrizione{Déchirure}{Déchirure: a tearing, a painful separation... This piece involves a number of 'déchirures', both musical as well as figurative, although the only literal 'tearing' is saved for the final phrase. This work was composed for Presque Rien 2013, in which it received a Special Mention. For this project, sounds from Luc Ferrari's archives were made available to composers for the composition of new works; all sound materials used in the piece are sourced and developed beginning from these recordings.}

\descrizione{Déchirure}{Déchirure: uno strappo, una dolorosa separazione…Questo brano comporta una serie di "déchirure" (strappi), sia musicali, così come figurativi, anche se l'unico letterale "strappo" viene lasciato per la frase finale. Questo lavoro è stato composto per Presque Rien 2013, in cui ha ricevuto una Menzione Speciale. Questo progetto, i suoni sono stati resi disponibili da archivi di Luc Ferrari ai compositori per la composizione di nuove opere; tutti i materiali sonori utilizzati nel brano sonori sono di provenienza e sviluppati a partire da queste registrazioni.}

% \descrizione{Rock}{It is a kind of homage to progressive rock, also called symphonic rock, which appeared in the 70s, and continues to accompany me in my inner world, though my compositions are always distinctly contemporary aesthetics. In my teens, groups like Yes, Pink Floyd, Emerson Lake & Palmer, injected me the virus of electronic sound, before than I discovered the existence of contemporary and electroacoustic music. "Rock" is not a rock, but is inspired by the energy and character of the music of my youth. Moreover, this work is based on a virtuous dialogue between the instrumentalist and the electroacoustic part.}

\descrizione{Rock}{Si tratta di una sorta di omaggio al rock progressivo, chiamato anche sinfonico, che è apparso negli anni '70, e continua ad accompagnarmi nel mio mondo interiore, anche se le mie composizioni sono sempre un'estetica decisamente contemporanea. Nei miei adolescenza, gruppi come Yes, Pink Floyd, Emerson Lake \& Palmer, ho iniettato il virus del suono elettronico, prima di scoprire l'esistenza di musica contemporanea ed elettroacustica.“Rock” non è una roccia, ma si ispira l'energia e il carattere.}

% \descrizione{Inhorep@emufest}{\textit{INHOREP} is a project of radical electroacoustic improvisation. The trio (first active as Improvviso) was formed in the course of Electronic Music of the  Conservatory of Naples under the impulse of  Elio Martusciello. The members immediately focused themselves on live exhibition, participating in several Italian festivals. Their performance is based on the interplay between the three musicians (guests are always welcome), strengthened through rehearsals and on the the attitude of listening to each other. The musicians improve theirs instruments with a DIY (Do it Yourself ) philosophy, recovering instruments and other sound objects, using the circuit bending and other techniques.}

\descrizione{Inhorep@emufest}{\textit{INHOREP} è un progetto di improvvisazione radicale elettroacustica. Il trio (prima attivo come Improvviso) si è formato nella classe di Musica Elettronica del Conservatorio di Napoli sotto l'impulso del M° Elio Martusciello. Fin da subito il gruppo ha deciso di puntare sull'esibizione dal vivo, partecipando a diversi festival italiani. Le loro performance si basano sull'interazione tra i tre musicisti (e gli eventuali ospiti, sempre assai graditi), affinata attraverso le prove e l'attitudine all'ascolto reciproco. Individualmente, gli strumentisti costruiscono e migliorano costantemente i loro strumenti attraverso la filosofia D.I.Y. (Do it Yourself), recuperando strumenti e altri oggetti sonori, ricorrendo alle possibilità del circuit bending ed altre tecniche.}



\end{multicols}

\clearpage




%----------------------------------------------------------------------------------------

% ---------------------------------------------------------------------------------------------
\section*{ }

\subsection*{\textsf{Autori ed Esecutori}\\}

{\fontsize{30}{30} \svolk{\emph{Biografie}}}

%\bigskip

\begin{multicols}{2}

% !TEX encoding = UTF-8 Unicode
% !TEX TS-program = XeLaTex
% !TEX root = EMU2015_booklet.tex

\biografia{James Andean}{Musicista e sound artist. È attivo sia come compositore che esecutore in vari campi, incluse composizioni elettroacustiche e performance, improvvisazioni, installazioni audio, e registrazioni. È membro fondatore del quartetto di improvvisazione e nuova musica Rank Ensemble, e fa parte del duo Plucié/DesAndes (audiovideo). Si è esibito per l'Europa e Nord America, e i suoi lavori sono stati presentati in tutto il mondo.}

\biografia{Damián Anache}{(1981, Buonos Aires) Musicista e sound artist. è attivo come compositore e performer nell'improvvisazione, istallazione sonora, sound recording. è membro fondatore di "improvisation and new music quartet Rank Ensemble, e fa parte del duo audiovisual Plucié/DesAndes. Si è esibito in tutta l'Europa e Nord America i suoi lavori sono stati presentati in tutto il mondo. Website: damiananache.com.ar}

\biografia{Alfredo Ardia}{Classe 1989, ha studiato al LEMS (Pesaro, Italia) e al CMT (Helsinki, Finlandia). È interessato al suono, alla sua percezione e a come esso si relaziona con altri media, esplorando i fenomeni sonori di entità elementari ed i loro comportamenti. È ispirato dalla bellezza della fisica.  Web: http://alfredoardia.altervista.org/}

\biografia{Luciano Azzigotti}{Compositore di Buones Aires, Argentina. La sua musica comprende differenti mezzi ed ascolti, gesti e interfacce di scrittura. Ha studiato composizione all'università di La Plata, in diversi master e partecipato a seminari internazionali con Gerardo Gandini, Mauricio Kagel, George Aperghis, Chaya Czernowin e Rebecca Saunders. La sua musica è stata eseguita in Argentina, Brasile, USA, Austria, Germania, Francia, Italia, da importanti ensemble e solisti. è fondatore del conDIT BsAs.}

\biografia{Christian Banasik}{(1963) ha studiato composizione alla Robert Schumann Academy of Music and Media di Dusseldorf e alla University of Music and Performing Arts di Francoforte. I suoi lavori strumentali ed elettronici sono stati eseguiti in concerti e programmi radio in tutta Europa, ma anche in America, Asia e Australia. È docente di Audio Visual Design presso la University for Applied Sciences e direttore artistico di Computer Music Studio "Studio 209" di Dusseldorf.}

\biografia{Carlo Barbagallo}{(1985) musicista, compositore e sound engineer.Sin da piccolo ha registrato la propria musica, sperimentando sulle (non) possibilità creative della registrazione casalinga:i lavori sono in rete tramite l'etichetta Noja Recordings.Dal 2012 è musicista elettroacustico con ricerca concentrata su: spazializzazione di musica acusmatica, feedback nelle sue diverse forme, improvvisazione e programmazione-tramite-ascolto come tecniche compositive, sviluppo di strumenti algoritmici per generare partiture e suoni a partire dalla struttura di testi linguistici, estetica delle forme incomplete, la composizione collettiva.Nel2013 è co-fondatore del Collettivo di Musica Elettroacustica di Torino (CoMET).Tra i festival: Premio Phonologia, ICMC, PNDA, 57Festival Internazionale di Musica Contemporanea (Biennale di Venezia), EMUFest, PIARS, XXCIM, Di\textunderscore Stanze. http://nojarecordings.tumblr.com}

\biografia{Antonella Barbarossa}{è nata e vive in Italia. Didatta del pianoforte, organista, direttrice d’orchestra, compositrice, filosofa e missionaria in Calabria per scelta dove assume la docenza di pianoforte al Conservatorio di stato di Cosenza nel 1976. Nel 1991 diviene direttore del Conservatorio di Vibo Valentia sino al 2013; nel 2003 fonda il politecnico internazionale “ scientia et ars” per la specializzazione in tecnologia del suono. In campo organistico ha eseguito in concerti pubblici l’opera integrale di Bach, Franck, Liszt e Messiaen, e in prima assoluta per la Rai e festivals internazionali, composizioni d’autori contemporanei. È vincitrice del primo premio al Concorso internazionale organistico di Roma nel 1981 e finalista al Concorso Internazionale di Lipsia per lo stesso strumento.}

\biografia{Cathy Berberian}{Mezzosoprano e compositrice statunitense d'origine armena (Attleboro 1925 - Roma 1983). Esordì a Napoli nel 1957 e si dedicò prevalentemente all'attività concertistica rivelandosi come una delle sostenitrici più valide della "nuova vocalità". Considerata una delle maggiori interpreti della musica contemporanea, ha coltivato anche il genere folcloristico, la musica da cabaret ed è stata apprezzata interprete di Monteverdi e di musica del periodo barocco. Hanno scritto per lei I. Stravinskij, L. Berio, S. Bussotti e J. Cage. Ha composto musica vocale e strumentale; ha pubblicato La nuova vocalità nell'opera contemporanea (1966).}

\biografia{Luciano Berio}{Allievo di G. F. Ghedini e G. C. Paribeni a Milano e di L. Dallapiccola a Tanglewood, nel 1954 ha fondato con B. Maderna lo Studio di fonologia musicale alla RAI di Milano. Esponente fra i più agguerriti e significativi dell'avanguardia musicale contemporanea, si è dedicato tra i primi all'esperienza elettronica. Ha scritto musiche di scena, pezzi orchestrali, da camera e vocali in cui ha alternato l'uso di strumenti tradizionali a quelli derivati dalla tecnica elettronica. Presidente e sovrintendente dell'Accademia di Santa Cecilia dal 2000, B. ha svolto una intensa attività didattica a Darmstadt, Colonia e in varie università degli USA; nel 2002 ha proposto un nuovo finale di Turandot di G. Puccini in sostituzione di quello realizzato da F. Alfano.}

\biografia{Francesco Bianco}{Laureato in Musicologia, frequenta il corso di Musica elettronica presso il Conservatorio di Roma Santa Cecilia. È stato visiting scholar al CRR de Boulogne-Billancourt (Parigi).  Musicista da sempre interessato alle profonde relazioni fra l'arte e la vita, ha esperienze musicali variegate di genere e stile, dalla composizione alla performance dal vivo, dall'azione scenica alla colonna sonora.}

\biografia{Isobel Blank}{Nasce a Pietrasanta in Toscana, si laurea con lode in filosofia estetica a Padova, vive e lavora a Torino. Le sue opere sono state esposte in numerose gallerie, musei e festivals, dagli Stati Uniti alla Cambogia, dal Messico alla Russia. Tra le esposizioni recenti, quella alla Triennale Fiberart International di Pittsburgh, al Museum of Modern Art di Mosca, a Palazzo Widmann di Venezia, alla Mumbai Art Room in India. Ha avuto diversi riconoscimenti tra cui il Primo Premio al Romaeuropa Webfactory nel 2009, sezione videoarte. http://www.isobelblank.com}

\biografia{Daniel Blinkhorn}{ è un compositore australiano. Attualmente impartisce lezione nel reparto di tecnologia, composizione e musica presso il Conservatorio dell' Università di Sidney. È anche un appassionato tecnico di field recording; ha intraprendeso spedizioni di registrazione in tutta l'Africa, Alaska, Amazon, Indie occidentali, Nord Europa, Medio Oriente, Australia e il Polo Nord. Le sue opere creative hanno ricevuto più di 25 citazioni di composizione internazionali e nazionali.È autodidatta in elettroacustica e ha studiato al tempo stesso composizione e arti creative in diverse università australiane, tra cui UOW, dove il suo dottorato ha ricevuto menzione speciale. Altre lauree: BMus (Hons ), MMus, e MA(r).}

\biografia{Elisabetta Braga}{Nata a Nardò, si diploma brillantemente in canto presso il Conservatorio di musica Santa Cecilia di Roma nel 2013. Comincia la sua attività concertistica esibendosi in vari teatri e sale da concerto Italia e all'estero, quali la Sala Accademica del Conservatorio Santa Cecilia di Roma, il Teatro Politeama Greco di Lecce, la Tchaikoskj Concert Hall di Mosca. Nel 2012 partecipa a EMUFest per l'esecuzione di "Anabasi" di G.Baggiani, diretta da T. Battista. Debutta nel 2015 a Roma come Mimì ne "La Bohème" di G. Puccini e a Rieti nel 2015 come Gilda in "Rigoletto" di G. Verdi. Si è perfezionata partecipando come allieva effettiva alla masterclass del soprano Sumi Jo tenuta a Roma in aprile. Attualmente sta per conseguire il Diploma Accademico di secondo livello presso il Conservatorio Santa Cecilia di Roma.}

\biografia{Daniele Buccio}{Diplomato in pianoforte presso il Conservatorio “A. Casella” dell’Aquila ed in composizione presso il Conservatorio “G. Verdi” di Torino, è dottore di ricerca in musicologia e beni musicali. Ha seguito corsi di alto perfezionamento in composizione presso l’Accademia Filarmonica di Bologna, l’Accademia “L. Perosi” di Biella e l’Accademia Musicale Chigiana. Si è esibito come solista presso il Teatro Regio di Parma, l’Auditorium Parco della Musica di Roma, la Deptford Town Hall per la Liszt Society di Londra, il Palazzo dei Congressi di Lugano.}

\biografia{Sylvano Bussotti}{Nato a Firenze il 1 ottobre 1931. Inizia lo studio del violino con Margherita Castellani ancora prima di compiere i cinque anni di età. Al Conservatorio "Luigi Cherubini" di Firenze studierà l'armonia e il contrappunto con Roberto Lupi e il pianoforte con Luigi Dallapiccola: studi che interromperà a causa della guerra, senza conseguire alcun titolo di studio. A Parigi, nel periodo che va da 1956 al 1958, frequenta i corsi privati di Max Deutsch, incontra Pierre Boulez e Heinz-Klaus Metzger, che lo condurrà a Darmstadt, dove conosce John Cage. Inizia in Germania, nel 1958, l'attività pubblica, con I'esecuzione delle sue musiche da parte del pianista David Tudor, seguita dalla presentazione a Parigi di brani eseguiti da Cathy Berberian sotto la direzione di Pierre Boulez.}

\biografia{Elisabetta Capurso}{pianista compositrice musicologa ha completato la formazione pianistica al Mozarteum di Salisburgo, gli studi di composizione a Darmstadt. Della sua formazione musicale fanno parte gli studi di Direzione d’orchestra composizione elettronica gli studi umanistici; è laureata con lode in Lettere Filosofia all’Università La Sapienza. Recentemente ha conseguito con lode  la seconda laurea in Musica elettronica al Conservatorio S. Cecilia. Concertista di livello internazionale, ha suonato in sale prestigiose per le maggiori associazioni italiane e straniere. Come compositrice ha scritto molti lavori di musica sinfonica cameristica elettronica, eseguite in teatri di rilievo in Italia e all’estero. Le sue opere sono state registrate da Rai Radio Tre, Radio Vaticana, pubblicate da Zanibon-Peters EdiPan AFM SEDAM. Diverse le commissioni di scrittura ricevute: Après alfa dai Solisti Aquilani Tre cose solamente dal Festival Internazionale Organo di Lecce. Ha un curriculum di Professore di Pianoforte nel Conservatorio G.Rossini di Pesaro, Professore di Pianoforte Semiografia della musica contemporanea Laboratorio contemporaneo nel Conservatorio S. Cecilia di Roma. Ha ricevuto premi e riconoscimenti per gli alti meriti artistici.}

\biografia{Simone Cardini}{Studia composizione con F. Telli, pianoforte con A. Torchiani; partecipa a corsi di perfezionamento e seminari tenuti da S. Sciarrino, M. André, T. Tulev, M. Trojahn, P. Manoury. Le sue composizioni sono state eseguite in Europa e Stati Uniti in mostre eminenti e festival come ArteScienza (2012), Contemporanea (2013), Nuova Consonanza (2013, 2014), Rondò (2014), NYCEMF (2015) da ensemble internazionali come Divertimento Ensemble, PMCE e sono stati premiati in concorsi illustri come AFAM (2013), Valentino Bucchi ed 37 °. (2015), ecc Il suo scritto Musica e Architettura - Estetica e implicazioni sociologiche è stato pubblicato nel libro Musica \& Architettura, Nuova Cultura Ed. (2012). La sua raccolta di lavori sarà pubblicata da Universal Edition (2015).}

\biografia{Antonio Carvallo}{A. Carvallo è nato in Cile nel 1972. Ha studiato contrappunto e armonia con Rodolfo Norambuena. Poi, presso l'Università del Cile, ha ottenuto il Diploma e il Master di Specializzazione in Composizione. Ha insegnato all' Università del Cile dal 2000 al 2002. Dopo di che, si è trasferito a Roma e ha studiato musica elettronica con R. Bianchini e G. Nottoli al Conservatorio "Santa Cecilia" ed ha ottenuto la Laurea di Primo e Secondo Livello. Attualmente insegna presso l'Università del Cile.}

\biografia{Pasquale Citera}{(1981) Ha studiato pianoforte con il M° Gemma D’Alessio, Composizione con i M° Luciano Pelosi e Giovanni Piazza e Musica Elettronica con il M° Giorgio Nottoli. Da anni collabora con diverse compagnie teatrali e case di produzione cinematografica oltre che con scultori e fotografi. Ha composto musiche di scena per spettacoli classici e contemporanei. Tra gli altri: L’Alcesti di Euripide, Lisistrata di Aristofane, Anfitrione di Plauto, la Locandiera di Goldoni, l’Avaro di Molière, Da quale parte del vetro di Silvio Nanni, Il dito sulla bocca di Donatella Ferrara, Certe Notti non accadono mai di Patrizia Masi. Ha scritto colonne sonore per la Nero-Film, è Assistente Musicale in diverse scuole di Roma ed è stato docente di Tecnologie Musicali. Dalla collaborazione con lo scultore Arturo Ianniello sono nate diverse sonorizzazioni di opere visuali raccolte in due esposizioni. È attualmente Compositore e Sound Designer per musiche di scena al Teatro Anfitrione ed all’Anfiteatro della Quercia del Tasso.}

\biografia{Cristiana Colaneri}{Nata a Roma ha conseguito il compimento medio di composizione vecchio ordinamento (sotto la guida del M. Pasquale Lucia); studia Composizione presso il Conservatorio Santa Cecilia di Roma, nella classe del M. Francesco Telli. Deve sostenere la Prova finale del diploma accademico di I livello. Finalista ai concorsi Mea 2010 e Bucchi 2015.}

\biografia{Valerio Cosmai}{Nasce a Roma nel 1983. Studia pianoforte esibendosi più volte come pianista solista nella sala Baldini di Roma e in varie ambasciate, tra cui quella americana ed indonesiana, specializzandosi nel repertorio mozartiano. Consegue nel 2008 la laurea in Lettere presso l'università degli studi "La Sapienza" di Roma. Si diploma in percussioni con il massimo dei voti nel 2014. Come percussionista collabora con l'orchestra del Conservatorio esibendosi anche in importanti festival di musica contemporanea. Dal 2012 lavora come insegnante di educazione musicale nella scuola Pio IX di Roma.}

\biografia{Giovanni Costantini}{(Corigliano d’Otranto - Lecce, 1965) Dal 1995 svolge attività di ricerca presso l'Università di Roma "Tor Vergata", dove è docente di Musica Elettronica. È direttore del Master in SONIC ARTS. Sue composizioni elettroacustiche sono state eseguite in numerosi concerti in Italia e all’estero e incise da Twilight Music (Roma) e IAEF (New York). La sua ricerca musicale è rivolta alla realizzazione della microstruttura e della macrostruttura del suono attraverso l’esplorazione e l’elaborazione in tempo reale di materiale acustico.}

\biografia{Elena D'Alò}{, flautista e ottavinista si laurea cum laude al biennio in Flauto, dopo un brillante diploma, presso il Conservatorio "Santa Cecilia" di Roma, con Deborah Kruzansky. Ha affiancato gli studi musicali con quelli scientifici, laureandosi in Fisica acustica presso "La Sapienza" con Paolo Camiz. Attualmente è iscritta al triennio di Musica Elettronica a Roma. Si esibisce in formazioni cameristiche e orchestrali, in un repertorio che va dal barocco al contemporaneo, per il quale ha suonato a festival come Nuova Consonanza, Atlante Sonoro XX secolo, ArteScienza ed EMUFest. Studia violoncello con Maurizio Massarelli.}

\biografia{Maria Cristina De Amicis}{(Avezzano, 1968) Ha compiuto studi di Composizione, Musica Elettronica, Organo e Composizione Organistica, diplomandosi con il massimo dei voti presso il Conservatorio di Musica “A. Casella” de L'Aquila. Le sue opere sono state eseguite in importanti manifestazioni di musica contemporanea in Italia e all’estero tra cui (Lione, Parigi, Barcellona, Aveiro, Madrid, Budapest, Atene, Salonicco, Berlino, Francoforte, Vienna). Dal 2012 è docente di Musica Elettronica presso il Conservatorio di Musica “A.Casella” dell’Aquila.}

\biografia{Vittoriana De Amicis}{(L’Aquila, 1992) a 15 anni intraprende lo studio del canto lirico presso il Conservatorio A.Casella sotto la guida di Antonella Cesari. Ha seguito numerosi corsi di perfezionamento in Italia e all’estero, nel 2013 viene selezionata dal Mozarteum di Salisburgo per prendere parte all’accademia estiva con Horiana Branisteanu. Sempre nel 2013 è assegnataria di una borsa di studio Erasmus e frequenta la classe di Anton Scharinger all’Universit\"at f\"ur Musik di Vienna. Nel 2014 si è diplomata a L’Aquila con il massimo dei voti e la lode e attualmente studia a Roma con Elizabeth Norberg-Schultz.}

\biografia{Domenico De Simone}{Diplomato in Pianoforte, Jazz, Composizione e Musica Elettronica. Ha conseguito il diploma del corso di perfezionamento di Composizione presso l’Accademia Nazionale di Santa Cecilia e, con il massimo dei voti e la lode, il diploma accademico di II livello in Musica Elettronica. Sue composizioni sono state eseguite in Italia e all’estero (Cina, Lettonia, Canada, Cile, Argentina, Romania, Malta, ecc.) e trasmesse da Radio3.}

\biografia{James Dashow}{Ha avuto commissioni, premi e borse di studio dal Bourges Festival Internazionale di Musica Sperimentale (Premio Magistere), il Guggenheim, Fromm e Fondazioni Koussevitzky, Linz Ars Electronica, la Biennale di Venezia, gli USA National Endowment for the Arts, RAI, l'American Academy and Institute of Arts \& Letters, Prague Musica Nova, etc. Nel 2011 gli è stato conferito del "CEMAT per la Musica" premio in riconoscimento della sua carriera di contributi eccezionali alla musica elettronica.}

\biografia{Gustavo Delgado}{Buenos Aires (1976). Diploma di Secondo Livello specialistico in “Musica Elettronica” presso il Conservatorio di Musica “Santa Cecilia” di Roma sotto la guida del M° Giorgio Nottoli con il massimo dei voti. Laurea in “Composizione di Musica Elettroacustica” presso l’Università Nazionale di Quilmes (Buenos Aires, Argentina). Compositore di musica acousmatica, live electronics e di musica applicata, interessato allo studio delle tecniche di missaggio on the box e sound design.}

\biografia{Dennis Deovides A. Reyes III}{Ha studiato composizione musicale nella sua città nativa Manila, Filippine, prima di trasferirsi negli Stati uniti nel 2006. Attualmente Dennis sta facendo il dottorato in composizione musicale all'università dell'Illinois all'Urbana-Champaign con Scott A. Wyatt. Le sue composizioni trovano ispirazione da una vasta gamma di argomenti, dalla musica Asiatica all'arte moderna, e incorpora anche elementi della tradizione Filippina.}

\biografia{Christian Eloy}{Nato ad Amiens, ha studiato flauto e composizione al Conservatorio Nazionale della regione ed al Conservatorio Nazionale Superiore di Parigi. Prima del suo incontro con Ivo Malec e l’emittente GRM / Groupe de Recherches Musicales di Radio France, è stato flautista in orchestra e direttore di una scuola di musica. Christian Eloy ha fondato l’associazione di compositori Octandre, è a capo del dipartimento di elettroacustica del Conservatoire National de Region a Bordeaux e dei workshop per il GRM a Parigi ed è direttore artistico dello studio di ricerca e creatività SCRIME all’Università di Bordeaux I. Vincitore di numerosi premi, tra cui il premio europeo per poesia e musica " François de Roubaix ", ha composto oltre quaranta brani strumentali, elettroacustici, vocali e didattici. I suoi lavori sono pubblicati dalla Billaudot, Fuzeau, Lemoine, Combre, Notissimo e Jobert e i suoi articoli scientifici dal PUF (Francia), Johnston Ed. (Irlanda), MIT press (USA), Le mensuel littéraire et poétique (Belgio) e Confluences (Francia).}

\biografia{Sara Ferrandino}{si è diplomata in pianoforte nel 2005 presso il Conservatorio di Perugia nella classe del Mº Tanganelli, conseguendo nel 2009, con votazione di 110 e Lode, la Laurea per il Biennio Specialistico. Nel 2012 ha ottenuto il diploma del Corso di Perfezionamento tenuto dal Mº Perticaroli, presso l’Accademia Nazionale di Santa Cecilia in Roma. Ha partecipato a numerosi concorsi nazionali e internazionali ottenendo sempre piazzamenti nelle prime posizioni. Si è esibita in molteplici concerti solistici e cameristici in prestigiose sale in Italia e all’estero. Collabora presso il Conservatorio di Perugia con le classi di corno, tromba, flauto, oboe e violino. È docente di pianoforte principale per i corsi pre-accademici presso il Conservatorio Santa Cecilia in Roma.}

\biografia{Enzo Filippetti}{è professore di Sassofono al Conservatorio “S. Cecilia” di Roma e da più di trent’anni tiene concerti in tutto il mondo. Si è esibito alla Biennale di Venezia, al Mozarteum di Salisburgo, a Roma, Milano, Parigi, Londra, Berlino, Vienna, Madrid, Bruxelles, Buenos Aires, Caracas, Riga, Birmingham, Köln, Lyon, St. Etienne (Francia), Principato di Monaco, Yeosu (Korea), Kawasaki, Adis Abeba, Chisnau, Taormina, Ravello. Ha collaborato con Claude Delangle, Alda Caiello e Bruno Canino e molti tra i più importanti compositori hanno scritto per lui più di cento opere e gli hanno affidato numerose prime esecuzioni. Come solista e con il Quartetto di Sassofoni Accademia ha inciso per Nuova Era, Dynamic, Rai Trade e Cesmel. Ha pubblicato studi per Riverberi Sonori e cura una collana per le edizioni Sconfinarte.}

\biografia{Alessia Forganni}{ (Brescia, 1982) si diploma in pianoforte presso il conservatorio Luca Marenzio, sotto la guida del M° M. Zana, e si laurea al D.A.M.S. - Indirizzo Cinema e Audiovisivi. Dal 2007 vive e insegna a Roma: negli ultimi anni all’attività classica ha affiancato un approccio moderno allo strumento: con il duo pianistico Duel, tra il 2009 al 2015 si è esibita in Europa, Russia, Libano e Sudafrica. Attualmente sta ultimando il triennio di Musica Elettronica presso il conservatorio Santa Cecilia: la sua ricerca compositiva è volta a una personale messa in relazione tra il background classico, l’esplorazione improvvisativa, l’utilizzo della voce e le istanze contemporanee.}

\biografia{FREI}{FREI è un progetto di Paolo Gatti e Francesco Bianco, nato nel 2014. Si sono esibiti al Circolo Dal Verme (Studiolo Laps Showcase), al teatro Tor Bella Monaca (Slaps-pourri.1 anteprima). L'improvvisazione è alla base della poetica di Frei. La performance live è basata su elementi preordinati, i quali, durante lo spettacolo, vengono elaborati e sviluppati. La strumentazione è costituita da due laptop sui quali sono vi sono sistemi digitali programmati degli stessi componenti del duo.}

\biografia{Javier Alejandro Garavaglia}{Compositore/Performer (Viola ed elettronica). Professore associato alla facoltà di CASS, Università di Londra. Le Composizioni e le performance eseguite in molti paesi dell'Europa, Dell'America, e dell'Asia, includono: Acusmatici, audio-visual, lavori per solo/camera/ensemble e orchestra, con o senza l'inclusione di elettronica e media interattivi. Alcune opere elettroacustiche sono disponibili su CD (Germania, USA, Argentina e Danimarca). Settori di ricerca: drammaturgia musicale; automazioni del live Electronics; diffusione speciale di sistemi audio.}

\biografia{Jorge García del Valle Méndez}{(1966) è cresciuto in spagna, dove ha studiato fagotto e composizione. Ora vive a Dresda (Germania) dove ha studiato composizione e musica elettronica. Le sue composizioni hanno avuto prime mondiali in tutto il mondo, commissioni da importanti istituzioni internazionali. Lavori di analisi digitale e "sound processing", applicati alla teoria e alla composizione. Premio di composizione Salvatore Martirano dellUniversità dell'Illinois, e premio di composizione Sächsischer Musikrat.}

\biografia{Paolo Gatti}{Laureato in ingegneria, consegue il master in ingegneria del suono presso l'università di Roma "Tor Vergata". Successivamente si laurea a pieni voti in musica elettronica presso il Conservatorio Santa Cecilia di Roma, sotto la guida di G.Nottoli,M.Lupone,N.Bernardini.Compositore, didatta e ricercatore, suoi lavori sono eseguiti in importanti manifestazioni e festival internazionali. Scrive musiche per spettacoli teatrali e rassegne poetiche. Nel 2015, la sua composizione Poltergeist, risulta fra i brani premiati al termine della finale nazionale del premio delle arti "Claudio Abbado".}

\biografia{Núria Giménez-Comas}{Ha studiato composizione presso la Escola Superior de Musica de Catalunya (ESMUC). Dopo due anni ha continuato la sua formazione presso il Conservatorio di Ginevra, studiando composizione con Luis Naon ed elettroacustica con Michael Jarrell. Ha studiato presso l'Institut de Recherche et Coordination Acoustique / Musique (IRCAM) per due anni, dove ha esplorato diversi tipi di sintesi e il nuovo sistema di spazializzazione in Ambisonics 3D. Núria ha lavorato con musicisti come Harry Sparnaay, Wien trio di Klangforum, Ensemble Contrechamps, Bruxelles Philharmonic, e Diotima Quartetto. È membro fondatore di Ensemble Matka.}

\biografia{Virginia Guidi}{Diplomata in Canto e in Musica Vocale da Camera al Conservatorio S. Cecilia, ivi specializzata con lode con S. Schiavoni con una tesi sperimentale sul rapporto tra interprete e compositore nella musica elettroacustica. Canta in Italia e all’estero (Pechino – National Centre of the Performing Arts; Roma –Accademia Filarmonica, Tecnopolo, GNAM, Macro, MAXXI; Napoli – Arena Flegrea; Catania – Teatro Metropolitan), su reti nazionali (RAI 1, RAI 2, RAI 5, Telepace) e in importanti Festival (EMUFest, ArteScienza). Con Voxnova Italia canta “In the Midst of Things” di Allora \& Calzadilla, musica di G. Colemann, per la Biennale Arte 2015.}

\biografia{Jan Jacob Hofmann}{Diplomato in architettura alla Fachhochschule di Francoforte sul Meno nel 1995. Nello stesso anno viene ammesso al corso di Peter Cook ed Enric Miralles alla Städelschule, Scuola Superiore di Formazione Artistica di Francoforte per un corso post- universitario di architettura e progettazione concettuale. Diplomato nel 1997. Lavora come compositore, architetto e successivamente come fotografo. Dall' estate 2005 è "Associate Researcher" alla "Signal Processing Applications Research Group", Universita di Derby, Inghilterra. È stato nominato al consiglio di amministrazione della Società Tesesca per la Musica Elettroacustica, DEGEM.}

\biografia{Sandro L'Abbate}{, classe 1988. Diplomato in fotografia all'Accademia di Belle Arti di Rimini in Italia. È interessato alla produzione audio- visuale usando sistemi elettronici ed interattivi per osservare fenomeni fisici. Al momento si trova di fronte al mare. Web:http://sandrolabbate.altervista.org}

\biografia{Silvia Lanzalone}{compositrice (Salerno 1970). Diploma di Flauto, Composizione e Musica Elettronica presso i Conservatori di Salerno, L’aquila e Roma. Sue composizioni sono edite da Ars Publica, Taukay e Suvini Zerboni e sono eseguite in festivals nazionali ed internazionali. Dal 1997 collabora con il CRM - Centro Ricerche Musicali di Roma. È Docente di Composizione Musicale Elettroacustica e Coordinatore del Dipartimento di Nuove Tecnologie e Linguaggi Musicali presso il Conservatorio “G. Martucci” di Salerno. (http://www.silvialanzalone.it/)}

\biografia{Jean-Francois Laporte}{Compositore, esecutore ed inventore di strumenti musicali. Attivo sulla scena artistica contemporanea dalla fine degli anni ’90, l’artista canadese ha un approccio creativo ibrido, che unisce assieme sound art, composizione, interpretazione, performance, installazione sonora e arte digitale. Artista piuttosto intuitivo, ha appreso l’arte attraverso sperimentazioni concrete con la materia, basando il suo approccio alla composizione sull’ascolto attivo e sull’attenta osservazione della realtà di ciascun fenomeno. Nel corso degli anni ha dedicato una grande quantità di energie all’invenzione, lo sviluppo e la costruzione di nuovi strumenti musicali. È fondatore e direttore artistico delle produzioni Totem Contemporain di Montreal. http://www.jflaporte.com}

\biografia{Gy\"orgy Ligeti}{Musicista di origine ungherese. Dopo aver preso la cittadinanza austriaca nel 1967, dal 1973 è insegnante di Composizione alla Hochschule für Musik di Amburgo ed è spesso invitato a tenere conferenze e seminari in importanti centri musicali di Europa e degli Stati Uniti. Nell'attività creativa, specialmente a partire dagli anni Settanta, ha continuato a sviluppare l'attitudine a una raffinata e inquieta manipolazione di tutti i parametri del suono, secondo un atteggiamento espressivo che non rifugge da ironiche bizzarrie e da violente accensioni timbriche, spesso legate a una ricerca esplicita di teatralità. Perciò le sue composizioni più recenti tendono ad allontanarsi dalle sottili vibrazioni materiche e dal senso di stupite e illusorie sonorità che caratterizzavano i suoi lavori nati nel clima di Darmstadt.}

\biografia{Jones Margarucci}{Ha studiato Composizione Elettroacustica presso il Conservatorio di Salerno e come exchange student presso la Royal Academy of Music (KMH) a Stoccolma. Sue musiche sono state eseguite in diversi festival in Europa e in Nord America e sono state selezionate per: Redshift Music - Postal Pieces (Vancouver – Canada – 2013); Vox Novus Fifteen Minutes of Fame - Yumi Suehiro (New York City – USA – 2014); Sonorities Festival 2015 (Belfast – North Ireland – 2015); SOUNDkitchen’s Earspace/Frontiers Festival 2015 (Birmingham – UK – 2015); Video Remakes - Call for Tape Music (La Fabbrica del Vedere) (Venice - Italy - 2015)}

\biografia{Marco Marinoni}{ nasce a Monza nel 1974. Nel 2007 si diploma con il massimo dei voti e la lode in Musica Elettronica con Alvise Vidolin. Nel 2009 consegue il Diploma Accademico Sperimentale di Secondo Livello in Live Electronics e Regia del Suono con 110 e Lode e nel 2013 il Diploma Accademico Sperimentale di Secondo Livello in Composizione con 110 e Lode. Dal 1999 è attivo come compositore in ambito contemporaneo. Prix du Trivium nel 29e Concours International de Musique et d'Art Sonore Electroacoustiques - Bourges 2002. Finalista dell'International Gaudeamus Composition Prize 2002 e 2003. Vincitore della Seconda Call per Opere Elettroacustiche indetta dalla Federazione CEMAT. Primo Premio nel Primo Concorso di Composizione per Iperviolino - Genova 2007. Primo Premio nel VIII Concorso Internazionale di Composizione “Città di Udine”. Come musicologo partecipa ai convegni indetti dall'AIMI e dal GATM. È membro del SIMC - Società Italiana Musica Contemporanea. Le partiture dei suoi brani sono edite da ARSPUBLICA EDIZIONI MUSICALI e da TAUKAY. Nel 2015 esce il suo primo romanzo, La Confraternita di Ecate - Cauda Draconis (ed. Nerocromo). È professore di Esecuzione e Interpretazione della Musica Elettroacustica e coordinatore del Dipartimento di Musica Elettronica presso il Conservatorio "G. Verdi" di Como. Vive a Finale Ligure.}

\biografia{Raffaele Marsicano}{ diplomato in trombone nel 2006 al conservatorio di Salerno, continua i suoi studi musicali diplomandosi anche in strumentazione per banda nel 2011 e composizione nel 2015 presso il conservatorio di Milano, dove attualmente frequenta il biennio specialistico di composizione. Considerata la sua duplice natura da trombonista e compositore, incentra le sue ricerche sulla sperimentazione di nuovi suoni degli ottoni applicata alla composizione.}

\biografia{Francesc Martí}{ è un matematico, informatico, compositore, sound and digital artist, nato a Barcellona e vive attualmente nel Regno Unito. Come compositore e digital media artist, i suoi lavori sono stati eseguiti o ha fatto esibizioni in tutto il mondo, inclusi feltival internazionali, eventi e manifestazioni. Attualmente, combina i suoi progetti artisti e tecnologici con l'insegnamento di Audio technology and Image alla open university della California, e Music Technology alla Montfort University di Leichester.}

\biografia{Mario Mary}{Mario MARY dottor in "Estetica, Scienza e Tecnologia delle Arti" (Università di Parigi VIII, Francia), Professore di Composizione di Musica Elettroacustica presso Academia Ranieri III di Monte-Carlo, e Direttore artistico di Monaco Electroacoustique - Incontri Internazionali di Musica Elettroacustica. Ha lavorato come ricercatore presso l'IRCAM e insegnato all'Università Parigi VIII, Ha vinto una ventina di premi in concorsi di composizione. Ha dato nomerosos conferenze e corsi in diversi paesi. http://ipt.univ-paris8.fr/mmary/}

\biografia{Massimiliano Mascaro}{Compositore. Nato a Roma nel 1986. Allievo del M° Michelangelo Lupone e del M° Nicola Bernardini, si è formato presso il Conservatorio “A. Casella” di L'Aquila e successivamente presso il Conservatorio “S. Cecilia” di Roma affrontando gli studi della Composizione elettroacustica e della Composizione classica. La musica elettroacustica è il settore nel quale svolge la sua principale attività musicale.}

\biografia{Massimo Massimi}{ si è formato musicalmente presso il Conservatorio Santa Cecilia di Roma, diplomato in liuto e musica elettronica, ha affrontando lo studio della musica antica e successivamente si è dedicato alla composizione elettroacustica con particolare attenzione all’interazione tra strumento e macchina.}

\biografia{Antonio Mazzotti}{ si è laureato in Ingegneria Elettronica presso il Politecnico di Bari e  specializzato in Signal Processing.  Ha proseguito gli studi accademici presso il Conservatorio di Bari, dove si è laureato con lode in Musica Elettronica, sotto la guida del M° F. Scagliola. I suoi interessi spaziano sulla composizione, assistita da calcolatore, di lavori elettroacustici e audiovisivi. Alcune sue composizioni sono state eseguite in vari festival internazionali come  ‘FIMU Festival 2012’, ‘Silence Festival 2012’, ‘New York City EMFestival 2013/14’, ‘ICMC-SMC 2014’, ‘ file.org.br 2015’, uvm2015.unb.br,  ICMC 2015.}

\biografia{Ursula Meyer-König}{Vive a Zurigo. Dopo una carriera come pediatra, ha intrapreso gli studi base di arte e media al HGKZ di Zurigo e la FH di Aarau, in Svizzera, seguito da un corso di composizione elettroacustica al Hochschule für Musik in Weimar, Germania, con il Prof. R. Minard. Attualmente studia composizione elettroacustica con il Prof. G. Toro- Pérez a ZHdk e ICST, Zurigo, Svizzera.}

\biografia{Enrico Minaglia}{Nasce a Bologna il 26 Dicembre del 1980. Studia composizione al conservatorio dell'Aquila con Alessandro Sbordoni, e al conservatorio di Milano con Fabio Vacchi e  Alessandro Solbiati.  Sempre a L'aquila ha svolto il ruolo di assistente del maestro Michelangelo Lupone nella classe di musica elettronica. In seguito si diploma in direzione d'orchestra. Ha arrangiato e condotto brani per film/tv/teatro musicale, As.Li.Co e Casa Ricordi.}

\biografia{Kenn Mouritzen}{Nato a Copenaghen (Danimarca) nel 1972. Vive e lavora a Vienna (A ) dal 2007. Ha studiato composizione elettroacustica con Germán Toro - Perez e Martin Neukom a ZHdK a Zurigo, Svizzera (fino al 2015). Ha inoltre conseguito un Master in letteratura comparata e Filosofia (2004). Recentemente la sua musica è stata presentata ai Festival UEM, Musicacustica Pechino, Noisefloor Festival, Festival Archipel, RIME, NYCEMF. È stato finanziato dalla Agenzia danese per la Cultura. Premio selezione a Bourges.}

\biografia{Roberto Musanti}{Roberto Musanti, musicista e media artist, autodidatta, diplomato in musica elettronica, insegna laboratorio di informatica e linguaggi di programmazione per la multimedialità. I suoi suoi lavori sono stati presentati, tra gli altri, ai festival “Zeppelin” Barcellona, “U.V.M.” Brasilia, “Kontakte” / “Music in touch” Cagliari, “Musica Viva” Lisbona, “Video Evening Photon Gallery” Lubjiana, ”MediaDepo” Lviv, “Electronicittà” Marseille, “Konsequenz” Napoli, “Decennale CEMAT”, “Saturazioni”, “EMUFest” Roma, “File Festival” Sao Paulo, “Simultan” Timisoara.}

\biografia{Giorgio Nottoli}{Giorgio Nottoli (compositore, nato a Cesena, Italia nel 1945) è stato docente di Musica Elettronica al Conservatorio di Roma “S.Cecilia” sino al 2013. Attualmente è docente di Composizione elettroacustica all’Università di Roma “Tor Vergata”. La maggior parte delle sue opere utilizza mezzi elettronici sia per la sintesi che per l'elaborazione del suono. Il centro della sua ricerca di musicista riguarda il timbro concepito quale parametro principale e "unità costruttiva" delle sue opere attraverso la composizione della microstruttura del suono. Nei suoi lavori per strumenti ed elettronica Giorgio Nottoli punta ad estendere la sonorità degli strumenti acustici mediante complesse elaborazioni del suono. Ha progettato vari sistemi elettronici per la musica utilizzando sia tecnologie analogiche che digitali in collaborazione con varie università e centri di ricerca.}

\biografia{Benjamin O'Brien}{compone, ricerca, ed esegue musica acustica e elettroacustica che si concentra su questioni di trasformazione e l'ascolto delle macchine. Ha conseguito un dottorato in musica presso l'Università della Florida, un MA in composizione musicale al Mills College, e una laurea in Matematica presso l'Università della Virginia. La sua opera è pubblicata dalla Oxford University Press, Taukay Edizioni Musicali, canadese Elettroacustica Comunità, e Seamus. Vive a Marsiglia, in Francia.}

\biografia{João Pedro Oliveira}{ ha completato il suo Dottorato di Ricerca in Musica all' Università Stony Brook di new York. ha ricevuto numerosi premi e riconoscimenti, inclusi tre premi alla Bourges Electroacoustic Music Competition, e il prestigioso Magisterium Prizes nella stessa competizione, il Giga-Hertz Special Award, il primo premio in Metamorphoses competition, etc.. He insegnante presso l'Università Federale di Minais Gerais (Brasile) e all'Università Aveiro (Portogallo).}

\biografia{Daniel Osorio}{Nato a Santiago del Cile. Nel 1996 inizia gli studi di composizione con il Prof. Pablo Aranda e di musica elettroacustica con il Prof. Edgardo Canton e Rolando Cori presso l'Università del Cile. Nel 2005 gli viene concessa una borsa di studio (Beca Presidente de la República - MIDEPLAN) da parte del Governo del Cile e si trasferisce a Saarbrücken / Germania, dove inizia i suoi studi post-laurea in Composizione con il Prof. Theo Brandmüller, il Dr. Prof. Stefan Litwin e Stefan Zintel alla Hochschule für Musik Saar.}

\biografia{Davide Palmentiero}{Nasce a Salerno il 19 Maggio 1993. Sei anni dopo inizia a suonare la chitarra classica, per poi passare alla chitarra elettrica all’età di 13 anni, iniziando a suonare e registrare con varie band e artisti senza distinzioni di genere. A 19 anni inizia ad affacciarsi alla Musica Elettronica e un anno dopo si iscrive al Conservatorio di Napoli; qui mostra particolare interesse per l’improvvisazione radicale, sperimentando soprattutto applicazioni e tecniche riguardanti la chitarra. Costruisce e sviluppa continuamente il proprio strumento, con il quale si esibisce in vari festival, rassegne e altri contesti sia in solo che con con diverse formazioni e diversi artisti, tra i quali Bob Ostertag.}

\biografia{Alessandro Pace}{laureato in Flauto con il M°Carlo Morena con la votazione di 110 e lode presso il Conservatorio di Santa Cecilia di Roma. Prosegue gli studi in flauto, affiancati dagli studi in Composizione tradizionale nello stesso conservatorio. Ha fatto e continua a fare molti concerti nei vari generi. Suona con diversi ensemble: Orchestra Ars Ludi Romana (anche come solista); Broadway Musical Orchestra (es. Festival di Todi); Indivenire Ensemble (repertorio contemporaneo). Ha suonato nell'orchestra nazionale di Panama a Panama City. Ha preso parte al festival Contaminazioni sia come flautista che come compositore. Ha seguito il progetto del M° Antonio Di Pofi sulla musica dei film muti (anche qui sia come flautista che compositore). Suona molta musica da camera in diverse formazione ed è in continua ricerca di nuove esperienze. Per la prima volta si esibirà ad EMUFest.}

\biografia{Carlos D. Perales}{Le sue opere sono state premiate ai concorsi internazionali 'Miniaturas Electroacústicas' - Confluencias (Huelva, 2008), Laboratorio del Espacio LIEM-CDMC (Madrid, 2010), XXII Concorso di Composizione SGAE (Madrid, 2011), Toy Piano Summit Mondiale (Lussemburgo, 2012), Musica Nova (Praga, Repubblica Ceca, 2012), Luigi Russolo (Francia, 2012), Fundación Destellos (Argentina, 2013). Ha conseguito il dottorato presso l'Universidad Politécnica di Valencia. Dal 2014 tiene lezioni di Composizione Elettroacustica al CSMCLM.}

\biografia{Alessandro Pirchio}{Studia presso il Conservatorio di Santa Cecilia con il M° Franz Albanese. Ha partecipato da solo o in formazioni cameristiche a la Rassegna “Musica a Roma per Roma”; il “Sutri Beethoven Festival; Stagione cameristica del Museo della ceramica di Viterbo. Ha suonato per lo spettacolo “La dodicesima notte” (Premio “Le maschere del teatro 2015” per le musiche originali del M° Piovani) in numerosi teatri italiani (Donizzetti di Bergamo, Ponchielli di Cremona, Verdi di Padova sono tra i più importanti). Attualmente ricopre la parte di Primo Flauto nella Banda della Gendarmeria Vaticana e dell’Ass. Nazionale Carabinieri.}

\biografia{Giuseppe Pisano}{Nato nel 1991. Inizia la sua attività musicale come batterista studiando da privatista con il M° Salvatore Tranchini. Da sempre interessato alle sonorità più estreme e rumorose, si concentra inizialmente sul black metal e sull'hardcore, per cambiare poi indirizzo in seguito alla sua permanenza in Norvegia dove si appassiona alla musica elettronica extra-colta ed inizia la militanza nel collettivo techno Stavanger Teknomune, alfieri della cultura rave che getta le sue basi nell'utilizzo di strumenti analogici e del vinile. Dopo due anni decide di proseguire i suoi studi musicali a livello accademico, tornando a Napoli e iscrivendosi al triennio di musica elettronica con il M° Elio Martusciello. Ad oggi è attivo nell' estetica del rumore che egli ricerca negli elementi della vita quotidiana. Attualmente suona la batteria con il gruppo La Bestia Carenne.}

\biografia{Maurizio Pisati}{Nato a Milano nel 1959, è presente con propri lavori in festival d’Europa, Australia, USA, Giappone, America Latina. Sue composizioni sono state premiate in concorsi nazionali e internazionali (tra cui: Bucchi’83; Contilli’83; Rass. B. Brecht’85; Gaudeamus’86; ICONS’86; Petrassi’89), sono pubblicate da Casa-Ricordi, trasmesse da emittenti radiofoniche europee ed extraeuropee, sono incise su CD Ricordi-Fonit Cetra, Edipan, BMG, CavalliRecordsBamberg, Victor, Limen, ArsPublica, SiltaClassics e LArecords, etichetta indipendente da lui fondata nel 1997. ha compiuto gli studi musicali al Conservatorio di Milano, oltre che ai corsi estivi di Darmstadt e all’Accademia di Città di Castello, diplomandosi con il massimo dei voti in Composizione con S.Sciarrino, A. Guarnieri e G.Manzoni, e in seguito anche in Chitarra svolgendo attività concertistica in Europa dal1983 al1989 col gruppo Laboratorio Trio. Al Conservatorio di Bologna insegna di Composizione per la Musica Applicata, Elementi di Composizione per la Didattica, Invenzione \& Interpretazione, e nella stessa sede nel 2014 fonda CRS - Centro di ricerche musicali.}

\biografia{Karen Power}{Ambienti quotidiani e rumori di ogni giorno si trovano al centro della pratica di Karen con un continuo interesse che porta ad offuscare la distinzione tra ciò che la maggior parte di noi chiama musica e il resto dei suoni. Ha trovato ispirazione nel mondo naturale e come noi rispondiamo agli spazi che occupiamo; utilizza continuamente la nostra familiarità inerente con tali suoni e spazi. www.karenpower.ie}

\biografia{Federico Ripanti}{Nato a Roma nel 1987, studia Musica Elettronica presso il Conservatorio "S. Cecilia" di Roma. Nel 2009 si diploma in Fonia e Music Technology presso la Saint Louis Music College. Ha studiato privatamente pianoforte, chitarra elettrica e percussioni africane.}

\biografia{Alessandro Ratoci}{Studi musicali di composizione, pianoforte e musica elettronica presso il Conservatorio di Bologna, Master of Arts in composizione acustica ed elettronica alla HEM di Ginevra, perfezionamento al cursus IRCAM 2014-2015 in Parigi. Compositore, interprete di musica elettronica e didatta, insegna alla HEMU di Losanna e al Conservatorio G.B.Martini di Bologna. Le sue musiche sono state eseguite dal Ictus Ensemble Trio, Modern Ensemble Academy, Orchestra de la HEM di Ginevra, Orchestra di Radio France}

\biografia{Matteo Rossi}{, percussionista, si diploma con il massimo dei voti presso il Conservatorio “S.Cecilia” di Roma con Gianluca Ruggeri. Segue il corso di perfezionamento presso l’Accademia Musicale Chigiana con Antonio Caggiano, e come membro del Chigiana Percussion Ensemble, si esibisce al CHIGIANA INTERNATIONAL FESTIVAL, RAVELLO FESTIVAL e MAXXI di Roma. Collabora con formazioni orchestrali e cameristiche quali PMCE, InDivenire Ensemble ed ensemble di percussioni quali Ars Ludi, Blow-Up Roma Percussion, Aere Silente con cui si esibisce in un repertorio percussionistico moderno e contemporaneo in diversi eventi quali Le esperienze del minimalismo, Le Forme del Suono, ArteScienza, EMUFest.}

\biografia{Demian Rudel Rey}{(Argentina - October 24, 1987) Compositore. Si è diplomato al conservatorio di Musica "Piazzolla" e presso L'università Nazionale delle Arti. è stato premiato al TRINAC, Trime, FINM, BIENALbahìaBlanca, SADAIC,CONDIT, ecc. È stato selezionato per il MUSLAB 2014 (Messsico), Interensemble2015 (Italia) e SIRGA Festival 2015 (Spagna). Ha partecipato come "LiveSamplingPlayer" a Les Chants de l'Amor di Grisey, Usina del arte e in Das Mädchen mit den Schwefelhölzern di Lachenmann al Teatro Colón.}

\biografia{Gianluca Ruggeri}{Performer, direttore, autore e didatta. Diplomato in Strumenti a percussione e Direzione di Coro. Dopo gli esordi come percussionista nelle orchestre lirico-sinfoniche di Roma, ha incentrato il suo lavoro sul repertorio solistico e cameristico contemporaneo concentrandosi sulla ricerca elettro-acustica (K. Stockhausen, B. Truax, Y. Taira, M. Lupone) e sulla “performance” (J. Cage, G. Battistelli, L. Hiller, L. Berio) Nel 1987 ha fondato con Antonio Caggiano, ARS LUDI, un ensemble modulare con cui si è esibito in tutto il mondo. In veste di direttore ha diretto opere di F. Evangelisti, K. Stockhausen, M. Betta, C. Crivelli, M. Fischione, L. Cinque, C. Cardew, L. Berio, S. Reich, B. Sorensen, De Machaut e I. Stravinsky. Attualmente si dedica in vari modi all’approfondimento dell’opera di S. Reich. È docente di Strumenti a Percussione presso il Conservatorio di Musica “S.Cecilia” di Roma.}

\biografia{Dimitrios Savva}{Nato a Cipro, 1987. È diplomato con lode in Composizione presso la Ionian University di Corfu e laureato (con lode) in Composizione Elettroacustica presso l'Università di Manchester. Attualmente è dottorando alla Scheffield University sotto la supervisione di Adrian Moore. Le sue composizioni sono state suonate in Grecia, Cipro, Regno Unito, Germania, Belgio, Francia, Italia, Portogallo, Brazile e Usa.}

\biografia{Dominique Schafer}{Nato in Svizzera a Friburgo, è un compositore la cui espressività musicale spazia da lavori per strumenti acustici a lavori multimediali elettroacustici. Le sue composizioni sono state eseguite dall’Arditti String Quartet, Dinosaur Annex Ensemble, Ensemble Fa, Boston Modern Orchestra Project (BMOP), Talea Ensemble, Frances Marie Uitti, Alarm will Sound, California EAR Unit, nel Musica Nova Finland Festival e nel June di Buffalo, tra i tanti.}

\biografia{Claudia Jane Scroccaro}{Laureata in musicologia all’Università “Tor Vergata” di Roma, ha studiato direzione d’orchestra con Piero Bellugi. Durante il dottorato in Teoria e Analisi Musicale alla McGill University di Montreal, decide di tornare in Italia per studiare composizione con Luigi Verdi al conservatorio di musica “S. Cecilia”. I suoi lavori sono stati eseguiti presso la British School of Rome, la M.K. Ciurlionis School of Arts in Lituania, l’Auditorium “Ennio Morricone” di Tor Vergata e il London College of Music. Ha composto le musiche per il film-documentario “I’m coming home” premiato al Sidney Film Festival e all’International Filmmaker Festival of World Cinema di Milano; è stata Composer in Residence “DAR 2015” per la Lithuanian Composer’s Union.}

\biografia{Giuseppe Silvi}{[Tivoli (RM) - 1981] Studia Saxofono e Musica Elettronica presso il Conservatorio "S. Cecilia" di Roma. Si diploma in Musica Elettronica nel 2013 con Giorgio Nottoli. Sue musiche vengono eseguite in diversi Festival di musica elettroacustica, nel 2013 il brano \textit{A. SAX.} (per sax e live electronics) viene eseguito al festival Internazionale "Monaco Electroacoustique" ed è finalista al Concorso Franco Evangelisti con il brano PS: \textit{Song \#04} (per mezzosoprano, percussioni ed elettronica). È Tecnico del Suono specializzato in registrazioni surround, incide per edizioni Tactus, Naxos, Brilliant Classic e Sony.}

\biografia{Arturo Tallini}{ ha suonato in tutta Europa, negli Stati  Uniti, in Egitto, Algeria e Tunisia. È docente al Conservatorio di Santa Cecilia in Roma e tiene regolarmente masterclass nei conservatori e italiani e università straniere. Considerato all'unanimità un riferimento per il repertorio contemporaneo, collabora con artisti di fama internazionale, Michiko Hirayama, il gruppo di musica contemporanea \textit{Modus Novus} di Madrid, il Coro dell’Accademia Nazionale di Santa il flautista Carlo Morena. È coordinatore del Master Annuale di II Livello in Interpretazione della Musica Contemporanea del Conservatorio di Santa Cecilia in cui è anche docente di chitarra.}

\biografia{Anna Terzaroli}{Laureata in Musica Elettronica presso il Conservatorio Santa Cecilia di Roma, attualmente sta concludendo il Biennio specialistico presso lo stesso Conservatorio. Come compositrice si dedica alla musica contemporanea acustica ed elettroacustica, suoi lavori sono selezionati e presentati in vari concerti e festival, in Italia e all'estero. Dal 2009 collabora a EMUFest, è membro del Consiglio Direttivo dell'AIMI (Associazione Informatica Musicale Italiana).}

\biografia{Gianni Trovalusci}{Diplomato in flauto al Conservatorio S. Cecilia, ha approfondito il repertorio contemporaneo con P.-Y. Artaud a Parigi e la prassi esecutiva della musica antica con J. Christensen e O. Peter presso la Schola Cantorum di Basilea. Dagli anni settanta è attivo nel campo della musica contemporanea, antica, nel teatro musicale e performance d'avanguardia; ha lavorato con importanti artisti e si è esibito nei più importanti festival europei e nazionali. È Segretario Artistico della Federazione Cemat.}

\biografia{Giovanni Ubertini}{A seguito del diploma in pianoforte, conseguito brillantemente (9.25/10) presso il Conservatorio "O. Respighi" di Latina, segue corsi di perfezionamento con lo statunitense Charles Rosen e con il M° Donella D'Alessio. Sotto la guida del M° Luigi Sacco, nel 2009 si diploma (10 cum laude) in organo e comp. organistica presso il Conservatorio di Latina e nel maggio 2014, sotto la guida del M° Alessandro Licata, conclude il biennio in organo e comp. organistica (110 cum laude e menzione d’onore) presso il Conservatorio “S. Cecilia”. Attualmente è triennalista nel corso di Direzione di coro e comp. corale nella classe del M° Mauro Bacherini presso il Conservatorio di Latina. Ai titoli musicali affianca la laurea in Giurisprudenza.}

\biografia{Kyle Vanderburg}{Kyle Vanderburg compone ecletticamente musica polistilistica, alimentata da unità ritmica e infatuazione melodica. Oltre ad essere compositore, è anche attivo programmatore di computer, scrive codici per performance interattive, servizi di pubblica utilità relativi alla automazione del workflow, e di controllori inusuali.}

\biografia{Daniele Vantaggio}{ (Roma, 1987) è un produttore, sound designer e sound engineer. Si diploma alla St.Louis di Roma nel 2006 con il M° Luca Spagnoletti, in sintesi del suono e HD Recording e ottiene attestato di Sound Engineering con Vittorio Nocenzi e Lorenzo Pozzi. Dal 2009 studia presso il dipartimento di Musica Elettronica del Conservatorio Santa Cecilia. Da sempre il suo interesse è rivolto al suono della scena underground. Si esibisce in Europa e Sud America. Attualmente si occupa di produzioni discografiche, post-produzione, produzioni cinematografiche e teatrali, tiene corsi di formazione e conduce un programma radiofonico nazionale.}

\biografia{Massimo Varchione}{Nasce in Svizzera nel 1979. Diplomato in Composizione presso il Conservatorio Nicola Sala con il M° Luigi Turaccio. Studia Musica Elettronica presso il Conservatorio San Pietro a Majella di Napoli prima con il M° Agostino Di Scipio ed ora con il M° Elio Martusciello. Ha composto brani strumentali, elettroacustici e realizzato installazioni. Ha iniziato di recente un nuovo percorso dedicato all'improvvisazione radicale con il mezzo elettroacustico e con gli strumenti. In duo con il clarinettista Agostino Napolitano, nel 2014 è stato selezionato dal centro Tempo Reale di Firenze per partecipare al loro festival dedicato all'elettronica.}

\biografia{Clemens Von Reusner}{è un compositore e soundartist residente in Germania, il cui lavoro si concentra sulla musica acusmatica. Le sue composizioni hanno ottenuto trasmissioni ed esecuzioni a livello internazionale.}

\biografia{Benjamin D. Whiting}{si diploma in Composizione e prende un master in Teoria musicale e composizione presso la Florida State University, ed è attualmente dottorando presso la University of Illinois in Urbana-Champaign. Compone sia musica acustica che elettroacustica, e i suoi lavori sono stati eseguiti negli Stati Uniti e all'estero, ed editi dall'etichetta Experimental Music Studios della University of Illinois.}

\vspace{1mm}

\biografia{Giuseppe Zampetti}{Compositore. Nato a Roma nel 1992. Allievo del M° Francesco Telli, studente di Composizione indirizzo Contemporaneo presso il Conservatorio “S. Cecilia” di Roma.}

\biografia{Francesco Ziello}{Si è laureato nel 2012 in Musica Elettronica al Conservatorio di Roma “S.Cecilia”, dove parallelamente ha studiato Pianoforte e Composizione. Polistrumentista attivo in diverse formazioni, partecipa come esecutore e performer in vari festivaxfl legati al mondo della musica contemporanea. Come compositore ha collaborato con l’Accademia Nazionale di Danza per uno spettacolo presentato al Centro di Ricerche Musicali nel Luglio 2015. Attualmente iscritto al Biennio di Musica Elettronica sotto la guida di Michelangelo Lupone.}


\end{multicols}

\clearpage

\section*{ }

%\subsection*{\textsf{}\\}

\hyphenation{Michele Andreotti Guido Capotosto Federico Coderoni Gianmarco Costa
Marco De Martino Simone Giudice Matteo Ilardo Leonardo Mammozzetti
Danilo Marro Massimiliano Mascaro Alessandro Pacetta
Federico Paganelli Ivo Papadopoulos Ivano Pecorini Susanna Rimondotto Federico Ripanti}

\begin{center}

{\fontsize{30}{30} \svolk{\emph{Organizzazione}}}
\medskip

{\fontsize{12}{12} \textsf{Dipartimento di Musica Elettronica}}\\
\vspace{.5cm}

\textbf{\textit{Comitato artistico EMUfest}}\\
Nicola Bernardini, Michelangelo Lupone, Alfredo Santoloci, Franco Sbacco
\medskip

\textbf{\textit{Comitato organizzatore}}\\
Francesco Bianco, Elena D'Alò, Paolo Gatti, Marco Giordano, Virginia Guidi, Luana Lunetta, Massimo Massimi, 	Luigi Pizzaleo, Federico Scalas, Giuseppe Silvi, Anna Terzaroli, Francesco Ziello
\medskip

\textbf{\textit{Supervisione tecnica}}\\
Federico Scalas
\medskip

\textbf{\textit{Responsabili di palco}}\\
Luana Lunetta, Massimo Massimi
\medskip

\textbf{\textit{Regia del suono}}\\
Giuseppe Silvi
\medskip

\textbf{\textit{Regia Conferenze}}\\
Anna Terzaroli
\medskip

\textbf{\textit{Tecnico di registrazione}}\\
Federico Coderoni
\medskip

\textbf{\textit{Tecnici luci}}\\
Massimiliano Mascaro, Simone Giudice
\medskip

\textbf{\textit{Ufficio Stampa}}\\
Francesco Bianco, Paolo Gatti
\medskip

\textbf{\textit{Staff esteso}}\\
Michele Andreotti, Guido Capotosto, Federico Coderoni, Gianmarco Costa,
Marco~De~Martino, Simone Giudice, Matteo Ilardo, Leonardo Mammozzetti,
Danilo Marro, Massimiliano Mascaro, Alessandro Pacetta,
Federico Paganelli, Ivo Papadopoulos, Ivano~Pecorini, Susanna Rimondotto, Federico Ripanti
\end{center}


%\begin{figure}[!h]
%\centering
%\includegraphics[width=6.4cm]{loghi.jpg}
%%\caption{}
%%\label{fig2_42}
%\end{figure}

\end{document}
