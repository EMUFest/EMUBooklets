% !TEX encoding = UTF-8 Unicode
% !TEX TS-program = XeLaTex


\documentclass[8pt, twoside, a5paper]{extreport}
\usepackage{lipsum} % ----------------------------------------------------------- cancellabile
\usepackage{blindtext}	% dummy text

\usepackage{latexsym}
\usepackage[polutonikogreek, italian, english]{babel}
\usepackage[T1]{fontenc}
%\usepackage[utf8]{inputenc}

\usepackage[margin=1cm]{geometry}

\usepackage{multicol}
\usepackage[hang, small,labelfont=bf,up,textfont=it,up]{caption} % Custom captions under/above floats in tables or figures
\usepackage{booktabs} % Horizontal rules in tables
\usepackage{float} % they need to be placed in specific locations with the [H] (e.g. \begin{table}[H])

\usepackage{graphicx}
\usepackage{amssymb}
\usepackage{epstopdf}

\usepackage{hyperref} % For hyperlinks in the PDF

\usepackage{titlesec} % Allows customization of titles
\renewcommand\thesection{\Roman{section}} % Roman numerals for the sections
\renewcommand\thesubsection{\Roman{subsection}} % Roman numerals for subsections
\titleformat{\section}[block]{\large\scshape\centering}{\thesection.}{1em}{} % Change the look of the section titles
\titleformat{\subsection}[block]{\large}{\thesubsection.}{1em}{} % Change the look of the section titles

% \usepackage{fancyhdr} % Headers and footers
% \pagestyle{fancy} % All pages have headers and footers
% \fancyhead{} % Blank out the default header
% \fancyfoot{} % Blank out the default footer
% \fancyhead{}%[C]{Sei volte EMUfest $\bullet$ Novembre 2013 }%$\bullet$ Vol. XXI, No. 1} % Custom header text
% \fancyfoot[RO,LE]{\thepage} % Custom footer text

\renewcommand{\thefootnote}{\textasteriskcentered}



\usepackage{mwe,tikz}
\usepackage{tcolorbox}

\usepackage{color}

\usepackage{fontspec,xltxtra,xunicode}
\defaultfontfeatures{Mapping=tex-text}
\setromanfont[Mapping=tex-text]{Source Sans Pro}
\setsansfont[Scale=MatchLowercase,Mapping=tex-text]{Source Sans Pro}
\setmonofont[]{Source Code Pro}

\newfontfamily{\svolk}{Volkhov}

\linespread{1.2}

\usepackage{lipsum}

\definecolor{supercolor}{RGB}{3,39,36}

%\topmargin -2.66in

\newtcbox{\mybox}{nobeforeafter,colframe=supercolor,colback=supercolor,boxrule=0.5pt,arc=8pt,
  boxsep=0pt,left=2pt,right=2pt,top=6pt,bottom=2pt,tcbox raise base}

\input{comandi2015.tex}

%----------------------------------------------------------------------------------------
%	TITLE SECTION
%----------------------------------------------------------------------------------------

\title{
	\svolk{CONSERVATORIO DI MUSICA S. CECILIA} \\
	%\vspace{-15mm}
	\fontsize{50}{50}
	\svolk{
		\emph{
			EMUFest 2015
			}
		}
	} % Article title

\author{
	\textsc{Conservatorio di Musica Santa Cecilia} \\
	5 -- 10 ottobre 2015 \\
	\textsc{Università di Roma Tor Vergata} \\
	13 e 14 ottobre 2015 \\
	Roma
}

\vfill

\date{}

%\dedica{.2\textwidth}{\small A Monica:\\
%per tutto ciò che mi hai insegnato\\
%e per tutto ciò che ancora avresti dovuto insegnarmi.}

%----------------------------------------------------------------------------------------

\begin{document}
\pagestyle{empty}
\maketitle 


%----------------------------------------------------------------------------------------

\section*{5 ottobre 2015 -- ore 17:00}

\subsection*{{\small Concerto Acusmatico in Cupola Ambisonic} \\
	\textsf{Aula Bianchini}}

{\fontsize{30}{30} \svolk{\emph{ATTO I}}}

\subsection*{\textsf{regia del suono di Francesco Ziello}}

\bigskip

\begin{multicols}{2}

% !TEX encoding = UTF-8 Unicode
% !TEX TS-program = XeLaTex
% !TEX root = EMU2015_booklet.tex

\acusmatici{Ursula Meyer-K\"onig}
{Allears}{2012-13} %8' 

\acusmatici{Benjamin O'Brien}
{Along the eaves}{2012-13} %8'20'' 

\acusmatici{Dennis Deovides A. Reyes III}
{Bolgia}{2014} %7'31''

\acusmatici{Dimitrios Savva}
{Balloon Theories}{2012-13} %14'30'' 

\acusmatici{Jones Margarucci}
{Inhabitated PlacesPart II}{2012-13} %5'52"


%\descrizione{Allears}{The inspiration for this work originally came from a series of intensive discussions with people who are deaf or have hearing impairments. We talked about the pros and cons of technical apparatuses such as hearing aids or cochlea implants, the different ethical and emotional responses people have to them, and the identity issues they raise. Wearing hearing aids also changes how sounds are perceived, sometimes causing interference, distortions, diminished spatial perception and noise overflow.}

\descrizione{Allears}{Originariamente l'ispirazione per questo lavoro  proveniva da una serie di intense discussioni con persone non udenti o che hanno problemi d'udito. Abbiamo parlato dei pro e dei contro di apparati tecnici, come apparecchi acustici o impianti cocleari, le diverse risposte etiche ed emotive che le persone sentono, e i problemi d'identità che sollevano. 
Indossare apparecchi acustici cambia anche come i suoni vengono percepiti, a volte causando interferenze, distorsioni,percezione spaziale ridotta e troppo pieno di rumore.}


%\descrizione{Along the Eaves}{takes its name from the following line in Franz Kafka’s “A Crossbreed” “On the moonlight nights its favorite promenade is along the eaves.” To compose the work, I developed custom software and used these programs in different ways to process and sequence my source materials, which, in this case, included audio recordings of water, babies, and string instruments. My interest is to create sonic coincidences that suggest relationships between sounds and the illusions they foster.}

\descrizione{Along the Eaves}{ prende il nome dalla riga che di Franz Kafka "Incrocio". "Sulle notti di luna la sua passeggiata preferita è lungo la grondaia" Per comporre l'opera, ho sviluppato software personalizzato e utilizzato questi programmi in modi diversi per elaborare e sequenziare i miei materiali di base, che, in questo caso, include registrazioni audio d’acqua, bambini, e di strumenti a corda. Il mio interesse è quello di creare coincidenze sonore che suggeriscono i rapporti tra i suoni e le illusioni che promuovono.}


%\descrizione{Bolgia}{Bolgia is an Italian word that means pocket or trench.  This term has been used by Dante Alighieri in his notable literary work Inferno.  Bolgia is a stereo fixed media electroacoustic composition, which depicts Alighieri’s journey to the eighth circle of hell, and his experiences to its horrific environment.  The musical gestures and sonic events of the piece evoke the different sounds and emotions of hell.}

\descrizione{Bolgia}{è una parola italiana che significa "fossa" e "luogo chiassoso in cui regna la confusione". Questo termine è stato usato da Dante Alighieri nel suo noto lavoro letterario "Inferno". Bolgia è un brano stereofonico fisso per composizioni elettroacustiche, che illustra il viaggio di Alighieri nell'ottavo girone dell'inferno, e la sua esperienza in questo posto terribile. I gesti musicali e l'evento sonoro del pezzo evocano i diversi suoni e le diverse emozioni dell'inferno.}


%\descrizione{Balloon Theories}{«I was always enjoying squeezing balloons, pressing them with my fingers until they pop… It has not been up until now that I realized why…»}

\descrizione{Balloon Theories}{«Ho sempre trovato divertente strizzare palloncini, premerli con le dita fino allo scoppio... Non mi è mai interessato fino a quando non ho capito perché...»}


%\descrizione{Inhabitated Places Part II}{Inhabitated Places part II is based on the concept of algorithmic composition. Although the general shape of this piece has been determined in a conventional way, every sound that one can hear are selected in real time by different algorithms written in SuperCollider. These algorithms choose randomly audio files from different folders and play them at different speeds and in different moments. It is as if we had placed several different objects in several boxes (that represent our shape), but every time we open one of these boxes the objects placed inside are positioned differently from how we had left them previously. The piece was also rendered in B-format Ambisonic.}

\descrizione{Inhabitated Places part II}{ è una composizione elettroacustica basata sul concetto di musica algoritmica. Sebbene la forma generale del brano sia stata determinata apriori e in modo convenzionale, tutti i suoni che ascoltiamo vengono scelti in tempo reale da vari algoritmi scritti in SuperCollider. Questi algoritmi selezionano in modo pseudocasuale dei samples da diverse cartelle e li riproducono a velocità diverse e in diversi momenti.  
È come se avessimo sistemato in una scatola (che in questo caso rappresenta la struttura dell’opera) degli oggetti in un dato ordine, ma ogni qual volta apriamo la scatola li troviamo disposti in modo differente da come li avevamo lasciati.}


\end{multicols}

\clearpage
%----------------------------------------------------------------------------------------

\section*{5 ottobre 2015 -- ore 20:30}

\subsection*{{\small Concerto Acusmatico \& Live Electronics\footnote{ Il concerto verrà trasmesso in diretta streaming da Radio Cemat}} \\
	\textsf{Sala Accademica}}

{\fontsize{30}{30} \svolk{\emph{ATTO II}}}

\subsection*{\textsf{Regia del suono di Federico Paganelli}}

\bigskip

\begin{multicols}{2}

%% !TEX encoding = UTF-8 Unicode
% !TEX TS-program = XeLaTex
% !TEX root = EMU2015_booklet.tex

\livel{Simone Cardini}
{Potlach}{2014 - vincitore Premio Bucchi} %9'30''
{per flauto basso ed elettronica su supporto}
{flauto basso}{Gianni Trovalusci}

\livel{Dominique Schafer}
{Cendre}{2008} %10'00''
{per flauto basso e live electronics}
{flauto basso}{Gianni Trovalusci}

\livel{Maria Cristina De Amicis}
{[re-spì-ro]}{2015 - prima esecuzione assoluta} %8'00''
{per flauti e supporto digitale}
{flauti}{Gianni Trovalusci}

\acusmatico{Christian Eloy}
{La cicatrice d'Ulysse}{new version 2015} %13'00''
{acusmatico}

\livel{Giorgio Nottoli}
{7 Isole}{2015 - prima esecuzione assoluta}%15'
{per flauto, percussioni e live electronics}
{flauto}{Gianni Trovalusci}
percussioni -- \textsc{Gianluca Ruggeri}
\\



%\descrizione{Potlach}{was a ritual ceremony performed among certain Native American tribes, during which the members of the community exchanged and squandered gifts, contributing to the cohesiveness of the social concord, whithin a totally inverted logic with respect to our free economy, following a reciprocity principle. The formal juxtaposition of the piece, disperses at once those elements that will constitute later on the concealed foundation that penetratrates the work.}

\descrizione{Potlach}{Il Potlach era una cerimonia rituale d’alcuni popoli di Nativi Americani, durante la quale i membri della tribù scambiavano e dilapidavano doni, in una logica completamente invertita rispetto alla nostra economia. La giustapposizione del brano \textit{dilapida} da subito quegli elementi che costituiranno il substrato che pervade l'opera. È la simbolizzazione dell'immagine che ne permette un'esegesi palesata; l'interprete realizza sé attraverso la sublimazione di gesti e momenti performativi, rappresentando la fenomenologia degli stessi. Da qui la necessità di superare il momento poietico ed estesico \textit{donando} la performance per un recupero del sistema di relazioni come tale.}

%\descrizione{Cendre}{draws on the idea from the impermanence of materiality. The title of the piece gives further meaning referring to a fragile and delicate state, but also the potential of ashes as a fertilizer for something new. The bass flute being suspended within the electronics extends space and timbre, at times detaching itself within its own identity, but repeatedly gets reabsorbed into space. The piece was written for and premiered by Mario Caroli during his Fromm Residency at Harvard University.}

\descrizione{Cendre}{prende ispirazione dall’idea della precarietà di ciò che è materiale. Il titolo del brano acquista ulteriore significato riferendosi ad uno stato di fragilità e delicatezza, ma anche alle potenzialità fertilizzanti della cenere per qualcosa di nuovo. Grazie alla sospensione del flauto basso attraverso l’elettronica, il suo spazio ed il suo timbro vengono estesi, a tratti allontanandosi dalla propria identità per poi essere riassorbito all’interno dello spazio. Il brano è stato composto ed eseguito in prima assoluta da Mario Caroli durante la Fromm Residency presso la Harvard University.}

%\descrizione{[re-spì-ro]}{Re-spi-ro (Breath) in this work, is the alternation of air movements which could produce real rhythmic and tonal structures. The sound of each breath adds and interacts itself with the previous one, drawing a music form which contracts and expands itself as well. Through the rhythmic articulation, inhalation and exhalation, the breath transforms and builds itself, removes and contradicts what has been just expressed. The digital support creates a constructive dialectic, supporting gently the concrete material. Breath transformation is described by the conduct of three different changing elements: localization, movement (direction and speed) and space size. In this way , the auditive perception of the space is transposed to the visual perception level. In this work I tried to realize the interaction between these changing elements on the acoustic level and associating with tones and pitches the breath expressive gestures of the performer.}

\descrizione{[re-spì-ro]}{Il re-spi-ro, in questa opera, è inteso come l’alternanza dei movimenti dell’aria in grado di generare vere e proprie strutture ritmiche e timbriche. Il suono di ogni respiro si aggiunge e interagisce con i precedenti, disegnando una forma musicale che si contrae e si espande.  Attraverso l’articolazione ritmica, l’inspirazione e l’espirazione, il respiro si trasforma e costruisce, dissolve o contraddice ciò che è stato appena espresso. Il supporto digitale crea una dialettica costruttiva sostenendo delicatamente il materiale concreto. La trasformazione del respiro è descritta attraverso il comportamento di tre variabili: localizzazione, movimento (direzione, velocità) e dimensione dell’ambiente. In questo modo la percezione uditiva dello spazio viene trasposta nel dominio della percezione visiva. In questo brano ho cercato di realizzare l’interazione tra queste variabili nel dominio acustico associando con timbri e altezze i gesti espressivi del respiro dell’esecutore.}

%\descrizione{La cicatrice d'Ulysse}{This title, borrowed from Erich Auerbach, German writer and critic deceased in 1957, immediately sets the scene on a plane where realism is depicted in the aesthetics of Western music. Electroacoustic music has the ability, using “concrete” sounds (with all the ambiguity implied by the word), of giving us an immediate sensation of reality, which we can all situate in relation to our references and private objects of reference; these are the profound notions of sublimitas and humilitas, which merge and unite in musical expression. Every listener will be able to decipher his or her own images of a collective mental universe from the essence of this kind of artistic creation, just like the scar by which Ulysses was recognised by his former servant woman.}

\descrizione{La cicatrice d'Ulysse}{Il titolo, preso in prestito da Erich Auerbach, scrittore e critico tedesco morto nel 1957, ambienta immediatamente la scena all’interno di un aereo, tratteggiandola con un realismo tipico dell’estetica musicale occidentale. Attraverso l’uso di suoni concreti (con tutte le ambiguità insite nella parola stessa), la musica elettroacustica possiede l’abilità di restituirci una percezione immediata della realtà, che è possibile collocare in relazione ai propri riferimenti e agli oggetti segreti cui intende riferirsi; questi sono i significati profondi di sublimitas e humilitas, che si fondono e riuniscono nell’espressione musicale. Dall’essenza di questo tipo di creazione artistica ogni ascoltatore è in grado di decodificare la propria immagine di un universo mentale collettivo, proprio come la cicatrice che permise alla serva di riconoscere Ulisse.}

%\descrizione{7 Isole}{I think the idea of the island is a fascinating metaphor, because I realize to think about many important things  as they are formed, in fact, by islands, which each contain a different world and a different feel, but together form a unitary context. In "7 Isole", each "island" is characterized by a particular combination of movement, pitches and colors of the sound. It consists of seven small pieces separated from one another, that can be performed in any order, however, strongly linked by way of forming the sound material and by a unitary constructive thinking. The instruments are used both with extended techniques that traditional, allowing different nuances both for the continuous sound as well as impulsive, both for what determined pitch and for what similar to the noise. The electronics supports and extends the sound of the instruments and, where necessary, becomes the instrument itself, completing the construction of the sound field. The dynamic distribution  in the listening space is obtained according to the principles of the method Ambisonic. Some of the processing and sound synthesis used have been developed by the author.}

\descrizione{7 Isole}{Quella dell'isola è per me una metafora affascinante, in quanto mi accorgo di pensare a molte cose importanti come fossero costituite, appunto, da isole, che contengono ciascuna un mondo diverso e quindi un diverso sentire, ma, insieme, costituiscono un contesto unitario. In 7Isole, ciascuna "isola" è caratterizzata da una particolare combinazione di movimento, altezze e colori del suono. Si tratta di sette piccoli pezzi fra loro separati, che si possono eseguire in un qualsiasi ordine, tuttavia fortemente legati dal modo di formare il materiale sonoro e da un pensiero costruttivo unitario. Gli strumenti sono utilizzati sia con tecniche estese che tradizionali, consentendo diverse sfumature sia per il suono continuo che per quello impulsivo, sia per quello ad altezza determinata e per quello simile al rumore. L'elettronica sostiene ed  estende il suono degli strumenti e, dove necessario, diviene strumento essa stessa, completando la costruzione del campo sonoro. La distribuzione dinamica nello spazio d'ascolto è ottenuta secondo i principi  del metodo Ambisonic. Alcune delle elaborazioni e sintesi del suono utilizzate sono state sviluppate dall'autore.}











\end{multicols}

\clearpage

%----------------------------------------------------------------------------------------

\section*{6 ottobre 2015 -- ore 17:00}

\subsection*{{\small Concerto Acusmatico in High Order Ambisonic Domain} \\
	\textsf{Aula Bianchini}}

{\fontsize{30}{30} \svolk{\emph{ATTO III}}}

\subsection*{\textsf{regia del suono di Federico Paganelli}}

\bigskip

\begin{multicols}{2}

%% !TEX encoding = UTF-8 Unicode
% !TEX TS-program = XeLaTex
% !TEX root = EMU2015_booklet.tex

\acusmatici{Ursula Meyer-K\"onig}
{Allears}{2012-13} %8' 

\acusmatici{Benjamin O'Brien}
{Along the eaves}{2012-13} %8'20'' 

\acusmatici{Dennis Deovides A. Reyes III}
{Bolgia}{2014} %7'31''

\acusmatici{Dimitrios Savva}
{Balloon Theories}{2012-13} %14'30'' 

\acusmatici{Jones Margarucci}
{Inhabitated PlacesPart II}{2012-13} %5'52"


%\descrizione{Allears}{The inspiration for this work originally came from a series of intensive discussions with people who are deaf or have hearing impairments. We talked about the pros and cons of technical apparatuses such as hearing aids or cochlea implants, the different ethical and emotional responses people have to them, and the identity issues they raise. Wearing hearing aids also changes how sounds are perceived, sometimes causing interference, distortions, diminished spatial perception and noise overflow.}

\descrizione{Allears}{Originariamente l'ispirazione per questo lavoro  proveniva da una serie di intense discussioni con persone non udenti o che hanno problemi d'udito. Abbiamo parlato dei pro e dei contro di apparati tecnici, come apparecchi acustici o impianti cocleari, le diverse risposte etiche ed emotive che le persone sentono, e i problemi d'identità che sollevano. 
Indossare apparecchi acustici cambia anche come i suoni vengono percepiti, a volte causando interferenze, distorsioni,percezione spaziale ridotta e troppo pieno di rumore.}


%\descrizione{Along the Eaves}{takes its name from the following line in Franz Kafka’s “A Crossbreed” “On the moonlight nights its favorite promenade is along the eaves.” To compose the work, I developed custom software and used these programs in different ways to process and sequence my source materials, which, in this case, included audio recordings of water, babies, and string instruments. My interest is to create sonic coincidences that suggest relationships between sounds and the illusions they foster.}

\descrizione{Along the Eaves}{ prende il nome dalla riga che di Franz Kafka "Incrocio". "Sulle notti di luna la sua passeggiata preferita è lungo la grondaia" Per comporre l'opera, ho sviluppato software personalizzato e utilizzato questi programmi in modi diversi per elaborare e sequenziare i miei materiali di base, che, in questo caso, include registrazioni audio d’acqua, bambini, e di strumenti a corda. Il mio interesse è quello di creare coincidenze sonore che suggeriscono i rapporti tra i suoni e le illusioni che promuovono.}


%\descrizione{Bolgia}{Bolgia is an Italian word that means pocket or trench.  This term has been used by Dante Alighieri in his notable literary work Inferno.  Bolgia is a stereo fixed media electroacoustic composition, which depicts Alighieri’s journey to the eighth circle of hell, and his experiences to its horrific environment.  The musical gestures and sonic events of the piece evoke the different sounds and emotions of hell.}

\descrizione{Bolgia}{è una parola italiana che significa "fossa" e "luogo chiassoso in cui regna la confusione". Questo termine è stato usato da Dante Alighieri nel suo noto lavoro letterario "Inferno". Bolgia è un brano stereofonico fisso per composizioni elettroacustiche, che illustra il viaggio di Alighieri nell'ottavo girone dell'inferno, e la sua esperienza in questo posto terribile. I gesti musicali e l'evento sonoro del pezzo evocano i diversi suoni e le diverse emozioni dell'inferno.}


%\descrizione{Balloon Theories}{«I was always enjoying squeezing balloons, pressing them with my fingers until they pop… It has not been up until now that I realized why…»}

\descrizione{Balloon Theories}{«Ho sempre trovato divertente strizzare palloncini, premerli con le dita fino allo scoppio... Non mi è mai interessato fino a quando non ho capito perché...»}


%\descrizione{Inhabitated Places Part II}{Inhabitated Places part II is based on the concept of algorithmic composition. Although the general shape of this piece has been determined in a conventional way, every sound that one can hear are selected in real time by different algorithms written in SuperCollider. These algorithms choose randomly audio files from different folders and play them at different speeds and in different moments. It is as if we had placed several different objects in several boxes (that represent our shape), but every time we open one of these boxes the objects placed inside are positioned differently from how we had left them previously. The piece was also rendered in B-format Ambisonic.}

\descrizione{Inhabitated Places part II}{ è una composizione elettroacustica basata sul concetto di musica algoritmica. Sebbene la forma generale del brano sia stata determinata apriori e in modo convenzionale, tutti i suoni che ascoltiamo vengono scelti in tempo reale da vari algoritmi scritti in SuperCollider. Questi algoritmi selezionano in modo pseudocasuale dei samples da diverse cartelle e li riproducono a velocità diverse e in diversi momenti.  
È come se avessimo sistemato in una scatola (che in questo caso rappresenta la struttura dell’opera) degli oggetti in un dato ordine, ma ogni qual volta apriamo la scatola li troviamo disposti in modo differente da come li avevamo lasciati.}


\end{multicols}

\clearpage
%----------------------------------------------------------------------------------------

\section*{6 ottobre 2015 -- ore 20:30}

\subsection*{{\small Concerto Acusmatico \& Live Electronics\footnote{ Il concerto verrà trasmesso in diretta streaming da Radio Cemat}} \\
	\textsf{Sala Accademica}}

{\fontsize{30}{30} \svolk{\emph{ATTO IV}}}

\subsection*{\textsf{Regia del suono di Luana Lunetta}}

\bigskip

\begin{multicols}{2}

%% !TEX encoding = UTF-8 Unicode
% !TEX TS-program = XeLaTex
% !TEX root = EMU2015_booklet.tex

\livel{Simone Cardini}
{Potlach}{2014 - vincitore Premio Bucchi} %9'30''
{per flauto basso ed elettronica su supporto}
{flauto basso}{Gianni Trovalusci}

\livel{Dominique Schafer}
{Cendre}{2008} %10'00''
{per flauto basso e live electronics}
{flauto basso}{Gianni Trovalusci}

\livel{Maria Cristina De Amicis}
{[re-spì-ro]}{2015 - prima esecuzione assoluta} %8'00''
{per flauti e supporto digitale}
{flauti}{Gianni Trovalusci}

\acusmatico{Christian Eloy}
{La cicatrice d'Ulysse}{new version 2015} %13'00''
{acusmatico}

\livel{Giorgio Nottoli}
{7 Isole}{2015 - prima esecuzione assoluta}%15'
{per flauto, percussioni e live electronics}
{flauto}{Gianni Trovalusci}
percussioni -- \textsc{Gianluca Ruggeri}
\\



%\descrizione{Potlach}{was a ritual ceremony performed among certain Native American tribes, during which the members of the community exchanged and squandered gifts, contributing to the cohesiveness of the social concord, whithin a totally inverted logic with respect to our free economy, following a reciprocity principle. The formal juxtaposition of the piece, disperses at once those elements that will constitute later on the concealed foundation that penetratrates the work.}

\descrizione{Potlach}{Il Potlach era una cerimonia rituale d’alcuni popoli di Nativi Americani, durante la quale i membri della tribù scambiavano e dilapidavano doni, in una logica completamente invertita rispetto alla nostra economia. La giustapposizione del brano \textit{dilapida} da subito quegli elementi che costituiranno il substrato che pervade l'opera. È la simbolizzazione dell'immagine che ne permette un'esegesi palesata; l'interprete realizza sé attraverso la sublimazione di gesti e momenti performativi, rappresentando la fenomenologia degli stessi. Da qui la necessità di superare il momento poietico ed estesico \textit{donando} la performance per un recupero del sistema di relazioni come tale.}

%\descrizione{Cendre}{draws on the idea from the impermanence of materiality. The title of the piece gives further meaning referring to a fragile and delicate state, but also the potential of ashes as a fertilizer for something new. The bass flute being suspended within the electronics extends space and timbre, at times detaching itself within its own identity, but repeatedly gets reabsorbed into space. The piece was written for and premiered by Mario Caroli during his Fromm Residency at Harvard University.}

\descrizione{Cendre}{prende ispirazione dall’idea della precarietà di ciò che è materiale. Il titolo del brano acquista ulteriore significato riferendosi ad uno stato di fragilità e delicatezza, ma anche alle potenzialità fertilizzanti della cenere per qualcosa di nuovo. Grazie alla sospensione del flauto basso attraverso l’elettronica, il suo spazio ed il suo timbro vengono estesi, a tratti allontanandosi dalla propria identità per poi essere riassorbito all’interno dello spazio. Il brano è stato composto ed eseguito in prima assoluta da Mario Caroli durante la Fromm Residency presso la Harvard University.}

%\descrizione{[re-spì-ro]}{Re-spi-ro (Breath) in this work, is the alternation of air movements which could produce real rhythmic and tonal structures. The sound of each breath adds and interacts itself with the previous one, drawing a music form which contracts and expands itself as well. Through the rhythmic articulation, inhalation and exhalation, the breath transforms and builds itself, removes and contradicts what has been just expressed. The digital support creates a constructive dialectic, supporting gently the concrete material. Breath transformation is described by the conduct of three different changing elements: localization, movement (direction and speed) and space size. In this way , the auditive perception of the space is transposed to the visual perception level. In this work I tried to realize the interaction between these changing elements on the acoustic level and associating with tones and pitches the breath expressive gestures of the performer.}

\descrizione{[re-spì-ro]}{Il re-spi-ro, in questa opera, è inteso come l’alternanza dei movimenti dell’aria in grado di generare vere e proprie strutture ritmiche e timbriche. Il suono di ogni respiro si aggiunge e interagisce con i precedenti, disegnando una forma musicale che si contrae e si espande.  Attraverso l’articolazione ritmica, l’inspirazione e l’espirazione, il respiro si trasforma e costruisce, dissolve o contraddice ciò che è stato appena espresso. Il supporto digitale crea una dialettica costruttiva sostenendo delicatamente il materiale concreto. La trasformazione del respiro è descritta attraverso il comportamento di tre variabili: localizzazione, movimento (direzione, velocità) e dimensione dell’ambiente. In questo modo la percezione uditiva dello spazio viene trasposta nel dominio della percezione visiva. In questo brano ho cercato di realizzare l’interazione tra queste variabili nel dominio acustico associando con timbri e altezze i gesti espressivi del respiro dell’esecutore.}

%\descrizione{La cicatrice d'Ulysse}{This title, borrowed from Erich Auerbach, German writer and critic deceased in 1957, immediately sets the scene on a plane where realism is depicted in the aesthetics of Western music. Electroacoustic music has the ability, using “concrete” sounds (with all the ambiguity implied by the word), of giving us an immediate sensation of reality, which we can all situate in relation to our references and private objects of reference; these are the profound notions of sublimitas and humilitas, which merge and unite in musical expression. Every listener will be able to decipher his or her own images of a collective mental universe from the essence of this kind of artistic creation, just like the scar by which Ulysses was recognised by his former servant woman.}

\descrizione{La cicatrice d'Ulysse}{Il titolo, preso in prestito da Erich Auerbach, scrittore e critico tedesco morto nel 1957, ambienta immediatamente la scena all’interno di un aereo, tratteggiandola con un realismo tipico dell’estetica musicale occidentale. Attraverso l’uso di suoni concreti (con tutte le ambiguità insite nella parola stessa), la musica elettroacustica possiede l’abilità di restituirci una percezione immediata della realtà, che è possibile collocare in relazione ai propri riferimenti e agli oggetti segreti cui intende riferirsi; questi sono i significati profondi di sublimitas e humilitas, che si fondono e riuniscono nell’espressione musicale. Dall’essenza di questo tipo di creazione artistica ogni ascoltatore è in grado di decodificare la propria immagine di un universo mentale collettivo, proprio come la cicatrice che permise alla serva di riconoscere Ulisse.}

%\descrizione{7 Isole}{I think the idea of the island is a fascinating metaphor, because I realize to think about many important things  as they are formed, in fact, by islands, which each contain a different world and a different feel, but together form a unitary context. In "7 Isole", each "island" is characterized by a particular combination of movement, pitches and colors of the sound. It consists of seven small pieces separated from one another, that can be performed in any order, however, strongly linked by way of forming the sound material and by a unitary constructive thinking. The instruments are used both with extended techniques that traditional, allowing different nuances both for the continuous sound as well as impulsive, both for what determined pitch and for what similar to the noise. The electronics supports and extends the sound of the instruments and, where necessary, becomes the instrument itself, completing the construction of the sound field. The dynamic distribution  in the listening space is obtained according to the principles of the method Ambisonic. Some of the processing and sound synthesis used have been developed by the author.}

\descrizione{7 Isole}{Quella dell'isola è per me una metafora affascinante, in quanto mi accorgo di pensare a molte cose importanti come fossero costituite, appunto, da isole, che contengono ciascuna un mondo diverso e quindi un diverso sentire, ma, insieme, costituiscono un contesto unitario. In 7Isole, ciascuna "isola" è caratterizzata da una particolare combinazione di movimento, altezze e colori del suono. Si tratta di sette piccoli pezzi fra loro separati, che si possono eseguire in un qualsiasi ordine, tuttavia fortemente legati dal modo di formare il materiale sonoro e da un pensiero costruttivo unitario. Gli strumenti sono utilizzati sia con tecniche estese che tradizionali, consentendo diverse sfumature sia per il suono continuo che per quello impulsivo, sia per quello ad altezza determinata e per quello simile al rumore. L'elettronica sostiene ed  estende il suono degli strumenti e, dove necessario, diviene strumento essa stessa, completando la costruzione del campo sonoro. La distribuzione dinamica nello spazio d'ascolto è ottenuta secondo i principi  del metodo Ambisonic. Alcune delle elaborazioni e sintesi del suono utilizzate sono state sviluppate dall'autore.}











\end{multicols}

\clearpage

%----------------------------------------------------------------------------------------

\section*{7 ottobre 2015 -- ore 12:30}

\subsection*{{\small Concerto con Spazializzazione Interattiva in HOA} \\
	\textsf{Aula Bianchini}}

{\fontsize{30}{30} \svolk{\emph{ATTO V}}}

\subsection*{\textsf{regia del suono di Balandino Di Donato}}

\bigskip

\begin{multicols}{2}

%% !TEX encoding = UTF-8 Unicode
% !TEX TS-program = XeLaTex
% !TEX root = EMU2015_booklet.tex

\acusmatici{Ursula Meyer-K\"onig}
{Allears}{2012-13} %8' 

\acusmatici{Benjamin O'Brien}
{Along the eaves}{2012-13} %8'20'' 

\acusmatici{Dennis Deovides A. Reyes III}
{Bolgia}{2014} %7'31''

\acusmatici{Dimitrios Savva}
{Balloon Theories}{2012-13} %14'30'' 

\acusmatici{Jones Margarucci}
{Inhabitated PlacesPart II}{2012-13} %5'52"


%\descrizione{Allears}{The inspiration for this work originally came from a series of intensive discussions with people who are deaf or have hearing impairments. We talked about the pros and cons of technical apparatuses such as hearing aids or cochlea implants, the different ethical and emotional responses people have to them, and the identity issues they raise. Wearing hearing aids also changes how sounds are perceived, sometimes causing interference, distortions, diminished spatial perception and noise overflow.}

\descrizione{Allears}{Originariamente l'ispirazione per questo lavoro  proveniva da una serie di intense discussioni con persone non udenti o che hanno problemi d'udito. Abbiamo parlato dei pro e dei contro di apparati tecnici, come apparecchi acustici o impianti cocleari, le diverse risposte etiche ed emotive che le persone sentono, e i problemi d'identità che sollevano. 
Indossare apparecchi acustici cambia anche come i suoni vengono percepiti, a volte causando interferenze, distorsioni,percezione spaziale ridotta e troppo pieno di rumore.}


%\descrizione{Along the Eaves}{takes its name from the following line in Franz Kafka’s “A Crossbreed” “On the moonlight nights its favorite promenade is along the eaves.” To compose the work, I developed custom software and used these programs in different ways to process and sequence my source materials, which, in this case, included audio recordings of water, babies, and string instruments. My interest is to create sonic coincidences that suggest relationships between sounds and the illusions they foster.}

\descrizione{Along the Eaves}{ prende il nome dalla riga che di Franz Kafka "Incrocio". "Sulle notti di luna la sua passeggiata preferita è lungo la grondaia" Per comporre l'opera, ho sviluppato software personalizzato e utilizzato questi programmi in modi diversi per elaborare e sequenziare i miei materiali di base, che, in questo caso, include registrazioni audio d’acqua, bambini, e di strumenti a corda. Il mio interesse è quello di creare coincidenze sonore che suggeriscono i rapporti tra i suoni e le illusioni che promuovono.}


%\descrizione{Bolgia}{Bolgia is an Italian word that means pocket or trench.  This term has been used by Dante Alighieri in his notable literary work Inferno.  Bolgia is a stereo fixed media electroacoustic composition, which depicts Alighieri’s journey to the eighth circle of hell, and his experiences to its horrific environment.  The musical gestures and sonic events of the piece evoke the different sounds and emotions of hell.}

\descrizione{Bolgia}{è una parola italiana che significa "fossa" e "luogo chiassoso in cui regna la confusione". Questo termine è stato usato da Dante Alighieri nel suo noto lavoro letterario "Inferno". Bolgia è un brano stereofonico fisso per composizioni elettroacustiche, che illustra il viaggio di Alighieri nell'ottavo girone dell'inferno, e la sua esperienza in questo posto terribile. I gesti musicali e l'evento sonoro del pezzo evocano i diversi suoni e le diverse emozioni dell'inferno.}


%\descrizione{Balloon Theories}{«I was always enjoying squeezing balloons, pressing them with my fingers until they pop… It has not been up until now that I realized why…»}

\descrizione{Balloon Theories}{«Ho sempre trovato divertente strizzare palloncini, premerli con le dita fino allo scoppio... Non mi è mai interessato fino a quando non ho capito perché...»}


%\descrizione{Inhabitated Places Part II}{Inhabitated Places part II is based on the concept of algorithmic composition. Although the general shape of this piece has been determined in a conventional way, every sound that one can hear are selected in real time by different algorithms written in SuperCollider. These algorithms choose randomly audio files from different folders and play them at different speeds and in different moments. It is as if we had placed several different objects in several boxes (that represent our shape), but every time we open one of these boxes the objects placed inside are positioned differently from how we had left them previously. The piece was also rendered in B-format Ambisonic.}

\descrizione{Inhabitated Places part II}{ è una composizione elettroacustica basata sul concetto di musica algoritmica. Sebbene la forma generale del brano sia stata determinata apriori e in modo convenzionale, tutti i suoni che ascoltiamo vengono scelti in tempo reale da vari algoritmi scritti in SuperCollider. Questi algoritmi selezionano in modo pseudocasuale dei samples da diverse cartelle e li riproducono a velocità diverse e in diversi momenti.  
È come se avessimo sistemato in una scatola (che in questo caso rappresenta la struttura dell’opera) degli oggetti in un dato ordine, ma ogni qual volta apriamo la scatola li troviamo disposti in modo differente da come li avevamo lasciati.}


\end{multicols}

\clearpage
%----------------------------------------------------------------------------------------

\section*{7 ottobre 2015 -- ore 20:30}

\subsection*{{\small Concerto Acusmatico \& Live Electronics\footnote{ Il concerto verrà trasmesso in diretta streaming da Radio Cemat}} \\
	\textsf{Sala Accademica}}

{\fontsize{30}{30} \svolk{\emph{ATTO VI}}}

\subsection*{\textsf{Regia del suono di Federico Paganelli}}

\bigskip

\begin{multicols}{2}

%% !TEX encoding = UTF-8 Unicode
% !TEX TS-program = XeLaTex
% !TEX root = EMU2015_booklet.tex

\livel{Simone Cardini}
{Potlach}{2014 - vincitore Premio Bucchi} %9'30''
{per flauto basso ed elettronica su supporto}
{flauto basso}{Gianni Trovalusci}

\livel{Dominique Schafer}
{Cendre}{2008} %10'00''
{per flauto basso e live electronics}
{flauto basso}{Gianni Trovalusci}

\livel{Maria Cristina De Amicis}
{[re-spì-ro]}{2015 - prima esecuzione assoluta} %8'00''
{per flauti e supporto digitale}
{flauti}{Gianni Trovalusci}

\acusmatico{Christian Eloy}
{La cicatrice d'Ulysse}{new version 2015} %13'00''
{acusmatico}

\livel{Giorgio Nottoli}
{7 Isole}{2015 - prima esecuzione assoluta}%15'
{per flauto, percussioni e live electronics}
{flauto}{Gianni Trovalusci}
percussioni -- \textsc{Gianluca Ruggeri}
\\



%\descrizione{Potlach}{was a ritual ceremony performed among certain Native American tribes, during which the members of the community exchanged and squandered gifts, contributing to the cohesiveness of the social concord, whithin a totally inverted logic with respect to our free economy, following a reciprocity principle. The formal juxtaposition of the piece, disperses at once those elements that will constitute later on the concealed foundation that penetratrates the work.}

\descrizione{Potlach}{Il Potlach era una cerimonia rituale d’alcuni popoli di Nativi Americani, durante la quale i membri della tribù scambiavano e dilapidavano doni, in una logica completamente invertita rispetto alla nostra economia. La giustapposizione del brano \textit{dilapida} da subito quegli elementi che costituiranno il substrato che pervade l'opera. È la simbolizzazione dell'immagine che ne permette un'esegesi palesata; l'interprete realizza sé attraverso la sublimazione di gesti e momenti performativi, rappresentando la fenomenologia degli stessi. Da qui la necessità di superare il momento poietico ed estesico \textit{donando} la performance per un recupero del sistema di relazioni come tale.}

%\descrizione{Cendre}{draws on the idea from the impermanence of materiality. The title of the piece gives further meaning referring to a fragile and delicate state, but also the potential of ashes as a fertilizer for something new. The bass flute being suspended within the electronics extends space and timbre, at times detaching itself within its own identity, but repeatedly gets reabsorbed into space. The piece was written for and premiered by Mario Caroli during his Fromm Residency at Harvard University.}

\descrizione{Cendre}{prende ispirazione dall’idea della precarietà di ciò che è materiale. Il titolo del brano acquista ulteriore significato riferendosi ad uno stato di fragilità e delicatezza, ma anche alle potenzialità fertilizzanti della cenere per qualcosa di nuovo. Grazie alla sospensione del flauto basso attraverso l’elettronica, il suo spazio ed il suo timbro vengono estesi, a tratti allontanandosi dalla propria identità per poi essere riassorbito all’interno dello spazio. Il brano è stato composto ed eseguito in prima assoluta da Mario Caroli durante la Fromm Residency presso la Harvard University.}

%\descrizione{[re-spì-ro]}{Re-spi-ro (Breath) in this work, is the alternation of air movements which could produce real rhythmic and tonal structures. The sound of each breath adds and interacts itself with the previous one, drawing a music form which contracts and expands itself as well. Through the rhythmic articulation, inhalation and exhalation, the breath transforms and builds itself, removes and contradicts what has been just expressed. The digital support creates a constructive dialectic, supporting gently the concrete material. Breath transformation is described by the conduct of three different changing elements: localization, movement (direction and speed) and space size. In this way , the auditive perception of the space is transposed to the visual perception level. In this work I tried to realize the interaction between these changing elements on the acoustic level and associating with tones and pitches the breath expressive gestures of the performer.}

\descrizione{[re-spì-ro]}{Il re-spi-ro, in questa opera, è inteso come l’alternanza dei movimenti dell’aria in grado di generare vere e proprie strutture ritmiche e timbriche. Il suono di ogni respiro si aggiunge e interagisce con i precedenti, disegnando una forma musicale che si contrae e si espande.  Attraverso l’articolazione ritmica, l’inspirazione e l’espirazione, il respiro si trasforma e costruisce, dissolve o contraddice ciò che è stato appena espresso. Il supporto digitale crea una dialettica costruttiva sostenendo delicatamente il materiale concreto. La trasformazione del respiro è descritta attraverso il comportamento di tre variabili: localizzazione, movimento (direzione, velocità) e dimensione dell’ambiente. In questo modo la percezione uditiva dello spazio viene trasposta nel dominio della percezione visiva. In questo brano ho cercato di realizzare l’interazione tra queste variabili nel dominio acustico associando con timbri e altezze i gesti espressivi del respiro dell’esecutore.}

%\descrizione{La cicatrice d'Ulysse}{This title, borrowed from Erich Auerbach, German writer and critic deceased in 1957, immediately sets the scene on a plane where realism is depicted in the aesthetics of Western music. Electroacoustic music has the ability, using “concrete” sounds (with all the ambiguity implied by the word), of giving us an immediate sensation of reality, which we can all situate in relation to our references and private objects of reference; these are the profound notions of sublimitas and humilitas, which merge and unite in musical expression. Every listener will be able to decipher his or her own images of a collective mental universe from the essence of this kind of artistic creation, just like the scar by which Ulysses was recognised by his former servant woman.}

\descrizione{La cicatrice d'Ulysse}{Il titolo, preso in prestito da Erich Auerbach, scrittore e critico tedesco morto nel 1957, ambienta immediatamente la scena all’interno di un aereo, tratteggiandola con un realismo tipico dell’estetica musicale occidentale. Attraverso l’uso di suoni concreti (con tutte le ambiguità insite nella parola stessa), la musica elettroacustica possiede l’abilità di restituirci una percezione immediata della realtà, che è possibile collocare in relazione ai propri riferimenti e agli oggetti segreti cui intende riferirsi; questi sono i significati profondi di sublimitas e humilitas, che si fondono e riuniscono nell’espressione musicale. Dall’essenza di questo tipo di creazione artistica ogni ascoltatore è in grado di decodificare la propria immagine di un universo mentale collettivo, proprio come la cicatrice che permise alla serva di riconoscere Ulisse.}

%\descrizione{7 Isole}{I think the idea of the island is a fascinating metaphor, because I realize to think about many important things  as they are formed, in fact, by islands, which each contain a different world and a different feel, but together form a unitary context. In "7 Isole", each "island" is characterized by a particular combination of movement, pitches and colors of the sound. It consists of seven small pieces separated from one another, that can be performed in any order, however, strongly linked by way of forming the sound material and by a unitary constructive thinking. The instruments are used both with extended techniques that traditional, allowing different nuances both for the continuous sound as well as impulsive, both for what determined pitch and for what similar to the noise. The electronics supports and extends the sound of the instruments and, where necessary, becomes the instrument itself, completing the construction of the sound field. The dynamic distribution  in the listening space is obtained according to the principles of the method Ambisonic. Some of the processing and sound synthesis used have been developed by the author.}

\descrizione{7 Isole}{Quella dell'isola è per me una metafora affascinante, in quanto mi accorgo di pensare a molte cose importanti come fossero costituite, appunto, da isole, che contengono ciascuna un mondo diverso e quindi un diverso sentire, ma, insieme, costituiscono un contesto unitario. In 7Isole, ciascuna "isola" è caratterizzata da una particolare combinazione di movimento, altezze e colori del suono. Si tratta di sette piccoli pezzi fra loro separati, che si possono eseguire in un qualsiasi ordine, tuttavia fortemente legati dal modo di formare il materiale sonoro e da un pensiero costruttivo unitario. Gli strumenti sono utilizzati sia con tecniche estese che tradizionali, consentendo diverse sfumature sia per il suono continuo che per quello impulsivo, sia per quello ad altezza determinata e per quello simile al rumore. L'elettronica sostiene ed  estende il suono degli strumenti e, dove necessario, diviene strumento essa stessa, completando la costruzione del campo sonoro. La distribuzione dinamica nello spazio d'ascolto è ottenuta secondo i principi  del metodo Ambisonic. Alcune delle elaborazioni e sintesi del suono utilizzate sono state sviluppate dall'autore.}











\end{multicols}

\clearpage

%----------------------------------------------------------------------------------------

\section*{8 ottobre 2015 -- ore 17:00}

\subsection*{{\small Concerto Acusmatico in Cupola Ambisonic} \\
	\textsf{Aula Bianchini}}

{\fontsize{30}{30} \svolk{\emph{ATTO VII}}}

\subsection*{\textsf{regia del suono di Francesco Ziello}}

\bigskip

\begin{multicols}{2}

%% !TEX encoding = UTF-8 Unicode
% !TEX TS-program = XeLaTex
% !TEX root = EMU2015_booklet.tex

\acusmatici{Ursula Meyer-K\"onig}
{Allears}{2012-13} %8' 

\acusmatici{Benjamin O'Brien}
{Along the eaves}{2012-13} %8'20'' 

\acusmatici{Dennis Deovides A. Reyes III}
{Bolgia}{2014} %7'31''

\acusmatici{Dimitrios Savva}
{Balloon Theories}{2012-13} %14'30'' 

\acusmatici{Jones Margarucci}
{Inhabitated PlacesPart II}{2012-13} %5'52"


%\descrizione{Allears}{The inspiration for this work originally came from a series of intensive discussions with people who are deaf or have hearing impairments. We talked about the pros and cons of technical apparatuses such as hearing aids or cochlea implants, the different ethical and emotional responses people have to them, and the identity issues they raise. Wearing hearing aids also changes how sounds are perceived, sometimes causing interference, distortions, diminished spatial perception and noise overflow.}

\descrizione{Allears}{Originariamente l'ispirazione per questo lavoro  proveniva da una serie di intense discussioni con persone non udenti o che hanno problemi d'udito. Abbiamo parlato dei pro e dei contro di apparati tecnici, come apparecchi acustici o impianti cocleari, le diverse risposte etiche ed emotive che le persone sentono, e i problemi d'identità che sollevano. 
Indossare apparecchi acustici cambia anche come i suoni vengono percepiti, a volte causando interferenze, distorsioni,percezione spaziale ridotta e troppo pieno di rumore.}


%\descrizione{Along the Eaves}{takes its name from the following line in Franz Kafka’s “A Crossbreed” “On the moonlight nights its favorite promenade is along the eaves.” To compose the work, I developed custom software and used these programs in different ways to process and sequence my source materials, which, in this case, included audio recordings of water, babies, and string instruments. My interest is to create sonic coincidences that suggest relationships between sounds and the illusions they foster.}

\descrizione{Along the Eaves}{ prende il nome dalla riga che di Franz Kafka "Incrocio". "Sulle notti di luna la sua passeggiata preferita è lungo la grondaia" Per comporre l'opera, ho sviluppato software personalizzato e utilizzato questi programmi in modi diversi per elaborare e sequenziare i miei materiali di base, che, in questo caso, include registrazioni audio d’acqua, bambini, e di strumenti a corda. Il mio interesse è quello di creare coincidenze sonore che suggeriscono i rapporti tra i suoni e le illusioni che promuovono.}


%\descrizione{Bolgia}{Bolgia is an Italian word that means pocket or trench.  This term has been used by Dante Alighieri in his notable literary work Inferno.  Bolgia is a stereo fixed media electroacoustic composition, which depicts Alighieri’s journey to the eighth circle of hell, and his experiences to its horrific environment.  The musical gestures and sonic events of the piece evoke the different sounds and emotions of hell.}

\descrizione{Bolgia}{è una parola italiana che significa "fossa" e "luogo chiassoso in cui regna la confusione". Questo termine è stato usato da Dante Alighieri nel suo noto lavoro letterario "Inferno". Bolgia è un brano stereofonico fisso per composizioni elettroacustiche, che illustra il viaggio di Alighieri nell'ottavo girone dell'inferno, e la sua esperienza in questo posto terribile. I gesti musicali e l'evento sonoro del pezzo evocano i diversi suoni e le diverse emozioni dell'inferno.}


%\descrizione{Balloon Theories}{«I was always enjoying squeezing balloons, pressing them with my fingers until they pop… It has not been up until now that I realized why…»}

\descrizione{Balloon Theories}{«Ho sempre trovato divertente strizzare palloncini, premerli con le dita fino allo scoppio... Non mi è mai interessato fino a quando non ho capito perché...»}


%\descrizione{Inhabitated Places Part II}{Inhabitated Places part II is based on the concept of algorithmic composition. Although the general shape of this piece has been determined in a conventional way, every sound that one can hear are selected in real time by different algorithms written in SuperCollider. These algorithms choose randomly audio files from different folders and play them at different speeds and in different moments. It is as if we had placed several different objects in several boxes (that represent our shape), but every time we open one of these boxes the objects placed inside are positioned differently from how we had left them previously. The piece was also rendered in B-format Ambisonic.}

\descrizione{Inhabitated Places part II}{ è una composizione elettroacustica basata sul concetto di musica algoritmica. Sebbene la forma generale del brano sia stata determinata apriori e in modo convenzionale, tutti i suoni che ascoltiamo vengono scelti in tempo reale da vari algoritmi scritti in SuperCollider. Questi algoritmi selezionano in modo pseudocasuale dei samples da diverse cartelle e li riproducono a velocità diverse e in diversi momenti.  
È come se avessimo sistemato in una scatola (che in questo caso rappresenta la struttura dell’opera) degli oggetti in un dato ordine, ma ogni qual volta apriamo la scatola li troviamo disposti in modo differente da come li avevamo lasciati.}


\end{multicols}

\clearpage
%----------------------------------------------------------------------------------------

\section*{8 ottobre 2015 -- ore 20:30}

\subsection*{{\small Concerto Acusmatico \& Live Electronics\footnote{ Il concerto verrà trasmesso in diretta streaming da Radio Cemat}} \\
	\textsf{Sala Accademica}}

{\fontsize{30}{30} \svolk{\emph{ATTO VIII}}}

\subsection*{\textsf{Regia del suono di Federico Paganelli}}

\bigskip

\begin{multicols}{2}

%% !TEX encoding = UTF-8 Unicode
% !TEX TS-program = XeLaTex
% !TEX root = EMU2015_booklet.tex

\livel{Simone Cardini}
{Potlach}{2014 - vincitore Premio Bucchi} %9'30''
{per flauto basso ed elettronica su supporto}
{flauto basso}{Gianni Trovalusci}

\livel{Dominique Schafer}
{Cendre}{2008} %10'00''
{per flauto basso e live electronics}
{flauto basso}{Gianni Trovalusci}

\livel{Maria Cristina De Amicis}
{[re-spì-ro]}{2015 - prima esecuzione assoluta} %8'00''
{per flauti e supporto digitale}
{flauti}{Gianni Trovalusci}

\acusmatico{Christian Eloy}
{La cicatrice d'Ulysse}{new version 2015} %13'00''
{acusmatico}

\livel{Giorgio Nottoli}
{7 Isole}{2015 - prima esecuzione assoluta}%15'
{per flauto, percussioni e live electronics}
{flauto}{Gianni Trovalusci}
percussioni -- \textsc{Gianluca Ruggeri}
\\



%\descrizione{Potlach}{was a ritual ceremony performed among certain Native American tribes, during which the members of the community exchanged and squandered gifts, contributing to the cohesiveness of the social concord, whithin a totally inverted logic with respect to our free economy, following a reciprocity principle. The formal juxtaposition of the piece, disperses at once those elements that will constitute later on the concealed foundation that penetratrates the work.}

\descrizione{Potlach}{Il Potlach era una cerimonia rituale d’alcuni popoli di Nativi Americani, durante la quale i membri della tribù scambiavano e dilapidavano doni, in una logica completamente invertita rispetto alla nostra economia. La giustapposizione del brano \textit{dilapida} da subito quegli elementi che costituiranno il substrato che pervade l'opera. È la simbolizzazione dell'immagine che ne permette un'esegesi palesata; l'interprete realizza sé attraverso la sublimazione di gesti e momenti performativi, rappresentando la fenomenologia degli stessi. Da qui la necessità di superare il momento poietico ed estesico \textit{donando} la performance per un recupero del sistema di relazioni come tale.}

%\descrizione{Cendre}{draws on the idea from the impermanence of materiality. The title of the piece gives further meaning referring to a fragile and delicate state, but also the potential of ashes as a fertilizer for something new. The bass flute being suspended within the electronics extends space and timbre, at times detaching itself within its own identity, but repeatedly gets reabsorbed into space. The piece was written for and premiered by Mario Caroli during his Fromm Residency at Harvard University.}

\descrizione{Cendre}{prende ispirazione dall’idea della precarietà di ciò che è materiale. Il titolo del brano acquista ulteriore significato riferendosi ad uno stato di fragilità e delicatezza, ma anche alle potenzialità fertilizzanti della cenere per qualcosa di nuovo. Grazie alla sospensione del flauto basso attraverso l’elettronica, il suo spazio ed il suo timbro vengono estesi, a tratti allontanandosi dalla propria identità per poi essere riassorbito all’interno dello spazio. Il brano è stato composto ed eseguito in prima assoluta da Mario Caroli durante la Fromm Residency presso la Harvard University.}

%\descrizione{[re-spì-ro]}{Re-spi-ro (Breath) in this work, is the alternation of air movements which could produce real rhythmic and tonal structures. The sound of each breath adds and interacts itself with the previous one, drawing a music form which contracts and expands itself as well. Through the rhythmic articulation, inhalation and exhalation, the breath transforms and builds itself, removes and contradicts what has been just expressed. The digital support creates a constructive dialectic, supporting gently the concrete material. Breath transformation is described by the conduct of three different changing elements: localization, movement (direction and speed) and space size. In this way , the auditive perception of the space is transposed to the visual perception level. In this work I tried to realize the interaction between these changing elements on the acoustic level and associating with tones and pitches the breath expressive gestures of the performer.}

\descrizione{[re-spì-ro]}{Il re-spi-ro, in questa opera, è inteso come l’alternanza dei movimenti dell’aria in grado di generare vere e proprie strutture ritmiche e timbriche. Il suono di ogni respiro si aggiunge e interagisce con i precedenti, disegnando una forma musicale che si contrae e si espande.  Attraverso l’articolazione ritmica, l’inspirazione e l’espirazione, il respiro si trasforma e costruisce, dissolve o contraddice ciò che è stato appena espresso. Il supporto digitale crea una dialettica costruttiva sostenendo delicatamente il materiale concreto. La trasformazione del respiro è descritta attraverso il comportamento di tre variabili: localizzazione, movimento (direzione, velocità) e dimensione dell’ambiente. In questo modo la percezione uditiva dello spazio viene trasposta nel dominio della percezione visiva. In questo brano ho cercato di realizzare l’interazione tra queste variabili nel dominio acustico associando con timbri e altezze i gesti espressivi del respiro dell’esecutore.}

%\descrizione{La cicatrice d'Ulysse}{This title, borrowed from Erich Auerbach, German writer and critic deceased in 1957, immediately sets the scene on a plane where realism is depicted in the aesthetics of Western music. Electroacoustic music has the ability, using “concrete” sounds (with all the ambiguity implied by the word), of giving us an immediate sensation of reality, which we can all situate in relation to our references and private objects of reference; these are the profound notions of sublimitas and humilitas, which merge and unite in musical expression. Every listener will be able to decipher his or her own images of a collective mental universe from the essence of this kind of artistic creation, just like the scar by which Ulysses was recognised by his former servant woman.}

\descrizione{La cicatrice d'Ulysse}{Il titolo, preso in prestito da Erich Auerbach, scrittore e critico tedesco morto nel 1957, ambienta immediatamente la scena all’interno di un aereo, tratteggiandola con un realismo tipico dell’estetica musicale occidentale. Attraverso l’uso di suoni concreti (con tutte le ambiguità insite nella parola stessa), la musica elettroacustica possiede l’abilità di restituirci una percezione immediata della realtà, che è possibile collocare in relazione ai propri riferimenti e agli oggetti segreti cui intende riferirsi; questi sono i significati profondi di sublimitas e humilitas, che si fondono e riuniscono nell’espressione musicale. Dall’essenza di questo tipo di creazione artistica ogni ascoltatore è in grado di decodificare la propria immagine di un universo mentale collettivo, proprio come la cicatrice che permise alla serva di riconoscere Ulisse.}

%\descrizione{7 Isole}{I think the idea of the island is a fascinating metaphor, because I realize to think about many important things  as they are formed, in fact, by islands, which each contain a different world and a different feel, but together form a unitary context. In "7 Isole", each "island" is characterized by a particular combination of movement, pitches and colors of the sound. It consists of seven small pieces separated from one another, that can be performed in any order, however, strongly linked by way of forming the sound material and by a unitary constructive thinking. The instruments are used both with extended techniques that traditional, allowing different nuances both for the continuous sound as well as impulsive, both for what determined pitch and for what similar to the noise. The electronics supports and extends the sound of the instruments and, where necessary, becomes the instrument itself, completing the construction of the sound field. The dynamic distribution  in the listening space is obtained according to the principles of the method Ambisonic. Some of the processing and sound synthesis used have been developed by the author.}

\descrizione{7 Isole}{Quella dell'isola è per me una metafora affascinante, in quanto mi accorgo di pensare a molte cose importanti come fossero costituite, appunto, da isole, che contengono ciascuna un mondo diverso e quindi un diverso sentire, ma, insieme, costituiscono un contesto unitario. In 7Isole, ciascuna "isola" è caratterizzata da una particolare combinazione di movimento, altezze e colori del suono. Si tratta di sette piccoli pezzi fra loro separati, che si possono eseguire in un qualsiasi ordine, tuttavia fortemente legati dal modo di formare il materiale sonoro e da un pensiero costruttivo unitario. Gli strumenti sono utilizzati sia con tecniche estese che tradizionali, consentendo diverse sfumature sia per il suono continuo che per quello impulsivo, sia per quello ad altezza determinata e per quello simile al rumore. L'elettronica sostiene ed  estende il suono degli strumenti e, dove necessario, diviene strumento essa stessa, completando la costruzione del campo sonoro. La distribuzione dinamica nello spazio d'ascolto è ottenuta secondo i principi  del metodo Ambisonic. Alcune delle elaborazioni e sintesi del suono utilizzate sono state sviluppate dall'autore.}











\end{multicols}

\clearpage

%----------------------------------------------------------------------------------------

\section*{9 ottobre 2015 -- ore 17:00}

\subsection*{{\small Concerto Acusmatico in Cupola Ambisonic} \\
	\textsf{Aula Bianchini}}

{\fontsize{30}{30} \svolk{\emph{ATTO IX}}}

\subsection*{\textsf{regia del suono di Francesco Ziello}}

\bigskip

\begin{multicols}{2}

%% !TEX encoding = UTF-8 Unicode
% !TEX TS-program = XeLaTex
% !TEX root = EMU2015_booklet.tex

\acusmatici{Ursula Meyer-K\"onig}
{Allears}{2012-13} %8' 

\acusmatici{Benjamin O'Brien}
{Along the eaves}{2012-13} %8'20'' 

\acusmatici{Dennis Deovides A. Reyes III}
{Bolgia}{2014} %7'31''

\acusmatici{Dimitrios Savva}
{Balloon Theories}{2012-13} %14'30'' 

\acusmatici{Jones Margarucci}
{Inhabitated PlacesPart II}{2012-13} %5'52"


%\descrizione{Allears}{The inspiration for this work originally came from a series of intensive discussions with people who are deaf or have hearing impairments. We talked about the pros and cons of technical apparatuses such as hearing aids or cochlea implants, the different ethical and emotional responses people have to them, and the identity issues they raise. Wearing hearing aids also changes how sounds are perceived, sometimes causing interference, distortions, diminished spatial perception and noise overflow.}

\descrizione{Allears}{Originariamente l'ispirazione per questo lavoro  proveniva da una serie di intense discussioni con persone non udenti o che hanno problemi d'udito. Abbiamo parlato dei pro e dei contro di apparati tecnici, come apparecchi acustici o impianti cocleari, le diverse risposte etiche ed emotive che le persone sentono, e i problemi d'identità che sollevano. 
Indossare apparecchi acustici cambia anche come i suoni vengono percepiti, a volte causando interferenze, distorsioni,percezione spaziale ridotta e troppo pieno di rumore.}


%\descrizione{Along the Eaves}{takes its name from the following line in Franz Kafka’s “A Crossbreed” “On the moonlight nights its favorite promenade is along the eaves.” To compose the work, I developed custom software and used these programs in different ways to process and sequence my source materials, which, in this case, included audio recordings of water, babies, and string instruments. My interest is to create sonic coincidences that suggest relationships between sounds and the illusions they foster.}

\descrizione{Along the Eaves}{ prende il nome dalla riga che di Franz Kafka "Incrocio". "Sulle notti di luna la sua passeggiata preferita è lungo la grondaia" Per comporre l'opera, ho sviluppato software personalizzato e utilizzato questi programmi in modi diversi per elaborare e sequenziare i miei materiali di base, che, in questo caso, include registrazioni audio d’acqua, bambini, e di strumenti a corda. Il mio interesse è quello di creare coincidenze sonore che suggeriscono i rapporti tra i suoni e le illusioni che promuovono.}


%\descrizione{Bolgia}{Bolgia is an Italian word that means pocket or trench.  This term has been used by Dante Alighieri in his notable literary work Inferno.  Bolgia is a stereo fixed media electroacoustic composition, which depicts Alighieri’s journey to the eighth circle of hell, and his experiences to its horrific environment.  The musical gestures and sonic events of the piece evoke the different sounds and emotions of hell.}

\descrizione{Bolgia}{è una parola italiana che significa "fossa" e "luogo chiassoso in cui regna la confusione". Questo termine è stato usato da Dante Alighieri nel suo noto lavoro letterario "Inferno". Bolgia è un brano stereofonico fisso per composizioni elettroacustiche, che illustra il viaggio di Alighieri nell'ottavo girone dell'inferno, e la sua esperienza in questo posto terribile. I gesti musicali e l'evento sonoro del pezzo evocano i diversi suoni e le diverse emozioni dell'inferno.}


%\descrizione{Balloon Theories}{«I was always enjoying squeezing balloons, pressing them with my fingers until they pop… It has not been up until now that I realized why…»}

\descrizione{Balloon Theories}{«Ho sempre trovato divertente strizzare palloncini, premerli con le dita fino allo scoppio... Non mi è mai interessato fino a quando non ho capito perché...»}


%\descrizione{Inhabitated Places Part II}{Inhabitated Places part II is based on the concept of algorithmic composition. Although the general shape of this piece has been determined in a conventional way, every sound that one can hear are selected in real time by different algorithms written in SuperCollider. These algorithms choose randomly audio files from different folders and play them at different speeds and in different moments. It is as if we had placed several different objects in several boxes (that represent our shape), but every time we open one of these boxes the objects placed inside are positioned differently from how we had left them previously. The piece was also rendered in B-format Ambisonic.}

\descrizione{Inhabitated Places part II}{ è una composizione elettroacustica basata sul concetto di musica algoritmica. Sebbene la forma generale del brano sia stata determinata apriori e in modo convenzionale, tutti i suoni che ascoltiamo vengono scelti in tempo reale da vari algoritmi scritti in SuperCollider. Questi algoritmi selezionano in modo pseudocasuale dei samples da diverse cartelle e li riproducono a velocità diverse e in diversi momenti.  
È come se avessimo sistemato in una scatola (che in questo caso rappresenta la struttura dell’opera) degli oggetti in un dato ordine, ma ogni qual volta apriamo la scatola li troviamo disposti in modo differente da come li avevamo lasciati.}


\end{multicols}

\clearpage
%----------------------------------------------------------------------------------------

\section*{9 ottobre 2015 -- ore 20:30}

\subsection*{{\small Concerto Acusmatico \& Live Electronics\footnote{ Il concerto verrà trasmesso in diretta streaming da Radio Cemat}} \\
	\textsf{Sala Accademica}}

{\fontsize{30}{30} \svolk{\emph{ATTO X}}}

\subsection*{\textsf{Regia del suono di Federico Paganelli}}

\bigskip

\begin{multicols}{2}

%% !TEX encoding = UTF-8 Unicode
% !TEX TS-program = XeLaTex
% !TEX root = EMU2015_booklet.tex

\livel{Simone Cardini}
{Potlach}{2014 - vincitore Premio Bucchi} %9'30''
{per flauto basso ed elettronica su supporto}
{flauto basso}{Gianni Trovalusci}

\livel{Dominique Schafer}
{Cendre}{2008} %10'00''
{per flauto basso e live electronics}
{flauto basso}{Gianni Trovalusci}

\livel{Maria Cristina De Amicis}
{[re-spì-ro]}{2015 - prima esecuzione assoluta} %8'00''
{per flauti e supporto digitale}
{flauti}{Gianni Trovalusci}

\acusmatico{Christian Eloy}
{La cicatrice d'Ulysse}{new version 2015} %13'00''
{acusmatico}

\livel{Giorgio Nottoli}
{7 Isole}{2015 - prima esecuzione assoluta}%15'
{per flauto, percussioni e live electronics}
{flauto}{Gianni Trovalusci}
percussioni -- \textsc{Gianluca Ruggeri}
\\



%\descrizione{Potlach}{was a ritual ceremony performed among certain Native American tribes, during which the members of the community exchanged and squandered gifts, contributing to the cohesiveness of the social concord, whithin a totally inverted logic with respect to our free economy, following a reciprocity principle. The formal juxtaposition of the piece, disperses at once those elements that will constitute later on the concealed foundation that penetratrates the work.}

\descrizione{Potlach}{Il Potlach era una cerimonia rituale d’alcuni popoli di Nativi Americani, durante la quale i membri della tribù scambiavano e dilapidavano doni, in una logica completamente invertita rispetto alla nostra economia. La giustapposizione del brano \textit{dilapida} da subito quegli elementi che costituiranno il substrato che pervade l'opera. È la simbolizzazione dell'immagine che ne permette un'esegesi palesata; l'interprete realizza sé attraverso la sublimazione di gesti e momenti performativi, rappresentando la fenomenologia degli stessi. Da qui la necessità di superare il momento poietico ed estesico \textit{donando} la performance per un recupero del sistema di relazioni come tale.}

%\descrizione{Cendre}{draws on the idea from the impermanence of materiality. The title of the piece gives further meaning referring to a fragile and delicate state, but also the potential of ashes as a fertilizer for something new. The bass flute being suspended within the electronics extends space and timbre, at times detaching itself within its own identity, but repeatedly gets reabsorbed into space. The piece was written for and premiered by Mario Caroli during his Fromm Residency at Harvard University.}

\descrizione{Cendre}{prende ispirazione dall’idea della precarietà di ciò che è materiale. Il titolo del brano acquista ulteriore significato riferendosi ad uno stato di fragilità e delicatezza, ma anche alle potenzialità fertilizzanti della cenere per qualcosa di nuovo. Grazie alla sospensione del flauto basso attraverso l’elettronica, il suo spazio ed il suo timbro vengono estesi, a tratti allontanandosi dalla propria identità per poi essere riassorbito all’interno dello spazio. Il brano è stato composto ed eseguito in prima assoluta da Mario Caroli durante la Fromm Residency presso la Harvard University.}

%\descrizione{[re-spì-ro]}{Re-spi-ro (Breath) in this work, is the alternation of air movements which could produce real rhythmic and tonal structures. The sound of each breath adds and interacts itself with the previous one, drawing a music form which contracts and expands itself as well. Through the rhythmic articulation, inhalation and exhalation, the breath transforms and builds itself, removes and contradicts what has been just expressed. The digital support creates a constructive dialectic, supporting gently the concrete material. Breath transformation is described by the conduct of three different changing elements: localization, movement (direction and speed) and space size. In this way , the auditive perception of the space is transposed to the visual perception level. In this work I tried to realize the interaction between these changing elements on the acoustic level and associating with tones and pitches the breath expressive gestures of the performer.}

\descrizione{[re-spì-ro]}{Il re-spi-ro, in questa opera, è inteso come l’alternanza dei movimenti dell’aria in grado di generare vere e proprie strutture ritmiche e timbriche. Il suono di ogni respiro si aggiunge e interagisce con i precedenti, disegnando una forma musicale che si contrae e si espande.  Attraverso l’articolazione ritmica, l’inspirazione e l’espirazione, il respiro si trasforma e costruisce, dissolve o contraddice ciò che è stato appena espresso. Il supporto digitale crea una dialettica costruttiva sostenendo delicatamente il materiale concreto. La trasformazione del respiro è descritta attraverso il comportamento di tre variabili: localizzazione, movimento (direzione, velocità) e dimensione dell’ambiente. In questo modo la percezione uditiva dello spazio viene trasposta nel dominio della percezione visiva. In questo brano ho cercato di realizzare l’interazione tra queste variabili nel dominio acustico associando con timbri e altezze i gesti espressivi del respiro dell’esecutore.}

%\descrizione{La cicatrice d'Ulysse}{This title, borrowed from Erich Auerbach, German writer and critic deceased in 1957, immediately sets the scene on a plane where realism is depicted in the aesthetics of Western music. Electroacoustic music has the ability, using “concrete” sounds (with all the ambiguity implied by the word), of giving us an immediate sensation of reality, which we can all situate in relation to our references and private objects of reference; these are the profound notions of sublimitas and humilitas, which merge and unite in musical expression. Every listener will be able to decipher his or her own images of a collective mental universe from the essence of this kind of artistic creation, just like the scar by which Ulysses was recognised by his former servant woman.}

\descrizione{La cicatrice d'Ulysse}{Il titolo, preso in prestito da Erich Auerbach, scrittore e critico tedesco morto nel 1957, ambienta immediatamente la scena all’interno di un aereo, tratteggiandola con un realismo tipico dell’estetica musicale occidentale. Attraverso l’uso di suoni concreti (con tutte le ambiguità insite nella parola stessa), la musica elettroacustica possiede l’abilità di restituirci una percezione immediata della realtà, che è possibile collocare in relazione ai propri riferimenti e agli oggetti segreti cui intende riferirsi; questi sono i significati profondi di sublimitas e humilitas, che si fondono e riuniscono nell’espressione musicale. Dall’essenza di questo tipo di creazione artistica ogni ascoltatore è in grado di decodificare la propria immagine di un universo mentale collettivo, proprio come la cicatrice che permise alla serva di riconoscere Ulisse.}

%\descrizione{7 Isole}{I think the idea of the island is a fascinating metaphor, because I realize to think about many important things  as they are formed, in fact, by islands, which each contain a different world and a different feel, but together form a unitary context. In "7 Isole", each "island" is characterized by a particular combination of movement, pitches and colors of the sound. It consists of seven small pieces separated from one another, that can be performed in any order, however, strongly linked by way of forming the sound material and by a unitary constructive thinking. The instruments are used both with extended techniques that traditional, allowing different nuances both for the continuous sound as well as impulsive, both for what determined pitch and for what similar to the noise. The electronics supports and extends the sound of the instruments and, where necessary, becomes the instrument itself, completing the construction of the sound field. The dynamic distribution  in the listening space is obtained according to the principles of the method Ambisonic. Some of the processing and sound synthesis used have been developed by the author.}

\descrizione{7 Isole}{Quella dell'isola è per me una metafora affascinante, in quanto mi accorgo di pensare a molte cose importanti come fossero costituite, appunto, da isole, che contengono ciascuna un mondo diverso e quindi un diverso sentire, ma, insieme, costituiscono un contesto unitario. In 7Isole, ciascuna "isola" è caratterizzata da una particolare combinazione di movimento, altezze e colori del suono. Si tratta di sette piccoli pezzi fra loro separati, che si possono eseguire in un qualsiasi ordine, tuttavia fortemente legati dal modo di formare il materiale sonoro e da un pensiero costruttivo unitario. Gli strumenti sono utilizzati sia con tecniche estese che tradizionali, consentendo diverse sfumature sia per il suono continuo che per quello impulsivo, sia per quello ad altezza determinata e per quello simile al rumore. L'elettronica sostiene ed  estende il suono degli strumenti e, dove necessario, diviene strumento essa stessa, completando la costruzione del campo sonoro. La distribuzione dinamica nello spazio d'ascolto è ottenuta secondo i principi  del metodo Ambisonic. Alcune delle elaborazioni e sintesi del suono utilizzate sono state sviluppate dall'autore.}











\end{multicols}

\clearpage
%----------------------------------------------------------------------------------------

\section*{12 ottobre 2015 -- ore 18:00}

\subsection*{{\small Concerto Acusmatico \& Live Electronics\footnote{ Il concerto verrà trasmesso in diretta streaming da Radio Cemat}} \\
	\textsf{Auditorium Ennio Morricone}}

{\fontsize{30}{30} \svolk{\emph{ATTO XI}}}

\subsection*{\textsf{regia del suono di Francesco Ziello}}

\bigskip

\begin{multicols}{2}

%% !TEX encoding = UTF-8 Unicode
% !TEX TS-program = XeLaTex
% !TEX root = EMU2015_booklet.tex

\acusmatici{Ursula Meyer-K\"onig}
{Allears}{2012-13} %8' 

\acusmatici{Benjamin O'Brien}
{Along the eaves}{2012-13} %8'20'' 

\acusmatici{Dennis Deovides A. Reyes III}
{Bolgia}{2014} %7'31''

\acusmatici{Dimitrios Savva}
{Balloon Theories}{2012-13} %14'30'' 

\acusmatici{Jones Margarucci}
{Inhabitated PlacesPart II}{2012-13} %5'52"


%\descrizione{Allears}{The inspiration for this work originally came from a series of intensive discussions with people who are deaf or have hearing impairments. We talked about the pros and cons of technical apparatuses such as hearing aids or cochlea implants, the different ethical and emotional responses people have to them, and the identity issues they raise. Wearing hearing aids also changes how sounds are perceived, sometimes causing interference, distortions, diminished spatial perception and noise overflow.}

\descrizione{Allears}{Originariamente l'ispirazione per questo lavoro  proveniva da una serie di intense discussioni con persone non udenti o che hanno problemi d'udito. Abbiamo parlato dei pro e dei contro di apparati tecnici, come apparecchi acustici o impianti cocleari, le diverse risposte etiche ed emotive che le persone sentono, e i problemi d'identità che sollevano. 
Indossare apparecchi acustici cambia anche come i suoni vengono percepiti, a volte causando interferenze, distorsioni,percezione spaziale ridotta e troppo pieno di rumore.}


%\descrizione{Along the Eaves}{takes its name from the following line in Franz Kafka’s “A Crossbreed” “On the moonlight nights its favorite promenade is along the eaves.” To compose the work, I developed custom software and used these programs in different ways to process and sequence my source materials, which, in this case, included audio recordings of water, babies, and string instruments. My interest is to create sonic coincidences that suggest relationships between sounds and the illusions they foster.}

\descrizione{Along the Eaves}{ prende il nome dalla riga che di Franz Kafka "Incrocio". "Sulle notti di luna la sua passeggiata preferita è lungo la grondaia" Per comporre l'opera, ho sviluppato software personalizzato e utilizzato questi programmi in modi diversi per elaborare e sequenziare i miei materiali di base, che, in questo caso, include registrazioni audio d’acqua, bambini, e di strumenti a corda. Il mio interesse è quello di creare coincidenze sonore che suggeriscono i rapporti tra i suoni e le illusioni che promuovono.}


%\descrizione{Bolgia}{Bolgia is an Italian word that means pocket or trench.  This term has been used by Dante Alighieri in his notable literary work Inferno.  Bolgia is a stereo fixed media electroacoustic composition, which depicts Alighieri’s journey to the eighth circle of hell, and his experiences to its horrific environment.  The musical gestures and sonic events of the piece evoke the different sounds and emotions of hell.}

\descrizione{Bolgia}{è una parola italiana che significa "fossa" e "luogo chiassoso in cui regna la confusione". Questo termine è stato usato da Dante Alighieri nel suo noto lavoro letterario "Inferno". Bolgia è un brano stereofonico fisso per composizioni elettroacustiche, che illustra il viaggio di Alighieri nell'ottavo girone dell'inferno, e la sua esperienza in questo posto terribile. I gesti musicali e l'evento sonoro del pezzo evocano i diversi suoni e le diverse emozioni dell'inferno.}


%\descrizione{Balloon Theories}{«I was always enjoying squeezing balloons, pressing them with my fingers until they pop… It has not been up until now that I realized why…»}

\descrizione{Balloon Theories}{«Ho sempre trovato divertente strizzare palloncini, premerli con le dita fino allo scoppio... Non mi è mai interessato fino a quando non ho capito perché...»}


%\descrizione{Inhabitated Places Part II}{Inhabitated Places part II is based on the concept of algorithmic composition. Although the general shape of this piece has been determined in a conventional way, every sound that one can hear are selected in real time by different algorithms written in SuperCollider. These algorithms choose randomly audio files from different folders and play them at different speeds and in different moments. It is as if we had placed several different objects in several boxes (that represent our shape), but every time we open one of these boxes the objects placed inside are positioned differently from how we had left them previously. The piece was also rendered in B-format Ambisonic.}

\descrizione{Inhabitated Places part II}{ è una composizione elettroacustica basata sul concetto di musica algoritmica. Sebbene la forma generale del brano sia stata determinata apriori e in modo convenzionale, tutti i suoni che ascoltiamo vengono scelti in tempo reale da vari algoritmi scritti in SuperCollider. Questi algoritmi selezionano in modo pseudocasuale dei samples da diverse cartelle e li riproducono a velocità diverse e in diversi momenti.  
È come se avessimo sistemato in una scatola (che in questo caso rappresenta la struttura dell’opera) degli oggetti in un dato ordine, ma ogni qual volta apriamo la scatola li troviamo disposti in modo differente da come li avevamo lasciati.}


\end{multicols}

\clearpage
%----------------------------------------------------------------------------------------

\section*{13 ottobre 2015 -- ore 18:00}

\subsection*{{\small Concerto Acusmatico \& Live Electronics\footnote{ Il concerto verrà trasmesso in diretta streaming da Radio Cemat}} \\
	\textsf{Auditorium Ennio Morricone}}

{\fontsize{30}{30} \svolk{\emph{ATTO XII}}}

\subsection*{\textsf{Regia del suono di Federico Paganelli}}

\bigskip

\begin{multicols}{2}

%% !TEX encoding = UTF-8 Unicode
% !TEX TS-program = XeLaTex
% !TEX root = EMU2015_booklet.tex

\livel{Simone Cardini}
{Potlach}{2014 - vincitore Premio Bucchi} %9'30''
{per flauto basso ed elettronica su supporto}
{flauto basso}{Gianni Trovalusci}

\livel{Dominique Schafer}
{Cendre}{2008} %10'00''
{per flauto basso e live electronics}
{flauto basso}{Gianni Trovalusci}

\livel{Maria Cristina De Amicis}
{[re-spì-ro]}{2015 - prima esecuzione assoluta} %8'00''
{per flauti e supporto digitale}
{flauti}{Gianni Trovalusci}

\acusmatico{Christian Eloy}
{La cicatrice d'Ulysse}{new version 2015} %13'00''
{acusmatico}

\livel{Giorgio Nottoli}
{7 Isole}{2015 - prima esecuzione assoluta}%15'
{per flauto, percussioni e live electronics}
{flauto}{Gianni Trovalusci}
percussioni -- \textsc{Gianluca Ruggeri}
\\



%\descrizione{Potlach}{was a ritual ceremony performed among certain Native American tribes, during which the members of the community exchanged and squandered gifts, contributing to the cohesiveness of the social concord, whithin a totally inverted logic with respect to our free economy, following a reciprocity principle. The formal juxtaposition of the piece, disperses at once those elements that will constitute later on the concealed foundation that penetratrates the work.}

\descrizione{Potlach}{Il Potlach era una cerimonia rituale d’alcuni popoli di Nativi Americani, durante la quale i membri della tribù scambiavano e dilapidavano doni, in una logica completamente invertita rispetto alla nostra economia. La giustapposizione del brano \textit{dilapida} da subito quegli elementi che costituiranno il substrato che pervade l'opera. È la simbolizzazione dell'immagine che ne permette un'esegesi palesata; l'interprete realizza sé attraverso la sublimazione di gesti e momenti performativi, rappresentando la fenomenologia degli stessi. Da qui la necessità di superare il momento poietico ed estesico \textit{donando} la performance per un recupero del sistema di relazioni come tale.}

%\descrizione{Cendre}{draws on the idea from the impermanence of materiality. The title of the piece gives further meaning referring to a fragile and delicate state, but also the potential of ashes as a fertilizer for something new. The bass flute being suspended within the electronics extends space and timbre, at times detaching itself within its own identity, but repeatedly gets reabsorbed into space. The piece was written for and premiered by Mario Caroli during his Fromm Residency at Harvard University.}

\descrizione{Cendre}{prende ispirazione dall’idea della precarietà di ciò che è materiale. Il titolo del brano acquista ulteriore significato riferendosi ad uno stato di fragilità e delicatezza, ma anche alle potenzialità fertilizzanti della cenere per qualcosa di nuovo. Grazie alla sospensione del flauto basso attraverso l’elettronica, il suo spazio ed il suo timbro vengono estesi, a tratti allontanandosi dalla propria identità per poi essere riassorbito all’interno dello spazio. Il brano è stato composto ed eseguito in prima assoluta da Mario Caroli durante la Fromm Residency presso la Harvard University.}

%\descrizione{[re-spì-ro]}{Re-spi-ro (Breath) in this work, is the alternation of air movements which could produce real rhythmic and tonal structures. The sound of each breath adds and interacts itself with the previous one, drawing a music form which contracts and expands itself as well. Through the rhythmic articulation, inhalation and exhalation, the breath transforms and builds itself, removes and contradicts what has been just expressed. The digital support creates a constructive dialectic, supporting gently the concrete material. Breath transformation is described by the conduct of three different changing elements: localization, movement (direction and speed) and space size. In this way , the auditive perception of the space is transposed to the visual perception level. In this work I tried to realize the interaction between these changing elements on the acoustic level and associating with tones and pitches the breath expressive gestures of the performer.}

\descrizione{[re-spì-ro]}{Il re-spi-ro, in questa opera, è inteso come l’alternanza dei movimenti dell’aria in grado di generare vere e proprie strutture ritmiche e timbriche. Il suono di ogni respiro si aggiunge e interagisce con i precedenti, disegnando una forma musicale che si contrae e si espande.  Attraverso l’articolazione ritmica, l’inspirazione e l’espirazione, il respiro si trasforma e costruisce, dissolve o contraddice ciò che è stato appena espresso. Il supporto digitale crea una dialettica costruttiva sostenendo delicatamente il materiale concreto. La trasformazione del respiro è descritta attraverso il comportamento di tre variabili: localizzazione, movimento (direzione, velocità) e dimensione dell’ambiente. In questo modo la percezione uditiva dello spazio viene trasposta nel dominio della percezione visiva. In questo brano ho cercato di realizzare l’interazione tra queste variabili nel dominio acustico associando con timbri e altezze i gesti espressivi del respiro dell’esecutore.}

%\descrizione{La cicatrice d'Ulysse}{This title, borrowed from Erich Auerbach, German writer and critic deceased in 1957, immediately sets the scene on a plane where realism is depicted in the aesthetics of Western music. Electroacoustic music has the ability, using “concrete” sounds (with all the ambiguity implied by the word), of giving us an immediate sensation of reality, which we can all situate in relation to our references and private objects of reference; these are the profound notions of sublimitas and humilitas, which merge and unite in musical expression. Every listener will be able to decipher his or her own images of a collective mental universe from the essence of this kind of artistic creation, just like the scar by which Ulysses was recognised by his former servant woman.}

\descrizione{La cicatrice d'Ulysse}{Il titolo, preso in prestito da Erich Auerbach, scrittore e critico tedesco morto nel 1957, ambienta immediatamente la scena all’interno di un aereo, tratteggiandola con un realismo tipico dell’estetica musicale occidentale. Attraverso l’uso di suoni concreti (con tutte le ambiguità insite nella parola stessa), la musica elettroacustica possiede l’abilità di restituirci una percezione immediata della realtà, che è possibile collocare in relazione ai propri riferimenti e agli oggetti segreti cui intende riferirsi; questi sono i significati profondi di sublimitas e humilitas, che si fondono e riuniscono nell’espressione musicale. Dall’essenza di questo tipo di creazione artistica ogni ascoltatore è in grado di decodificare la propria immagine di un universo mentale collettivo, proprio come la cicatrice che permise alla serva di riconoscere Ulisse.}

%\descrizione{7 Isole}{I think the idea of the island is a fascinating metaphor, because I realize to think about many important things  as they are formed, in fact, by islands, which each contain a different world and a different feel, but together form a unitary context. In "7 Isole", each "island" is characterized by a particular combination of movement, pitches and colors of the sound. It consists of seven small pieces separated from one another, that can be performed in any order, however, strongly linked by way of forming the sound material and by a unitary constructive thinking. The instruments are used both with extended techniques that traditional, allowing different nuances both for the continuous sound as well as impulsive, both for what determined pitch and for what similar to the noise. The electronics supports and extends the sound of the instruments and, where necessary, becomes the instrument itself, completing the construction of the sound field. The dynamic distribution  in the listening space is obtained according to the principles of the method Ambisonic. Some of the processing and sound synthesis used have been developed by the author.}

\descrizione{7 Isole}{Quella dell'isola è per me una metafora affascinante, in quanto mi accorgo di pensare a molte cose importanti come fossero costituite, appunto, da isole, che contengono ciascuna un mondo diverso e quindi un diverso sentire, ma, insieme, costituiscono un contesto unitario. In 7Isole, ciascuna "isola" è caratterizzata da una particolare combinazione di movimento, altezze e colori del suono. Si tratta di sette piccoli pezzi fra loro separati, che si possono eseguire in un qualsiasi ordine, tuttavia fortemente legati dal modo di formare il materiale sonoro e da un pensiero costruttivo unitario. Gli strumenti sono utilizzati sia con tecniche estese che tradizionali, consentendo diverse sfumature sia per il suono continuo che per quello impulsivo, sia per quello ad altezza determinata e per quello simile al rumore. L'elettronica sostiene ed  estende il suono degli strumenti e, dove necessario, diviene strumento essa stessa, completando la costruzione del campo sonoro. La distribuzione dinamica nello spazio d'ascolto è ottenuta secondo i principi  del metodo Ambisonic. Alcune delle elaborazioni e sintesi del suono utilizzate sono state sviluppate dall'autore.}











\end{multicols}

\clearpage




%----------------------------------------------------------------------------------------

% ---------------------------------------------------------------------------------------------
\section*{ }

\subsection*{\textsf{Autori ed Esecutori}\\}

{\fontsize{30}{30} \svolk{\emph{Biografie}}}

\bigskip

\begin{multicols}{2}

\biografia{Natasha Barrett}{Compositrice freelance che lavora con la musica, la ricerca e l'uso creativo del suono. Ha conseguito in Inghilterra master e dottorato in composizione elettroacustica, dopo di che, nel 1999, si trasferisce in Norvegia, dove ha vissuto. Si concentrò sulla musica acusmatica e sulla musica strumentale con live electronics. Dal 1999 il suo lavoro con il suono si è ampliato fino a comprendere sound-art, installazioni sonore-architettoniche, tecniche interattive, la collaborazione con i progettisti e scienziati sperimentali e performance live e l'improvvisazione. Esempi recenti di questo includono l'uso di dati scientifici e dei processi geologici in sound-art, composizione spaziale per altoparlante a matrice emisferica e uno speciale interesse nell’HOA, e il suo terzo progetto d’installazione con il gruppo Oceano Design Research Association.

Le sue opere sono eseguite e commissionate in tutto il mondo e ha ricevuto numerosi riconoscimenti , in particolare il Music Prize Consiglio nordico (Norden / Scandinavia, 2006), Giga - Hertz Award (Germania, 2008), Edvard Prize (2004, Norvegia), Noroit- Leonce Petitot (Arras, Francia, 2002 e 1998), Bourges International Electroacoustic Music Awards (Francia del 2001, 1998 e 1995), Musica Nova (2001), IV CIMESP 2001 Concours Scrime, (Francia 2000), International Electroacoustic Creation Competition of Ciberart (Italia 2000), Concours Luigi Russolo (Italia 1995 e 1998), Prix Ars Electronica (Linz , Austria 1998), 9th International Rostrum for electoacoustic music (2002). Le sue installazioni includono un lavoro importante per la Commissione dello Stato Norvegese per l'arte negli spazi pubblici}


\end{multicols}

\clearpage

\section*{ }

%\subsection*{\textsf{}\\}

\hyphenation{Michele Andreotti Guido Capotosto Federico Coderoni Gianmarco Costa
Marco De Martino Simone Giudice Matteo Ilardo Leonardo Mammozzetti
Danilo Marro Massimiliano Mascaro Alessandro Pacetta
Federico Paganelli Ivo Papadopoulos Ivano Pecorini Susanna Rimondotto Federico Ripanti}

\begin{center}

{\fontsize{30}{30} \svolk{\emph{Organizzazione}}}
\medskip

{\fontsize{12}{12} \textsf{Dipartimento di Musica Elettronica}}\\
\vspace{.5cm}

\textbf{\textit{Comitato artistico EMUfest}}\\
Nicola Bernardini, Michelangelo Lupone, Alfredo Santoloci, Franco Sbacco
\medskip

\textbf{\textit{Comitato organizzatore}}\\
Francesco Bianco, Elena D'Alò, Paolo Gatti, Marco Giordano, Virginia Guidi, Luana Lunetta, Massimo Massimi, 	Luigi Pizzaleo, Federico Scalas, Giuseppe Silvi, Anna Terzaroli, Francesco Ziello
\medskip

\textbf{\textit{Supervisione tecnica}}\\
Federico Scalas
\medskip

\textbf{\textit{Responsabili di palco}}\\
Luana Lunetta, Massimo Massimi
\medskip

\textbf{\textit{Regia del suono}}\\
Giuseppe Silvi
\medskip

\textbf{\textit{Regia Conferenze}}\\
Anna Terzaroli
\medskip

\textbf{\textit{Tecnico di registrazione}}\\
Federico Coderoni
\medskip

\textbf{\textit{Tecnici luci}}\\
Massimiliano Mascaro, Simone Giudice
\medskip

\textbf{\textit{Ufficio Stampa}}\\
Francesco Bianco, Paolo Gatti
\medskip

\textbf{\textit{Staff esteso}}\\
Michele Andreotti, Guido Capotosto, Federico Coderoni, Gianmarco Costa,
Marco~De~Martino, Simone Giudice, Matteo Ilardo, Leonardo Mammozzetti,
Danilo Marro, Massimiliano Mascaro, Alessandro Pacetta,
Federico Paganelli, Ivo Papadopoulos, Ivano~Pecorini, Susanna Rimondotto, Federico Ripanti
\end{center}


%\begin{figure}[!h]
%\centering
%\includegraphics[width=6.4cm]{loghi.jpg}
%%\caption{}
%%\label{fig2_42}
%\end{figure}

\end{document}
