% !TEX encoding = UTF-8 Unicode
% !TEX TS-program = XeLaTex
% !TEX root = EMU2015_booklet.tex

\livel{Frei}
{Poetics}{16'50''}
{per due laptop}
{2015}

\livel{Christian Banasik}
{Ik ́}{9'}
{per flauto e live electronics}
{2013}

\livel{Marco Marinoni}
{Finita è la terra}{4'16''} %
{pianoforte e live electronics}
{2015}

\livel{Domenico De Simone}
{A[LIVE]}{5'30''}
{pianoforte, vibrafono  ed elettronica su supporto}
{2015}

\brano{Benjamin D. Whiting}
{Illumina! Arabidopsis thaliana}{9'18''}
{acusmatico}{2015}\\

\esecutore{flauto}{Elena D'Alò}
\esecutore{pianoforte}{Sara Ferrandino}
\esecutore{vibrafono}{Matteo Rossi}

\noindent \textsc{Frei}:\\
\esecutore{due laptop}{Francesco Bianco, Paolo Gatti}

% \descrizione{Poetics}{Work based on the concepts of entropy, redundancy, and noise and how these play a role on messages and information in a musical discourse. From an aesthetic point of view, everything is resolved, here, in an alternation between punctiform moments, almost static, and growing dynamic more or less sudden.}

\descrizione{Poetics}{Lavoro basato sull'improvvisazione giocata sull'alternanza fra momenti puntiformi, quasi statici, e dinamiche di crescendo più o meno improvvise. Altro tema trattato è il rapporto fra comunicabilità e incomunicabilità tramite l'indagare delle modalità di trasmissione dei messaggi musicali e vocali e i concetti di entropia, ridondanza e di rumore.}

% \descrizione{Ik ́}{\textit{Ik ́} is the name of the 2nd day in the ritual calendar of the Maya associated with breath and wind.The liturgical year consisted of twenty cycles and their glyphs, each of them thirteen days long, and had 260 days in all. For the tone material I use a contemporary folk song from Central America and a virtually generated original song of the Maya.The flute score consists of virtuoso abstracted variations mixed with the graphical-audio analysis of this particular glyph. It is divided in 20 parts.}

\descrizione{Ik ́}{\textit{Ik ́} è il nome del secondo giorno del calendario rituale dei Maya, associato al respiro e al vento. L'anno liturgico era composto da venti cicli e i loro glifi, ognuno dei quali durava tredici giorni, avendo in totale 260 giorni. Per il materiale tonale ho usato una canzone popolare contemporanea dell'America Centrale, e una canzone dei Maya generata virtualmente. La parte del flauto consiste in variazioni astratte miste all'analisi grafica-audio di questo particolare glifo. È diviso in 20 parti.}

% \descrizione{Finita è la terra}{is a small formal experiment that makes use of three piano sound objects, variants of a single musical image, which remain identical to themselves during the entire piece, arranged in a formal course partly iterative and partly structured through microvariations. These objects interact along the time axis with material fixed on support, agglomerations deriving from their possible electroacoustic processing – a live-electronics frozen, or even more than a live-electronics, an image of it, captured and distilled through an alchemical process which aims to turn something that was only movement, dispersion, probability (grain clouds applied to the three sound events of origin) in its \greco{εἴδωλον}, playing this way with the perception and memory in a continuum consisting of appearance / disappearance / memory / expectation / desire.}

\descrizione{Finita è la terra}{è un piccolo esperimento formale che fa uso di tre oggetti sonori pianistici, varianti di un'unica immagine musicale, che permangono identici a se stessi durante tutto il brano, disposti secondo un decorso formale in parte iterativo e in parte microvariato. Tali oggetti interagiscono lungo l’asse del tempo con materiali fissati su supporto, agglomerazioni derivanti da una loro possibile processazione elettroacustica – un live-electronics ghiacciato, o ancora, più che un live-electronics, una sua immagine, catturata e distillata, mediante un processo alchemico volto a trasformare qualcosa che era solo movimento, dispersione, probabilità (le nubi di granulazioni applicate ai tre eventi sonori di partenza) nel suo \greco{εἴδωλον}, giocando in questo modo con la percezione e la memoria in un continuum fatto di apparizione/sparizione/ricordo/attesa / desiderio.}

% \descrizione{A[LIVE]}{I AM NOT THERE \\ Do not stand at my grave and weep. \\ I am not there. I do not sleep. \\ I am a thousand winds that blow. \\ I am the diamond glints on snow. \\ I am the sunlight on ripened grain.  \\ I am the gentle autumn rain. \\ When you awaken in the morning’s hush \\ I am the swift uplifting rush \\ Of quiet birds in circled flight. \\ I am the soft stars that shine at night. \\ Do not stand at my grave and cry; \\ I am not there. I did not die. \\ Mary Elizabeth Frye}

\descrizione{A[LIVE]}{IO NON SONO LÌ \\ Non piangere sulla mia tomba.  \\ Io non sono lì. Io non sto dormendo.  \\ Io sono mille venti che soffiano. \\ Sono lo scintillio del diamante sulla neve. \\ Sono il sole che brilla sul grano maturo. \\ Sono la lieve pioggia d'autunno.  \\ Quando tu ti svegli nel silenzio del mattino, \\ io sono il rapido fruscio \\  degli uccelli che volano in cerchio. \\ Io sono la piccola stella che brilla di notte. \\ Non piangere sulla mia tomba;  \\ io non sono lì, la mia anima non è morta.   \\ (libera traduzione della poesia I AM NOT THERE di Mary Elizabeth Frye)}

% \descrizione{Illumina! Arabidopsis thaliana}{This piece represents the ongoing artistic and scientific collaboration between genomic biologist Aleel K. Grennan and myself. Grennan is studying the rate of photosynthesis between a natural wild type of Arabidopsis thaliana leaf and three genetically engineered mutants with different sizes of chloroplasts. I designed the majority of the sonic material in DISSCO and KYMA, incorporating Grennan’s data into several parameters, thus creating a wealth of stylized sounds.}

\descrizione{Illumina! Arabidopsis thaliana}{Questo brano rappresenta l'attiva collaborazione artistica e scientifica tra il biologo genetico Aleel K. Grennan. Grennan sta studiando l'indice di fotosintesi tra un tipo naturale selvatico di foglia Arabidopsis thaliana e tre diverse mutazioni geneticamente modificate con diverse grandezze di cloroplasti. La maggior parte dei materiali sonori sono stati progettati con DISSCO and KYMA, incorporando i dati di Grennans attraverso alcuni parametri, sono stati creati così un'abbondanza di suoni stilizzati.}

