% !TEX encoding = UTF-8 Unicode
% !TEX TS-program = XeLaTex
% !TEX root = EMU2015_booklet.tex

\livel{Javier Alejandro Garavaglia}
{Hoquetus}{9'}
{per sassofono soprano ed elettronica multi-traccia}
{2005-6}

\brano{Gustavo Delgado}
{Permanente e transitorio}{7'20''}
{acusmatico}{2015}\\

\livel{Sylvano Bussotti}
{Ultima RARA}{8'}
{per un chitarrista recitante}
{versione 2015}

\livel{James Dashow}
{Soundings in Pure Duration n.7}{10'}
{per sassofono contralto e live electronics}
{2015}

\livel{Maurizio Pisati}
{SENTI!}{15'}
{chitarra, percussione, orchestra d’archi e live electronics}
{versione 2015}

%\vspace{5mm}

\esecutore{sassofoni}{Enzo Filippetti}
\esecutore{chitarra}{Arturo Tallini}
\esecutore{percussioni}{Gianluca Ruggeri}
\esecutore{orchestra d'archi}{Orchestra Europa}

% \descrizione{Hoquetus}{The first version of the piece was commissioned by Prof. Esther Lamneck (NYU), who also premiered it at the 14th Florida Electroacoustic Music Festival on the Tárogató (April 2005). The current version is a further development, with changes in the instrumental part (including its adaptation for Saxophone) as well as in the electronics, which have been completely reprogrammed since the world premiere.Basically, the work tries to recreate in a different context the medieval technique of the Hocket. Being this technique polyphonic, the Tárogató part delivers a sort of “hiccup” technique with itself by the use of different registers. Moreover, the computer makes a Hocket-like response to the Tárogató at certain moments.}

\vspace{-5mm}
\descrizione{Hoquetus}{La prima versione di questo brano fu commissionata dal prof. Esther Lamneck (NYU), e fu premiato alla quattordicesima edizione del Florida Electroacoustic Music Festival (Aprile 2015). La versione revisionata ha avuto diversi sviluppi, con il cambio da parte degli strumenti (compreso il suo adattamento per sassofono) non che nell'elettronica, dove è stata completamente riprogrammata rispetto alla prima mondiale. Il lavoro ha cercato di ricreare in un differente contesto la tecnica medievale dell' Hoquetus. Essendo una tecnica polifonica, le parti di Tárogató fornisce una sorta di tecnica del "singhiozzo" mediante l uso di diversi registri. Inoltre, in certi momenti  il computer esegue una risposta simil-hoquetus al Tárogató.}

% \descrizione{Permanente e transitorio}{The composition deals with two seemingly opposite concepts in a such closer relationship: permanent and temporary. By the combination of different mixing and editing techniques very often used in the cinema and video games sfx, there were created very complex "impact sounds" classified by their dynamic structure as short (temporaries). Resonance frequencies have been isolated and emphasized from those materials in order to obtain stable sounds (permanents) to be used as starting points or bridges to the original impact sounds and viceversa.}

\descrizione{Permanente e transitorio}{Mediante il montaggio e la combinazione di diverse tecniche di missaggio e sound design utilizzate al cinema, ai videogiochi e alla musica elettronica, sono stati creati numerosi materiali di tipo “impact sounds” (materiale di tipo transitorio) da cui sono estratte delle risonanze (materiale di tipo permanente) dinamicamente modulate in frequenza (FM) su una patch in MaxMSP creata dall’autore.}

% \descrizione{Ultima Rara}{eng}

\descrizione{Ultima Rara}{Ultima Rara è pensata “per voce e da una a tre chitarre” la versione di questa sera riunisce le 4 figure e infatti viene proposta come versione “per un chitarrista recitante” così come proposta da Arturo Tallini con l’approvazione del compositore. Il testo e conseguentemente la musica hanno un carattere multiforme, fortemente teso all’espressione intima, seppure non disgiunta da un’ironia che qua e là fa capolino. Il brano ci racconta la fine di una storia passando dal dolore, al delirio alla malinconia. Il tipo di recitazione non è indicata in alcun modo dunque la versione di stasera è stata interamente elaborata dall’interprete.}

% \descrizione{Soundings in Pure Duration n.7}{The seventh in the series of Soundings in Pure Duration provides a wealth of timbres and trajectories in octophonic space designed to place the alto saxophone soloist in the middle of varying degrees of transformative motion, musical and  physical.  No, the saxophonist doesn't fly around the hall, but the three-dimensional temporally structured spatialization of the constantly evolving sounds creates the impression of everything, including the soloist, moving in contrappuntal complexities.}

\descrizione{Soundings in Pure Duration n.7}{Il settimo brano della serie Soundings in Pure Duration propone un'abbondanza di timbri e tragitti nello spazio ottofonico, creati per collocare il sassofono contralto al centro di una varietà di movimenti trasformativi, musicali e fisici.  No, il sassofonista non vola attraverso la sala, ma la spazializzazione dei suoni in evoluzione è strutturata  temporalmente nelle tre dimensioni così da creare l'impressione che tutto, incluso il solista, si muova nelle complessità contrappuntistiche.}

% \descrizione{SENTI!}{In SENTI! (listen up!) the sounds and gestures born in Senti? (can you hear?) and Sì. Cos’è? (Yes. What is this?) came to an end, closing a triptyque made in the two previous operas. Now the final disposition of the compositions is transformed into an essential dialogue: «Senti? Sì, cos’è? SENTI!». The triptyque is prompted by a short story by August Strindberg “Stora grusharpan” here adapted at the composer will. This leads us into the sea bottom, and from here it tells of sounds never heard before: a Piano falls into water, a storm deforms it and the fishes explore it lightly touching it’s string, while on the shore two guys are listening. The title of the first track –Senti?– it’s the question that one of the two asks to the other after hearing that sounds. In the second track the friend answers, asking too “Sì. Cos’è?”.  In the latter part, after wondering briefly on the nature of the sounds, no one asks anything or waits for an answer, until one of the two states: “sì, ma senti!”. From the brief dialogue of the two guys, still able to appreciate the marvel, is born the third score, which owes his shape to the imaginary breathe of things and sounds sinked in unreachable places.}

\descrizione{SENTI!}{In SENTI! giungono alla conclusione i gesti e alcune sonorità nate in Senti? e Sì. Cos’è?, i due primi pezzi di un trittico. Ora quindi l’ordine finale delle tre composizioni è diventato anche un dialogo essenziale: «Senti? Sì, cos’è? Senti!» Il trittico prende le mosse dalla novella di August Strindberg “Stora grusharpan” qui liberamente adattata. Ci conduce su un fondale marino e da lì racconta di sonorità mai sentite: un Pianoforte cade in acqua, una tempesta lo deforma e i pesci lo esplorano sfiorando le corde, mentre sulla riva due ragazzi ascoltano. Il titolo del primo brano –Senti?– è la domanda che uno rivolge all’altro dopo aver udito quei suoni. Nel secondo pezzo l’amico risponde, a sua volta chiedendo: “Sì, cos’è?”. Infine qui, nel terzo brano, dopo un breve scambio di supposizioni sulla natura di quei suoni, nessuno più domanda o attende risposta, ma uno dei due formula la frase forse più importante, l’esortazione ad ascoltare: “sì, ma senti!”. Dal breve dialogo dei due ragazzi che ancora sanno godere della meraviglia, nasce questa terza partitura, che deve la sua forma al respiro immaginario delle cose e dei suoni sprofondati in un luogo di solito inaccessibile.}