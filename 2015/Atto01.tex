% !TEX encoding = UTF-8 Unicode
% !TEX TS-program = XeLaTex
% !TEX root = EMU2015_booklet.tex

\acusmatici{Ursula Meyer-K\"onig}
{Allears}{2012-13}{8'}

\acusmatici{Benjamin O'Brien}
{Along the eaves}{2012-13}{8'20''}

\acusmatici{Dennis Deovides A. Reyes III}
{Bolgia}{2014}{7'31''}

\acusmatici{Dimitrios Savva}
{Balloon Theories}{2012-13}{14'30''}

\acusmatici{Jones Margarucci}
{Inhabitated Places\textunderscore Part II}{2012-13}{5'52"}


%\descrizione{Allears}{The inspiration for this work originally came from a series of intensive discussions with people who are deaf or have hearing impairments. We talked about the pros and cons of technical apparatuses such as hearing aids or cochlea implants, the different ethical and emotional responses people have to them, and the identity issues they raise. Wearing hearing aids also changes how sounds are perceived, sometimes causing interference, distortions, diminished spatial perception and noise overflow.}

\descrizione{Allears}{Originariamente l'ispirazione per questo lavoro  proveniva da una serie di intense discussioni con persone non udenti o che hanno problemi d'udito. Abbiamo parlato dei pro e dei contro di apparati tecnici, come apparecchi acustici o impianti cocleari, le diverse risposte etiche ed emotive che le persone sentono, e i problemi d'identità che sollevano. 
Indossare apparecchi acustici cambia anche come i suoni vengono percepiti, a volte causando interferenze, distorsioni,percezione spaziale ridotta e troppo pieno di rumore.}


%\descrizione{Along the Eaves}{takes its name from the following line in Franz Kafka’s “A Crossbreed” “On the moonlight nights its favorite promenade is along the eaves.” To compose the work, I developed custom software and used these programs in different ways to process and sequence my source materials, which, in this case, included audio recordings of water, babies, and string instruments. My interest is to create sonic coincidences that suggest relationships between sounds and the illusions they foster.}

\descrizione{Along the Eaves}{ prende il nome dalla riga che di Franz Kafka "Incrocio". "Sulle notti di luna la sua passeggiata preferita è lungo la grondaia" Per comporre l'opera, ho sviluppato software personalizzato e utilizzato questi programmi in modi diversi per elaborare e sequenziare i miei materiali di base, che, in questo caso, include registrazioni audio d’acqua, bambini, e di strumenti a corda. Il mio interesse è quello di creare coincidenze sonore che suggeriscono i rapporti tra i suoni e le illusioni che promuovono.}


%\descrizione{Bolgia}{Bolgia is an Italian word that means pocket or trench.  This term has been used by Dante Alighieri in his notable literary work Inferno.  Bolgia is a stereo fixed media electroacoustic composition, which depicts Alighieri’s journey to the eighth circle of hell, and his experiences to its horrific environment.  The musical gestures and sonic events of the piece evoke the different sounds and emotions of hell.}

\descrizione{Bolgia}{è una parola italiana che significa "fossa" e "luogo chiassoso in cui regna la confusione". Questo termine è stato usato da Dante Alighieri nel suo noto lavoro letterario "Inferno". Bolgia è un brano stereofonico fisso per composizioni elettroacustiche, che illustra il viaggio di Alighieri nell'ottavo girone dell'inferno, e la sua esperienza in questo posto terribile. I gesti musicali e l'evento sonoro del pezzo evocano i diversi suoni e le diverse emozioni dell'inferno.}


%\descrizione{Balloon Theories}{«I was always enjoying squeezing balloons, pressing them with my fingers until they pop… It has not been up until now that I realized why…»}

\descrizione{Balloon Theories}{«Ho sempre trovato divertente strizzare palloncini, premerli con le dita fino allo scoppio... Non mi è mai interessato fino a quando non ho capito perché...»}


%\descrizione{Inhabitated Places\textunderscore Part II}{Inhabitated Places part II is based on the concept of algorithmic composition. Although the general shape of this piece has been determined in a conventional way, every sound that one can hear are selected in real time by different algorithms written in SuperCollider. These algorithms choose randomly audio files from different folders and play them at different speeds and in different moments. It is as if we had placed several different objects in several boxes (that represent our shape), but every time we open one of these boxes the objects placed inside are positioned differently from how we had left them previously. The piece was also rendered in B-format Ambisonic.}

\descrizione{Inhabitated Places\textunderscore part II}{ è una composizione elettroacustica basata sul concetto di musica algoritmica. Sebbene la forma generale del brano sia stata determinata apriori e in modo convenzionale, tutti i suoni che ascoltiamo vengono scelti in tempo reale da vari algoritmi scritti in SuperCollider. Questi algoritmi selezionano in modo pseudocasuale dei samples da diverse cartelle e li riproducono a velocità diverse e in diversi momenti.  
È come se avessimo sistemato in una scatola (che in questo caso rappresenta la struttura dell’opera) degli oggetti in un dato ordine, ma ogni qual volta apriamo la scatola li troviamo disposti in modo differente da come li avevamo lasciati.}