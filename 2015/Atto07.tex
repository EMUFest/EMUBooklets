% !TEX encoding = UTF-8 Unicode
% !TEX TS-program = XeLaTex
% !TEX root = EMU2015_booklet.tex

\acusmatici{Antonio Mazzotti}
{I haven't seen you on the jumbotrons \\at Time Square}{2014}{10'44''}

\acusmatici{Francesc Martí}
{Speech 2}{2015}{7'53''}

\acusmatici{Alfredo Ardia, Sandro L'Abbate}
{Studio N.1}{2015}{2'}

\acusmatici{Carlo Barbagallo}
{Drowning}{2015}{8'00''}

\acusmatici{Roberto Musanti}
{Rotational Chaos}{2014}{3'50''}

\acusmatici{João Pedro Oliveira}
{Hydatos}{2012}{13'00''}

\vspace{8mm}

% \descrizione{I haven't seen you on the jumbotrons at Time Square}{'I have not seen you on the jumbotrons at TimeSSquare' was realized with the ComputerAidedAlgorithmComposition.It was conceived as a study for the computational models to produce musically meaningful results.The model which I am designing has been adopted as a tool of composition, investigating on the deep connection between sound and emotional meaning.It was implemented in “Mathematica”, “Csound” and Kyma, that uses the Pacarana as audio accelerator. "Processing" for the rendering video.}

\descrizione{I haven't seen you on the jumbotrons at Time Square}{Il modello che ho progettato indaga a fondo il legame tra le composizioni multimediali audio/video ed il significato emotivo. Il video e  l'audio in sincronia sono generati utilizzando lo stessa algoritmo. La composizione si sviluppa basandosi su pratiche di tipo algoritmico per progettazione assistita dal computer. L’ insieme di simboli gestiti da una serie di regole inserite nell’algoritmo producono espressioni complesse. Gli errori, imperfezioni e limitazioni dei particolari mezzi compositivi sono elementi costitutivi del pezzo. L’implementazione per il tramite di "Mathematica", "Csound" e’ Kyma’, che utilizza Pacarana come acceleratore audio. "Processing" per il video rendering.}

% \descrizione{Speech 2}{Speech 2 is an experimental audiovisual piece created from a series of old clips from the public affairs interview program The Open Mind. Technically, in this piece, the author has been experimenting how granular sound synthesis techniques, and pseudo-random number generators can be used for audiovisual creative works. All the piece sounds and images come from that series of clip, in other words, no other sound samples or images have been used to create the final result.}

\descrizione{Speech 2}{è un brano audio-video sperimentale ricavato da una serie di vecchi video presi da interviste del programma di affari pubblici \emph{The Open Mind}. Tecnicamente, in questo brano, l'autore ha sperimentato in quali modi le tecniche di sintesi granulare e i generatori di numeri pseudorandomici possono essere utilizzati in lavori creativi audiovisuali. Tutti i frammenti di suono e le immagini provengono da queste serie di video, in altre parole, nessun altro campione di suono o di immagini sono state utilizzate per ricreare il risultato finale.}

% \descrizione{Studio N.1}{A disoriented individual interacts with a virtual space. The body becomes an instrument, therefore able to respond to stimuli according to a specific sound logic. As any computer program, it is a system where the defects, bugs, disruptions and any kind of interferences, lead to a fragmented pace of the narrative itself. This work was born with the intention of creating a connection between what we see and what we hear, it's a study about connections and interactions of these processes. The simple sound material, sine waves and glitches, presents complex internal movements based on beats phenomenon linked to the dynamics of the video.}

\descrizione{Studio N.1}{Un individuo interagisce all’interno di uno spazio virtuale. Il corpo diviene strumento in grado di rispondere agli stimoli, secondo precise logiche sonore. Come ogni programma informatico, esso è un sistema in parte difettoso dove interferenze e interruzioni di vario genere determinano un andatura frammentata della narrazione. Il lavoro è nato con l’intenzione di creare una relazione tra ciò che si osserva e ciò che si ascolta, uno studio sul legame e sull’interazione di questi due aspetti. Il materiale sonoro utilizzato, sinusoidi e glitch, apparentemente semplice e di carattere minimalista, presenta in realtà complessi movimenti interni, dovuti al fenomeno dei battimenti, che seguono le dinamiche del video secondo precisi criteri compositivi.}

% \descrizione{Drowning}{Drowning is a three movements piece written by Carlo Barbagallo for Jean Francois Laporte's Table de Babel. Video directed & performed by Isobel Blank. Drowning: I: Surprise / Involuntary Breath Holding II: Unconsciousness III: Hypoxic Convulsions / Clinical Death}

\descrizione{Drowning}{I: Surprise/Involuntary Breath Holding II: Unconsciousness III: Hypoxic Convulsions/Clinical Death - Drowning, un invito ad annegare, è un brano in tre movimenti composto per la Babel Table, strumento musicale che utilizza l’aria compressa come sorgente di energia, d’invenzione di Jean- Francois Laporte. Ispirata alla fasi dell’annegamento, la composizione del brano è stata commissionata da Productions Totem Contemporain in collaborazione con la Scuola di Musica Elettronica del Conservatorio di Torino. Il video è stato diretto e realizzato dall’artista Isobel Blank. Il brano, nella sua versione dal vivo, è stato già eseguito a Giugno 2015 al Conservatorio G.Verdi di Torino, a Settembre alla Fountain School of Performing Arts della Dalhousie University (Halifax, Canada) e verrà eseguito a Novembre all’università di Huddersfield (UK). Musica: Carlo Barbagallo Video: Isobel Blank Babel Table: Jean-Francois Laporte}


% \descrizione{Rotational Chaos}{"Chaos rotation" is an audiovisual work that explores the relationship between images and sounds. Although the composition is abstract, because it is based primarily on the relationship between the forms, and between them and the sounds, the assembly of the material has the effect of a kind of narrative that we can define Geo/Math fiction.}

\descrizione{Rotational Chaos}{è un lavoro audiovisuale che esplora il rapporto tra suono e immagine. In primo piano, solidi di rotazione, il cui profilo è influenzato dal contenuto armonico di suoni da generatori caotici,  evidenziano il contrasto tra le simmetrie delle loro forme e il timbro sonoro. In secondo piano un sistema particellare funge da "sfondo" alla composizione. Benché si tratti di un lavoro astratto, i materiali grafici scelti e il loro montaggio, anche in rapporto al suono, invitano a una lettura “narrativa”, una sorta di geo/math-fiction.}

% \descrizione{Hydatos}{Hydatos is a greek word that means “water”. This piece is inspired on the first verses of the Old Testament (Genesis Chapter 1:2) “And the Spirit of God moved upon the face of the waters.” The audio part of this piece was commissioned by Gulbenkian Foundation, and was composed in the composer’s personal studio and at the NOVARS Center in Manchester. The video part was done at the composer’s personal studio.}

\descrizione{Hydatos}{Hydatos è una parola greca che significa “acqua”. Questo brano trae ispirazione dai primi versi del Vecchio Testamento (Genesi Capitolo 1:2) “E lo spirito di Dio aleggiava sulla superficie delle acque.” La parte sonora di questo video è stata commissionata dalla Fondazione Gulbenkian, ed è stata composta nello studio personale del compositore e al NOVARS Center a Manchester. La parte video è stata realizzata nello stesso studio personale del compositore.}

